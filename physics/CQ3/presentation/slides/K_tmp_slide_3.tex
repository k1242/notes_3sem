
13.
Next, we measured the dependence of the arc coordinate y on the time at different 
opening angles (between electrodes) 
and rotation angle (between electrodes plane and Gravitational acceleration). 
The coordinate was correspondingly measured along the electrodes. 

Using Python, the alpha and beta values ​​were taken from the captured videos, as well as the arc coordinate.


14.
First of all, we confirmed the hypothesis that the reason for the rise of the arc is the Archimedes force. The graph shows the dependence of the arc rise speed on time with different values of the rotation angles beta. Average values ​​are highlighted. In the limit of 90 degrees beta, no arc movement is observed. 
Thus, the hydrodynamic nature of motion is confirmed.



15.
In the article that inspired us, it was argued that at some point, the speed of the arc comes to a constant value. Sufficiently accurate measurements are required to verify this. The graph shows the results of the dependence of the arc coordinate on the time with different opening angles alpha.

16.
So, at some values ​​of the alpha a mode with a constant speed is reached.
I would like to see the plot of the speed versus time.
It is shown on the slide for an alpha value of 34 degrees.
The speed becomes constant, indeed.


17.
As we measured, the constant speed mode is not achieved at an angle of less than 14 degrees, which is clearly shown in the graph. Arc break occurs before stabilization on the top graph.


18.
For various values ​​of the opening angle, we've founded the average value of the speed after stabilization.


19.
This allowed us to plot the dependence of the average arc speed on the opening angle. The curve here is just a spline approximation. So, it can be seen that with increasing angle the constant speed increases.


20.
This clearly contradicts the values ​​obtained in the referenced earlier article,
which undoubtedly deserves attention. 
Note that the article did not stipulate the repeatability of the experiments, it is not known whether averaging over different launches took place, perhaps this is the reason.


21. 
To sum up, 

We worked through theory that is enough for modelling high
voltage electric arc.

We built the experimental setup and gathered and processed
the data.

We shoved that the key role in up movement of the arc is
presented by pressure gradient.

We measured dependence of the average stabilized velocity to
the opening angle, that conflicts with the results of Almazova's
article.

Algorithms