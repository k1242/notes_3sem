\subsection{Принцип наименьшего действия}

В силу необходимости инвариантности интеграла, действие для свободной частицы должно иметь вид
$$
    S = - \alpha \int_{a}^{b} ds,
$$
где интеграл берется вдоль мировой линии. В силу экстремальности интеграла $\alpha > 0$. 

Действие можно представить в виде интеграла по времени, из значения собственного времени найдём
\begin{equation}
    S = \int_{t_1}^{t_2} L \d t,
    \hspace{0.5cm} \bigg/
        dt' = \frac{ds}{c} = dt \sqrt{1 - \frac{v^2}{c^2}}.
    \bigg/ \hspace{0.5cm} 
    S = - \int_{t_1}^{t_2} \alpha c \sqrt{1 - \frac{v^2}{c^2} }.
\end{equation}
При $c \to \infty$  выражение должно перейти в классическое выражение, тогда
$$
    L = -\alpha c \sqrt{1 - \frac{v^2}{c^2} } \approx - \alpha c + \frac{\alpha v^2}{2c}
    \hspace{0.5cm} 
        \Longrightarrow
    \hspace{0.5cm} 
    \alpha = mc.
$$
Таким образом, действие для свободной материальнной точки равно 
\begin{equation}
\label{8_1}
    S = -mc \int_{a}^{b } \d s,
\end{equation}
а функция Лагранжа
\begin{equation}
\label{8_2}
    L = -mc^2 \sqrt{1 - \frac{v^2}{c^2} }.
\end{equation}

