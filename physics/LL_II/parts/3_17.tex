\subsection{Уравнения движения заряда в поле}




Допустим, заряд $e$ не велик, тогда его действием на поле можно пренебречь. То есть поле не зависит ни от координат, ни от скорости заряда.

Итак, надо бы найти уравнение движения заряда в заданном электромагнитном поле. Эти уравнения получаются варьированием действия, т.е. даются уравнения Лагранжа
\begin{equation}
    \frac{d}{dt} \frac{\partial L}{\partial \vc{v}} = \frac{\partial L}{\partial \vc{r}},
\end{equation}
где $L$ определяется формулой \eqref{16_4}.

Производная ${\partial L} / {\partial \vc{v}}$ есть обобщенный импульс частицы \eqref{16_5}. Далее имеем 
\begin{equation}
    \frac{\partial L}{\partial \vc{r}} \equiv \grad L = \frac{e}{c} \grad \vc{A} \vc{v} - e \grad \varphi.
\end{equation}
Но по \textbf{известной} формуле векторного анализа
$$
    \grad \vc{ab} = (\vc{a} \nabla) \vc{b} + (\vc{b} \nabla) \vc{a} + [\vc{b} \rot \vc{a}] + [\vc{a} \rot \vc{b}],
$$
где $\vc{a}$ и $\vc{b}$ -- любые два вектора. Применяем эту формулу к $\vc{Av}$ и помня, что \red{дифференцирование по $\vc{r}$ производится при постоянном $\vc{v}$}, находим
$$
    \frac{\partial L}{\partial \vc{r}} = \frac{e}{c} \left(\vc{v} \nabla \right) \vc{A} + \frac{e}{c} [\vc{v} \rot \vc{A}] - e \grad \varphi.
$$
 Уравнения Лагранжа, следовательно, имеют вид
 $$
     \frac{d}{dt} \left(\vc{p} + \frac{e}{c} \vc{A}\right) = \frac{e}{c} \left(\vc{v} \nabla \right) \vc{A} + \frac{e}{c} [\vc{v} \rot \vc{A}] - e \grad \varphi.
 $$
Но полный дифференциал $(d \vc{A} / d t) \d t$ складывается из двух вещей:
из изменения $(\partial \vc{A} / \partial t) \d t$ векторного потенциала со временем в данной точке пространства и из изменения при переходе от одной точки пространства к другой на расстояние $d \vc{r}$. Это вторая часть равна\footnote{
    \red{Почему???}
} $(d \vc{r} \nabla) \vc{A}$. Таким образом,
$$
    \frac{d \vc{A}}{dt} = \frac{\partial \vc{A}}{\partial t} + (\vc{v} \nabla) \vc{A}.
$$
Подставляя это в предыдущее уравнение, получаем
\begin{equation}
\label{17_2}
    \frac{d \vc{p}}{dt} = \underbrace{
        - \frac{e}{c} \frac{\partial \vc{A}}{\partial t} - e \grad \varphi
    }_{\text{I}} +
    \underbrace{
        \frac{e}{c} \left[ \vc{v} \rot \vc{A} \right]
    }_{\text{II}}
\end{equation}
где I часть не зависит от скорости частицы. II часть зависит от этой скорости:
пропорциональна величине скорости и перпендикулярна к ней.

\begin{to_def} 
    Силу первого рода, отнесенную к заряду, равному единицу, называют \textit{напряженностью электрического поля} -- $\vc{E}$.
    \begin{equation}
    \label{17_3}
        \vc{E} = - \frac{1}{c} \frac{\partial \vc{A}}{\partial t} - \grad \varphi.
    \end{equation}
\end{to_def}

\begin{to_def} 
    Множитель при скорости, точнее при $\vc{v}/c$, в силе II рода, действующей на единичный заряд, называют \textit{напряженностью магнитного поля} -- $\vc{H}$.
    \begin{equation}
    \label{17_4}
         \vc{H} = \rot \vc{A}.
    \end{equation} 
\end{to_def}

\begin{to_def} 
    Если в электромагнитном поле $\vc{E} \neq 0$, а $\vc{H} = 0$, то говорят об \textit{электрическом поле}; если же $\vc{E} = 0$, $\vc{H} \neq 0$, то поле называют \textit{магнитным}. В общем случае электромагнитное поле является наложением полей электрического и магнитного. 
\end{to_def}

\red{Отметим, что $\vc{E}$ представляет собой полярный, а $\vc{H}$ -- аксиальный вектор.} Уравнение движения заряда в электромагнитном поле можно теперь написать в виде
\begin{equation}
\label{17_5}
    \frac{d \vc{p}}{dt} = e \vc{E} + \frac{e}{c} \left[\vc{v} \vc{H}\right].
\end{equation}
Стоящее справа выражение носит название \textit{лоренцевой силы}. Первая её части -- сила, с которой действует электрическое поле на заряд, -- не зависит от скорости заряда и ориентирована по направлению поля $\vc{E}$. Вторая часть -- сила, оказываемая магнитным полем на зарядЮ -- пропорциональная скорости заряда и направлена перпендикулярно к этой скорости и к направлению магнитного поля $\vc{H}$.

Для скоростей $\ll c$, импульс $\vc{p} \approx m \vc{v}$, и уравнение движения переходит в 
\begin{equation}
    m \frac{d \vc{v}}{dt} = e \vc{E} + \frac{e}{c} \left[\vc{v} \vc{H}\right].
\end{equation}

Выведем ещё уравнение, определяющее изменение кинетической энергии частицы со временем:
$$
    \frac{d \mathscr{E}_{\text{кин}}}{dt} = \frac{d}{dt} \frac{mc^2}{\sqrt{1 - v^2 / c^2}}.
$$
Так как
$$
    \frac{d \mathscr{E}_{\text{кин}}}{dt} = \vc{v} \frac{d \vc{p}}{dt};
$$
подставляя $d \vc{p} / d t$ из \eqref{17_5} и замечая, что $[\vc{v} \vc{H}] \vc{v} = 0$, имеем
\begin{equation}
    \frac{d \mathscr{E}_{\text{кин} }}{dt}  = e \vc{E} \vc{v}.
\end{equation}
Изменение кинетической энергии со временем есть работа поля над частицей в единицу времени. Видно, что работа равна произведению скорости заряда на силу, с которой действует на него электрическое поле. Работа поля за время $dt$, т.е. при перемещении заряда на $d \vc{r}$, равна $e \vc{E} \d \vc{r}$.

Подчеркнем, что работу над зарядом производит только
электрическое поле; магнитное поле не производит работы над
движущимся в нем зарядом. Последнее связано с тем, что сила, с
которой магнитное поле действует на частицу, всегда 
перпендикулярна к ее скорости.


\phantom{42}

\noindent
\textbf{Про обращение времени:}

\phantom{42}    

Уравнения механики инвариантны по отношению к перемене знака у времени, т.е. по отношению к замене будущего прошедшим. Легко видеть, что то же самое имеет место и в ЭМ поле в теории относительности. При этом, однако, вместе с заменой $t$  на $-t$надо изменить знак магнитного поля. Действительно, уравнения движения \eqref{17_5} не меняются, если произвести замену
\begin{equation}
    t \to -t, \hspace{0.5cm} \vc{E} \to \vc{E}, \hspace{0.5cm} \vc{H} \to - \vc{H}.
\end{equation}
При этом, согласно \eqref{17_3}, \eqref{17_4}, скалярный потенциал не меняется, а векторный меняет знак:
\begin{equation}
    \varphi \to \varphi, \vc{A} \to - \vc{A}.
\end{equation}

Таким образом, если в электромагнитном поле возможно
некоторое движение, то возможно и обратное движение в поле с
обратным направлением $\vc{H}$.