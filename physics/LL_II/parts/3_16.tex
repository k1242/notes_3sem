\subsection{Четырёхмерный потенциал поля}




Действие для частицы в ЭМ поле -- \eqref{8_1} + взаимодействие частицы с полем. \textbf{Оказывается}, что это определяется одним параметром -- \textit{зарядом}\footnote{
    О единицах измерения см. \S 4.
} частицы $e$. Свойства поля характеризуются 4-вектором $A_i$, так называемым 4\textit{-потенциалом}, компоненты которого -- $f(\vc{x}, t)$. Эти величины входят в действие в виде члена
$$
    - \frac{e}{c} \int_{a}^{b} A_i dx^i,
$$
где функции $A_i(x_i)$ берутся в точках мировой линии частицы. Множитель $1/c$ -- для удобства. 

Таким образом, действие для заряда в ЭМ поле имеет вид
\begin{equation}
    S = \int_{a}^{b} \left(
        -mc \d s - \frac{e}{c} A_i \d x^i
    \right).
\end{equation}

\begin{to_def} 
     Три пространственные компоненты 4-вектора $A^i$ образуют трёхмерный вектор $\vc{A}$, называемый \textit{векторным потенциалом} поля. Временную же компоненту называю \textit{скалярным потенциалом}, обозначим её как $A^0 = \varphi$. Таким образом,
    \begin{equation}
        A^i = \left(\varphi, \vc{A} \right).
    \end{equation}
\end{to_def}

Поэтому интеграл действия можно написать в виде
$$
    S = \int_{a}^{b} \left(-mc \d s + \frac{e}{c} \vc{A} \d \vc{r} - e \varphi \d t\right),
$$
или, вводя скорость частицы $\vc{v} = d \vc{r} / dt$ и переходя к интегрированию по времени, в виде
\begin{equation}
    S = \int_{t_1}^{t_2} \left(
    - mc^2 \sqrt{1 - \frac{v^2}{c^2} } + \frac{e}{c}  \vc{A} \vc{v} - e \varphi
    \right) \d t.
\end{equation}
Подынтегральное выражение есть функция Лагранжа для заряда в ЭМ поле:
\begin{equation}
\label{16_4}
    L = - mc^2\sqrt{1 - \frac{v^2}{c^2} } + \frac{e}{c} \vc{A} \vc{v} - e \varphi.
\end{equation}
Отличие от \eqref{8_2} для свободной частицы в члене $\frac{e}{c} \vc{A} \vc{v} - e \varphi$, который и описывают взаимодействие заряда с полем.


Производная $\partial L / \partial \vc{v}$ есть \textit{обобщенный импульс} частицы; обозначим его $\vc{P}$. Находим
\begin{equation}
\label{16_5}
    \vc{P} = \frac{
    m \vc{v}
    }{
    \sqrt{1 - v^2 / c^2}
    } + \frac{e}{c} \vc{A} = \vc{p} + \frac{e}{c} \vc{A}.
\end{equation}

%%%%%%%%%%%%%%%%%%%%%%%%%%%%%%%%%%%%%%%%%%%%%%%%%%%%%%%%%%%%%%%%%%%%%%%%%%%%%%%%%%%

\noindent
\textbf{Про Гамильтониан}:

\phantom{42}

Из функции Лагранжа можно найти функцию Гамильтона частицы в поле по \textbf{известной общей} формуле 
\begin{equation}
    \mathscr{H} = \vc{v} \frac{\partial L}{\partial \vc{v}} - L.
\end{equation}
Подставляя сюда \eqref{16_4}
\begin{equation}
    \mathscr{H} = \frac{mc^2}{\sqrt{1 - v^2/c^2}} + e \varphi.
\end{equation}
Функция Гамильтона, однако, должна быть выражена не через скорость, а через обобщенный импульс частицы. По предыдущим двум формулам видно, что соотношение между $\mathscr{H}  - e\varphi$ и $\vc{P} - \frac{e}{c} \vc{A}$ -- такое же, как и в отсутствие поля (совпадение?), т.е.
\begin{equation}
\label{16_7}
    \left(\frac{\mathscr{H} - e\varphi}{c} \right)^2 = m^2c^2 + \left(\vc{P} - \frac{e}{c} \vc{A}\right)^2,
\end{equation}
или иначе:
\begin{equation}
    \mathscr{H}  = \sqrt{ m^2 c^4 + c^2 \left(\vc{P} - \frac{e}{c} \vc{A} \right)^2} + e\varphi.
\end{equation}

Для малых скоростей, т.е. в классической механике, функция Лагранжа \eqref{16_4} переходит в 
\begin{equation}
    L = \frac{mv^2}{2}  + \frac{e}{c} \vc{A} \vc{v} - e \varphi.
\end{equation}
В этом приближении
$$
    \vc{p} = m \vc{v} = \vc{P} - \frac{e}{c} \vc{A},
$$
и мы находим следующее выражение для функции Гамильтона:
\begin{equation}
    \mathscr{H} = \frac{1}{2m} \left(\vc{P} - \frac{e}{c} \vc{A}\right)^2 + e \varphi.
\end{equation}

Наконец, выпишем уравнение Гамильтона--Якоби для частицы в электромагнитном поле. Оно получается заменой функции Гамильтона обобщенного импульса $\vc{P}$ на $\partial S / \partial \vc{r}$, а самого $\mathscr{H}$ -- на $-\partial S/\partial t$. Таким образом, получим из \eqref{16_7} 
\begin{equation}
    \left(
        \text{grad} \, S - \frac{e}{c} \vc{A}
    \right)^2 - \frac{1}{c^2} \left(\frac{\partial S}{\partial t} + e \varphi \right)^2 + m^2 c^2 = 0 
\end{equation}

%%%%%%%%%%%%%%%%%%%%%%%%%%%%%%%%%%%%%%%%%%%%%%%%%%%%%%%%%%%%%%%%%%%%%%%%%%%%%%%%%%%





