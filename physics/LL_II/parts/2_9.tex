\subsection{Энергия и импульс}

Вспомним, что $\vc{p} = \partial L / \partial \vc{v}$ -- \textit{импульс} частицы. Из \eqref{8_2} находим 
\begin{equation}
\label{9_1}
    \vc{p} = \frac{m \vc{v}}{\sqrt{1 - \frac{v^2}{c^2} }} .
\end{equation}
Производная от импульса по времени есть сила, действуящая на частицу. Пусть $\vc{F} \bot \vc{v}$ , тогда 
\begin{equation}
    \frac{d \vc{p}}{d t}  = \frac{m}{\sqrt{1 - \frac{v^2}{c^2} }}  \frac{d \vc{v}}{dt}.
\end{equation}
Если же $\vc{F} || \vc{v}$, тогда 
\begin{equation}
    \frac{d \vc{p}}{dt} = \frac{m}{\left(1 - \frac{v^2}{c^2} \right)^{3/2}} \frac{d \vc{v}}{d t} .
\end{equation}
Собственно, отношения силы к ускорению различно.

\textit{Энергией} $\E$ частицы называется величина $\E = \vc{p} \vc{v} - L$. Подставляя выражения \eqref{8_2} и \eqref{9_1} для $L$ и $\vc{p}$, получим
\begin{equation}
\label{9_4}
    \E = \frac{mc^2}{\sqrt{1 - \frac{v^2}{c^2}}}.
\end{equation}
При малых скоростях, разлагая по степеням $v/c$
$$
    \E = mc^2 + \frac{mv^2}{2} ,
$$
что соответсвует классическому случаю.

$$
    \left(\frac{ac}{1 + a^2} \right)
$$

