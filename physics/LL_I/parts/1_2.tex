\subsection{Принцип наименьшего действия}

\begin{to_thr}[Принцип Гамильтона / принцип наименьшего действия]
     Каждая механическая система характеризуется\footnote{
         \red{Почему нет более высоких производных и зачем нужны скорости?}
     } определенной функцией
     $$
         L(q_1, \ldots, q_s, \dot q_1, \ldots, \dot q_s, t) \equiv L(q, \dot q, t),
     $$
     удовлетворяющая следующему условию.

     Пусть $t_1$ и $t_2$ -- два момента времени. Тогда между этими положениями система движется таким образом, что 
     \begin{equation}
     \label{2_1}
         S = \int_{t_1}^{t_2} L(q, \dot q, t) 
     \end{equation}
     имеет наименьшее возможное значение на малых участках и экстремальное для всей траектории. Функция $L$ называется \textit{функцией Лагранжа} данной системы, а интеграл \eqref{2_1} --- \textit{действием}.
\end{to_thr}


Рассмотрим сначала одномерный случай. Пусть $q = q(t)$ есть как раз та функция, для которой $S$ имеет минимум. Тогда $S$ возрастает при $q \leadsto q + \delta q$, где $\delta q (t)$ -- функция, малая во всем интервале от $t_1$ до $t_2$, -- \textit{вариация} функции $q(t)$. Также учтём, что\footnote{
    \red{Почему?} 
}
\begin{equation}
    \label{2_3}
    \delta q(t_1) = \delta q(t_2) = 0
\end{equation}

Изменение $S$ при замене $q$ на $q+\delta q$ дается разностью
$$
    \int_{t_1}^{t_2} L(q + \delta q, \dot{q}, t) \d t - \int_{t_1}^{t_2} L(q, \dot{q}, t) \d t.
$$

Во-первых, при разложение по степеням $\delta q$ и $\delta \dot{q}$, получим обращение в нуль совокупности членов первого порядка --- \textit{первая вариация / вариация} интеграла. Тогда принцип наименьшего действия запишется в виде
\begin{equation}
    \delta S = \delta \int_{t_1}^{t_2} L(q, \dot{q}, t) \d t = 0
\end{equation}
или, произведя варьирование\footnote{
    \red{Проделать аккуратно ручками!}
}:
% нужны ли "\bigg/"
$$
    \int_{t_1}^{t_2} \left(\frac{\partial L}{\partial q} \partial q  + \frac{\partial L}{\partial \dot{q}} \delta \dot{q}\right) = 0 
    \hspace{0.5cm} \bigg/
    \overset{\delta \dot{q} = \frac{d}{dt} \delta q}{\underset{\Longrightarrow}{}}
    \bigg/ \hspace{0.5cm}
    \delta S = 
    \cancel{\frac{\partial L }{\partial \dot{q}} \delta q \bigg|_{t_1}^{t_2}}
    + \int_{t_1}^{t_2} \left(
        \frac{\partial L}{\partial q} - \frac{d}{dt} \frac{\partial L}{\partial \dot{q}}
    \right) \delta q \d t = 0.
$$
Оставшийся интеграл равен нулю при произвольных $\delta q$. Это возможно, только если подынтегральное выражение $\equiv 0$. Таким образом, получаем
\begin{equation}
    \frac{d}{dt} \frac{\partial L}{\partial \dot{q}} - \frac{\partial L}{\partial q} = 0.
\end{equation}
При наличии нескольких степеней свободы в принципе наименьшего действия должны независимо варьироваться $s$ различных функций $q_i (t)$. Тогда мы получаем $s$ уравнений:
\begin{equation}
\label{2_6}
    \boxed{
        \frac{d}{dt} \frac{\partial L}{\partial \dot{q_i}} - \frac{\partial L}{\partial q_i} = 0,
    } \hspace{0.25cm} 
    \left(i = 1, 2, \ldots, s\right) \text{ --- \textit{уравнения Лагранжа}}
\end{equation}
Но только в механике\footnote{
    В вариационном исчислении -- \textit{уравнения Эйлера}.
}. Если функция Лагранжа данной механической системы известна, то уравнения устанавливают связь между ускорениями\footnote{
    \red{Почему?}
}, скоростями и координатами -- соответствуют уравнениям движения системы.


% пропущены абзацы

Стоит заметить, что функция Лагранжа определена лишь с точностью до прибавления к ней полной производной от любой функции координат и времени.
