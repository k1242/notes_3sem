

\begin{to_thr}[теорема Эйлера об однородных функциях]
\label{thr:E}
     Пусть дифференцируемая функция $f \colon \mathbb{R}^n \ni X \to \mathbb{R}$ такая, что $\forall \vc{r} \in X$, $\forall \lambda \in \mathbb{R}$ выполняется:
     \begin{equation}
     \label{E1}
         f(\lambda \vc{r}) = \lambda^k f(\vc{r}), \text{ --- однородная функция, тогда} \hspace{0.5cm} 
          \boxed{
         \sum\nolimits_i r_i \frac{\partial f}{\partial r_i} = k f(\vc{r})
         }.
     \end{equation}
\end{to_thr}

\begin{proof}[$\triangle$]
    Дифференцируя условие \eqref{E1}, по $\lambda$, получим соотношение
    $$
        \sum\nolimits_i \frac{\partial f}{\partial (\lambda r_i)} 
        \frac{\partial (\lambda r_i)}{\partial \lambda} =
        k \lambda^{k-1} f(\vc{r}),
    $$
    верное $\forall \lambda$. Достаточно принять $\lambda = 1$.
\end{proof}




