\subsection{Функция Лагранжа системы материальных точек}

\begin{to_def} 
    Система материальных точек, взаимодействующих только друг с другом, --- \textit{замкнутая}. 
\end{to_def}

Считая, что мы не выходим за пределы классической механики, введём функцию координат $U$~---~\textit{потенциальная энергия}.
\begin{equation}
\label{5_1}
     L = \sum_{a} \frac{m_a}{2} v_a^2 - U(\vc{r}_1, \vc{r}_2, \ldots) 
 \end{equation} 

Тот факт, что $U$ -- функция только координат, означает, что взаимодействие <<распространяются>> мгновенно -- проделки абсолютности времени и принципа относительности Галилея. Также стоит заметить, что $L(\ldots, t) = L(\ldots, -t)$. То есть если в система возможно движение, то всегда возможно и обратное движение. 

Зная функцию Лагранжа, мы можем составить уравнения движения
\begin{equation}
\label{5_2} % and 5_3
    \frac{d}{dt} \frac{\partial L}{\partial \vc{v}_a} = \frac{\partial L}{\partial \vc{r}_a} 
    \hspace{0.5cm} 
        \overset{\text{\eqref{5_1}}}{\underset{\Longrightarrow}{}} 
    \hspace{0.5cm} 
    \boxed{
        m_a \frac{d \vc{v}_a}{dt}  = - \frac{\partial U}{\partial \vc{r}_a}. 
    } \text{ --- \textit{уравнения Ньютона}}
\end{equation}

\begin{to_def} 
\label{5_4}
    Вектор $\vc{F}_a = - \dfrac{\partial U}{\partial \vc{r}_a}$ -- \textit{сила}, действующая на $a$-ю точку.
\end{to_def}

Если для описания движения используются не декартовы координаты точек, а произвольные обобщенные координаты $q_i$, то для получения $L$ надо произвести соответствующее преобразование
$$
    x_a = f_a(q_1, \ldots, q_s), \hspace{0.1cm} \dot{x}_a = \sum_k \frac{\partial f_a}{\partial q_k} \dot{q}_k, \hspace{0.1cm} \ldots 
    \hspace{0.25cm} \Rightarrow \bigg/ L = \frac{1}{2} \sum_a m_a \left(\dot{x}_a^2 + \dot{y}_a^2 + \dot{z}_a^2\right) \bigg/ \Rightarrow \hspace{0.25cm} 
    \boxed{L = \frac{1}{2} \sum_{i, k} a_{ik}(q) \cdot \dot{q}_i \dot{q}_k - U(q)},
$$
где $a_{ik}$ -- функции только координат. Кинетическа энергия $= f(\dot{\vc{q}}, \vc{q})$.


%%%%%%%%%%%%%%%%%%%%%%%%%%%%%%%%%%%%%%%%%%%%%%%%%%%%%%%%%%%%%%%%%%%%%%%%%%%%%%%%%%%

рассмотрим теперь незамкнутую систему $A$, взаимодействующую с другой системой $B$, совершающей заданное движение. В таком случае $A$ движется в заданном внешнем поле (системы $B$). Для нахождения $L_A$ системы $A$ воспользуемся $L_{A+B}$, заменив $q_B$ функциями времени. Предполагая систему $A + B$ замкнутой,
$$
    L = T_A(q_A, \dot{q}_A) + T_B (q_b, \dot{q}_B) - U(q_A, q_B).
$$
Т.к. $T_B (q_b, \dot{q}_B)$ зависит только от времени (и потому\footnote{
    \red{Почему?}
}-- полная производная функции времени), то
$$
    L_A = T_A (q_A, \dot{q}_A) - U(q_A, q_B(t)).
$$
Получается единственно отличие -- явная зависимость от времени для $U$.

Так, для одной частицы во внешнем поле общий вид функции Лагранжа
\begin{equation}
    L = \frac{m}{2} v^2 - U(\vc{r}, t), \hspace{0.25cm} \text{и уравнения движения}
    \hspace{0.25cm} m \dot{\vc{v}} = - \frac{\partial U}{\partial \vc{r}} .
\end{equation}

\begin{to_def} 
    \textit{Однородным} называют поле, во всех точках которого действует $\vc{F} \equiv \const$, соотвественно $U = - \vc{F} \vc{r}$. 
\end{to_def}

