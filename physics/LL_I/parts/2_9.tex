\subsection{Момент импульса}

Поговорим про закон сохранения от \textit{изотропии пространства}. Введём $\delta \vc{\varphi}$, такой, что $(\delta \vc{\varphi})^2 = (\delta \varphi)^2$ и направление совпадает с осью поворота. Тогда 
\begin{equation}
    \delta \vc{r} = [\delta \vc{\varphi} \cdot \vc{r}], \hspace{0.5cm} 
    \delta \vc{v} = [\delta \vc{\varphi} \cdot \vc{v}].
\end{equation}
При повороте $L \equiv \const$, т.е.
$$
    \delta L = \sum_a \left(
        \frac{\partial L}{\partial \vc{r}_a} \delta \vc{r}_a + \frac{\partial L}{\partial \vc{v}_a} \delta \vc{v}_a
    \right) = 0
    \hspace{0.5cm}  
        \overset{\partial L/ \partial {v}_a = {p}_a}{\underset{\partial L / \partial {r}_a = {\dot{p}}_a}{\Longrightarrow}} 
    \hspace{0.5cm} 
    \sum_a \left(
        \vc{\dot{p}}_a [\delta \vc{\varphi} \cdot \vc{r}_a] 
        + \vc{p}_a [\delta \vc{\varphi} \cdot \vc{v}_a]
    \right) = 0,
$$
или, \red{производя циклическую перестановку множителей}  и вынося $\delta \vc{\varphi}$ за знак суммы, имеем
$$
    \delta \vc{\varphi} \sum\nolimits_a \left(
        [\vc{r}_a \cdot \vc{\dot{p}}_a] + [\vc{v}_a \cdot \vc{p}_a]
    \right) =
    \delta \varphi \frac{d}{dt} \sum\nolimits_a [\vc{r}_a \cdot \vc{p}_a] = 0
    \hspace{0.5cm} \overset{\text{т.к. } \forall \delta \vc{\varphi}}{\underset{\Longrightarrow}{}} \hspace{0.5cm} 
    \frac{d}{dt} \sum\nolimits_a [\vc{r}_a \cdot \vc{p}_a] = 0.
$$

\begin{to_def} 
    При движении системы сохраняется векторная величина
    \begin{equation}
         \vc{M} = \sum\nolimits_a [\vc{r}_a \cdot \vc{p}_a],
     \end{equation} 
     называемая \textit{моментом импульса} (или \textit{моментом}, \textit{вращательным моментом}, \textit{угловым моментом}) системы.
\end{to_def}

\noindent
\fbox{
    \begin{minipage}[t]{1\textwidth}
        Этим исчерпываются (\red{почему?}) \textbf{аддитивные} интегралы движения. Таким образом, всякая замкнутая система имеет всего \textbf{семь} таких интегралов: $E$ и три компоненты векторов $\vc{P}$ и $\vc{M}$.
    \end{minipage}
}

\phantom{42} 

% на 32 странице доказательство
Стоит заметить, что в другой смещённой СО ($\vc{r}_a = \vc{r}_a' + \vc{a}$)
\begin{equation}
    \vc{M} = \vc{M}' + [\vc{a} \cdot \vc{P}].
\end{equation}
Из этой формулы видно, что в случае, когда $\vc{P} = 0$, момент системы не зависит от выбора начала координат. 

В случае же, когда $K'$ движется относительно $K$ со скоростью $V$ верно, что
\begin{equation}
    \vc{M} = \vc{M}' + \mu [\vc{R} \cdot \vc{V}].
\end{equation}

Если система $K'$ есть та, в которой система покоится как целое, то $\vc{V}$ есть скорость центра инерции, а $\mu \vc{V}$ -- её полный импульс $\vc{P}$ (относительно $K$). Тогда
\begin{equation}
    \vc{M} = \vc{M}' + [\vc{R} \cdot \vc{P}].
\end{equation}
Другими словами, $\vc{M}$ механической системы складывается из её <<собственного момента>> относительно СО: $V_{\text{ц.м.}}=0$, и момента $[\vc{R} \cdot \vc{P}]$, связанного с её движением как целого.

Во внешнем же поле $\vc{M}$ сохраняется в проекции на такую ось, относительно которой данное поле симметрично. 

% тут пропущен абзац, см. 34 страницу.

Проекция момента на какую-либо ось ($z$) может быть найдена по формуле
\begin{equation}
    M_z = \sum\limits_a \frac{\partial L}{\partial \dot{\varphi}_a} = \sum\limits_a m_a r_a^2 \dot{\varphi}_a,
\end{equation}
где координата $\varphi$ есть угол поворота вокруг оси $z$. 