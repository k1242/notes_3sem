\subsection{Принцип относительности Галилея}

\begin{to_def} 
    По первому закону Ньютона (\red{да?}) всегда можно найти такую систему отсчёта, по отношению к которой пространство однородно и изотропно, а время -- однородно. Такая система называется \textit{инерциальной}.
\end{to_def}

Что ж, тогда $L \neq f(\vc{r}, \vc{v}, t)$, т.е. $L = L(v)$. Тогда для $L$ верно, что $\partial L / \partial \vc{r} = 0$, и потому из уравнений Лагранжа можно получить
$$
    \frac{d}{dt} \frac{\partial L}{\partial \vc{v}} = 0, 
    \hspace{0.5cm} 
    \Rightarrow
    \bigg/
    \frac{\partial L}{\partial \vc{v}} = \const,
    \hspace{0.25cm} 
    \frac{\partial L}{\partial \vc{v}} \equiv \frac{\partial L}{\partial \vc{v}} (\vc{v})
    \bigg/
    \Rightarrow
    \hspace{0.5cm} 
    \vc{v} = \const    
$$

Получается в ИСО всякое свободное движение происходит с $\vc{v} = \const$, --- \textit{закон инерции}.

Тот замечательный факт, что таких ИСО много, и что законы там одинаковые формирует \textit{принцип относительности Галилея}.
