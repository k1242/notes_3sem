\subsection{Импульс}

Теперь о сохранении, связанном с однородностью пространства (\red{наверное}).

В силу этого свойства замкнутой системы не меняются при любом параллельном переносе системы. В таком случае рассмотрим сдвиг на бесконечно малые $\vc{\varepsilon}$ каждой точки системы, т.е. $\vc{r}_a \leadsto \vc{r}_a + \vc{\varepsilon}$. Изменение функции $L$ в результате бесконечно малого изменения координат при неизменных скоростях есть
$$
    \delta L = \sum_A \frac{\partial L}{\partial \vc{r}_a} \delta \vc{r}_a =
    \vc{\varepsilon} \sum_a \frac{\partial L}{\partial \vc{r}_a}.
$$
Т.к. это верно $\forall \vc{\varepsilon}$, то 
\begin{equation}
\label{7_1}
    \delta L = 0 \hspace{0.25cm} \Longleftrightarrow \hspace{0.25cm} 
    \sum_a \frac{\partial L}{\partial \vc{r}_a} = 0,
    \hspace{0.25cm} \overset{\eqref{5_2}}{\underset{\Longrightarrow}{}} 
    \hspace{0.25cm} 
    \sum_a \frac{d}{dt} \frac{\partial L}{\partial \vc{v}_a} =
    \frac{d}{dt} \sum_a \frac{\partial L}{\partial \vc{v}_a} = 0.
\end{equation}

\begin{to_def} 
    В замкнутой системе векторная величина 
    \begin{equation}
         \vc{P} = \sum_a \frac{\partial L}{\partial \vc{v}_a} 
         \overset{\eqref{5_1}}{\underset{=}{}} 
         \sum_a m_a \vc{v}_a
     \end{equation}
    остается неизменной при движении. Вектор $\vc{P}$ называется \textit{импульсом} системы. 
\end{to_def}

Стоит обратить внимание, что \eqref{7_1} о том, что $\partial L / \partial \vc{r}_a = - \partial U / \partial \vc{r}_a = \vc{F}_a$. То есть $\eqref{7_1} \Leftrightarrow \sum_a \vc{F}_a = 0$. Это, в принципе, и есть третий закон Ньютона.



Если движении описывается $q_i$, то 
\begin{equation}
    p_i = \frac{\partial L}{\partial \dot{q}_i} 
\end{equation}
называются \textit{обобщенными импульсами}, а производные
\begin{equation}
    F_i = \frac{\partial L}{\partial q_i} 
\end{equation}
называются \textit{обобщенными силами}. Так уравнения Лагранжа имеют вид
\begin{equation}
    \dot{p}_i = F_i.
\end{equation}










