% %%%%%%%%%%%%%%%%%%%%%%%%%%%%%%%%%%%%%%%%%%%%%%%%%%%%%%%%%%%%%%%%%%%%%%%%%%%%
% \section{Principal variables and constants}
% %%%%%%%%%%%%%%%%%%%%%%%%%%%%%%%%%%%%%%%%%%%%%%%%%%%%%%%%%%%%%%%%%%%%%%%%%%%%


% The heat flux potential (потенциал теплового потока) -- 

% The Prandtl and turbulent Prandtl numbers -- 

% The eddy diffusivity for momentum (вихревой коэффициент диффузии для импульса) -- 

% The specific enthalpy (удельная энтальпия) --



%%%%%%%%%%%%%%%%%%%%%%%%%%%%%%%%%%%%%%%%%%%%%%%%%%%%%%%%%%%%%%%%%%%%%%%%%%%%
\section{Principal equations (I article)}
%%%%%%%%%%%%%%%%%%%%%%%%%%%%%%%%%%%%%%%%%%%%%%%%%%%%%%%%%%%%%%%%%%%%%%%%%%%%


\subsection{Hand waving}

I would like to write the energy equation: the rate of change of energy per unit volume. To do this, we look at heat exchange with the environment and at movement in an electromagnetic potential field.

In fact $E^2\sigma$ -- specific energy density per unit time. We also have some heat exchange with the environment.
The heat flux potential can be defined by
\begin{equation}
    S = \int_0^T \varkappa \d T,
\end{equation}
So we characterize it as $\div \grad S = \nabla^2 S$, --  specific heat exchange. Then
\begin{equation*}
    P = \sigma E^2 + \div \grad S.
\end{equation*}
We expect to see something like that.

\subsection{According to the article}


Taking into consideration only a small part of the arc, we
define a set of axes $(r,x)$ in cylindrical coordinates.
\textbf{The energy equation} for a \textbf{steady arc} is
\begin{equation}
    \rho U \frac{\partial h}{\partial x} + \rho V \frac{\partial h}{\partial r} 
    =
    \frac{1}{r} \frac{\partial }{\partial r} 
    \left[
        r \left(\frac{\mu}{P_r} + \frac{\rho \varepsilon_m}{P_{rt}} \right) 
        \frac{\partial h}{\partial r} 
    \right] + 
    \sigma E^2,
\end{equation}
and \textbf{the continuity equation} by
\begin{equation} % вроде готовы в это верить. 
    \div \vc{j}_m = \frac{1}{r} \frac{\partial }{\partial r} (r \rho V) + \frac{\partial }{\partial x} (\rho U) = 0,
\end{equation}
where $\rho$ is the density, $U$ the axial velocity, $V$ the radial velocity, $h$ the specific enthalpy, $\mu$ the viscosity, $\varepsilon_m$ the eddy diffusivity for momentum, $P_\text{t}$ and $P_\text{rt}$ the Prandtl and turbulent Prandtl numbers, respectively, $\sigma$ the electrical conductivity, and $E$ the voltage gradient.

The heat flux potential can be defined by
\begin{equation*}
    S_{(T)} = \int_0^T \varkappa_{(T)} \d T,
\end{equation*}
so that the \textbf{molecular diffusion term}
\begin{equation*}
    \frac{1}{r} \frac{\partial }{\partial r} \left[
        r \left(\frac{\mu}{P_r} \right) \frac{\partial h}{\partial r} 
    \right] \equiv \nabla^2 S
\end{equation*}
in the energy equation can be replaced.


%%%%%%%%%%%%%%%%%%%%%%%%%%%%%%%%%%%%%%%%%%%%%%%%%%%%%%%%%%%%%%%%%%%%%%%%%%%%
\subsection{Assumptions}
%%%%%%%%%%%%%%%%%%%%%%%%%%%%%%%%%%%%%%%%%%%%%%%%%%%%%%%%%%%%%%%%%%%%%%%%%%%%


To describe the arc, we use a simplified model in which
rough assumptions are made. These assumptions were used 
by Maecker and Frind to describe cylindrical arc behavior. 
\begin{enumerate*}
    \item Flow is axially symmetrical.
    \item Viscous dissipation and Lorentz forces can be neglected.
    \item  Radial pressure gradient is negligible compared to the
        static pressure.
    \item  Radiation is neglected.
    \item  Principal transfer of energy is produced by conduction
        and convection.
\end{enumerate*}


We consider two arc regions separately -- 
\textbf{the plasma core}, 
also called plasma string, 
and 
\textbf{the outer flame} or weak ionized ring.




%%%%%%%%%%%%%%%%%%%%%%%%%%%%%%%%%%%%%%%%%%%%%%%%%%%%%%%%%%%%%%%%%%%%%%%%%%%%
\subsection{The plasma string model}
%%%%%%%%%%%%%%%%%%%%%%%%%%%%%%%%%%%%%%%%%%%%%%%%%%%%%%%%%%%%%%%%%%%%%%%%%%%%

Based on experimental results, although the arc is moving, the plasma channel can be considered to be \textbf{in a steady state with a constant radius}. Physical properties along the center line of the plasma string are assumed to be the same at
every point on the line. As a result of the longitudinal convection affecting the arc, the form is assumed to be cylindrical.

Let us consider a small part of the plasma string which is assumed to be identical to any other section of the plasma arc.

The emission of light being the same all along the string,
we assume
\begin{equation}
    \frac{\partial h}{\partial x} = 0.
\end{equation}
Turbulent effects and radial convection can be neglected for heat exchanges
\begin{equation}
    V \frac{\partial h}{\partial r} = 0.
\end{equation}
The energy equation is rewritten
\begin{equation}
    0   =
    \frac{1}{r} \frac{\partial }{\partial r} 
    \bigg[
        r \bigg(\frac{\mu}{P_r} + 
        \underbrace{\frac{\rho \varepsilon_m}{P_{rt}}}_{0?}
        \bigg) 
        \frac{\partial h}{\partial r} 
    \bigg] + \sigma E^2,
\end{equation}
rewriting in a different form,
\begin{equation}
    \nabla^2 S + \sigma E^2 = 0,
\end{equation}
which corresponds to the well known (авторами статьи) \textbf{Elenbaas–Heller equation}.


%%%%%%%%%%%%%%%%%%%%%%%%%%%%%%%%%%%%%%%%%%%%%%%%%%%%%%%%%%%%%%%%%%%%%%%%%%%%
\subsection{The arc core-outer flame transition}
%%%%%%%%%%%%%%%%%%%%%%%%%%%%%%%%%%%%%%%%%%%%%%%%%%%%%%%%%%%%%%%%%%%%%%%%%%%%

At the boundary of the arc core and the outer flame, the temperature profile in the plasma string is assumed to be invariable with a constant conducting radius $r_c$.


Convection is, therefore, the main cause of plasma string  cooling. It constricts the arc as the difference in velocity between arc–core and the surrounding region increases. The constant plasma string radius is evidence of an equivalent
constant convection effect expressed in terms of an axially symmetrical convection flow.

The convection term does not appear in the energy balance equation of the plasma string, but is implicitly taken into account in the input value of the electric field and power per unit of length deduced from experiments. 

In the conductive inner part of the discharge, the electrical conductivity is given as a linear function of heat flux potential
\begin{equation}
    \sigma = \beta (S - S_{\text{c}}),
\end{equation}
where $S_C$ is the heat flux potential for $\sigma=0$ correspondings to the conduction radius $r_c$.

 The conduction radius is 
\begin{equation}
    \vc{r}_c \sim \frac{1}{\sqrt{\beta} E} .
\end{equation}
The power per unit of discharge length $w$ is related to $S_0$ which corresponds to the axis heat flux potential ($r=0$), on which the temperature is $T_0$, according to the following equation:
\begin{equation*}
    S_0 = \frac{w}{2\pi} + S_c.
\end{equation*}
