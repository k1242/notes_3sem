Магнетизм вещества обусловлен тремя причинами: 
1) орбитальным движением электронов вокруг атомных ядер; 
2) собственным вращением, или спином, электронов; 
3) собственным вращением, или спином, атомных ядер. 

Атомы вещества, совершая беспорядочное тепловое движение, в отсутствие внешнего магнитного поля обычно ориентированы хаотически. 
При наложении внешнего магнитного поля атомы полностью или частично ориентируются в направлении этого поля, и тогда компенсация нарушается --- тело \textit{намагничивается}.
Тела способные к намагничиванию называют \textit{магнетиками}, а очень способные к этому называются ещё \textit{ферромагнетиками}.
В теории Максвелла речь идёт об усредненном по времени поле от частиц совершающих финитное движение.

Молекулярные токи принято характеризовать вектором \textit{намагначиваемости} $\vc{M}$. Рассмотрим косой цилиндр:

Пусть поверхностные токи текут в плоскости параллельной торцам цилиндра. Если $\vc{m}$ -- средний магнитный момент молекулы, то $\vc{M} = n \vc{m}$. То есть $\vc{M}$ будет перпендикулярен плоскостям торцов цилиндра. 
Пусть $\alpha$ -- угол между $\vc{M}$ и осью цилиндра $\vc{L}$, тогда как для магнитного момента: $V \vc{M} = S L \vc{M} \cos \alpha$.
 С другой стороны $V \vc{M} = \vc{I}_\text{мол} \vc{S}/c = L i_\text{мол} \vc{S}/c$.
 Приравнивая получаем (где $\vc{l}$ -- единичный вектор по оси):

 \begin{equation}
 	i_\text{мол}= i_\text{m} = c \vc{M} \cos \alpha = c (\vc{M}\cdot \vc{l}) = c \vc{M}_\text{l}.
 \end{equation}

 Орбитальные и спиновые вращения электронов и атомных ядер в отношении возбуждаемого ими магнитного поля эквивалентны \textit{орбитальным} токам --- токам циркулирующим в атомах вещества. Учтём эти токи в циркуляции индукции магнитного поля:
\begin{equation}
	\oint_{(L)} \vc{B} \cdot \d \vc{l} = \frac{4 \pi}{c} (I_{\text{проводов}} + I_{\text{молекулярный}}).
\end{equation}
Выразим молекулярный ток через вектор намагничивания $\vc{M}$: 
\begin{equation}
	i_\text{m} \d \vc{l} = c \vc{M}_\text{l} \d \vc{l} = c (\vc{M} \d \vc{l}) \hspace*{1 cm} \leadsto \hspace*{1 cm} I_\text{мол} = c \oint_{(L)} (\vc{M} \d \vc{l}).
\end{equation}
Внесём это выражение в циркуляцию вектора магнитной индукции:
\begin{equation}
	\oint \underbrace{(\vc{B} - 4 \pi \vc{M})}_{\vc{H}}\d \vc{l} = \frac{4 \pi}{c} I_\text{проводов} \hspace*{2 cm} \rot{\vc{B}} = \frac{4 \pi}{c}\vc{j} + c \rot{\vc{M}}.
\end{equation}
Ввели вектор напряженности магнитного поля: $\vc{H} = \vc{B} - 4 \pi \vc{M}$. Который подобно индукции электрического поля будет включать в себя молекулярные токи оставляя ``свободные''. Тогда уравнения Максвелла примут более удобный вид:
\begin{equation}
	\oint H \d \vc{l} = \frac{4 \pi}{c}I \hspace*{2 cm} \rot{H} = \frac{4 \pi}{c} \vc{j}.
\end{equation}

Из уравнений Максвелла так же получим граничные условия. 
\begin{align}
	&\div{\vc{B}} = 0 \hspace*{1 cm} &\leadsto& \hspace*{1 cm}&B_{\text{1n}} = B_{\text{2n}}\\
	&\rot \vc{H} = \frac{4 \pi}{c}I \hspace*{1 cm} &\leadsto& \hspace*{1 cm}&H_\text{2t} - H_\text{1t} = \frac{4 \pi}{c} I.
\end{align}

В выводе циркуляции будем предполагать, что вдоль границы раздела течет поверхностный ток проводимости с линейной плотностью $i$. Применим теорему о циркуляции к бесконечно малому прямоугольному контуру, высота которого пренебрежимо мала по сравнению 
с длиной основания. Тогда можно пренебречь вкладом в циркуляцию, который вносят боковые стороны. И получить написанную выше формулу.
