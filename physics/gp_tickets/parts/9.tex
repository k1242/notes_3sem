Майкл Фарадей в 1831 году открыл \textit{электромагнитную индукцию}.

На движущуюся перемычку в замкнутом контуре, через которую проходит магнитное поле, действует сила Лоренца, которая будет ускорят электроны в ней:
\begin{equation}
	e\vc{E} = \vc{F} = \frac{e}{c}[\vc{v} \times \vc{B}].
\end{equation}
Cила создаваемая этим полем $\vc{E}$ называется \textit{электродвижущей силой}. В целом для магнитного потока пронизывающего рамку:
\begin{equation}
	\varepsilon = -\frac{1}{c}\frac{d \Phi}{d t}.
\end{equation}

Джеймс Кларк Максвелл же расширил понятие электромагнитной индукции, из его уравнений следует, что всякое изменение потока магнитного поля во времени возбуждает электрическое поле в пространстве, даже в отсутствии проводников:
\begin{equation}
	\oint_{S}(\vc{E} \d \vc{s}) = -\frac{1}{c} \frac{\partial \Phi}{\partial t} \hspace*{2 cm} \rot{\vc{E}} = -\frac{1}{c} \frac{\partial \vc{B}}{\partial t}
\end{equation}
Электрическое поле возбуждаемое магнитным полем, в силу отсутствия зарядов является не потенциальным, а вихревым.

\textbf{Правило Ленца}: индукционный ток всегда имеет такое направление, что он ослабляет действие причины, возбуждающей этот ток.

Что видно из выражения для электромагнитной индукции, которая всегда действует против изменения магнитного потока. Позже Ле Шателье и Браун обобщили правило Ленца на все физические явления.

Продемонстировать правило можно, как это сделал Элиу Томсон. Катушка с сердечником ставится теперь вертикально. На железный сердечник надевается широкое толстое алюминиевое кольцо. 
В катушку посылается переменный ток от городской сети, возбуждающий в кольце индукционный ток противоположного направления. Эти токи отталкиваются. Если замкнуть ток в катушке, то сила отталкивания подбросит алюминиевое кольцо.

В массивных проводниках, движущихся в магнитных полях или помещенных в переменные магнитные поля, возбуждаются вихревые индукционные токи, называемые токами Фуко. 
По физической природе они ничем не отличаются от индукционных токов, возбуждаемых в линейных проводах. 