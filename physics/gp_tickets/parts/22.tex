Пусть есть диэлектрик без свободных зарядов и токов проводимости  $\vc{j} = 0$ с $\mu, \varepsilon$. 

Однаков при постоянных полях, как мы уже выяснили, уравнения Максвелл обратятся в тождесвенный 0, соотвественно рассмотрим переменные поля. 
Давайте выберем 4-потенциал так, чтобы $\varphi = 0$. Тогда
\begin{equation*}
    \vc{E} = - \frac{1}{c} \frac{\partial \vc{A}}{\partial t}, \hspace{0.5cm} 
    \vc{H} = \rot \vc{A}.
\end{equation*}
Далее,
\begin{equation*}
    \rot \vc{H} = \frac{1}{c} \frac{\partial \vc{E}}{\partial t}, \hspace{0.5cm} \Rightarrow \hspace{0.5cm} 
    \rot \rot A = - \nabla \vc{A} + \grad \div \vc{A} = - \frac{1}{c^2} \frac{\partial^2 \vc{A}}{\partial t^2}.
\end{equation*}
Выберем потенциал так, чтобы
\begin{equation*}
    \div \vc{A} = 0.
\end{equation*}
Тогда уравнение приобретает вид
\begin{equation}
    \nabla \vc{A} - \frac{1}{c^2} \frac{\partial^2 \vc{A}}{\partial t^2} = 0,
    \text{ \ \ --- уравнение \textit{д'Аламбера} или \textit{волнове уравнение}}.
\end{equation}
Применяя к нему $\rot$ и $\partial/\partial t$, можем убедиться, что напряженности $\vc{E}$ и $\vc{H}$ удоволетворяют таким же волновым уравнениям.  
\begin{equation}
    \left\{\begin{aligned}
        \rot \vc{E} &= - \frac{1}{c} \frac{\partial \vc{B}}{\partial t} \\
        \rot \vc{H} &= \frac{1}{c} \frac{\partial \vc{D}}{\partial t}
    \end{aligned}\right.;
    \hspace{0.1cm} 
    \left\{\begin{aligned}
        \vc{B} &= \mu \vc{H} \vphantom{\frac{1}{2}}\\
        \vc{D} &= \varepsilon \vc{E}  \vphantom{\frac{1}{2}}
    \end{aligned}\right.;
    \hspace{0.2cm} \Rightarrow \hspace{0.2cm} 
    \left\{\begin{aligned}
        \nabla \times \left[ \nabla \times \vc{E} \right] &= \cancel{\nabla \left(\nabla \cdot \vc{E}\right)} - \Delta \vc{E} \vphantom{\frac{1}{2}} \\
        \nabla \times \left[ \nabla \times \vc{E} \right] &= -\frac{\partial }{\partial t} \rot \vc{B}
    \end{aligned}\right. ;
    \hspace{0.5cm} \Rightarrow \hspace{0.5cm} 
    \boxed{
        \Delta \vc{E} = \frac{\varepsilon \mu}{c^2} \frac{\partial^2 E}{\partial^2 t}  = 0
    }.
\end{equation}
Аналогично для $\vc{B}$ мы можем записать, что
\begin{equation*}
    \frac{\partial^2 \vc{B}}{\partial^2 t}  = \frac{c^2}{\varepsilon \mu}  \Delta \vc{B}.
\end{equation*}
Общее решение такого уравнения имеет вид
\begin{equation*}
    \xi = f_1(z-vt) + f_2 (z+vt).
\end{equation*}
При произвольном виде $f_1, \ f_2$ они удоволетворяют волновому уравнению. Тогда
\begin{equation*}
    v_1^2 f_1'' = \frac{c^2}{\varepsilon \mu} f_1'',
    \hspace{0.5cm} \Rightarrow \hspace{0.5cm} 
    \boxed{
        v = c / \sqrt{\varepsilon \mu}   
    }.
\end{equation*}
Получается, что любое возмущение распростроняется с конечной скоростью $v$. Какие ещё у нас следствия?
\begin{enumerate*}
    \item Существуют электромагнитные волны;
    \item Они поперечны: $\vc{E}, \ \vc{H} \colon \vc{k} \bot \{\vc{E}, \vc{H}\}$;
    \item Распространение происходит с конечной скоростью $c / \sqrt{\varepsilon \mu}$;
    \item В вакууме $v  \approx 3 \cdot 10^{10} \approx c$, таким образом свет -- ЭМ волна.
\end{enumerate*}

Найдём связь между электрической и магнитными компонентами волны. Пусть
\begin{equation*}
    E_x = f(x \mp v t); \hspace{0.5cm} 
    B_y = g(z \mp vt),
    \hspace{0.5cm} \Rightarrow \hspace{0.5cm} 
    f'' = \pm \frac{v}{c} g'',
    \hspace{0.5cm} \Rightarrow \hspace{0.5cm} 
    f = \pm \frac{v}{c} g + \const.
\end{equation*}
Прямой подставновкой убедились, что
\begin{equation*}
    E_x = \pm \frac{1}{\sqrt{\varepsilon\mu}} B_y, \hspace{0.5cm} \Leftrightarrow 
    \hspace{0.5cm} 
    \boxed{
        \varepsilon E_x^2 = \mu H_y^2.   
    }
\end{equation*}

\subsubsection*{Плоские волны}

Рассмотрим частный случай ЭМ волн, когда поле зависит только от одной координаты, скажем $x$ (и от времени) -- \textit{плоская} волна. Тогад уравнения поля примут вид
\begin{equation*}
    \frac{\partial^2 f}{\partial t^2}  - c^2 \frac{\partial^2 f}{\partial x^2}  = 0,
\end{equation*}
где под $f$ подразумевается любая компонения $\vc{E}$ или $\vc{H}$. Перепишем уравнение в виде
\begin{equation*}
    \left(
        \frac{\partial }{\partial t} - c \frac{\partial }{\partial x} 
    \right)
    \left(
        \frac{\partial }{\partial t} + c \frac{\partial }{\partial x} 
    \right) f = 0
\end{equation*}
и введем новые переменные 
\begin{equation*}
    \xi = t - \frac{x}{c}, \hspace{0.5cm} \eta = t + \frac{x}{c},
    \hspace{0.5cm} \Rightarrow \hspace{0.5cm} 
    t  = \frac{1}{2} \left(\vph \eta + \xi \right),
    \hspace{0.5cm} 
    x = \frac{c}{2} \left(
        \vph \eta - \xi
    \right).
\end{equation*}
Тогда
\begin{equation*}
    \frac{\partial }{\partial \xi} = \frac{1}{2} \left(
        \frac{\partial }{\partial t} - c \frac{\partial }{\partial x} 
    \right),
    \hspace{0.5cm} 
    \frac{\partial }{\partial \eta} = \frac{1}{2} \left(
        \frac{\partial }{\partial t}  + c \frac{\partial }{\partial x} 
    \right),
    \hspace{0.5cm} \Rightarrow \hspace{0.5cm} 
    \frac{\partial^2 f}{\partial \xi \partial \eta} = 0.
\end{equation*}
Очевидно, что его решение имеет вид
\begin{equation*}
    f = f_1(\xi) + f_2(\eta),
\end{equation*}
где $f_1$ и $f_2$ -- произвольные функции. Таким образом
\begin{equation}
    f = f_1 \left(t - \frac{x}{c} \right) + f_2 \left(t + \frac{x}{c} \right).
\end{equation}
Пусть, например, $f_2 = 0$. Выясним смысл этого решения. Очевидно, что для 
\begin{equation*}
    t - \frac{x}{c} = \const \ \Rightarrow \ f = \const
    \hspace{0.5cm} \Rightarrow \hspace{0.5cm} 
    \text{при \ }x = \const + ct
\end{equation*}
волна имеет одинаковые значения, то есть распростроняется вдоль $x$ со  скоростью, равной скорости света $c$. Пусть $\vc{n}$ -- единичный вектор вдоль волны, тогда
\begin{equation}
    \vc{H} = \left[\vc{n} \times \vc{E}\right]. 
\end{equation}
Повторимся, именно поэтому эм волны называют \textit{поперечными}. Поток энергии в плоской волне
\begin{equation*}
    \vc{S} = \frac{c}{4\pi} \left[\vc{E} \times \vc{H}\right] = \frac{c}{4\pi} \left[ 
        \vc{E} \times \left[\vc{n} \times \vc{E}\right]
    \right] = \frac{c}{4\pi} E^2 \vc{n} = \frac{c}{4\pi} H^2 \vc{n} = c W \vc{n},
\end{equation*}
где $W = (E^2+H^2)/8\pi = E^2/4\pi$, в согласии с распространением со скоростью света. 



\subsubsection*{Монохроматическая плоская волна}
Важный частный случай ЭМ волн -- волны, в которых поле является простой периодической функцией, такая волна называется \textit{монохроматической}. Все компоненты зависят от времени посредством множителя $\cos (\omega t + \alpha)$, где $\omega$ -- \textit{циклическая частота}. Тогда волновое уравнение примет вид
\begin{equation*}
    \nabla f + \frac{\omega^2}{c^2} f = 0,
    \hspace{1cm} \left(
            \text{а для сред \textit{уравнение Гельмгольца} }
            \left\{\begin{aligned}
                \Delta \vc{E} + k^2 \vc{E} &= 0 \\        
                \Delta \vc{B} + k^2 \vc{B} &= 0 \\
            \end{aligned}\right.,
            \hspace{0.5cm} 
            k^2 = \mu \varepsilon \frac{\omega^2}{c^2}.\right)
\end{equation*}
В плоской волне поле является функцией только $t - x/c$, поэтому если плоская волна монохроматична, то ее поле является простой периодической функцией. Векторный потенциал такой волны удобнее всего написать в виде вещественной части комплексного выражения:
\begin{equation*}
    \vc{A} = \Re \{
        \vc{A}_0 \exp(-i\omega(t-x/c))
    \}.
\end{equation*}
Здесь $\vc{A}_0$ -- некоторый постоянный комплексный вектор. Очевидно, что и напряженности $\vc{E}$ и $\vc{H}$ в такой волне будут иметь аналогичный вид с той же частотой $\omega$. Величины
\begin{equation}
    \lambda = \frac{2\pi c}{\omega} \text{\ \ --- \ \ \textit{длина волны}}.
\end{equation}
По сути это период изменения поля координате $x$ в заданный момент времени $t$. Вектор
\begin{equation}
    \vc{k} = \frac{\omega}{c} \vc{n}, \text{\ \ --- \ \ \textit{волновой вектор}}.
\end{equation}
где $\vc{n}$ -- единичный вектор в направлении распространения волны. С его помощью уравнение можем переписать в виде
\begin{equation*}
    \vc{A} = \Re\{
        \vc{A}_0 \exp\left(
            i(\vc{k} \cdot \vc{r} - \omega t)
        \right)
    \},
\end{equation*}
инвариантном к выбору системы координат. Величину, с множителем $i$ называют \textit{фазой волны}. 

До тех пор, пока нащи операции линейны, можем опустить знак взятия вещественной части и получить
\begin{equation*}
    \vc{A} = \vc{A}_0 \exp\left(
        i (\vc{k} \cdot \vc{r} - \omega t)
    \right),
\end{equation*}
подставив в волновое уравнение, получим связь между напряженностями и векторным потенциалом плоской монохроматической волны в виде
\begin{equation}
    \vc{E} = ik \vc{A},  \hspace{0.5cm} \vc{H} = i \left[\vc{k} \times \vc{A} \right].
\end{equation}
Рассмотрим отдельно $\vc{E}$. Нетрудно получить, что
\begin{equation*}
    E_y = b_1 \cos \left(
        \omega t - \vc{k} \cdot \vc{r} + \alpha
    \right),
    \hspace{0.5cm} 
    E_z = \pm b_2 \sin \left(
        \omega t - \vc{k} \cdot \vc{r} + \alpha
    \right).
\end{equation*}
Отсюда следует, что
\begin{equation*}
    \frac{E_y^2}{b_1^2} + \frac{E_z^2}{b_2^2}  = 1.
\end{equation*}
Получается, что в каждой точке пространства вектор $\vc{E}$ вращается в плоскости, нормально к распространению волны, при чём его конец описывает эллипс. Такая волна называется \textit{эллиптически поляризованной}. 

Если $b_1$ или $b_2$ равны 0, то волну называют \textit{линейно поляризованной}. 



% \subsubsection*{Волноводы}


% Если мы хотим видеть в волноводе целое число волн, то
% \begin{equation*}
%     \omega^2 = \frac{c^2 \pi^2}{\varepsilon \mu} \left[
%         \left(\frac{n_x}{a_x}\right)^2 + 
%         \left(\frac{n_y}{a_y}\right)^2 + 
%         \left(\frac{n_z}{a_z}\right)^2  
%     \right].
% \end{equation*}