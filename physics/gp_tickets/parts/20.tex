Естественно ввести \textit{тензор электромагнитного поля}:
\begin{equation}
{F}_{ik} = 
    \begin{pmatrix}
        0    & E_x & E_y  & E_z   \\
        -E_x  & 0    & -B_z & B_y   \\
        -E_y & B_z  & 0    & -B_x  \\
        -E_z & -B_y & B_x & 0     \\
    \end{pmatrix}_{ik},
    \hspace{2cm} 
{F}^{ik} = 
    \begin{pmatrix}
        0    & -E_x & -E_y  & -E_z   \\
        E_x  & 0    & -B_z & B_y   \\
        E_y & B_z  & 0    & -B_x  \\
        E_z & -B_y & B_x & 0     \\
    \end{pmatrix}^{ik}
\end{equation}
Тогда уравнения Максвелла запишутся в виде
\begin{equation}
    \boxed{
        \varepsilon^{iklm} \partial_k F_{lm} = 0, \hspace{0.5cm} 
        \partial_k F^{ik} = - \frac{4\pi}{c} j^i
    },
\end{equation}
где $j^i = (\rho c, \vc{j})$. Прямой подстановкой тензора ЭМ поля нетрудно убедиться, что

\phantom{42}

\noindent
\begin{minipage}[t]{0.45\textwidth}
\noindent
Дифференциальная форма в СГС:
    \begin{align}
        \vphantom{\oint }
        \div \vc{D} &= 4 \pi \rho \\
        \vphantom{\oint }
        \div \vc{B} &= 0 \\
        \vphantom{\oint }
        \rot \vc{E} &= -\frac{1}{c} \frac{\partial \vc{B}}{\partial t}  \\
        \vphantom{\oint }
        \rot \vc{H} &= \frac{4\pi}{c} \vc{j} + \frac{1}{c} \frac{\partial \vc{D}}{\partial t} 
    \end{align}
\end{minipage}
\hfill
\begin{minipage}[t]{0.45\textwidth}
Интегральная форма в СГС:
    \begin{align}
       \oint \vc{D} \cdot d \vc{s} &= 4\pi Q \\
       \oint \vc{B} \cdot d \vc{s} &= 0 \\
       \oint \vc{E} \cdot d \vc{l} &= - \frac{1}{c} \frac{d}{dt} \int \vc{B} \cdot d \vc{s} \\
        \oint \vc{H} \cdot \d \vc{l} &= \frac{4\pi}{c} I + \frac{1}{c} \frac{d}{dt} \int \vc{D} \cdot \d \vc{s}.
    \end{align}
\end{minipage}

\begin{description*}
    \item[$\vc{E}$]  --- напряженность электрического поля;
    \item[$\vc{H}$]  --- напряженность магнитного поля;
    \item[$\vc{D}$]  --- электрическая индукция;
    \item[$\vc{B}$]  --- магнитная индукция.
\end{description*}

\subsubsection*{Материальные уравнения}

В проводниках связь между плотностью тока и напряжённостью электрического поля выражается в линейном приближении \textit{законом Ома}:
\begin{equation*}
    \vc{j} = \sigma \vc{E},
\end{equation*}
где $\sigma$ -- \textit{удельная проводимость среды}.

В среде сторонние электрические и магнитные поля вызывают поляризация $\vc{P}$ и намагничивание вещества $\vc{M}$.
Тогда
\begin{align*}
    \rho_\text{b} &= - \nabla \cdot \vc{P} \\
    \vc{j}_\text{b} &= c \nabla \times \vc{M} + \frac{\partial \vc{P}}{\partial t} ,
\end{align*}
Далее, по определению
\begin{align*}
    \vc{D} &= \vc{E} + 4\pi \vc{P}, &\vc{B} &= \vc{H} + 4 \pi \vc{M} \\
\end{align*}
Что в случае линейной поляризации или линейной намагниченности можно записать, как  
$$
    \left\{\begin{aligned}
        \vc{P} &= \chi_{\text{e}} \vc{E}, \\
        \vc{M} &= \chi_{\text{m}} \vc{H}, \\
    \end{aligned}\right.
    \hspace{0.5cm} \Rightarrow \hspace{0.5cm} 
    \left\{\begin{aligned}
         \vc{D} &= \varepsilon \vc{E} = (1 + 4\pi \chi_{\text{e}}) \vc{E}, \\
       \vc{B} &= \mu \vc{H} = (1 + 4 \pi \chi_{\text{m}}) \vc{H}.
    \end{aligned}\right.
$$
где $\varepsilon$ -- \textit{\cancel{относительная} диэлектрическая проницаемость}, $\mu$  — \textit{относительная магнитная проницаемость}, $\chi_{\text{e}}$  -- \textit{диэлектрическая восприимчивость}, $\chi_{\text{m}}$ -- \textit{магнитная восприимчивость}.

Наконец, в однородных средах верно, что
\begin{equation*}
    \left\{\begin{aligned}
        \div \vc{E} &= 4\pi \frac{\rho}{\varepsilon},  \\
        \div \vc{B} &= 0,
    \end{aligned}\right.
    \hspace{1cm} 
    \left\{\begin{aligned}
        \rot \vc{E} &= - \frac{1}{c} \frac{\partial \vc{B}}{\partial t}, \\
        \rot \vc{B} &= \frac{4\pi}{c} \mu \vc{j} + \frac{\varepsilon \mu}{c} \frac{\partial \vc{E}}{\partial t},
    \end{aligned}\right.
\end{equation*}
где в оптическом диапазоне принято $n = \sqrt{\varepsilon \mu}$.

\subsubsection*{Граничные условия}
Опять же, в СГС,
\begin{equation*}
    \left\{\begin{aligned}
        (\vc{E}_1 - \vc{E}_2) \times \vc{n}_{1,2} &= 0, \\
        (\vc{H}_1 - \vc{H}_2) \times \vc{n}_{1,2} &= \frac{4\pi}{c} \vc{j}_\text{s},
    \end{aligned}\right.
    \hspace{1cm} 
    \left\{\begin{aligned}
        \left(\vc{D}_1 - \vc{D}_2\right) \cdot \vc{n}_{1,2} &= - 4\pi \rho_\text{s}, \\
        \left(\vc{B}_1 - \vc{B}_2\right) \cdot \vc{n}_{1, 2} &= 0,
    \end{aligned}\right.
\end{equation*}
где $\rho_{\text{s}}$ -- поверхностная плотность свободных зарядов, $\vc{j}_\text{s}$ -- плотность поверхностных свободных токов вдоль границы. 

Эти граничные условия показывают непрерывность нормальной компоненты вектора магнитной индукции, и непрерывность на границе областей тангенциальных компонент напряжённости электрического поля. 

\subsubsection*{Уравнение непрерывности}

Источники полей $\rho, \vc{j}$ не могут быть заданы произвольным образом. Применяя операцию дивергенции к четвёртому уравнению (закон Ампера—Максвелла) и используя первое уравнение (закон Гаусса), получаем уравнение непрерывности
\begin{equation*}
    \nabla \cdot \vc{j} + \frac{\partial \rho}{\partial t} = 0,
    \hspace{0.5cm} \Leftrightarrow \hspace{0.5cm} 
    \oint_S \vc{j} \cdot \d \vc{s} = - \frac{d }{d t} \int_V \rho \d V.
\end{equation*}


