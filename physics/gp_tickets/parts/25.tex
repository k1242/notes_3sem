\textit{Волновод} -- цилиндр с идеально проводящими стенками, внутри диэлектрик с $\varepsilon, \mu$. Формально есть следующая система
\begin{equation*}
    \left\{\begin{aligned}
        \div \vc{E} &= 0, \\
        \div \vc{B} &= 0,
    \end{aligned}\right.
    \hspace{0.5cm} 
    \left\{\begin{aligned}
        \rot \vc{E} = - \frac{1}{c} \frac{\partial \vc{B}}{\partial t}, 
        \rot \vc{B} = \frac{\varepsilon \mu}{c} \frac{\partial \vc{E}}{\partial t},
    \end{aligned}\right.
    \hspace{0.5cm} 
    \left\{\begin{aligned}
        E_{\bot}, H_{\nparallel} (x=0) &= 0 \ (\text{скачком}),\\
        E_{\nparallel}, H_{\bot} (x=0) &= 0 \ (\text{плавно}),
    \end{aligned}\right.
\end{equation*}
решение которой хотим найти.

Решение будем искать в виде монохроматической плоской волны $\vc{E}(x,y,z) \cdot \exp\left(-i(\omega t- \vc{k} \cdot \vc{r})\right)$, тогда
\begin{equation*}
    \left\{\begin{aligned}
        \frac{\partial^2 E_z}{\partial x^2}  +  \frac{\partial^2 E_z}{\partial y^2} + \left(
            \frac{\varepsilon \mu}{c^2} \omega^2 - k^2
        \right) \cdot E_z = 0 \\
        \frac{\partial^2 B_z}{\partial x^2}  +  \frac{\partial^2 B_z}{\partial y^2} + \left(
            \frac{\varepsilon \mu}{c^2} \omega^2 - k^2
        \right) \cdot B_z = 0
    \end{aligned}\right.
    \hspace{1cm} 
    \left[\begin{aligned}
        E_z |_S &= 0 \\
        \left(\partial B_z / \partial n \right)\big|_S &= 0
    \end{aligned}\right.
\end{equation*}
Эти граничные условия могут быть выполнены поочереди. 

Для начала рассмотрим TM волну, или E-волну, -- поперечно-магнитные волны:
\begin{equation*}
    B_z = 0 \ \forall z, \hspace{1cm} 
    E_z |_S = 0.
\end{equation*}
Возможны TEM поперечные электромагнитные волны,
\begin{equation*}
    E_z = 0, \hspace{0.5cm} 
    B_z = 0, \hspace{0.5cm} 
    \left(
        B_\tau \neq 0, \ E_\tau \neq 0
    \right),
    \hspace{0.5cm} \Rightarrow \hspace{0.5cm} 
    k^2 = \varepsilon \mu \frac{\omega^2}{c^2}.
\end{equation*}
Действительно, давайте введем $\gamma^2 = \frac{\varepsilon \mu}{c^2} \omega^2 - k^2$. Тогда
\begin{equation*}
    \frac{\partial^2 E_z}{\partial x^2} + \frac{\partial^2 E_z}{\partial y^2} + \gamma^2 E_z = 0.
\end{equation*}
Решения существуют только для некоторых $\gamma$, при некоторых $\gamma_\lambda$. Во-первых $\gamma_\lambda^2 > 0$. Аналогично $k^2 > 0$,   
\begin{equation*}
    k^2_\lambda = \frac{\varepsilon \mu}{c^2} \omega^2 - \gamma^2_\lambda > 0,
    \hspace{0.5cm} \Rightarrow \hspace{0.5cm} 
    \omega_{\text{гр }} = c\frac{\gamma_{\lambda}}{\sqrt{\varepsilon \mu}}.
\end{equation*}
Поясним, почему так. Пусть $f'' + \gamma^2 \varphi = 0$ и $f|_{\text{границе}} = 0$. Тогда $T = 2 \pi / \gamma$ такой, чтобы помещалось целое число волн. Важно, что $\gamma_\lambda$ определяется геометрией, а
\begin{equation*}
    k = \frac{\sqrt{\varepsilon \mu}}{c} \sqrt{\vph \omega^2 - \omega_{\text{гр}}^2}.
\end{equation*}
что и является искомым дисперсионным соотношением. Можем найти фазовую скорость
\begin{equation*}
    v_{\text{ф}} = \frac{\omega}{k} = \frac{c}{\sqrt{\varepsilon \mu}} \cdot \frac{
    1
    }{
    \sqrt{
        1 - \frac{\omega^2_{\text{гр}}}{\omega^2} 
    }
    }  > \frac{c}{\sqrt{\varepsilon \mu}} .
\end{equation*}
Поэтому введем величину групповой скорости, с которой возможен перенос информации
\begin{equation*}
    v_{\text{гр}} = \frac{d \omega}{dk} = \frac{c}{\sqrt{\varepsilon \mu}} \sqrt{
    1  -\frac{\omega_{\text{гр}}^2}{\omega^2} 
    } < \frac{c}{\sqrt{\varepsilon \mu}}.
\end{equation*}

\subsubsection*{Прямоугольный волновод}

Рассмотрим ТЕ волну. 
Решение может быть найдено в виде
\begin{equation}
    B_{z_{nm}} (x, y) = B_0 \cos \left(
        \frac{m \pi x}{a} 
    \right) \cos \left(
        \frac{n\pi x}{b} 
    \right),
\end{equation}
а для частоты можем найти, что
\begin{equation}
    \gamma^2_{mn} = \pi^2  \left(
        \frac{m^2}{a^2} + \frac{n^2}{b^2} 
    \right),
    \hspace{1cm} 
    \omega_{mn} = \frac{c}{\sqrt{\varepsilon \mu}} \pi \sqrt{
    \frac{m^2}{a^2} + \frac{n^2}{b^2} 
    }.
\end{equation}