\subsubsection*{Давление света}
При полном поглощении запишем следующее:
\begin{equation*}
    w = \frac{E^2}{4\pi} = pc, \hspace{0.5cm} p = \frac{E^2}{4\pi c} 
    \hspace{0.5cm} \Rightarrow \hspace{0.5cm} 
    P = \frac{1}{dS} \frac{d (p\d V)}{d t} = \frac{E^2}{4\pi},
    \hspace{0.5cm} 
    \langle P\rangle = \frac{\langle E^2\rangle}{4\pi}.
\end{equation*}
Или, по-другому
\begin{equation*}
    \left\{\begin{aligned}
        \frac{\partial \vc{E}}{\partial z} &= \rot \vc{E} = - \frac{1}{c} \frac{\partial \vc{H}}{\partial t}, \\
        \frac{\partial \vc{H}}{\partial z} &= \rot \vc{H} = \frac{4\pi}{c} \vc{j} + \frac{1}{c} \frac{\partial E}{\partial t}.
    \end{aligned}\right.
    \hspace{0.5cm} \Rightarrow \hspace{0.5cm} 
    \frac{\partial }{\partial z} \left(\frac{E^2+H^2}{8\pi} \right) = - \frac{1}{c} \frac{\partial \vc{H}}{\partial t} \cdot \frac{\vc{E}}{4\pi}  + \frac{4\pi}{c} \vc{j} \cdot \frac{\vc{B}}{4\pi} + \frac{1}{c} \frac{\partial \vc{E}}{\partial t} \frac{\partial \vc{H}}{\partial 4\pi} = \vc{f} + \frac{1}{8\pi c} \frac{\partial }{\partial t} \left(\vc{E} \cdot \vc{H}\right).
\end{equation*}
Таким образом
\begin{equation*}
    \frac{\partial }{\partial z} 
    \vph
    \langle w\rangle= \langle f \rangle
    \hspace{0.5cm} \Rightarrow \hspace{0.5cm} 
    \langle P\rangle = \int_0^{\infty} \langle f\rangle\d z = w(0) - w(\infty) = w(0) = \frac{\langle E^2\rangle}{4\pi}.
\end{equation*}
Что и требовалось доказать!) Кстати,
\begin{equation*}
\overline{S} = c \cdot \overline{\mathcal W_{\text{эл}}}, \hspace{0.5cm} \Rightarrow \hspace{0.5cm} 
    \vc{g}_{\text{эм}} = \frac{1}{c^2} \vc{S} = \frac{1}{4\pi c} \left[
        \vc{E} \times \vc{H}
    \right].
\end{equation*}

\subsubsection*{Формулы Снеллиуса}

Скорость распространения света
\begin{equation*}
    n = \frac{c}{v} = \sqrt{\varepsilon \mu}.
\end{equation*}
Для нормальной и тангенциальной компоненты воспользуемся граничными условиями
\begin{align*}
    E_2 \cos \psi &= E_1 \cos \varphi - E_1' \cos \varphi \\
    \varepsilon_2 E_2 \sin \psi &= \varepsilon_1 \left(
        E_1 \sin \varphi + E_1' \sin \varphi'
    \right),
\end{align*}
при чём $\varepsilon_2 = n_2^2$. И для $B$ верно, что
\begin{equation*}
    B_1 + B_1' = B_2.
\end{equation*}
Верно, что на границе
\begin{equation*}
    \vc{k} \cdot \vc{r} = k x \sin \varphi,
    \hspace{0.5cm} \Rightarrow \hspace{0.5cm} 
    \vc{E}_1 = \vc{E}_{10} e^{i\omega t} e^{-ik_1 x \sin \varphi},
    \hspace{0.5cm} \Rightarrow \hspace{0.5cm} 
    E_1 \sim \cos (k_1 x \sin \varphi).
\end{equation*}
Вспомним, что $\sqrt{\varepsilon} E = H \sqrt{\mu} = H$, тогда
\begin{equation*}
    \boxed{n_1 (E_1 + E_1') = n_2 E_2}, \hspace{0.5cm} 
    \boxed{
        \varphi = \varphi'
    }.
\end{equation*}
Можем сказать, что
\begin{equation*}
    n_1 \left(
        \hat{E}_{10} e^{ik_1 x\sin \varphi} + \hat{E}_{10} e^{-ik_1' x \sin \varphi' }
    \right) = 
    n_2 \hat{E}_{20} e^{i k_2 x \sin \psi}.
\end{equation*}
Мы знаем связь для $k = 2\pi / \lambda$, соответсвенно $k_1 = k_1'$.  Приходим к системе
\begin{equation*}
        k_1 \sin \varphi = k_2 \sin \psi, \hspace{0.5cm} 
    \frac{n_1 (\hat{E}_{10} + \hat{E}_{10}')}{n_2 \hat{E}_{20}} = \exp\left(
        -ix(
        \underbrace{k_2 \sin \psi - k_1 \sin \varphi}_{=0}
        )
    \right).
\end{equation*}
Ещё можем записать, что
\begin{equation*}
    n = \frac{\varepsilon \mu \omega}{c k}.
\end{equation*}
Получили формулы Снеллиуса.
\begin{equation*}
    \boxed{
        n_1 \sin \varphi = n_2 \sin \psi = \frac{n^2 \omega}{ck} .
    }
\end{equation*}


% \subsubsection*{Формулы Френе}

Запишем формулы Френе. Для $P$-поляризации
\begin{equation*}
    E_2 = E_1 \cdot \left(
        \frac{2 \sin \psi \cos \varphi}{\sin(\varphi + \theta)\cos (\varphi - \psi)} 
    \right),
    \hspace{1cm} 
    E_1' = E_1 \cdot \left(
    \frac{\tg(\varphi - \psi)}{\tg(\varphi + \psi)} 
    \right).
\end{equation*}
Для $S$-поляризации
\begin{equation*}
    E_1' = -E_1 \frac{\sin(\varphi-\psi)}{\sin(\varphi+\psi)},
    \hspace{1cm} 
    E_2 = E_1 \frac{2 \sin \psi \cos \varphi}{\sin(\varphi+\psi)}.
\end{equation*}

При угле между $\vc{k}'$ и $\vc{k}_2$ равным $\pi/2$ волна полностью проникает внутрь. 
Что ж, действительно, $\varphi + \psi = \pi/2$, тогда $n_1 \sin \varphi = n_2 \sin \psi = n_2 \cos \varphi$, тогда $\tg \varphi = n_2 / n_1$.

% \begin{equation*}
%     \varepsilon_2 \sin \psi E_{20} \cos \left(k_2 x \sin \psi \right) = 
%     \varepsilon_1 \left(
%         E_{10} \sin \varphi \cdot \cos \left(
%             k_1 x \sin \varphi
%         \right) + 
%         E_{10}' \sin \varphi' \cos \left(k_1 x \sin \varphi\right)
%     \right)
% \end{equation*}

\subsubsection*{Излучение диполя}

Рассмотрим антенку-диполь, $l \ll \lambda$. Два шарика перезаряжаются, и 
\begin{equation*}
    p = ql = p_0 \cos \omega t.
\end{equation*}
Ток будет равным
\begin{equation*}
    I = \dot{q} = \frac{\dot{p}}{l} = -\frac{1}{l} \omega p_0 \sin \omega t =I_0 \sin \omega t 
    .
\end{equation*}
Окружим диполь сферой. При $r \ll \lambda$ можем пренебречь запаздыванием, тогда можем говорить про статические формулы и $E \sim r^{-3}$ и $B \sim r^{-2}$. 

Однако наиболее интересен второй случай при $r \gg \lambda$. Тогда поле имеет вид сформировавшейся бегущей волны $\vc{k} \nparallel \vc{r}$. Вектор $\vc{E}$ ориентирован по меридиану сферы, вектор $\vc{B}$ ориентирован по широте, $\vc{E}, \vc{B}, \vc{k}$ образуют правую тройку. 
\begin{equation}
    E = B = \frac{
        \ddot{p} \left(t - \frac{r}{c}\right) \sin \theta
    }{
        c^2 r
    }.
\end{equation}
Рассмотрим это подробнее. 
\begin{equation*}
    \ddot{p}\left(t - \frac{r}{c} \right) = \omega_0^2 p_0 \cos \left(
        \omega \left(t - \frac{r}{c} \right)
    \right),
    \hspace{0.5cm} \Rightarrow \hspace{0.5cm} 
    \ddot{p} = - \omega^2 p_0 \cos (\omega t - kr).
\end{equation*}
Найдём среднее значение для вектора Пойтинга
\begin{equation*}
    \overline{S} = \frac{1}{8\pi} E_0^2 n,
    \hspace{0.5cm} \Rightarrow \hspace{0.5cm} 
    \overline{S} = \frac{1}{8\pi c^3} \frac{p_0^2 \omega^4 \sin^2 \theta}{r^2}.
\end{equation*}
Найдём интеграл по поверхности -- полный поток энергии
\begin{equation*}
    \int \overline{S} \d \sigma  = \int \overline{S} 
    2 \pi r \sin \theta r \d \theta = \frac{p_0^2 \omega^4}{3 c^3} 
    =
    \frac{l^2 \omega^2}{3 c^3}  I_0^2.
    =
    \frac{1}{2} R_{\text{изл}} I_0^2.
    .
\end{equation*}





% Для проводник верен закон Ома $\vc{j} = \sigma \vc{E}$, тогда по силе Лоренца
% \begin{equation*}
%     \vc{f} = \frac{1}{c} \left[
%         \vc{j} \times \vc{B}
%     \right] = \frac{1}{c} \left[\sigma \vc{E} \times \vc{B}\right]
%     ,
%     \hspace{0.5cm} \Rightarrow \hspace{0.5cm}

% \end{equation*}









