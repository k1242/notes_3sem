В 1820 году датский ученый Ганс Христиан Эрстед свершил выдающееся открытие – магнитное действие электрического тока... Как было показано раннее, 
\begin{equation}
    \frac{d \vc{p}}{d t} = - e \vc{E} + \frac{e}{c} \left[\vc{v} \times \vc{H} \right],
    \text{\ \ --- \ \ \textit{сила Лоренца}.}
\end{equation}
Здесь введено обозначение $\vc{H} = \rot \vc{A}$ -- напряженность магнитного поля. Это поле, 
действующее на движущиеся электрические заряды и на тела, обладающие магнитным моментом. 

Если взять провод, по которому течёт ток $I$ и засунуть его в магнитное поле, то погонная сила будет иметь вид
\begin{equation}
    \d \vc{F} = \frac{1}{c} I \left[
        d \vc{l} \times \vc{B}
    \right], \text{\ \ --- \ \ \textit{сила Ампера}.}
\end{equation}

%%%%%%%%%%%%%%%%%%%%%%%%%%%%%%%%%%%%%%%%%%%%%%%%%%%%%%%%%%%%%%%%%%%%%%%%%%%%%%%%%%%
\subsubsection*{Закон Био-Савара-Лапласа}
%%%%%%%%%%%%%%%%%%%%%%%%%%%%%%%%%%%%%%%%%%%%%%%%%%%%%%%%%%%%%%%%%%%%%%%%%%%%%%%%%%%

Рассмотри поле $\vc{H}$, создаваемое зарядами, совершающими \textit{финитное} движение -- остаются в конечной области пространства, сохраняя конечный импульс. Такое движение будет иметь стационарный характер и интересно будет рассмотреть усредненное по $t$ магнитное поле, создаваемое зарядами, соответсвенно $\vc{H} \neq \vc{H} (t)$. 

Усредненные по времени уравнения Максвелла примут вид
\begin{equation}
    \div \langle \vc{H}\rangle = 0, \hspace{0.5cm} 
    \rot \vc{H} = \frac{1}{c} \frac{\partial \vc{E}}{\partial t} + \frac{4\pi}{c} \vc{j},
    \hspace{0.2cm} \Rightarrow \hspace{0.2cm} 
    \rot \langle \vc{H}\rangle = \frac{4\pi}{c} \langle \vc{j} \rangle.
\end{equation}
Вспомним, что
\begin{equation*}
    \rot \langle \vc{A}\rangle = \langle \vc{H}\rangle,
    \hspace{0.5cm} \Rightarrow \hspace{0.5cm} 
    \grad \div \langle \vc{A}\rangle - \Delta \langle \vc{A}\rangle = \frac{4\pi}{c} \langle \vc{j}\rangle,
\end{equation*}
теперь по калибровочной инвариантности выберем $\langle \vc{A}\rangle : \div \langle \vc{A} \rangle = 0$. Тогда
\begin{equation}
    \Delta \langle \vc{A} \rangle = - \frac{4\pi}{c} \langle \vc{j} \rangle,
\end{equation}
что \textit{очень похоже на уравнение Пуассона}, но вместо $\rho$ стоит $\langle \vc{j}\rangle / c$. Аналогично поиску $\varphi$, находим
\begin{equation*}
    \varphi = \int \frac{\rho}{R} \d V, \hspace{0.5cm} \Rightarrow \hspace{0.5cm} 
    \langle  \vc{A} \rangle = \frac{1}{c} \int \frac{\langle \vc{j}\rangle}{R} \d V.
\end{equation*}
Теперь, зная $\langle  \vc{A}\rangle$, можно найти 
\begin{equation*}
    \langle \vc{H}\rangle = \rot \langle \vc{A}\rangle = \frac{1}{c} \rot \int \frac{\langle \vc{j}\rangle}{R} \d V.
\end{equation*}
Так как $\rot$ происходит по координатам точки наблюдения, то можем занести его под $\int$, считая $\vc{j} = \const$. Вспомним, что
\begin{equation*}
    \rot f \vc{a} = f \rot \vc{a} + \left[
        \grad f \times \vc{a}
    \right],
    \hspace{0.5cm} \Rightarrow \hspace{0.5cm} 
    \rot \frac{\langle \vc{j}\rangle}{R} = \left[
        \grad \frac{1}{R} \times \langle \vc{j} \rangle
    \right] = \frac{\left[
        \langle \vc{j} \rangle \times \vc{R}
    \right]}{R^3},
\end{equation*}
подставляя в формулу для $\langle \vc{H} \rangle$, получаем \textit{закон Био-Савара-Лапласа}
\begin{equation}
    \langle \vc{H} \rangle = \frac{1}{c} \int \frac{\left[\langle \vc{j}\rangle \times \vc{R}\right]}{R^3} \d V,
\end{equation}
где $\vc{R}$ -- направление из $dV$ в точку наблюдения. 

%%%%%%%%%%%%%%%%%%%%%%%%%%%%%%%%%%%%%%%%%%%%%%%%%%%%%%%%%%%%%%%%%%%%%%%%%%%%%%%%%%%
\subsubsection*{Теорема Гаусса для магнитного поля}
%%%%%%%%%%%%%%%%%%%%%%%%%%%%%%%%%%%%%%%%%%%%%%%%%%%%%%%%%%%%%%%%%%%%%%%%%%%%%%%%%%%

И вновь обращаемся к уравнениям Максвелла, а именно
\begin{equation}
    \div \vc{H} = 0,
    \hspace{0.5cm} \Leftrightarrow \hspace{0.5cm} 
    \int_V \div \vc{H} \d V = 
    \oint \vc{H} \cdot \d \vc{f} = 0.
\end{equation}
И, из другуго уравнения
\begin{equation}
    \rot \vc{H} = \frac{4\pi}{\vc{j}} + \frac{1}{c} \frac{\partial \vc{E}}{\partial t}.
\end{equation}
Интегрируя по некоторой поверхности, по формуле Стокса,
\begin{equation*}
    \int \rot \vc{H} \d \vc{f} = \oint \vc{H} \cdot \d \vc{l} = 
    \frac{1}{c} \frac{\partial }{\partial t} \int \vc{E} \d \vc{f} + \frac{4\pi}{c} \int \vc{j} \d \vc{f}.
\end{equation*}
Теперь естественно ввести величину \textit{тока смещения} $\frac{\partial }{\partial t} \vc{E} / 4\pi$, и получить уравнение вида
\begin{equation*}
    \oint \vc{H} \cdot \d \vc{l} = \frac{4\pi}{c} \int \left(
        \vc{j} + \frac{1}{4\pi} \frac{\partial \vc{E}}{\partial t} 
    \right) \d \vc{f}.
\end{equation*}


%%%%%%%%%%%%%%%%%%%%%%%%%%%%%%%%%%%%%%%%%%%%%%%%%%%%%%%%%%%%%%%%%%%%%%%%%%%%%%%%%%%
\subsubsection*{Коллекционирование магнитных бабочек}
%%%%%%%%%%%%%%%%%%%%%%%%%%%%%%%%%%%%%%%%%%%%%%%%%%%%%%%%%%%%%%%%%%%%%%%%%%%%%%%%%%%


Из теореме о циркуляции $\vc{H}$ можем найти, прямой провод с током окружают самозамкнутый кружочки с 
\begin{equation*}
    H = \frac{2I}{cr}.
\end{equation*}
Интересен случай с линейной плотностью тока $i = I/l$, тогда
\begin{equation*}
    d B_{\tau} = \frac{i}{c} \d \Omega, \hspace{0.5cm} \Omega = 2 \pi (1 - \cos \alpha).
\end{equation*}
где $\Omega$ -- телесный угол. Теперь легко получить поле внутри соленодиа:
\begin{equation*}
    B = \frac{4\pi}{c} i.
\end{equation*}
Аналогичное значение мы получим для <<магнитного конденсатора>>.  Для тороидальной катушки с разрывом, по теореме о циркуляции
\begin{equation*}
    H_0 \Delta l + H (l-\Delta l) = NI, \hspace{0.5cm} \Rightarrow \hspace{0.5cm} 
    H_0 = \frac{\mu N I }{(\mu-1) \Delta l + l}, \hspace{0.5cm} H \approx \frac{IN}{l}.
\end{equation*}