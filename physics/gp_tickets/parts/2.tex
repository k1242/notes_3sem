Вместо поиска $\vc{E}$ достаточно найти $\varphi$,
\begin{equation*}
    \left\{\begin{aligned}
        &\vc{E} = -\frac{1}{c}\frac{\partial \vc{A}}{\partial t} - \grad \varphi \\
        &\div \vc{E} = 4 \pi \rho \\
    \end{aligned}\right.
    \hspace{0.5cm} \Rightarrow \hspace{0.5cm} 
    \div \grad \varphi \equiv \Delta \varphi = 
    \left\{\begin{aligned}
        &- 4 \pi \rho &\text{ур. Пуассона} \\
        &\phantom{42} 0 &\text{ур. Лапласа}
    \end{aligned}\right.
\end{equation*}


Как может быть поставлена задача? Заданы граничные значения, найти распределения зарядов. Заданы заряды, найти распределения. Что-то задано, что-то не задано. Во всех трёх случаях \textbf{решение уравнения Пуассона единственно}. 


К слову, так как $\varphi = \varphi(\vc{r})$, а при $A_i = \const$ верно, что $\vc{E} = -\grad \varphi$, то электростатическое поле потенциально. Также это можно увидеть в работе ЭМ сил, при перемещении заряда по замкнутому контуру:
\begin{equation*}
    A_{\text{замкн}}/q = \oint_{(L)} \vc{E} \cdot \d \vc{l} = - \frac{1}{c} \frac{\partial }{\partial t} \int \vc{H} \cdot \d \vc{f} = 0.
\end{equation*}

\vspace{-10pt}

\subsubsection*{Разность потенциалов}

\begin{to_def} 
      Если на участке цепи не действуют сторонние силы, работа по перемещению включает только работу потенциального электрического поля и \textit{электрическое напряжение} $U_{AB}$ между $A$ и $B$ совпадает с разностью потенциалов $\varphi_A - \varphi_B = A^{\text{el}}_{AB}/q$. В общем случае $U_{AB} = \varphi_A - \varphi_B + \mathscr{E}_{AB}$.
\end{to_def}

\subsubsection*{Граничные условия на заряженной поверхности}

\vspace{-10pt}

\begin{minipage}[c]{0.45\textwidth}
\noindent
    По теореме Гаусса верно, что
    \begin{align*}
        E_{2n_2} \cancel{\Delta S} + E_{1n_1} \cancel{\Delta S} &= 4\pi\sigma \cancel{\Delta S}, \\
        E_{2n} - E_{1n} &= 4\pi\sigma
    \end{align*}
\noindent
    По теореме циркуляции верно, что
    \begin{align*}
        E_{2l} \cancel{\Delta} l - E_{1l} \cancel{\Delta} l &= 0 \\
        E_{2l} - E_{1l} &= 0.
    \end{align*}
\end{minipage}
\hfill
\begin{minipage}[c]{0.45\textwidth}
    \incfig{3}
\end{minipage}

\vspace{-10pt}

\subsubsection*{Проводники}

\begin{to_def}[пусть так]
    \textit{Проводник} -- костяк частиц, окруженных \textit{свободными} электронами, которые в пределах тела могут перемещаться на какие угодно
    расстояния. 
\end{to_def}

\vspace{-10pt}

\begin{minipage}[c]{0.55\textwidth}
    В частности, для проводников, верно, что
    \begin{align*}
        E_n &= 4 \pi \sigma \\
        E_\tau &= 0
    \end{align*}
\end{minipage}
\hfill
\begin{minipage}[c]{0.35\textwidth}
    \incfig{4}
\end{minipage}

Собственно, объёмных зарядов в проводнике нет, поверхностные есть и компенсируют внешнее поле. Аналогично работает решетка Фарадея, электростатическое поле не проникает в проводники.


\subsubsection*{Метод изображений}

Если существует некоторая эквипотенциальная поверхность разделяющая пространство на два полупространства, то можем считать что эта поверхность является проводящей. И наоборот, проводящую поверхность можно заменить, на системы зарядов в полупространстве, ей ограниченном, создающих эквипотенциальную поверхность.

