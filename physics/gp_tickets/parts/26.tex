\textbf{Плазма} --- а-ля ионизированный квазинейтральный газ, но с коллективными явлениями. 
В плазме молекулярных токов нет: $\vc{H} = \vc{B}$; обладает высокой электропроводностью.
Если в каком-то веществе протекает ток с объёмной плотностью $\vc{j}$, то возникает сила на единицу объёма с током: $\vc{f} = [\vc{j} \times \vc{B}]/c$.

Давление, которое может быть созданно на плазму магнитным поле -- \textbf{магнитное давление} = $B^2 /8\pi $. По плазменному шнуру, чтобы он не расползался, можно пустить такой ток, чтобы возникло магнитное поле
\begin{equation*}
  \vc{B} \colon n k T = B^2/8\pi.   
\end{equation*}
Оценим размеры области, в которой могут происходить заметные
нарушения квазинейтральности. Пусть отрицательные заряды сместились на $l$, тогда возникнет $\sigma = nle$, где $n$ -- концетрация частиц одного знака. Напряженность будет $E = 4\pi \sigma = 4 \pi nle,$ а плотность ЭМ энергии $E^2/8\pi = 2 \pi (nle)^2$. Энергия ЭМ поля возникает за счёт теплового движения, соотвественно не превосходит $3nk_{\text{Б}}T/2 \times 2$. Так приходим к 
 \begin{equation}
    2 \pi (nle)^2 < nkT \Leftrightarrow l < r_D, \hspace{0.5cm} 
    r_D  = \sqrt{\frac{k T}{8 \pi e^2 n}} 
    \text{\ \ --- \ \ \textit{дебаевский радиус}}. 
 \end{equation}

 Имеет смысл радиуса сферы вокруг внесённого в плазму заряда, в которой нарушается квазинейтральность плазмы.
 Количество частиц в дебаевской сфере, по которой различают ионизированный газ и плазму: $N \approx n r_D^3$. И если $N_D
  \lesim
  1$, то это называется ионизованным газом, а если $N_D \gg 1$ -- называется плазмой.

Самый простой пример коллективного взаимодействия -- плазменные колебания. Ещё раз предположили, что всё сместилось, $\sigma = nex$, $E = 4\pi nex$, соотвественно найдём плазменную частоту:
\begin{equation*}
    m \ddot{x} = - eE = -4 \pi e^2 n x, \hspace{0.5cm} \Rightarrow \hspace{0.5cm} 
    \omega_\text{pl} = \sqrt{\frac{4 \pi n e^2}{m}},
\end{equation*}
Энергия таких колебаний: 
\begin{equation}
    W_\text{э} = \frac{E^2}{8 \pi} = 2 \pi (n e x)^2 
    \lesim
    n k T 
    \hspace{0.5cm} \Rightarrow \hspace{0.5cm} 
    x_{\text{max}} \approx 2 r_D,
\end{equation}
где $x_\text{max} $ -- максимальная амплитуда таких колебаний (может служить определением дебаевского радиуса).

Для плазмы в переменном электрическом поле:
\begin{equation}
    m \ddot{r} = - e E_0 \cos \omega t 
    \hspace{0.5cm} \Rightarrow \hspace{0.5cm} 
    r = \frac{e}{m \omega^2} E_0 \cos \omega t,
\end{equation}
по такому закону будет колебаться каждый электрон. Это значит, что есть дипольный момент. Запишем сразу дипольный момент единицы объёма с такими электронами:
\begin{equation}
    P = - n e r = - \frac{n e^2}{m \omega^2} E = \alpha E 
    \hspace{0.5cm} \Rightarrow \hspace{0.5cm} 
    \varepsilon = 1+ 4 \pi \alpha
    \hspace{0.5cm} \Rightarrow \hspace{0.5cm} 
    n^2 =  \varepsilon = 
    1 - \frac{4\pi n e^2}{m \omega^2} 
    =
    1 -  \frac{\omega_\text{pl}^2}{\omega^2},
\end{equation}
таким образом получили дисперсионное выражение для плазмы. От возбуждаемой частоты будут зависит оптические (электрические) свойства плазмы. При $\omega_{\text{pl}} > \omega$ увидим, что $\varepsilon < 0$, и, соотвественно, $n = i n'$, тогда произойдёт отражение от плазмы. Например, для $n \approx 10^6$, $\nu_{\text{pl}} \approx 20$ МГц.

Естественно ввести величину \textit{фазовой скорости} в плазме:
\begin{equation}
    v = \frac{c}{n} = \frac{c}{\sqrt{\varepsilon}}= \frac{\omega}{k}.
\end{equation}
Нетрудно получить, что
\begin{equation*}
c^2 k^2 = \varepsilon \omega^2 = \varepsilon \left(\omega^2 - \omega_\text{pl}^2\right), \hspace{0.5cm} \Rightarrow \hspace{0.5cm} 
     \boxed{\omega = \sqrt{\omega_\text{pl}^2 + k^2 c^2}}.
\end{equation*}
Получим дисперсионное выражение для плазмы. Или, выражая $k$,
\begin{equation*}
    k = \sqrt{\frac{\omega^2}{c^2} - \frac{\omega^2_\text{pl}}{c^2 }}.
\end{equation*}


Центральное выражение называется дисперсионной формулой для плазмы. Её можно выразить в виде последнего выражения, которое похоже на критическую частоту для волновода. Оказывается, что дисперсионное соотношение для волн в волноводе и в плазме одинаковое.