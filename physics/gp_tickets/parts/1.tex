Взаимодействие частиц друг с другом можно описывать с
помощью понятия силового поля.Частица создает вокруг себя поле; на всякую другую частицу, находящуюся в этом поле, действует некоторая сила.

Оказывается, что свойства частицы в отношении ее взаимодействия с электромагнитным полем определяются всего одним параметром -- так называемым зарядом частицы.


\subsubsection*{Уравнение движения пробного заряда}

Как было показано в введение, уравнение движения частицы в электромагнитном поле запишется, как
\begin{equation*}
    \frac{d \vc{p}}{d t} = 
    \underbrace{
        -\frac{e}{c} \frac{\partial \vc{A}}{\partial t} - e 
    }_{
        \text{про } E
    }
    + 
    \underbrace{
        \frac{e}{c} \left[\vc{v} \times \rot \vc{A} \right]
    }_{
        \text{про } H   
    }
\end{equation*}
Так как первая часть не зависит от скорости, то естественно ввести следующие величины
\begin{align*}
    \vc{E} &= -\frac{1}{c} \frac{\partial \vc{A}}{\partial t} - \grad \varphi 
    & \text{--- \textit{напряженность электрического поля}}
    \\
    \vc{H} &= \rot \vc{A} 
    & \text{--- \textit{напряженность магнитного поля}}
\end{align*}
Теперь уравнение движения в ЭМ поле примет вид
\begin{equation}
    \frac{d \vc{p}}{d t} = - e \vc{E} + \frac{e}{c} \left[\vc{v} \times \vc{H} \right]
    \text{\ \ --- \ \ \textit{сила Лоренца}.}
\end{equation}
Таким образом может определить $\vc{E}$, как отношение силы $\vc{F}$, действующей на неподвижный заряд, помещенный в данную точку поля к величине этого заряда $q$: $\vc{E} = \vc{F} / q$. 

\subsubsection*{Теорема Гаусса}

Как было показано в введение, верно что
\begin{equation*}
    \div E = 4 \pi \rho.
\end{equation*}
Или, в интегральной форме
\begin{equation*}
    i
\end{equation*}


\subsubsection*{Закон сохранения заряда}
Изменение заряда в объеме в единицу времени:
\begin{equation*}
    \oint \rho \vc{v} \cdot \d \vc{f} = 
     - \frac{\partial }{\partial t} \int \rho \d V,
\end{equation*}
где минус появился из-за выбора направления нормали к поверхности. Это уравнение можно переписать в виде
\begin{equation}
    \frac{\partial }{\partial t} \int \rho \d V = - \oint \vc{j} \cdot \d \vc{f} = - \int \div \vc{j} \d V,
    \hspace{0.25cm} \Rightarrow \hspace{0.25cm} 
    \int \left(
        \div \vc{j} + \frac{\partial \rho}{\partial t} \d V
    \right) = 0,
    \hspace{0.25cm} \Rightarrow \hspace{0.25cm} 
    \boxed{
        \div \vc{j} + \frac{\partial \rho}{\partial t}  = 0
    }.
\end{equation}
Таким образом алгебраическая сумма зарядов электрически замкнутой системы сохраняется. Требование релятивистской инвариантности приводит к тому, что закон сохранения заряда имеет локальный характер: изменение заряда в любом наперёд заданном объёме равно потоку заряда через его границу.


\subsubsection*{Принцип суперпозиции}

Как показывает опыт, электромагнитное поле подчиняется так называемому принципу суперпозиции: поле, создаваемое системой зарядов, представляет собой результат простого сложения полей, которые создаются каждым из зарядов в отдельности. Это значит, что напряженности результирующего поля в каждой точке равны сумме (векторной) напряженностей в этой точке каждого из полей в отдельности.

Всякое решение уравнений поля является полем, которое может быть осуществлено в природе. Согласно принципу суперпозиции сумма любых таких полей тоже должна быть полем,
которое может быть осуществлено в природе, т. е. должно удовлетворять уравнениям поля.

Как известно, линейные дифференциальные уравнения как
раз отличаются тем свойством, что сумма любых его решений
тоже является решением. Следовательно, уравнения для поля
должны быть линейными дифференциальными уравнениями.



\subsubsection*{Закон Кулона}

Рассмотрим точечный заряд, окружим его сферой и запишем для неё теорему Гаусса:
\begin{equation*}
    \oint_{\partial S} \vc{E} \cdot \d \vc{f} = 4 \pi e,
    \hspace{0.5cm} \Rightarrow \hspace{0.5cm} 
    E = \frac{e}{R^2},
    \hspace{0.5cm} \Leftrightarrow \hspace{0.5cm} 
    \vc{E} = \frac{e}{R^2} \frac{\vc{R}}{R}, 
\end{equation*}
что верно в силу радиальной симметрии. Так мы получили \textit{закон Кулона}. 

Потенциал же $\varphi = e / R$, или, для системы зарядов,
\begin{equation*}
    \varphi = \sum_a \frac{e_a}{R_a}, \text{\ \ или \ \ }
    \varphi = \int \frac{\rho}{R} \d V.
\end{equation*}
\textit{Забавный факт}: подстановка в уравнение пуассона приводит к уравнению вида $\Delta (1/R) = -4\pi \cdot \delta(R)$. 

Простейшим примером системы зарядов является \textit{диполь} $q_1 + q_2 = 0$, для которого верно, что
\begin{equation}
    \vc{E} = \frac{q}{r_1^2} \frac{\vc{r}_1}{r_1} - \frac{q}{r_1^2} \frac{\vc{r}_2}{r_2} 
    \hspace{0.5cm} \overset{l \ll r_2, r_1}{\underset{\Longrightarrow}{}} \hspace{0.5cm} 
    \vc{E} = \frac{3 (\vc{p} \cdot \vc{n}) \vc{n}}{r^3}  - \frac{\vc{p}}{r^3},
\end{equation}
где $\vc{n}$ -- единичный вектор направления к рассматриваемой точке, а $\vc{p}=q \vc{l}$ -- \textit{дипольный момент}.
Чуть подробнее остановимся на системе зарядов и дипольном моменте.
