\subsubsection*{Метод комплексных амплитуд}


Рассмтрим $U = U_0 \cos (\Omega t + \varphi)$. По методу комплексных амплитуд
\begin{equation*}
    U = U_0 e^{i\left(\Omega t + \varphi \right)}.
\end{equation*}
Он применим всегда, когда $\Re$ и $\Im$ не смешиваются. 

Можем ввести импеданс -- комплексное сопротивление, при чём $\Re(z)$ -- активное сопротивление, $\Im z$ -- реактивное споротивление. 

\subsubsection*{Мощность переменного тока}
Рассмотрим некоторый элемент
\begin{equation*}
    I = I_0 \cos \left(\Omega t + \psi\right) = \Re I_0 e^{i \psi} e^{i \Omega t}, \hspace{0.5cm} 
    U = U_0 \cos (\Omega t + \varphi) = \Re U_0 e^{i\varphi} e^{i \Omega t}.
\end{equation*}
Мгновенное значение мощности
\begin{equation*}
    P = UI = 
    \frac{1}{2} I_0 U_0 \cos (\psi - \varphi) + \frac{1}{2} I_0 U_0 \cos (
    2 \Omega t + \varphi + \psi
    ),
    \hspace{0.5cm} \Rightarrow \hspace{0.5cm} 
    \langle P\rangle_T = \frac{1}{T} \int P \d t = \frac{1}{2} I_0 U_0 \cos (\psi - \varphi).
\end{equation*}
Можно исправить метод комплексных амплитуд сопряжением.
\begin{equation*}
    \langle P\rangle_T = \frac{1}{2} \tilde I \tilde U^*,
    \hspace{0.5cm} 
    I_{\text{эфф}} = \frac{1}{T} \int_0^T I^2 \d t = \frac{I_0}{\sqrt{2}},
    \hspace{0.5cm} 
    U_{\text{эфф}} = \frac{1}{T} \int_0^T U^2 \d t.
\end{equation*}


