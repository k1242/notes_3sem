Рассмотрим $\langle \vc{H} \rangle$, как предыдущем билете, перейдя от $\vc{j}$ к $\rho \vc{v}$.
\begin{equation*}
    \langle \vc{A} \rangle = \frac{1}{c} \sum 
    \left\langle \frac{e_a \vc{v}_a}{R_a} \right\rangle
    = \frac{1}{c} \sum 
    \left\langle 
    \frac{e_a \vc{v}_a}{\|\vc{R}_0 - \vc{r}_a\|} 
    \right\rangle
    ,
\end{equation*}
и аналогично раскладывая в ряд до первого порядка $\vc{r}_a/R_0$:
\begin{equation*}
    \left\langle \vc{A}\right\rangle = 
    \frac{1}{cR_0} \sum e \left\langle \vc{v}\right\rangle - 
    \frac{1}{c} \sum e
    \left\langle \left(\vc{r} \cdot \grad \frac{1}{R_0} \right) \vc{v}\right\rangle = 
    \frac{1}{c} \sum e
    \left\langle \left(\vc{r} \cdot \grad \frac{1}{R_0} \right) \vc{v}\right\rangle.
\end{equation*}
Вспомнив, что $\vc{v} = \dot{\vc{r}}$, считая $\vc{R}_0 = \const$  -- 
\begin{equation*}
    \sum e \left(R_0 \cdot \vc{r} \right) \vc{v} = \frac{1}{2} 
    \underbrace{\frac{d }{d t} 
    \sum e \vc{r} \left(\vc{r} \cdot \vc{R}_0\right)}_{=0}
     + 
    \frac{1}{2} \sum e 
    \underbrace{\{
        \vc{v} \left(\vc{r} \cdot \vc{R}_0\right) - \vc{r} (\vc{v} \cdot \vc{R}_0)
    \}}_{
        \left[\vc{r} \times \vc{v} \right] \times \vc{R}_0
    }\ ,
\end{equation*}
и теперь введём
\begin{equation}
    \vc{m} =  \frac{1}{2c} \sum e \left[\vc{r} \times \vc{v}\right], \text{\ \ --- \textit{магнитный момент системы}.}
\end{equation}
Тогда
\begin{equation*}
    \left\langle \vc{A}\right\rangle = \frac{1}{R_0^3} 
    \left[ \vph
        \left\langle \vc{m}\right\rangle \times \vc{R}_0
    \right] = 
    \left[
        \grad \frac{1}{R_0} \times \left\langle \vc{m}\right\rangle.
    \right]
\end{equation*}
Вспомним, что
\begin{equation*}
    \rot \left[\vc{a} \times \vc{b}\right] = \left(\vc{b} \cdot \nabla \right) \vc{a} - 
    \left(\vc{a} \cdot \nabla \right) \vc{b} + a\div \vc{b} - \vc{b} \div \vc{a}, 
\end{equation*}
и получим
\begin{equation*}
    \left\langle \vc{H}\right\rangle = \rot \left\langle \vc{A}\right\rangle = 
    \rot \left[
        \left\langle \vc{m}\right\rangle \frac{\vc{R}_0}{R_0^3} 
    \right] = \left\langle \vc{m}\right\rangle \div \frac{\vc{R}_0}{R_0^3} - 
    \left(
        \left\langle \vc{m}\right\rangle \cdot \nabla
    \right) \frac{\vc{R}_0}{R_0^3}.
\end{equation*}
Раскрыв по правилу Лейбница
\begin{equation*}
    \left( \vph
        \left\langle \vc{m}\right\rangle \cdot \nabla
    \right) \frac{\vc{R}_0}{R_0^3} = 
    \frac{1}{R_0^3} \left(
        \left\langle \vc{m}\right\rangle \cdot \nabla
    \right) \vc{R}_0 + \vc{R}_0 
    \left(
        \left\langle \vc{m}\right\rangle \cdot \nabla \frac{1}{R_0^3} 
    \right) =
    \frac{1}{R_0^3} \left( \vph
        \left\langle \vc{m}\right\rangle - 3 \vc{R}_0 \left(
            \left\langle \vc{m}\right\rangle \cdot \vc{R}_0
        \right) / R_0^2
    \right).
\end{equation*}
И наконец, сквозь тернии, находим очень знакому формулу
\begin{equation}
    \boxed{
        \left\langle \vc{H}\right\rangle = \frac{1}{R_0^3} 
        \left( \vph
            3 \vc{n} \left(
                \left\langle \vc{m}\right\rangle \cdot \vc{n}
            \right) - \left\langle \vc{m}\right\rangle
        \right), \hspace{1cm} 
        \vc{n} = \frac{\vc{R}_0}{R_0},
    }
\end{equation}
таким образом мы нашли вектор напряженности магнитного поля вокруг магнитного диполя!

\subsubsection*{Сила и момент сил}

Результирующая сила, дейтсвующая на виток с током в постоянном магнитном поле, дается выражением
\begin{equation}
    F = \frac{I}{c} \oint \left[
        \d \vc{l} \times \vc{B}
    \right].
\end{equation}
В случае плоского витка с током, разобьем его на $I \d \vc{l}_1$ и $I \d \vc{l}_2$.  Действующие на них силы Ампера нормальны к плоскости витка и противоположны по направлению. 
\begin{equation*}
    F_1 = \frac{I}{c} B \d l_1 \sin \alpha = \frac{I}{c} B \d h,
\end{equation*}
аналогчино для $F_2$, соотвественно эти силы образуют момент
\begin{equation*}
    dM = \frac{I}{c}  B a \d h = \frac{I}{c} B\d S, 
    \hspace{0.5cm} \Rightarrow \hspace{0.5cm} 
    M = \frac{I}{c} B , 
\end{equation*}
где $S$ -- площадь витка. Введем вектро площади контура $S$, тогда
\begin{equation}
    \vc{M} = \left[
        \vc{m} \times \vc{B}
    \right], 
    \hspace{0.5cm} 
    \vc{m} = \frac{I}{c} \vc{S},
\end{equation}
где вектор $\vc{m}$ -- \textit{магнитный момент тока}. В случае нормального направления к магнитному полю, сила ампера будет только растягивать или сжимать виток. В случае криволинейной поверхности, следует ввести $\vc{S} = \int \d \vc{S}$, где интегрирование производится по поверхности. 

Если расположить катушку (много витков) так, чтобы $\vc{m} \bot \vc{B}$, то получим
\begin{equation*}
    \vc{B} = \frac{1}{m^2} \left[\vph
        \vc{M} \times \vc{m}
    \right],
\end{equation*}
что позволят на практике проверить закон Би-Савара-Лапласа.