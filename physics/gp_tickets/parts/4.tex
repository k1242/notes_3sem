Рассказ про объёмную плотность энергии сводится к повторению билета №21, что приведёт к объёмной плотности энергии ЭМ поля

\vspace{-18pt}
\begin{equation*}
    \mathcal W = \frac{E^2 + H^2}{8\pi}.
\end{equation*}
Найдём теперь энергию системы зарядов. Считая поле постоянным, а заряды неподвижными
\begin{equation*}
    \mathcal U = \frac{1}{8\pi} \int E^2 \d V, \text{\ \ --- энергия системы зарядов,}
\end{equation*}
где $\vc{E}$ -- поле от всех зарядов, а интеграл берется по всему пространству. Вспомнив уравнения Максвелла, немного перепишем выражение в 
\begin{equation*}
    \mathcal U = - \frac{1}{8\pi} \int \vc{E} \grad \varphi \d V = 
    - \frac{1}{8\pi} 
    \underbrace{
        \int \div \left(\vc{E} \cdot \varphi\right) \d V
    }_{
        \oint \vv{E} \varphi \cdot \d \vv{f} = 0
    } + 
    \frac{1}{8\pi} \int \varphi \div \vc{E} \d V.
\end{equation*}
Таким образом
\begin{equation*}
    \mathcal U = \frac{1}{2} \int \rho \varphi \d V,
    \text{ \ \ или, в случае дискретного распределения, \ \ }
    \mathcal U = \frac{1}{2} \sum e_a \varphi_a.
\end{equation*}
Тут мы встречаем проблему заряда, находящегося в потенциале собственного поля, но, так как эта величина не зависит от конфигурации зарядов, то его мы проигнорируем, от греха подальше
\begin{equation}
    \mathcal U' = \frac{1}{2} \sum e_a \varphi_a', \text{ \ \ где \ \ }
    \varphi_a' = \sum_{b (\neq a) } \frac{e_b}{R_{ab}},
    \hspace{0.5cm} \Leftrightarrow \hspace{0.5cm} 
    U' = \frac{1}{2} \sum_{a \neq b} \frac{e_a e_b}{R_{ab}}.
\end{equation}

\vspace{-10pt}

\subsubsection*{Проводники}


В случае \textit{проводников} заряд распределен по поверхности, потенциал вдоль которой не меняется, тогда энергия такой системы
\begin{equation*}
    \mathcal W = 
    \frac{1}{2} \sum_i \oint_{S_i} \varphi_i \sigma \d S = 
    \frac{1}{2} \sum_i \varphi_i \oint \sigma \d S = \frac{1}{2} \sum_i \varphi_i Q_i,
\end{equation*}
где $\varphi_i$ и $Q_i$ -- потенциал и заряд $i$-го проводника, $\sigma$ -- плотность заряда.

Так как решение задачи Пуассона единственно, то $\varphi(\vc{r})$ однозначно определяется потенциалами проводников $\varphi_i$, поэтому $Q_i$ -- некоторые однозначные функции $\varphi_i$. Из-за линейности уравнений поля, эти функции могут быть только линейными, поэтому существует связь вида
\begin{equation*}
    Q_i = \sum_k C_{ik} \varphi_k, \hspace{1cm} 
    \varphi_i = \sum_k S_{ik} Q_k.
\end{equation*}
Постоянные коэффициенты $C_{ik}$ называются \textit{емкостными коэффициентами} системы проводников, а $S_{ik}$ -- \textit{потенциальными коэффициентами}. При этом $C_{ii}$ называют \textit{собственными емкостями}, а $C_{ik}, \ i\neq k$ -- \textit{коэффициенами взаимной емкости} или \textit{коэффициентами электростатической индукции}.

Можно показать, что $C_{ik} = C_{ki}$, что ещё называют \textit{соотношением взаимности}. Это позволяет прийти к следующей теореме

\begin{to_thr}[теорема взаимности Грина]
     Если есть две конфигурации системы: $\{\varphi_i; Q_i\}$ и $\{\varphi_i'; Q_i'\}$, то всегда  верно, что
     $\sum Q_i' \varphi_i = \sum Q_k \varphi_k'$.
\end{to_thr}

Для одиночного проводника $Q = C\varphi$, где $C$ -- \textit{собственная емкость} проводника. В частности, для металлического шарика радиуса $a$ с зарядом $Q$, находящегося в среде с $\varepsilon$, получится $\varphi = Q / (\varepsilon a)$, т.е. $C = \varepsilon a$. 

\subsubsection*{Конденсатор}

В случае двух проводников, с зарядами $Q_1 = -Q_2 = Q$, имеем \textit{конденсатор}, коэффициент $C$ которого называется \textit{емкостью}.

Найдём емкость конденсатора. Из №3 билета понятно, что поле внутри $E = 2 \pi \sigma / \varepsilon \times 2$. Далее, по определению, вспомнив что $C_{12}=C_{21}$, и $C_{11}=C_{22} = C$, получим
\begin{equation}
    \left\{\begin{aligned}
        Q_1 &= Q = C_{11}\varphi_1 + C_{12}\varphi_2 \\
        Q_2 &= -Q= C_{21}\varphi_1 + C_{21}\varphi_2 \\
    \end{aligned}\right.
    \hspace{0.25cm} \overset{}{\Rightarrow}  \hspace{0.25cm} 
    Q = C_{12} (\varphi_2 - \varphi_1) = C l \frac{4\pi}{\varepsilon} \left(\frac{Q}{S} \right),
    \hspace{0.5cm} \Rightarrow \hspace{0.5cm} 
    C = \frac{\varepsilon}{4\pi} \frac{S}{l}.
\end{equation}

\phantom{42}

\noindent
\textit{Комментарий}: 
Энергию диполя во внешнем поле смотреть в первом билете. 

