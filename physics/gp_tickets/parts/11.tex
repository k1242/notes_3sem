Для диамагнитных и парамагнитных сред зависимость $\vc{M}$ и $\vc{B}$ линейна и для изотропных сред запишется в виде $\vc{M} = \chi_{1} \vc{B}$. Однако исторически сложилось линейно связывать $\vc{M} =\chi \vc{H}$, что равносильно.

Имея $B = 4 \pi \vc{M} + H$, подставим, имея $\mu= 1 + 4 \pi \chi$, получим:
$\vc{B} = \mu \vc{H}$.
Величина $\chi$ -- \textit{магнитная восприимчивость}, а $\mu$ -- \textit{магнитная проницаемость}.


\subsection{Диамагнетики}
\textit{Диамагнетики} --- вещества для которых $\chi < 0$, а следовательно $\mu<1$.
Диамагнетизм наблюдается у веществ у атомов которых в отсутствие магнитного поля не обладают магнитными моментами. 

При наличии внешнего постоянного магнитного поля внутреннее движение электронов атомов не изменяется, но атом в целом получает дополнительное вращение с угловой скоростью $\vc{\Omega} = \frac{e}{2 m c} \vc{B}$.
Этот результат называется \textit{теоремой Лармора}, а сама частота -- ларморовская.


\subsection{Парамагнетики}
\textit{Парамагнетики} --- вещества для которых $\chi > 0$, а следовательно $\mu>1$.
Парамагнетизм наблюдается у тех веществ, у которых атомы обладают магнитными моментами в отсутствие внешнего магнитного поля. В силу хаотичности тело всё равно остаётся не намагниченным.

\subsection{Ферромагнетики}





\subsection{Магнитные свойства сверхпроводников}
При низких температурах металлы переходят к состоянию, при котором у них наблюдается \textit{сверхпроводимость} --- полное отсутствие электрического сопротивления постоянному току. Температурная точка сверхпроводящего перехода -- фазовый переход второго рода.

Магнитное поле никогда не проникает в толщу сверхпроводника. То есть $\vc{B} = 0$, это свойства появляется независимо от того в каких условиях произошёл переход в сверхпроводящее состояние. В действительности магнитное поле проникает в сверхпроводник на небольшую глубину, большую по сравнению с междуатомными расстояниями.

По граничным условиям тогда получаем, что на поверхности нормальная составляющая внешнего поля тоже равна нулю. То есть магнитное поле снаружи касательно к сверхпроводнику, магнитные силовые линии огибают его.



