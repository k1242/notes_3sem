\subsubsection*{Спектральное разложение}

Представим $\mathscr{E}$, как
\begin{equation*}
    \mathscr{E} = \sum_j a_j \cos(\omega_j t + \varphi_j).
\end{equation*}
Тогда мы в линейной системе всегда можем работать с синусоидальной возбуждающей силой. 

Пусть функция $f(t)$ периодическая с периодом $T$, тогда 
\begin{equation*}
    f(t) = \sum_{n=-\infty}^{+\infty} \frac{1}{2\pi} C_n \exp(i\omega_n t),
    \hspace{0.5cm} 
    C_n = \frac{1}{T} \int_{-T/2}^{T/2} f(t) \cdot e^{-i \omega_n t} \d t,
    \hspace{0.5cm} 
    \omega_n = \omega_0 n, \hspace{0.5cm} \omega_0 = \frac{2\pi}{T}.
\end{equation*}
В случае, если $f(t)$ не является периодической, то спект перестает быть дискретным
\begin{equation*}
    f(t) = \frac{1}{2\pi} A(\omega) \exp(i\omega t) \d \omega,
    \hspace{0.5cm} 
    A(\omega) = \int_{-\infty}^{\infty} f(t) \exp(-i\omega t) \d t.
\end{equation*}

\subsubsection*{Прямоугольный сигнал}

Рассмотрим прямоугольный сигнал. Для него
\begin{equation*}
    A(\omega) = \int_{\infty}^{\infty} f(t) \exp(-i\omega t)\d t = 
    \int_{-\tau/2}^{\tau/2} f_0 \exp(-i \omega t) \d t = 
    f_0 \tau \frac{1}{\omega \tau/2} 
    \left(
        \frac{e^{i \omega \tau/2 - e^{-i\omega \tau/2}}}{2i} 
    \right)= f_0 \tau \cdot  \left(
        \frac{
        \sin \omega \tau / 2
        }{
        \omega \tau /2 
        } 
    \right)
\end{equation*}
Основная часть спектра лежит в 
\begin{equation*}
    \Delta \omega = \frac{2\pi}{\tau}, \hspace{0.5cm} \Delta \omega \cdot \tau = 2 \pi.
\end{equation*}
Рассмотрим теперь последовательность прямоугольных сигналов с шириной $\tau$ и периодом $T$. Тогда
\begin{equation*}
    C_n = \frac{1}{T} \int_{-T/2}^{T/2} f(t) e^{-i \omega t } \d t = 
    \frac{1}{T} \int_{-\tau/2}^{\tau/2} f_0 \exp(-in \Omega t) \d t =
    \frac{2f_0 \tau}{T} \cdot 
    \frac{
        \sin \omega \tau / 2
        }{
        \omega \tau /2 
        } 
\end{equation*}


\subsubsection*{Амплитудная модуляция}

Рассмотрим амплитудную модуляцию
\begin{equation*}
    U = A(1 + m \cos \Omega t) \cdot \cos \omega_0 t, \hspace{0.5cm} 
    \Omega \ll \omega_0, \hspace{0.5cm} m<1.
\end{equation*}
Тогда
\begin{equation*}
    U = A\bigg(
        1 + m \underbrace{
        \frac{1}{2} \left(
            e^{i\Omega t} + e^{-i\Omega t}
        \right)
        }_{\cos \Omega t}
    \bigg) \cdot 
    \underbrace{
    \exp\left(i \omega_0 t\right)
    }_{
    \cos \omega_0 t
    }
\end{equation*}
Раскрыв скобки получаем, что
\begin{equation*}
    U = A e^{i \omega_0 t} + \frac{Am}{2} e^{i \left(\omega_0-\Omega\right) t} + \frac{Am}{2} e^{i \left(\omega_0+\Omega\right) t}
\end{equation*}
Вообще может быть $U = F(\tau) \cdot \cos \omega_0 t, \ \tau \gg \frac{2\pi}{\omega_0}$. 

\subsubsection*{Фазовая модуляция}

Рассмотрим фазовую модуляцию
\begin{equation*}
    U = A \cos \left(
        \omega_0 t + m \cos \Omega t
    \right),
    \hspace{0.5cm} \Omega \ll \omega_0, \hspace{0.5cm} m \ll 1.
\end{equation*}
Перепишем комплексными амплитудами
\begin{equation*}
    U = A \cdot e^{i \omega_0 t} \cdot \exp\left(
        im \frac{1}{2} \left(
            e^{i \Omega t} + e^{-i\Omega t}
        \right)
    \right) = 
    A e^{i \omega_0 t}
    \left(
        1 + \frac{m}{2} i e^{i \Omega t} + \frac{m}{2} i e^{-i\Omega t}
    \right) = 
    A e^{i\omega_0 t} + \frac{Am}{2} i e^{i(\omega_0 - \Omega)t}
    + \frac{Am}{2} i e^{i(\omega_0 + \Omega)t}.
\end{equation*}
То есть теперь есть сдвиг по фазе на $\pi/2$. 


\subsubsection*{Детектирование сигнала}
Пусть есть некоторое преобразование из $f(t)$ в $g(t)$, при чем
\begin{equation*}
    g(t) = \frac{1}{\Delta t} \int_{\Delta t} f^2(t) \d t.
\end{equation*}
Давайте $\Delta t \gg T_0 = 2\pi/\omega_0$, при этом $\Delta t \ll \tau$, то мы сохраним информацию об огибающей, где $\tau$ -- характерное время изменения модулирующей функции.
Рассмотрим $f(t) = A(1+m\cos \Omega t) \cdot \cos \omega_0 t$ -- амплитудную модуляцию. 
\begin{equation*}
    g(t) = \frac{A^2}{\Delta t} (1 + 2m \cos \Omega t).
\end{equation*}
Или, методом комплексных амплитуд,
\begin{align*}
    f(t) = A(1 + m \cos \Omega t) \cos \omega_0 t,
    \hspace{0.5cm} 
    g(t) = \frac{A^2}{2} (1 + 2 m \cos \Omega t), \\
    g(t) = \frac{1}{2} f(t) \cdot f^*(t).
\end{align*}

Посмотри теперь историю с фазовой модуляцией, без несущей. Тогда
\begin{equation*}
    g(t) = \left(
        \frac{mA}{2} 
    \right)^2 \cdot (1 + \cos 2 \Omega t).
\end{equation*}
Или можно сдвинуть фазу несущей на $\pi/2$, тогда ситуауия аналогична амплитудной модуляции, тогда
\begin{equation*}
    g(t) = \frac{A}{2} (1 + 2m \cos \Omega t). 
\end{equation*}