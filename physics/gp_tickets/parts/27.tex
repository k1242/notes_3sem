\subsubsection*{Преломление и поглощение волн}

Пусть есть некоторое внешнее поле $E = E_0 e^{i(\omega t - kz)}$. Посмотрим на электрон на орбите атома,
\begin{equation*}
    m \ddot{z} = - k z - \beta \dot{z} + E_0 \cdot e^{-i(\omega t - kz)},
    \hspace{0.5cm} \overset{p = e z}{\Rightarrow}  \hspace{0.5cm} 
    \ddot{p} + 2 \delta \dot{p} + \omega_0^2 p = \frac{eE_0}{m} e^{-i(\omega t - kz)},
\end{equation*}
будем искать решения вида $p = p_0 e^{i\omega t}$, тогда
\begin{equation*}
    p = \frac{eE_0}{m} \frac{
    1
    }{
    (\omega_0^2 - \omega_2) - 2 \delta \omega i 
    } 
    e^{-i(\omega t - kz)}.
\end{equation*}
Вспомним, что $D$ равно
\begin{equation*}
     D = E + 4 \pi N p = 
     \bigg(
     \underbrace{
     1 + \frac{4\pi N e^2}{m} \cdot 
     \frac{1}{(\omega_0^2 - \omega^2) - 2 \delta \omega i} 
     }_{\varepsilon}
    \bigg) E.
\end{equation*}
По определению $n = \sqrt{\varepsilon}$, то есть
\begin{equation*}
    n^2 = \varepsilon = 1 + \frac{4\pi N e^2}{m} \cdot 
     \frac{1}{(\omega_0^2 - \omega^2) - 2 \delta \omega i}.
\end{equation*}

В случае проводника $\omega_0 = 0$, к тому же $2 \delta = 1 / \tau$, где $\tau$ -- характерное время затухания. Запишем, что
\begin{equation*}
    j = \sigma E, \hspace{0.5cm} j = N e v,
    \hspace{0.5cm} 
    v = \frac{eM}{m} \tau
    \hspace{0.5cm} \Rightarrow \hspace{0.5cm} 
    \tau = \frac{m\sigma}{Ne^2}. 
\end{equation*}
Тогда для проводника
\begin{equation*}
    n^2 = 1 - \frac{4\pi\sigma}{\omega i (1 -i \omega \tau)}.
\end{equation*}
Рассмотрим два случая, когда $\omega \tau \ll 1$, и $\omega \tau \gg 1$. 

\subsubsection*{Скин-эффект}


Первый случай приводит нас к явлению скин-эффекта.
\begin{equation*}
    n^2 = 1 - \frac{4\pi\sigma}{\omega i (1 -i \omega \tau)} 
    \approx
    1 - \frac{4\pi \sigma}{\omega i} \approx i \frac{4 \pi \sigma}{\omega},
    \hspace{0.5cm} \Rightarrow \hspace{0.5cm} 
    n = \left(\frac{1 + i}{\sqrt{2}}\right) \sqrt{\frac{4\pi\sigma}{\omega}}
    = \sqrt{\frac{2\pi\sigma}{\omega} } (1 + i)
    .
\end{equation*}
Посмотрим к чему это приводит. 
\begin{equation*}
    E = E_0 e^{-i(\omega t - kz)}, \hspace{0.5cm} 
    k = \frac{\omega}{c} \sqrt{\varepsilon \mu} = \frac{\omega}{c} n.
\end{equation*}
Тогда
\begin{equation*}
    E = E_0 e^{-i(\omega t - \frac{n}{c} z)} = E_0
    e^{-i \omega t} \exp\left(
        i \frac{\omega}{c} \sqrt{\frac{2\pi\sigma}{\omega} }z
    \right)
    \exp\left(
        - \frac{\omega}{c} \sqrt{\frac{2\pi\sigma}{\omega} }
    \right).
\end{equation*}
Появился множитель, соответсвующий экспоненциальному затуханию
\begin{equation*}
    z_0 = \sqrt{\frac{c^2}{2\pi \sigma \omega} },
\end{equation*}
где на $z_0$ происходит затухание в $e$ раз. 


\subsubsection*{Прозрачность металлов}

Первый случай приводит к прозрачности металлов.
\begin{equation*}
    n^2 = 1 - \frac{4\pi\sigma}{\omega i (1 -i \omega \tau)} 
    \approx
    1 - \frac{4\pi\sigma}{\omega^2 \tau} = 1 - \frac{\omega_{\text{p}}^2}{\omega^2} 
    ,
    \omega_{\text{p}}:2 = 
    \frac{4 \pi \sigma}{\tau} = \frac{4 \pi N e^2}{m}.
\end{equation*}
То есть при $\omega < \omega_{\text{p}}$ происходит отражение, а при $\omega > \omega_{\text{p}}$ металл становится прозрачным, то есть похожим на диэлектрик. 






% \subsubsection*{Бонус о плазме}





