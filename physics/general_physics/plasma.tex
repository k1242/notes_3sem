\textbf{Плазма} --- ионизированный квазинейтральный газ.

В плазме молекулярных токов нет $\vc{H} = \vc{B}$, обладает высокой электропроводностью.

Если в каком-то веществе протекает ток с объёмной плотностью $\vc{j}$, то возникает сила на единицу объёма с током: $\vc{f} = 1/c [\vc{j} \times \vc{B}]$.

Давление, которое может быть созданно на плазму магнитным поле -- \textbf{магнитное давление} = $B^2 /8\pi $. По плазменному шнуру, чтобы он не расползался, можно пустить такой ток, чтобы возникло магнитное поле $\vc{B}$: $n k T = B^2/8\pi $.

 \textbf{Дебаевский радиус}: 
 \begin{equation}
 	r_D  = \sqrt{\frac{k T}{8 \pi e^2 n}}. 
 \end{equation}

 Имеет смысл радиуса сферы вокруг внесённого в плазму заряда, в которой нарушается квазинейтральность плазмы.
 Количество частиц в дебаевской сфере, по которой различают ионизированный газ и плазму: $N \approx n r_D^3$. И если $N_D \,\raisebox{1.2pt}{\scalebox{0.55}{$\lesssim$}}\, 1$, то это называется ионизованным газом, а если $N_D \gg 1$ -- называется плазмой.

 Плазменная частота:
\begin{equation}
	\omega_\text{pl} = \sqrt{\frac{4 \pi n e^2}{m}},
\end{equation}
Например для колебаний смещенных электронов в плазме. Энергия таких колебаний: 
\begin{equation}
	W_\text{э} = \frac{E^2}{8 \pi} = 2 \pi (n e x)^2 \,\raisebox{1.2pt}{\scalebox{0.55}{$\lesssim$}}\, n k T \hspace*{1 cm} \leadsto \hspace*{1 cm} x_{\text{max}} \approx 2 r_D,
\end{equation}
где $x_\text{max} $ -- максимальная амплитуда таких колебаний (может служить определением дебаевского радиуса).

Для плазмы в переменном электрическом поле:
\begin{equation}
	m \ddot{r} = - e E_0 \cos \omega t \hspace*{1 cm} \leadsto \hspace*{1 cm} r = \frac{e}{m \omega^2} E_0 \cos \omega t,
\end{equation}
по такому закону будет колебаться каждый электрон. 

Запишем сразу дипольный момент единицы объёма с такими электронами:
\begin{equation}
	P = - n e r = - \frac{n e^2}{m \omega^2} E = \alpha E \hspace*{0.5 cm} \leadsto \hspace*{0.5 cm} (\varepsilon = 1+ 4 \pi \alpha) \hspace*{0.5 cm} \leadsto \hspace*{0.5 cm} \varepsilon = 1 -  \frac{\omega_\text{pl}^2}{\omega^2},
\end{equation}
от возбуждаемой частоты будут зависит оптические (электрические) свойства плазмы.

\textbf{Фазовая скорость в плазме}:
\begin{equation}
	v = \frac{c}{n} = \frac{c}{\sqrt{\varepsilon}}= \frac{\omega}{k} \hspace*{0.5 cm} \leadsto \hspace*{0.5 cm} \boxed{\omega = \sqrt{\omega_\text{pl}^2 + k^2 c^2}} \hspace*{0.5 cm} \leadsto \hspace*{0.5 cm} k = \sqrt{\frac{\omega^2}{c^2} - \frac{\omega^2_\text{pl}}{c^2 }}.
\end{equation}


Центральное выражение называется дисперсионной формулой для плазмы. Её можно выразить в виде последнего выражения, которое похоже на критическую частоту для волноводы. Оказывается, что дисперсионное соотношение для волн в волноводе и в плазме -- одинаковое.