\noindent
Считая в среде $\vc{j} = 0$, можно написать, что
\begin{equation}
    \left\{\begin{aligned}
        \rot \vc{E} &= - \frac{1}{c} \frac{\partial \vc{B}}{\partial t} \\
        \rot \vc{H} &= \frac{1}{c} \frac{\partial \vc{D}}{\partial t}
    \end{aligned}\right.;
    \hspace{0.1cm} 
    \left\{\begin{aligned}
        \vc{B} &= \mu \vc{H} \vphantom{\frac{1}{2}}\\
        \vc{D} &= \varepsilon \vc{E}  \vphantom{\frac{1}{2}}
    \end{aligned}\right.;
    \hspace{0.2cm} \Rightarrow \hspace{0.2cm} 
    \left\{\begin{aligned}
        \nabla \times \left[ \nabla \times \vc{E} \right] &= \cancel{\nabla \left(\nabla \cdot \vc{E}\right)} - \Delta \vc{E} \vphantom{\frac{1}{2}} \\
        \nabla \times \left[ \nabla \times \vc{E} \right] &= -\frac{\partial }{\partial t} \rot \vc{B}
    \end{aligned}\right. ;
    \hspace{0.5cm} \Rightarrow \hspace{0.5cm} 
    \boxed{
        \Delta \vc{E} = \frac{\varepsilon \mu}{c^2} \frac{\partial^2 E}{\partial^2 t}  = 0
    }.
\end{equation}
Аналогично для $\vc{B}$ мы можем записать, что
\begin{equation*}
    \frac{\partial^2 \vc{B}}{\partial^2 t}  = \frac{c^2}{\varepsilon \mu}  \Delta \vc{B}.
\end{equation*}
Если мы хотим видеть в волноводе целое число волн, то
\begin{equation*}
    \omega^2 = \frac{c^2 \pi^2}{\varepsilon \mu} \left[
        \left(\frac{n_x}{a_x}\right)^2 + 
        \left(\frac{n_y}{a_y}\right)^2 + 
        \left(\frac{n_z}{a_z}\right)^2  
    \right].
\end{equation*}