\section{Закон Кулона и теорема Гаусса}

Здесь попробуем индуктивно построить содержательную теорию, \textbf{начнём с двух эксперементальных фактов}, положенных в основу теории. Закона Кулона (сгсэ)
\begin{equation}
    \vc{F} = \frac{q_1 q_2 }{r^2} \frac{\vc{r}}{r},
\end{equation}
и, введя вектор напряженности электростатического поля $\vc{E} = \vc{F} / q$, принцип суперпозиции:
\begin{equation}
    \vc{E} = \sum \vc{E}_i.  
\end{equation}

\subsubsection*{Дипольный момент}
Простейшим примером системы зарядов является диполь $q_1 + q_2 = 0$, для которого введём $\vc{p}=q \vc{l}$:
$$
    \vc{E} = \frac{q}{r_1^2} \frac{\vc{r}_1}{r_1} - \frac{q}{r_1^2} \frac{\vc{r}_2}{r_2} 
    \hspace{0.5cm} \overset{l \ll r_2, r_1}{\underset{\Longrightarrow}{}} \hspace{0.5cm} 
    \vc{E} = \frac{3 (\vc{p} \cdot \vc{n}) \vc{n}}{r^3}  - \frac{\vc{p}}{r^3}
$$

Для заряженной нити верно, что
$$
    E = 2 \frac{\varkappa}{r}. 
$$



Теперь дойдём до двух теорем (кусочки уравнений Максвелла), описывающих электростатическое поле. 

\phantom{42}

\begin{minipage}[h]{0.6\textwidth}
    \begin{to_thr}[теорема Гаусса]
    Для потока $\vc{E}$ через замкнутую поверхность $S$ верно, что
    \begin{align}
        \oint_S E_n \d S =
        \boxed{
         \upoint_S (\vc{E} \d \vc{S}) 
            = 4 \pi q_{\textnormal{вн}}.   
        }
    \end{align}
\end{to_thr}
\end{minipage}
\hfill
\begin{minipage}{0.3\textwidth}
\begin{center}
    \incfig{1}
\end{center}
\end{minipage}


\begin{proof}[$\triangle$]
\textcolor{grey}{
    \begin{minipage}[t]{0.9\textwidth}
        \begin{enumerate}[label = \Roman*.]
            \item Доказательство (из закона Кулона) для сферы вокруг точечного заряда очевидно. 
            \item Рассмотрим произвольную поверхность $\Omega$, содержащую заряд, и телесный угол в онной:
            $$
                E_n \d S = E \cos \alpha \d S = E \d S' 
            $$
            То есть поток через наклонную площадку равен потоку через тот же телесный угол через некоторую вспомогательную сферу. Так как $s_1 / s_2 = r_1^2 / r_2^2$ и $E_1 / E_2 = r_2^2/r_1^2$, получается интегрировать по $\Omega$ то же самое, что и интегрировать по выбранной хорошей сфере. 
            \item Рассмотрим теперь некоторую $\Omega$, не содержащую заряд. Посмотрим на телесный угол от $q$. По модулю потоки через них одинаковые, а знаки противоположны, следовательно вклада в поток через $\Omega$ нет.
            \item Для сложного распределения зарядов, по принципу суперпозиции верно, что
            $$
                \vc{E} = \sum_i \vc{E}_i
                \hspace{0.5cm} \Rightarrow \hspace{0.5cm} 
                \oint_S E_n \d S = \sum_i \oint_S \vc{E}_i \d S.
            $$
        \end{enumerate}
    \end{minipage}
}

\phantom{42}
\end{proof}


