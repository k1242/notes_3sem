\section{Проводники}

\begin{to_def}[пусть так]
    \textit{Проводник} -- костяк частиц, окруженных \textit{свободными} электронами, которые в пределах тела могут перемещаться на какие угодно
    расстояния. 
\end{to_def}

\begin{minipage}[c]{0.55\textwidth}
    В частности, для проводников, верно, что
    \begin{align}
        E_n &= 4 \pi \sigma \\
        E_\tau &= 0
    \end{align}
\end{minipage}
\hfill
\begin{minipage}[c]{0.35\textwidth}
    \incfig{4}
\end{minipage}

Собственно, объёмных зарядов в проводнике нет, поверхностные есть и компенсируют внешнее поле. Аналогично работает решетка Фарадея, электростатическое поле не проникает в проводники.