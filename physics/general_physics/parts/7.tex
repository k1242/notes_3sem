\section{Магнитное поле в вакууме}

\subsection{Сила Ампера}


Ампер ввел элементы тока, тогда
\begin{equation}
    d \vc{F} = K_1 I \left[d \vc{l} \times \vc{B}\right],
\end{equation}
где $\vc{B}$ -- \textit{индукция магнитного поля}, силовая характеристика магнитного поля.


\subsection{Закон Био-Савара}
Ещё один эксперементальный факт:
\begin{equation}
    d \vc{B} = K_2 I \frac{\left[d \vc{l} \times \vc{r}\right]}{r^3} .
\end{equation}
Осталось поговорить про коэффициенты\footnote{
        В системе Гаусса $I, q, \Delta \varphi$ измерется в СГСЭ, а $B, H, L, M$ в СГСМ.
    } $K_1$ и $K_2$. 
Из $I^{(\text{М})} = \frac{1}{c} I^{(\text{Э})}$, получим

\begin{minipage}[t]{0.45\textwidth}
     в СГСМ:
    \begin{align*}
        d \vc{F} &= I \left[ \vc{l} \times \vc{B}\right] \\
        d \vc{B} &= I \ \frac{\left[d \vc{l} \times \vc{r}\right]}{r^3} .
    \end{align*}    
\end{minipage}
\hfill
\begin{minipage}[t]{0.45\textwidth}
    в системе Гаусса:
    \begin{align*}
        d \vc{F} &= \frac{I}{c} 
         \left[d \vc{l} \times \vc{B}\right] \\
        d \vc{B} &= \frac{I}{c} 
         \ \frac{\left[d \vc{l} \times \vc{r}\right]}{r^3} .
    \end{align*}    
\end{minipage}

Подставляя $\vc{j} = n q \vc{v}$, получим формулу следующего раздела.


\subsection{Сила Лоренца}


\begin{to_law} 
    Сила, действующая на движущийся точечный заряд $q$ в магнитном поле, получен обобщением опытных фактов,
    \begin{equation}
         \vc{F}_m = \frac{q}{c} \left[\vc{v} \times \vc{B}\right],
     \end{equation} 
    где вектор $\vc{B} \neq f(q, v)$ характеризует магнитное поле, напряженность магнитного поля.
\end{to_law}

Из этого можем найти, что
$$
    \vc{B} = \frac{c}{q \vc{v}^2_{\bot}} \left[
        \vc{F}_m \times \vc{v}_{\bot}
    \right],
$$
что однозначно определяет $\vc{B}$.

В предположение, что электрическое и магнитное поля действуют независимо, то $\vc{F} = \vc{F}_e + \vc{F}_m$, т.е.
$$
    \vc{F} = q \left(
        \vc{E} + \frac{1}{c} \left[\vc{v} \times \vc{B}\right]
    \right),
$$
где $\vc{F}$ -- сила Лоренца.




