\section{Магнитное поле в вакууме}


\begin{to_law} 
    Сила, действующая на движущийся точечный заряд $q$ в магнитном поле, получен обобщением опытных фактов,
    \begin{equation}
         \vc{F}_m = \frac{q}{c} \left[\vc{v} \times \vc{B}\right],
     \end{equation} 
    где вектор $\vc{B} \neq f(q, v)$ характеризует магнитное поле, напряженность магнитного поля.
\end{to_law}

Из этого можем найти, что
$$
    \vc{B} = \frac{c}{q \vc{v}^2_{\bot}} \left[
        \vc{F}_m \times \vc{v}_{\bot}
    \right],
$$
что однозначно определяет $\vc{B}$.

В предположение, что электрическое и магнитное поля действуют независимо, то $\vc{F} = \vc{F}_e + \vc{F}_m$, т.е.
$$
    \vc{F} = q \left(
        \vc{E} + \frac{1}{c} \left[\vc{v} \times \vc{B}\right]
    \right),
$$
где $\vc{F}$ -- сила Лоренца.

