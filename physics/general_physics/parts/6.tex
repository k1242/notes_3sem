\section{Виды диэлектриков}

Посмотрим на энергию внутри вакуума и диэлектрика, $E^2 / 8\pi$ и $E^2 / \varepsilon 8 \pi$. Энергия электрического поля определяется через работу внешних сил, которую необходимо затратить, чтобы это поле создать. Собственно, во втором случае есть ещё добавки. рассмотрим диэлектрик с упругими диполями, то есть пусть
$$
    F = \varkappa l.
$$
Пусть диполь попал во внешнее поле, тогда
$$
    Eq \cdot \frac{l}{2} = \varkappa l \cdot \frac{l}{2} = \frac{1}{2} E p.
$$
Тогда вся энергия, чтобы создать в этой среде поле
\begin{align*}
    W = \frac{E^2}{8\pi} + \frac{E P}{2} = \frac{E^2}{8\pi} + \frac{1}{2} E^2 \alpha 
    =  \frac{E^2}{8\pi} +  \frac{E^2}{8\pi} (\varepsilon - 1) = \frac{\varepsilon E^2}{8 \pi} .
\end{align*}

А если работать с диэлектриками с собственным дипольным моментом? Тогда ещё появиться некоторое тепло, которое необходимо отдать термостату, увеличивая упорядоченность системы. Постараемся обобщить, для этого вспомним, что

\begin{to_def} 
     \textit{Свободная энергия} -- функция состояния, приращение которой в обратимом изотермическом процессе равно совершаемой работе внешних сил.
\end{to_def}

Так вот, то что мы называем энергией электрического поля (в диэлектриках), на самом деле это объёмная плотность свободной энергии $\Psi = U - TS$.


%%%%%%%%%%%%%%%%%%%%%%%%%%%%%%%%%%%%%%%%%%%%%%%%%%%%%%%%%%%%%%%%%%%%%%%%%%%%%%%%%%%
\section{Теория постоянных токов}

\begin{to_def} 
    \textit{Сила тока} -- заряд, протекший через сечений проводника в единицу времени, 
    \begin{equation}
        I = \frac{dq}{dt}.
    \end{equation}
    \textit{Плотность тока} -- ток, протекающий через единичное сечение.
    \begin{equation}
        \vc{j} = n e \vc{u}.
    \end{equation}
\end{to_def}


\begin{to_law}[закон Ома]
     Для класса линейных проводников верно, что при наличии разности потенциалов $U$
     \begin{equation}
         I = \frac{U}{R} 
         \hspace{0.5cm} \Leftrightarrow \hspace{0.5cm} 
         \vc{j} = \lambda \vc{E},
     \end{equation}
     где $\lambda = 1 / \rho$, обратное удельное сопротивление.
\end{to_law}

В СГСЭ, кстати, $\dim \rho = \text{с}$, а в СИ 1 ед. СГСЭ  $\rho = 9 \cdot 10^9$ Ом.

\subsubsection*{Условие стационарности}

Пусть в некоторый узел втекает $I_1, \ldots, I_n$, тогда
$$
    \oint_{(S)} j_n \d S = - \dot{Q}.
$$
Это <<закон сохранения заряда>>, или уравнение непрерывности. В частности, в стационарном случае
\begin{equation}
    \boxed{
        \oint j_n \d S = 0
    }.
\end{equation}
Получается (??), что поле зарядов, которые участвуют в протекании постоянных токов можно описывает с помощью электростатических формул, то есть применять теорему Гаусса и теорему о циркуляции.

По теореме Гаусса и условия стационарности,
$$
    0 = \oint j_n \d S = \lambda \oint E_n \d S = \lambda 4 \pi q,
$$
то есть для проводников с постоянным током всё ещё верно, что внутреннего заряда в проводниках нет, а есть только поверхностный.

Невозможна стационарная ситуация с постоянных током только на потенциальных силах.
Для участка цепи, в котором действуют сторонние силы, можно записать
\begin{equation}
    \vc{j} = \lambda \left(\vc{E} + \vc{E}^{\text{стор}}\right).
\end{equation}

\begin{to_def} 
    \textit{ЭДС} -- электро-движущая сила, работа совершаемая сторонними силами при перемещении единичного заряда по рассматриваемому участку, 
    \begin{equation}
        \mathcal E = \int_{(I)} E_l^{\text{стор}} \d l.
    \end{equation}
\end{to_def}


\subsubsection*{Правила Кирхгофа}

\noindent
\begin{minipage}[c]{0.7\textwidth}
\phantom{42} \indent
    Рассмотрим узел, в который втекает $I_1, \ldots, I_n$. Из условия стационарности получим (I). Рассмотрев замкнутый участок цепи, получим (II) правило Кирхгофа. Действительно, $j_l = \lambda \left(E_l+E_l^{\text{стор}}\right)$, или
    $$
        \oint \frac{I \d l}{\lambda S} = \oint \left(E_l+E_l^{\text{стор}}\right) \d l,
        \hspace{0.25cm} \text{где} \hspace{0.25cm} 
        \oint \frac{I \d l}{\lambda S} = I R.
    $$
    Но для каждого участка $I_i R_i = \Delta \varphi_i + \mathcal E_i$
    . Это с учётом направления тока.
\end{minipage}
\hfill
\begin{minipage}[c]{0.25\textwidth}
    \begin{enumerate}[label = \Roman*.]
        \item $\sum I_i = 0$.
        \item $\encircled{\sum} I_i R_i = \encircled{\sum} \mathcal E_i$
    \end{enumerate}
\end{minipage}

\phantom{42}

\textit{Оказывается}, для любой цепи, записав уравнения Кирхгофа для всех узлов и всех независимых контуров, получим разрешимую единственным образом систему уравнений (ну или хотя бы столько, сколько можно). 



