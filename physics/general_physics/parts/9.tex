\section{Магнитное поле в намагничивающихся средах}

\subsection{Уравнения максвелла для магнитного поля в веществе}

Посмотрим на рамку с током в магнитном поле. Для неё верно, что суммарная сила, действующая на рамку,
$$
    \vc{B} \oint \d \vc{F} = \frac{I}{c} \oint [\d \vc{l} \times \vc{B}] = 
    \frac{I}{c}  \left[\oint \d \vc{l} \times \vc{B}\right] =
     0.
$$
Однако момент, действующий на рамку, не равен 0,
$$
    S = a b, \hspace{0.5cm} F = \frac{I}{c} b B 
    \hspace{0.5cm} \Rightarrow \hspace{0.5cm} 
    M = \frac{IS}{c} \sin \alpha
    \hspace{0.5cm} \Rightarrow \hspace{0.5cm} 
    \vc{p}_m = \frac{IS}{c} \vc{n}
    \hspace{0.5cm} \Rightarrow \hspace{0.5cm} 
    \vc{M} = \left[\vc{p}_m \times \vc{B}\right].
$$

Посмотрим теперь на рамку в неоднородном магнитном поле. Рассмотрим рамку такую, что $\vc{p}_m \parallel \vc{B}$, тогда $\vc{I} \d \vc{l}$ имеет проекцию на $\vc{n}$, получается, что
$$
    F_x = (p_m)_x \frac{\partial B_x}{\partial x}.
$$


Возвращая к полю, предполагается, что внутри молекул формируются \textit{молекулярные токи}, создающие дополнительный магнитный момент, а при наличие внешнего поля происходит ориентация этих моментов. Тогда теорема о циркуляции магнитного поля в веществе запишется, как
\begin{equation}
    \oint_{(L)}
    \left(\vc{B} \cdot \d \vc{l} \right)
    = \frac{4\pi}{c} \left(
        I_{\text{пров}} + I_{\text{мол}}
    \right).
\end{equation}
Стоит заметить, что в теории Максвелла имеется ввиду, что
$$
    \vc{B} = \vc{<B}_{\mu}>.
$$

Характеристика, описывающая состояние намагниченного вещества в точке -- магнитный дипольный момент, единице объема:
$$
    \vc{\mathcal I} = \frac{1}{\Delta V} \sum_{\Delta V}
    \left(\vc{p}_m\right)_i.
$$

Можем записать, что
\begin{equation}
    \oint \mathcal I_l \d l = \frac{I_{\text{мол}}}{c} .
\end{equation}
Тогда уравнение перепишется, как
\begin{equation}
    \oint_{(L)} \left( \vc{B} \d \vc{l} \right) = \frac{4\pi}{c} I_{\text{пров}} + 4\pi \oint_{(L)} \left(\vc{\mathcal  I} \d \vc{l} \right).
\end{equation}

\begin{equation}
    \oint_{(L)}
    \underbrace{\left( 
    \vc{B} - 4 \pi \vc{\mathcal I} 
    \right)}_{\vc{H}}
     \d \vc{l} = \frac{4\pi}{c} I_{\text{пров}}, 
\end{equation}

здесь принимается определение $\boxed{\vc{H} = \vc{B} - 4 \pi  \vc{\mathcal  I}}$ -- \textit{напряженность магнитного поля}. 

Далее нас интересует линейная намагничиваемость:
$$
    \vc{\mathcal I} = \varkappa \vc{H},
$$
где $\varkappa$ -- магнитная восприимчивость. Тогда можем записать, что
\begin{equation}
    \vc{H} \underbrace{\left(
        1 + 4 \pi \varkappa
    \right)}_{\mu} = \vc{B},
\end{equation}
что записано в системе Гаусса. В СИ верно, что
$$
    \vc{H} = \frac{\vc{B}}{\mu_0} - \vc{\mathcal I} .
$$

\subsection{Различные вещества}

\begin{enumerate}[label = \Roman*.]
    \item Парамагнетики, $\varkappa \in [10^{-3}, 10^{-6}]$, пример: алюминий. 
    \item Диамагнетики, $\varkappa < 0$ , пример: золото, серебро, см. модель Ланжевена.
    \item Ферромагнетики, $\varkappa \in [10^3, 10^6]$, пример: железо, никель.
\end{enumerate}




\subsection{Граничные условия}

Рассмотрим границу двух веществ с $\mu_1$ и $\mu_2$. Тогда
$$
    \vc{B}_{1n} = \vc{B}_{2n},
$$
а для тангенциальной компоненты 
$$
    H_{2\tau} - H_{1\tau} = \frac{4\pi}{c} i_N.
$$

 
