\section{Потенциал электрического поля.}

\begin{minipage}{0.6\textwidth}
\begin{to_thr}[Теорема о циркуляции]
    Для заряда, при квазистатическом перемещении, верно, что
    \begin{align}
      A_{\text{замкн}} = \boxed{
            \oint_{(L)} (\vc{E} \cdot d \vc{l}) = 0
        }  \\
        \label{rote}
        \rot \vc{E} = 0
    \end{align}
\end{to_thr}

\end{minipage}
\hfill
\begin{minipage}[contentpos=t]{0.3\textwidth}
\begin{center}
        \incfig{2}
\end{center}
\end{minipage}


\begin{proof}[$\triangle$]
\textcolor{grey}{
    \begin{minipage}[t]{0.9\textwidth}
        \begin{enumerate}[label = \Roman*.]
            \item Рассмотрим поле точечного заряда $Q$ и перемещение с $\vc{r}$ до $\vc{r} + d \vc{l}_r + (d \vc{l} - d \vc{l}_r)$. Тогда $d A = (\vc{E} \cdot d \vc{l}) = \frac{Q}{r^2} dl_r$, то есть $A \equiv A(r_1, r_2)$.
            \item Для поля в принципе вышесказанное верно по принципу суперпозиции.
        \end{enumerate}
    \end{minipage}
}

\phantom{42}
\end{proof}


\begin{to_def} 
    \textit{Разностью потенциалов} $\varphi_1 - \varphi_2$ между точками $\vc{r}_1$ и $\vc{r}_2$ называется $A = \int_{r_1}^{r_2} \vc{E} d \vc{l}$,  при перемещении единичного положительного заряда. Потенциал определен с точностью до произвольной аддитивной постоянной. 
\end{to_def}

В частности, для точечного заряда, при $\varphi_{\infty} = 0$, верно
$$
    \varphi (r) =
    \int_r^\infty \frac{Q(r)}{r^2} \d r
    = \frac{Q}{r} .
$$
А для двух зарядов, $+q, -q$
\begin{align*}
    \varphi = -\frac{q}{r_1} + \frac{q}{r_2} = q \frac{(r_1-r_2)}{r_1 r_2} 
    \hspace{0.5cm} \overset{r \gg l}{\underset{\Rightarrow}{}} \hspace{0.5cm} 
    \varphi = \frac{1}{r^2}  \frac{(\vc{p} \cdot \vc{r})}{r} 
\end{align*}


\subsection{Дифференциальная форма записи}
Вектор напряженности электростатического поля
\begin{equation}
\label{gradphi}
\boxed{
    \vc{E} = - \grad \varphi.
}
\end{equation}
Действительно,
$$
    d \varphi = - (\vc{E} \cdot d \vc{l}) = \frac{\partial \varphi}{\partial x^i} d x_i = d \vc{l} \cdot \nabla \varphi, \text{ где } \nabla \varphi \equiv  \grad \varphi.
$$
А теперь рассмотрим некоторый элементарный параллелепипед. Тогда поток через левую грань это $- E_x \d y \d z$, а через правую это $\left(E_x + \frac{\partial E_x}{\partial x} dx\right) \d y \d z$. Тогда суммарный поток через мааленький параллелепипед равен $dV \, \partial E / \partial x$, а теорема Гаусса примет вид
\begin{equation}
\label{dive}
    \left(\frac{\partial E}{\partial x} + \frac{\partial E}{\partial y} + \frac{\partial E}{\partial z} \right) \d V = 4 \pi \rho \d V
    \hspace{0.5cm} \Rightarrow \hspace{0.5cm} 
    \boxed{
        \div \vc{E} = 4 \pi \rho
    }.
\end{equation}

\subsection{Граничные условия на заряженной поверхности}
\begin{minipage}[c]{0.45\textwidth}
\noindent
    По теореме Гаусса верно, что
    \begin{align*}
        E_{2n_2} \cancel{\Delta S} + E_{1n_1} \cancel{\Delta S} &= 4\pi\sigma \cancel{\Delta S}, \\
        E_{2n} - E_{1n} &= 4\pi\sigma
    \end{align*}
\noindent
    По теореме циркуляции верно, что
    \begin{align*}
        E_{2l} \cancel{\Delta} l - E_{1l} \cancel{\Delta} l &= 0 \\
        E_{2l} - E_{1l} &= 0.
    \end{align*}
\end{minipage}
\hfill
\begin{minipage}[c]{0.45\textwidth}
    \incfig{3}
\end{minipage}

