\subsection{Уравнения Максвелла}


\begin{minipage}[t]{0.2\textwidth}
\phantom{42}    
Ди-форма в СГС:
    \begin{align*}
        \vphantom{\oint }
        \div \vc{D} &= 4 \pi \rho \\
        \vphantom{\oint }
        \div \vc{B} &= 0 \\
        \vphantom{\oint }
        \rot \vc{E} &= -\frac{1}{c} \frac{\partial \vc{B}}{\partial t}  \\
        \vphantom{\oint }
        \rot \vc{H} &= \frac{4\pi}{c} \vc{j} + \frac{1}{c} \frac{\partial \vc{D}}{\partial t} 
    \end{align*}
\end{minipage}
\vline
\hfill
\begin{minipage}[t]{0.23\textwidth}
\phantom{42}
Ин-форма в СГС:
    \begin{align*}
       \oint \vc{D} \cdot d \vc{s} &= 4\pi Q \\
       \oint \vc{B} \cdot d \vc{s} &= 0 \\
       \oint \vc{E} \cdot d \vc{l} &= - \frac{1}{c} \frac{d}{dt} \int \vc{B} \cdot d \vc{s} \\
        \oint \vc{H} \cdot \d \vc{l} &= \frac{4\pi}{c} I + \frac{1}{c} \frac{d}{dt} \int \vc{D} \cdot \d \vc{s}.
    \end{align*}
\end{minipage}
\vline
\hfill
\begin{minipage}[t]{0.2\textwidth}
\phantom{42} Ди-форма в СИ:
    \begin{align*}
        \vphantom{\oint }
        \div \vc{D} &= \rho \\
        \vphantom{\oint }
        \div \vc{B} &= 0 \\
        \vphantom{\oint }
        \rot \vc{E} &= - \frac{\partial \vc{B}}{\partial t}  \\
        \vphantom{\oint }
        \rot \vc{H} &= \vc{j} +  \frac{\partial \vc{D}}{\partial t} 
    \end{align*}
\end{minipage}
\vline
\hfill
\begin{minipage}[t]{0.23\textwidth}
\phantom{42}    Ин-форма в СГС:
    \begin{align*}
       \oint \vc{D} \cdot d \vc{s} &=  Q \\
       \oint \vc{B} \cdot d \vc{s} &= 0 \\
       \oint \vc{E} \cdot d \vc{l} &= - \frac{d}{dt} \int \vc{B} \cdot d \vc{s} \\
        \oint \vc{H} \cdot \d \vc{l} &=  I + \frac{d}{dt} \int \vc{D} \cdot \d \vc{s}.
    \end{align*}
\end{minipage}


Поле Диполя:
\begin{equation}
    \vc{E} = \frac{1}{R_0^3}  \left(
        3 (\vc{n} \vc{d}) \vc{n} - \vc{d}
    \right)
\end{equation}