\subsection{Основная задача электростатики}

Вместо поиска $\vc{E}$ достаточно найти $\varphi$, воспользовавшись \eqref{gradphi} и \eqref{dive}, получим
$$
    \div \grad \varphi \equiv \delta \varphi = 
    \left\{\begin{aligned}
        &- 4 \pi \rho &\text{ур. Пуассона} \\
        &\phantom{42} 0 &\text{ур. Лапласа}
    \end{aligned}\right.
$$

Как может быть поставлена задача? Заданы граничные значение, найти распределения зарядов. Заданы заряды, найти распределения. Что-то задано, что-то не задано. Во всех трёх случаях \textbf{решение уравнения Лапласа единственно}.

\subsubsection*{Метод изображений}

Если существует некоторая эквипотенциальная поверхность разделяющая пространство на два полупространства, то можем считать что эта поверхность является проводящей. 

% И?

\section{Диэлектрики}

\begin{to_def} 
    \textit{Диэлектрики} -- непроводники электричества. В них возбуждаются индукционные заряды, привязанные к кастету частиц, -- \textit{поляризационные}, или \textit{связанные заряды}. 

    Альтернативный вариант, -- наличие дипольного момента у молекул. При наличии электрического поля дипольные моменты ориентируются, диэлектрик попользуется. 
\end{to_def}


\begin{to_def} 
    \textit{Вектор поляризации} -- дипольный момент единицы объема диэлектрика, возникающий при его поляризации.  
\end{to_def}


Рассмотрим скошенный параллелепипед. На основаниях параллелепипеда возникнут поляризационные заряды с  поверхностной плотностью $\sigma_{\text{пол}}$. Взяв его площадь за $S$, найдём дипольный момент равный $\sigma_{\text{пол}} S \vc{l}$. Тогда вектор поляризации будет 
\begin{equation}
    \vc{P} = \frac{\sigma_{\text{пол}} S}{V} \vc{l},
\end{equation}
что верно и для анизотропных кристаллов где $\vc{E} \nparallel \vc{P}$. 

Пусть $\vc{n}$ -- единичный вектор внешней нормали к основанию параллелепипеда, тогда $V = S (\vc{l} \cdot \vc{n})$.

\noindent
\begin{minipage}[с]{0.55\textwidth}
Подставив $V$ в предыдущую формулу, получим, что
\begin{equation}
    \sigma_{\text{пол}} = (\vc{P} \cdot \vc{n}) = P_n
\end{equation}
Или, более общо,
$$
    \vc{P} = \frac{1}{\Delta V} \sum_{\Delta V} \vc{p}_i
$$
\end{minipage}
\hfill
\begin{minipage}[с]{0.35\textwidth}
    \incfig{5}
\end{minipage}


В случае неоднородной поляризации верно, что поляризационные заряды могут появиться и на поверхности. Выделим $V$, ограниченный $S$, смещённый заряд равен $-P_n \d S$, тогда через $S$ поступает
\begin{equation}
    q_{\text{пол}} = - \oint P_n \d S = - \oint \left(\vc{P} \times \d \vc{S}\right).
\end{equation}
Стоит заметить, что в теорему о циркуляции не входят заряды, соотвественно для диэлектриков верно, что 
$$
    \oint_{(L)} E_l \d l = 0.   
$$

Далее чаще всего мы будем сталкиваться с линейной поляризацией, когда
$$
    \vc{P} = \alpha \vc{E},
    \hspace{0.5cm} \Rightarrow \hspace{0.5cm} 
    \vc{D} = \vc{E} \underbrace{(1 + 4\pi\alpha)}_{\varepsilon} = \varepsilon \vc{E},
$$
где $\alpha$ -- \textit{поляризуемость диэлектрика}, а $\varepsilon$ -- \textit{диэлектрическая проницаемость}.

 \subsection{Теорема Гаусса}
Запишем теорему Гаусса для электрического поля в диэлектрике. Знаем, что $\vc{E} = \vc{E}_{\text{пол}} + \vc{E}_{\text{св}}$.
\begin{equation}
    \oint E_n \d S = 4 \pi (q + q_{\text{пол}})
    \hspace{0.5cm} \Rightarrow \hspace{0.5cm} 
    \oint\underbrace{\left(
        E_n + 4 \pi P_n
    \right)}_{D_n} \d S= 4 \pi q.
    \hspace{0.5cm} \Rightarrow \hspace{0.5cm} 
    \boxed{
        \oint D_n \d S = 4 \pi q_{\text{св}}   
    }
\end{equation}
где $\vc{D} = \vc{E} + 4 \pi \vc{P}.$ -- вектор \textit{электрической индукции}, или \textit{электрического смещения}. Поток вектора $\vc{D}$ определяется только свободными зарядами. 

Можно посмотреть на это в дифференциальной форме:
\begin{align*}
    \div \vc{D} &= 4 \pi \rho, \\
    \div \vc{E} &= 4 \pi \left(\rho - \div \vc{P} \right), \\
    \div \vc{E} &= 4 \pi (\rho + \rho_{\text{пол}})
    \hspace{0.5cm} \Rightarrow \hspace{0.5cm} 
    \rho_{\text{пол}} = - \div \vc{P}.
\end{align*}




%%%%%%%%%%%%%%%%%%%%%%%%%%%%%%%%%%%%%%%%%%%%%%%%%%%%%%%%%%%%%%%%%%%%%%%%%%%%%%%%%%
\subsection{Граничные условия на границе двух диэлектриков}


Повторя рассуждения для проводников, найдём, что
$$
    D_{1n} = D_{2n},
$$
а в случае линейных диэлектриков верно
$$
    \varepsilon_1 E_{1n} = \varepsilon_2 E_{2n}.
$$
Или
$$
    E_{2n} - E_{1n} = 4 \pi \sigma_{\text{пол}}.
$$
Аналогично, из теоремы о циркуляции получим, что
$$
    E_{1\tau} - E_{2\tau} = 0.
$$

\subsubsection*{Плоский конденсатор}

\begin{center}
    \incfig{6}    
\end{center}

\phantom{42}    

То есть на грани пластинки $\sigma_{\text{пол}} = \sigma \left(1 - \frac{1}{\varepsilon} \right)$.


\subsection{Поле системы зарядов в однородном диэлектрике}

Для точечного заряда в однородном диэлектрике, по теореме Гаусса
$$
\left.\begin{aligned}
    D &\cdot 4 \pi r^2 = 4 \pi \\
    D &= \varepsilon E
\end{aligned}\right\} 
\hspace{0.5cm} \Rightarrow \hspace{0.5cm} 
\boxed{
    E = \frac{q}{\varepsilon r^2} .
}
$$
То ест в общем случае, по принципу суперпозиции, в диэлектрике
$$
    \vc{E} = \frac{1}{\varepsilon} \vc{E}_0.
$$


