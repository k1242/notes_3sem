Есть такое определение магнитного поля $\vc{B}$:
\begin{equation*}
    \vc{B} = \frac{q}{cr^3} \left[\vc{v} \times \vc{r}\right] = \frac{1}{c} \left[\vc{v} \times \vc{E} \right],
\end{equation*}
в СГСЭ.

Можно из $I \d \vc{l} = \vc{j} \d V$, перейти к закону 
\begin{equation*}
    d \vc{B} = \frac{I}{c} \left[d \vc{l} \times \vc{r} \right],
    \hspace{0.5cm} 
    d \vc{B} = \frac{I}{c} \left[ \vc{j} \times \vc{r} \right].
\end{equation*}
Например, для прямого провода есть самозамкнутые кружочки вокруг онного с модулем
\begin{equation*}
    B = \frac{2I}{cr}.
\end{equation*}

Если взять маленький виток с проводом, то конфигурация полей аналогична полю диполю. В центре витка поле будет
\begin{equation*}
    B =  \frac{2\pi I}{cr}.
\end{equation*}
Так вот для витка 
\begin{equation*}
    \vc{\mathfrak{M}} = \frac{1}{c} I \vc{S}.
\end{equation*}
Поле вокруг магнитного диполя
\begin{equation*}
    \vc{B} = \frac{3}{5} \frac{\left(\vc{\mathfrak{M}} \cdot \vc{r} \right)}{r^5} \vc{r} - \frac{1}{r^3} \vc{\mathfrak{M}}.
\end{equation*}
Сила Лоренца:
\begin{equation}
    \vc{F} = e \left(\vc{E} + \frac{1}{c} [\vc{v} \times \vc{B}]\right).
\end{equation}
Введём линейную плотность тока и запишем, где $\Omega$ -- телесный угол площадочки, 
\begin{equation*}
    i = \frac{I}{l},
    \hspace{0.5cm} 
    d B_{\tau} = \frac{i}{c} \d \Omega.
\end{equation*}
Тогда поле внутри соленодиа
\begin{equation*}
    B = \frac{i}{c} 4 \pi, \hspace{0.5cm} i = \frac{IN}{l}.
\end{equation*}
Для плоскости по которой течёт $i$, 
\begin{equation*}
    B = \frac{2\pi}{c} i.
\end{equation*}
Для двух плоскостей аналогично, (а-ля магнитный конденсатор)
\begin{equation*}
    B = \frac{4\pi}{c} i.
\end{equation*}
Кстати, для телесного угла,
\begin{equation*}
    \Omega = 2 \pi (1 - \cos \alpha).
\end{equation*}
Для погнутого в окружность радиуса $R$ соленоида, площади $S$, с $i$
\begin{equation*}
    B_O = \frac{2iS}{cR^2} = \frac{2\pi i}{c} \left(\frac{r}{R} \right)^2.
\end{equation*}

\subsubsection* {Магнетики -- магнитное поле в веществе}

Магнитный момент молекулярных токов:
\begin{equation*}
    \vc{I} = \frac{\vc{\mathfrak{M}}}{V} = \frac{1}{c} i_{\text{мол}} \vc{l}, \hspace{0.5cm} \Rightarrow \hspace{0.5cm} 
    i_{\text{мол}} = c \left(\vc{I} \cdot \vc{l} \right),
    \hspace{0.5cm} 
    i_{\text{мол}} = c \oint_L \left(\vc{I} \cdot \d \vc{l} \right).
\end{equation*}
Кстати,
\begin{equation*}
    \vc{I} = \varkappa \vc{H},
\end{equation*}
где $\varkappa$ -- магнитная восприимчивость.
Также 
\begin{equation*}
    \vc{B} = \mu \vc{H},
\end{equation*}
где $\mu = 1 + 4 \pi \varkappa$ -- магнитная проницаемость.

