Поле диполя:
\begin{equation*}
    \vc{E} = \frac{3 \left(\vc{p} \cdot \vc{r}\right)}{r^5}   \vc{r} - \frac{ 1}{ r^3 } \vc{p}.  
\end{equation*}
Или, считая $\vc{n}$ -- единичным вектором вдоль $r$:
\begin{equation*}
    \vc{E} = \frac{1}{r^3}  \left(
        3 (\vc{n} \cdot   \vc{p}) \vc{n} - \vc{p}
    \right)
\end{equation*}
Момент сил, действующих на диполь в однородном магнитном поле
\begin{equation*}
    \vc{M} = \left[\vc{p} \times \vc{E}\right].
\end{equation*}
Энрегия диполя:
\begin{equation*}
    W = \int \delta A = \int M \d \alpha = 
    pE \cos \alpha \bigg|_{\alpha}^{\alpha_0 = \pi/2} = - \left(\vc{p} \cdot \vc{E}\right).
\end{equation*}
Теперь для неупругого диполя:
\begin{equation*}
    kl = qE, \hspace{0.25cm} \Rightarrow \hspace{0.25cm}   
    U = \frac{1}{2} \left(\vc{p} \cdot \vc{E}\right). 
\end{equation*}
Для диполя в неоднородном поле:
\begin{equation*}
    \vc{F} = q (\vc{E}_2 - \vc{E}_1) \approx q \d \vc{E},
    \hspace{0.5cm} d \vc{E} = l^i \partial_i \vc{E},
    \hspace{0.5cm} \Rightarrow \hspace{0.5cm} 
    \vc{F} = p^i \partial_i \vc{E} = \left(\vc{p} \cdot \nabla\right) \vc{E}
\end{equation*}
Взаимодействие двух сонаправленных диполей:
\begin{equation*}
     E_1 = -6\frac{p_1}{d^4}, \hspace{0.5cm} \Rightarrow \hspace{0.5cm} 
     F = p_2 \frac{d E_1}{d x} = - \frac{6p_1p_2}{x^4}.
\end{equation*}
Для двух равномерно заряженных сфер, смещённых на $\vc{l}$, поле внутри области пересечения
\begin{equation*}
    \vc{E}(A) = - \frac{4}{3} \pi \rho \vc{l}.
\end{equation*}
Найдём теперь такое распредление заряда, чтобы поле внутри всей сферы было $\vc{E}_0$. Толщина заряженной части $l' = l \cos \theta$, тогда
\begin{equation*}
    \rho = \frac{3}{4} \frac{E_0}{l}, \hspace{0.5cm} \Rightarrow \hspace{0.5cm} 
    \sigma(\theta) = \rho l' = \rho l \cos \theta =  \frac{3}{4\pi} E_0 \cos \theta,
\end{equation*}
при чём для этой сферы дипольный момент
\begin{equation*}
    \vc{P} = q \vc{l} = \frac{4}{3} \pi R^3 \rho \vc{l} = - R^3 \vc{E}_0.
\end{equation*}



