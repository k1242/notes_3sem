\section{Электрическое поле в веществе}

\begin{to_def} 
    \textit{Диэлектрик} -- совокупность заряженных частиц в вакууме, без свободного заряда. Все заряды <<привязаны>> к частицам, однако может возникать избыточный заряд.
\end{to_def}

Немного поправим\footnote{
    Ещё раз, почему возникают поляризационные заряды? Потому что мы достаточно далеко разнесли состовялющие некогда нейтральной частицы?
} теорему Гаусса:
$$
    \oint_S E_n \d S = 4\pi (q + q_{\text{пол}})
$$
Введём вектор поляризованности
$$
    \vc{P} = \frac{1}{\Delta V} \sum_{\Delta V} \vc{p}_i.
$$
Или, говоря о поверхностных зарядах:
$$
    P_n = \sigma_{\text{пол}}, \hspace{0.5cm} \oint_S P_n \d S = - q_{\text{пол}}.
$$
Тогда
\begin{equation}
    \oint_S
    \underbrace{\left(\vc{E} + r\pi \vc{P} \right)}_{\vc{D}}
    \d S = 4\pi q 
\end{equation}
Ну и, не забывая о циркуляции,
$$
    \oint_L E_l \d l = 0.
$$
А рассматривая класс линейных диэлектриков $\vc{P} = \alpha \vc{E}$, получим, что
$$
    \vc{D} = \vc{E} (\underbrace{1 + 4 \pi \alpha}_{\varepsilon}),
$$
где
$
    \alpha \text{ -- коэффициент поляризации},$ $
    \varepsilon \text{ -- диэлектрическая проницаемость}.
$



\subsubsection*{Граничные условия на границе двух диэлектриков}

Для границы, рассмотрев область без свободных зарядов, с учётом теормы Гаусса, можно записать, что $D_{1n} = D_{2n}$, или
$$
    \varepsilon_1 E_{1n} = \varepsilon_2 E_{2n},
$$
или
$$
    E_{2n} - E_{1n} = 4 \pi \sigma_{\text{пол}}.
$$

Посмотрим на тангенсальную состовляющую. Рассмотрим вытянутый прямоугольник вдоль границы, тогда получим, что
$$
    E_{1\tau} = E_{2\tau}.
$$

\subsubsection*{Плоский конденсатор}


Рассмотрим плоский конденсатор, засунем в него диэлектрическую пластинку. Возникнут поляризационные заряды, тогда
$$
    \vc{D}_1 = \vc{D}_2 = E_0 = 4\pi\sigma.
$$
Поле внутри диэлектрика тогда
$$
    E_2 = \frac{4\pi\sigma}{\varepsilon}; \hspace{0.5cm} 
    E_1 - E_2 = 4\pi\sigma\left(1- \frac{1}{\varepsilon} \right)  = 4\pi\sigma_{\text{пол}}
    \hspace{0.5cm} \Rightarrow \hspace{0.5cm} 
    \sigma_{\text{пол}} = \sigma \frac{\varepsilon-1}{\varepsilon}.f
$$


\subsubsection*{Поле системы зарядов в диэлектрике}
По теореме Гаусса
$$
    D \cdot 4\pi r^2 = 4 \pi q, \hspace{0.25cm}  D = \varepsilon E
    \hspace{0.5cm} \Rightarrow \hspace{0.5cm} 
    E = \frac{q}{\varepsilon r^2} 
$$
По принципу суперпозиции
$$
    \vc{E} = \sum \vc{E}_i = \sum \frac{\vc{E}_{i0}}{\varepsilon} 
    = \frac{\vc{E}_0}{\varepsilon} .
$$
Но только для линейной поляризации.

Получается, для конденсатора верно, что при заполнение онного диэлектриком, напряжение $U_0$ изменится так, что
$$
    U =  \frac{U_0}{\varepsilon} , \hspace{0.5cm} C = \frac{q}{U} 
    \hspace{0.25cm} \Rightarrow \hspace{0.25cm} C = \varepsilon C_0.
$$  

