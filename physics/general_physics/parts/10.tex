% \section{Электромагнитная индукция}


 \section{Электромагнитная индукция}

\subsection{Понимания}

\begin{to_def}[Понимание Фарадея]
    Для движущейся перемычки в замкнутом контуре, помещенного в магнитное поле, можно записать силу лоренца, которая будет толкать каждый носитель заряда в ней как: 
    \begin{equation*}
        \vc{E} = \frac{\vc{F}}{e} = \frac{1}{c} [\vc{v} \times \vc{B}]. 
    \end{equation*}
    
    Электродвижущая сила, создаваемая этим полем называется \textit{электродвижущей силой}. И для магнитного потока пронизывающего площадь рамки:

    \begin{equation}
        \mathcal{E} = -\frac{1}{c} \frac{\d \Phi}{\d t}
    \end{equation}
\end{to_def}


\begin{to_def}[Понимание Максвелла]
    Всякое изменение магнитного поля во времени возбуждает в окружающем пространстве электрическое поле. Циркуляция $E$ по $\forall$ замкнутому контуру определяется:

    \begin{equation}
        \oint_s (E \d s) = - \frac{1}{c} \frac{\partial \Phi}{\partial t}, \hspace*{3cm} \text{rot}\, \vc{E} = -\frac{1}{c} \frac{\partial \vc{B}}{\partial t}
    \end{equation}
    где $\Phi = \oint_s \vc{B} \d \vc{S}$ --- магнитный поток, пронизывающий неподвижный контур $s$.
\end{to_def}
    
Сущность в таком понимании прежде всего в возбуждении электрического поля, а не тока. Электромагнитная индукция может наблюдаться и тогда, когда в пространстве нет проводников вообще.

\subsection{Сила Лоренца}
\begin{to_def}
    \textit{Сила Лоренца} для проводника движущегося в переменном магнитном поле ток возбуждается как магнитной, так и электрической силами:
    \begin{equation}
        \vc{F} = e \left(\vc{E} + \frac{1}{c} [\vc{v} \times \vc{B}]\right)
    \end{equation}
\end{to_def}

От выбора системы отсчета зависит, какая часть индукционного тока вызывается электрической, а какая магнитной составляющей силы Лоренца. Деление электромагнитного поля на электрическое и магнитное определяется системой отсчета, в которой рассматриваются явления. 

С помощью пересадок, в общем случае, нельзя добиться того, чтобы электромагнитное поле сделалось либо чисто электрическим, либо чисто магнитным.

\subsection{Индуктивность проводов}
\begin{to_def}
    Предполагаем, что ферромагнетиков нет, тогда $\vc{B}$ и $\Phi$ пропорциональны току:    

    \begin{equation}
        \Phi = L I^{(m)} = \frac{1}{c} L I,
    \end{equation}

    где $I^{(m)}$ -- сила тока в СГСМ, а $I$ -- сила того же тока в СИ, $L$ же не зависит от силы тока и называется \textit{индуктивностью провода}. Чем тоньше провод, тем болше его индуктивность.
\end{to_def}





\subsection{Магнитная энергия}

Для витка с током, в котором с помощью внешних сил потечёт ток, а значит будет нарастать и магнитный поток через него, возникнет ЭДС, тогда элементарная работа внешний сил:

\begin{equation*}
    \delta A^\text{внеш} = - \mathcal{E}^\text{инд} I \d t = \frac{1}{c} I \d \Phi.
\end{equation*}

\begin{to_def}
    Из верхнего, достаточно общего утверждения, если работа внешняя работы пойдёт только на увеличение \textit{магнитной энергии}, то есть диа- или парамагнетик, в частности.

    \begin{equation}
        \d W_m = \frac{1}{c} \d \Phi
        \hspace{0.5cm} 
        \overset{\nexists \text{ферромагнетиков}}{\leadsto} 
        \hspace{0.5cm}
        W_m = \frac{L}{2}\left(\frac{I}{c}\right)^2 = \frac{1}{2 c} I \Phi = \frac{\Phi^2}{2 L},
    \end{equation}
    где $L$ -- самоиндукция проводника с током и константа. Также, для справедливости последней формулы не обязательно виток должен оставаться неподвижным.
\end{to_def}

Важно, что $\mu$ остается постоянной, или же, если проницаемость зависит от температуры, то в процессе намагничивания, чтобы формула работала, надо поддерживать $T$ постоянной.

Тогда $W_m$ будет иметь смысл свободной магнитной энергии системы.

Можно перейти к другому виду записи энергии магнитного поля, энергия, которую запас соленоид, используя:

\begin{equation}
    \left\{
    \begin{aligned}
        H &= 4 \pi I / (c l)\\
        \Phi &= B S
    \end{aligned}
    \right.
    \leadsto
    \d W_m = \frac{I}{c} \d \Phi = \frac{V}{4 \pi} (\vc{H} \cdot \d \vc{B})
\end{equation}

Если $w_m$ -- плотность магнитной энергии в соленоиде, то в общем случае можно записать $W_m = \int w_m \d V$,
где плотность определяется: $w_m = (\vc{H} \cdot \d \vc{B})/(4 \pi)$.

В случае пара- и диамагнитный сред $\vc{B} = \mu \vc{H}$ получаем: $w_m = \mu H^2 /(8 \pi)$



Рассмотрим два витка с током по которым текут токи $I_1$  и $I_2$. В отсутствии ферромагнетиков запишется:

\begin{align*}
    \Phi_1 = \frac{1}{c} L_{1 1} I_1 + \frac{1}{c} L_{ 1 2} I_2, \\
     \Phi_2 = \frac{1}{c} L_{2 1} I_1 + \frac{1}{c} L_{2 2} I_2.
\end{align*}

Можно сформулировать \textbf{теорему о взаимности} $L_{i k} = L_{k i}$

\begin{to_thr}[О сохранении магнитного потока]
Проводник с током в $\forall$ магнитном поле, движется и деформируется, тогда в нём возбуждается:
    \begin{equation*}
        I = \frac{\mathcal{E}^\text{инд}}{R} = - \frac{1}{c R} \frac{\d \Phi}{\d t}.
    \end{equation*}

Если $R =0$, то $\d \Phi/ \d t = 0$, то есть при движении идеально проводящего замкнутого провода в магнитном поле остается постоянным магнитный поток, пронизывающий контур провода. 
\end{to_thr}