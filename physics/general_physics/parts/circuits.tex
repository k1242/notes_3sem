
Колебательный контур описывается дифференциальным уравнением второго порядка:
\begin{equation}
	L \ddot{q} + R \dot{q} + \frac{q}{C} = \mathscr{E}
\end{equation}

Если же <<внешних сил>> $\varepsilon$ или $F$ нет, то уравнения линейны и однородны по времени, описывая, так называемые, свободные колебания. Введём обозначения для таких линейных колебательных систем:
\begin{equation}
	\omega_0^2 = \frac{1}{L C}, \hspace*{1 cm} 2 \gamma = \frac{R}{L}, \hspace*{1 cm} X = \frac{\mathscr{E}}{L}.
\end{equation}


Тогда уравнения преобразуются в более удобный вид, в котором $\omega_0 $ -- собственная частота, а $\gamma$ -- коэффициент затухания. 

\subsubsection*{Виды колебаний в электрических цепях}

\textbf{Свободные колебания гармонического осциллятора} характеризуется отсутствием омического сопротивления:
\begin{equation}
	\ddot{q} + \omega_0^2 q = 0, \hspace*{1 cm} \Rightarrow \hspace*{1 cm} 
	\begin{aligned}
		&q = q_0 \cos(\omega_0 t + \delta),\\
		&I = \omega_0 q_0 \cos\left(\omega_0 t + \delta + {\pi}/{2}\right).
	\end{aligned}
\end{equation}
\textbf{Затухающие колебания} характеризуются наличием тормозящей силы:
\begin{equation}
	\ddot{q} + 2 \gamma \dot{q} + \omega_0^2 q = 0 \hspace*{1 cm} \Rightarrow \hspace*{1 cm} q = a e^{- \gamma t} \cos(\omega t + \delta), \text{ где } \omega^2 = \omega_0^2 - \gamma^2.
\end{equation}
Выпишем несколько важных определений:

\begin{minipage}[t]{0.05\textwidth}
\end{minipage}
\hfill
\begin{minipage}[t]{0.95\textwidth}
    \begin{description*}
    \item[Период колебаний] --- величина $T = 2 \pi / \omega = 2\pi / \sqrt{\omega_0^2 - \gamma^2} = T_0 / \sqrt{1 - (\gamma/\omega_0)^2}$;
    \item[Амплитуда]  --- величина $A = a e^{- \gamma t} $;
    \item[Время затухания] --- величина $\tau = 1/\gamma$ за которое $A$ убывает в $e$ раз;
    \item[Логарифмический декремент затухания] ---величина  $d = \gamma T$ (безбожно устарел);
    \item[Добротность] --- величина $Q = \pi / d = \omega / 2 \gamma = \pi N$, где $N = \tau/T = \frac{1}{\gamma T} (= 1/d)$. 
    Также $Q = \Delta W / W$. 
\end{description*}
\end{minipage}

\subsubsection*{Вынужденный колебания затухающего осциллятора под действием синусоидальной силы}

Запишем уравнение колебаний в самом простом случае:
\begin{equation}
	\ddot{q} + 2 \gamma \dot{q} + \omega_0^2 q = X(t) = X_0 \cos (\omega t)
\end{equation}
Частное решение ищем в виде $q = q_0 e^{i \omega t}$, откуда $\dot{q} = i \omega q$, $\ddot{q} = \omega^{2} q $, тогда:
\begin{equation}
	q = \frac{X}{\omega_0^2 - \omega^2 + 2 i \omega \gamma} e^{i \omega t} + e^{-\gamma t}(C_1 \cos \omega_0 t + C_2 \sin \omega_0 t)
\end{equation}
Если $t \gg \tau$, то свободные колебания практически затухнут и останутся только вынужденные (первый член выражения для $q$).

Оставим только вещественную часть решения:
\begin{equation}
	q = \frac{X}{\omega_0^2 - \omega^2 + 2 i \omega \gamma} e^{i \omega t} 
	\hspace{0.5 cm } 	\leadsto \hspace{0.5 cm}
	q = \frac{X_0}{\sqrt{(\omega_0^2 - \omega^2)^2 + 4 \gamma^2 \omega^2}} \cos\left(\omega t - \Delta \varphi\right), \hspace{0.5cm} \Delta \varphi = \arctan\left[\frac{2 \gamma \omega}{\omega_0^2 - \omega^2}\right].
\end{equation}
В наиболее важном случае, когда затухание невелико, положения всех максимумов почти не отличаются друг от друга. Поэтому за максимум аплитуды смещения можно принять её значение при $\omega = \omega_0 $:
\begin{equation}
	a_{max} = \frac{X_0}{2 \omega_0 \gamma} = \frac{\omega_0}{2 \gamma} a_0,
	 \hspace*{1 cm} 
	\leadsto 
	\hspace*{1 cm} \frac{a_{max}}{a_0} = Q = \frac{\omega_0}{2 \gamma}.
\end{equation}
