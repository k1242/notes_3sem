\section{Семинар про электрические цепи}
\subsection{Общие сведения}
Колебательный контур описывается дифференциальным уравнением второго порядка:

\begin{equation}
	L \ddot{q} + R \dot{q} + \frac{q}{C} = \mathscr{E}
\end{equation}

Если же <<внешних сил>> $\varepsilon$ или $F$ нет, то уравнения линейны и однородны по времени, описывая, так называемые, свободные колебания. Введём обозначения для таких линейных колебательных систем:

\begin{equation}
	\omega_0^2 = \frac{1}{L C} \hspace*{1 cm} 2 \gamma = \frac{R}{L} \hspace*{1 cm} X = \frac{\mathscr{E}}{L}
\end{equation}


Тогда уравнения преобразуются в более удобный вид, в котором $\omega_0 $ -- собственная частота, а $\gamma$ -- коэффициент затухания. 

\subsection{Виды колебаний}

\textbf{Свободные колебания гармонического осциллятора} характеризуется отсутствием омического сопротивления:

\begin{equation}
	\ddot{q} + \omega_0^2 q = 0 \hspace*{1 cm} \Longrightarrow \hspace*{1 cm} 
	\begin{aligned}
		&q = q_0 \cos(\omega_0 t + \delta)\\
		&I = \omega_0 q_0 \cos\left(\omega_0 t + \delta + \frac{\pi}{2}\right)
	\end{aligned}
\end{equation}

\textbf{Затухающие колебания} характеризуются наличием тормозящей силы.

\begin{equation}
	\ddot{q} + 2 \gamma \dot{q} + \omega_0^2 q = 0 \hspace*{1 cm} \Longrightarrow \hspace*{1 cm} q = a e^{- \gamma t} \cos(\omega t + \delta), \text{ где } \omega^2 = \omega_0^2 - \gamma^2
\end{equation}

\phantom{239}

\noindent<<\textbf{Периодом}>> колебаний в таком случае называется величина $T = 2 \pi / \omega = 2\pi / \sqrt{\omega_0^2 - \gamma^2} = T_0 / \sqrt{1 - (\gamma/\omega_0)^2}$
\textbf{Амплитуда} -- $A = a e^{- \gamma t} $, 
\textbf{время затухания} -- $\tau = 1/\gamma$ за которое $A$ убывает в $e$ раз.
\textbf{Логарифмический декремент затухания} $d = \gamma T$ (безбожно устарел).

\noindent Число колебаний совершаемых за $\tau$: $N = \tau/T = \frac{1}{\gamma T} (= 1/d)$.
\textbf{Добротностью} называется величина $Q = \pi N$

\phantom{239} 

\textbf{Вынужденный колебания затухающего осциллятора под действием синусоидальной силы.}

\begin{equation}
	\ddot{q} + 2 \gamma \dot{q} + \omega_0^2 q = X(t) = X_0 \cos (\omega t)
\end{equation}
Частное решение ищем в виде $q = q_0 e^{i \omega t}$, откуда $\dot{q} = i \omega q$, $\ddot{q} -\omega^{2} q $, тогда:

\begin{equation}
	q = \frac{X}{\omega_0^2 - \omega^2 + 2 i \omega \gamma} e^{i \omega t} + e^{-\gamma t}(C_1 cos \omega_0 t + C_2 sin \omega_0 t)
\end{equation}
Если $t \gg \tau$, то свободные колебания практически затухнуть и останутся только вынужденные (первый член выражения для $q$).

Оставим только вещественную часть решения:

\begin{equation}
	q = \frac{X}{\omega_0^2 - \omega^2 + 2 i \omega \gamma} e^{i \omega t} \hspace*{1 cm } 
	\scalebox{1.5}{$\leadsto$} \hspace*{1 cm}
	q = \frac{X_0}{\sqrt{(\omega_0^2 - \omega^2) + 4 \gamma^2 \omega^2}} \cos(\omega t - \arctan\frac{2 \gamma \omega}{\omega_0^2 - \omega^2}), 
\end{equation}
В наиболее важном случае, когда затухание невелико, положения всех максимумов почти не отличаются друг от друга. Поэтому за максимум аплитуды смещения можно принять её значение при $\omega = \omega_0 $:

\begin{equation}
	a_{max} = \frac{X_0}{2 \omega_0 \gamma} = \frac{\omega_0}{2 \gamma} a_0 \hspace*{1 cm} \scalebox{1.5}{$\leadsto$} \hspace*{1 cm} \frac{a_{max}}{a_0} = Q = \frac{\omega_0}{2 \gamma}
\end{equation}
