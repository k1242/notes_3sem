Многие вещи раньше происходили исключительно в локальных координатах. 
Пусть есть некоторая $`\Sigma \in \mathbb{R}^n$. Область $D$ через $r$  параметризует некоторый кусок нашей поверхности. Другая область $D'$ через $\hat{r}$ параметризует другой кусок. Что происходит на их пересечениях?
У нашей поверхности есть условияе регулярности.  Хочется от него уйти, имея при этом возможность замены координат и склейки карт.

Рассмотрим любое множество, как аналог поверхности и потребовать, чтобы $r$ и $\hat{r}$ были взаимнооднозначны. Но этого мало. Нужно чтобы они были и непрерывны, а чтобы говорить про непрерывность нужно работать с топологическими пространствами. Хорошо, рассмотрим топологическое пространство $M$, такое, что:

\vspace{-10pt}
\begin{enumerate*}
    \item $M$ хаусдорфово.
    \item $M$ удоволетворяет второй аксиоме счётности: $M$ обладает счётной базой топологии.
\end{enumerate*}
\vspace{-10pt}

\begin{to_exm} Для прямой в качестве базы топологии можно брать открытые интервалы, при чём с рациональными концами. Аналогично для $\mathbb{R}^n$. Из счётности базы выведем разбиение единицы. Если мы хотим интегрировать, то хотим разбивать единицу, и иначе никак. 
\end{to_exm}

\begin{to_def} 
    Возьмём пару $(U, \ \varphi)$, где $U \in M$ -- открытая область, а $\varphi \colon U \mapsto \mathbb{R}^k$ -- гомеоморфизм на образ.
\begin{equation*}
    U \overset{\varphi}{\mapsto} \varphi(U) \subseteq \mathbb{R}^k.
\end{equation*}
В таком случае $u^1 \circ \varphi, \ldots, u^k \circ \varphi$ -- локальные координаты на $M$. Такую пару назовём \textit{картой} на $M$.
\end{to_def}

Рассмотрим ситуацию пересечения двух разных карт. Выберем некоторые $U, \ V \subset M$, так чтобы $U \cap V \neq \varnothing$ и $\varphi \colon U \mapsto \varphi(U) \subseteq \mathbb{R}^k, \ \psi \colon V \mapsto \psi(V) \subseteq \mathbb{R}^l$. Заметим, что 
\begin{equation*}
    \varphi(U \cap V) \home \psi (U \cap V),
    \hspace{0.5cm} \Rightarrow \hspace{0.5cm} 
    k = l.
\end{equation*}
\texttt{Есть такая шутка -- топологичская теория размерности}. Мораль в том, что \texttt{размерность можно опредлелить чисто в топологических терминах}. Если $M$ -- связно, то все карты принимают значения в $\mathbb{R}^k$ , это $k$ назывют размерностью $M$, $\dim M$. 


Пусть на многообразие $M$ есть
    \begin{equation*}
        U_\alpha, \varphi_\alpha, \ \ U_\beta, \varphi_\beta, \ \ \mapsto \mathbb{R}^n 
        \hspace{0.5cm} 
        \varphi_\beta \circ \varphi_\alpha^-1 \colon \mathbb{R}^n \home \mathbb{R}^n.
    \end{equation*}
    И даже больше, есть
    \begin{equation*}
        \left\{\begin{aligned}
            y^1 &= y^1\left(\set{x}{n}\right), \\
            &\ldots \\
            y^2 &= y^2\left(\set{x}{n}\right)
        \end{aligned}\right.
        \text{ --- замена локальных координат или \textit{отображение склейки}}
    \end{equation*}
    Раньше мы выводили их гладкость и регулярность из гладкости и регулярности самой поверхности, но на многообразие их вывести неоткуда, соответсвенно будем требовать для разных карт одного и того же атласа, чтобы эти замены были гладкими: 

\begin{to_def} 
    Набор карт $\{\left(U_{\alpha}, \varphi_\alpha\right)\}$  на $M$ называется \textit{атласом} класса гладкости $C^l$, если эти карты согласованы в следующем смысле: отображения 
    $\varphi_\beta \circ \varphi_\alpha^{-1} \colon \mathbb{R}^n  \mapsto \mathbb{R}^n $ -- являются гладкими класса $C^l$ на области определения. А также $M = \bigcup_\alpha U_\alpha$.
\end{to_def}


Почему отображение склейки? Ну, пусть $V_\alpha = \varphi_\alpha (U_\alpha) \subset \mathbb{R}^n$. Тогда
\begin{equation}
    \bigsqcup_\alpha V_\alpha \ \big/_\sim = M, \text{ --- гомеоморфизм топологических пространств.}
\end{equation}
где $\forall p \in V_\alpha, \ q \in V_\beta \ p \sim q \ \Leftrightarrow \ q = \varphi_\beta \circ \varphi_\alpha^-1 (p)$.


\begin{to_def} 
    Атласы $A_1 = \{\left(U_\alpha, \ \varphi_\alpha\right)\}$ и $A_2 = \{\left(U_\beta, \ \varphi_\beta\right)\}$ \textit{эквивалентны}, если $A_1 \bigcup A_2$ тоже атлас (класса гладкости $C^l$).
\end{to_def}

Что это значит? Если $A_1$ атлас, то все карты внутри согласованы. Если их объединения тоже атлас, что любые две карты из них согласованы. 
\texttt{Но есть неэкивалентные атласы, с этим нужно смериться.} 


\begin{to_def} 
    Класс эквивалентности атласов на $M$ называется \textit{гладкой} (или дифференцируемой) \textit{структурой} на $M$ (класса гладкости $C^l$).
\end{to_def}

\begin{to_def} 
    \textit{Гладкое многообразие} (класса гладкости $C^l$) --- это топологическое пространство $M$ (хаусдорфово, со счётной топологической базой) с гладкой структурой класса $C^l$. 
\end{to_def}


\begin{to_exm} 
    РАссмотрим нехаусдорфово пространство. Рассмотрим 
    \begin{equation*}
         \mathbb{R} \bigsqcup \mathbb{R}/\sim, \hspace{0.5cm} x \sim y \Leftrightarrow x = y \neq 0.
     \end{equation*} 
     Было две прямых, склеиваем одинаковые точки -- все, кроме нуля, получили прямую с двойным нулем. 
     Как мы вводим тополгию на фактормножестве? У нас открыто такое множество, чтобы его прообраз до факторизации открыт. 

    Множество $U$  в $M$ с верхним $0+$ открыто, множество $V$  с $0-$ также открыто в $M$. 
    Есть две карты $U \overset{\varphi = \id}{\mapsto} \mathbb{R}$ и $V\overset{\psi = \id}{\mapsto} \mathbb{R}$ -- гомеоморфизмы. Посмотрим на $\psi \circ \varphi^{-1}$ -- прямую без нуля отображаем тождественно в прямую без нуля, что гладко. Это атлас. \texttt{Это уродливо.}
\end{to_exm}

\begin{to_exm} 
    Посмотрим на сферу $S^2$. Легко было бы получить стереографическую проекцию, вида
    \begin{equation}
        (u, v) = \varphi_N (x, y, z) = 
        \left(\frac{x}{1-z} , \frac{y}{1-z} \right),
    \end{equation}
    и, в обратную сторону,
    \begin{equation}
        x = \frac{2u}{u^2+v^2+1}, \hspace{0.5cm} 
        y = \frac{2v}{u^2+v^2+1} , \hspace{0.5cm} 
        z = \frac{u^2+v^2-1}{u^2+v^2+1},
    \end{equation}
    собирая всё вместе,
    \begin{equation*}
        \varphi_N^{-1}(u, v) = 
        \left(
            \frac{2u}{u^2+v^2+1},
            \frac{2v}{u^2+v^2+1},
            \frac{u^2+v^2-1}{u^2+v^2+1}
        \right).        
    \end{equation*}

    Аналогично можем построить проекцию из южного полюча $\varphi_S (A) = (s, t)$. Можно заметить, что проецировать из северного, это то же самое, что и проецировать из южного.
    \begin{equation*}
        \varphi_s (x, y,z) = \varphi_N (x, y, -z),
    \end{equation*}
    поэтому
    \begin{equation*}
        (s, t) = \varphi_S (x, y, z) = 
        \left(\frac{x}{1+z} , \frac{y}{1+z} \right),
    \end{equation*}
    получили вторую карту $\left( S^2 \backslash \{S\}, \varphi_S\right)$.

    Как выглядит замена координат?
\begin{equation*}
    \varphi_S \circ \varphi_N^{-1} (u, v) = 
    \varphi_S \left(
            \frac{2u}{u^2+v^2+1},
            \frac{2v}{u^2+v^2+1},
            \frac{u^2+v^2-1}{u^2+v^2+1}
        \right) = 
         \left(
            \frac{u}{u^2+v^2} , \ \frac{v}{u^2+v^2} 
         \right).
\end{equation*}
    Но есть тонкости: пересечение карт это $S^2 \backslash \langle \text{N, \ S}\rangle$. 
    Посмотрим на $\varphi_N (S^2 \backslash \langle \text{N, \ S}\rangle) = \mathbb{R}^2 \backslash \{0, 0\}$. Отображение склейки гладкое на области определения.  
\end{to_exm}


\noindent
Очень забавный факт:
\begin{to_thr}[теорема Уитни]
     Многообразие размерности $k$ можно засунуть в $\mathbb{R}^{2k+1}$. 
\end{to_thr}


\begin{to_exm} 
    Рассмотрим $\mathbb{RP}^n$. Фиксируем $O$ и говорим, что $\mathbb{RP}^n$ -- множество прямых, проходящих через $O$. Заметим, что $\left(\mathbb{RP}^n, \ \alpha\right)$ -- метрическое пространство, следовательно хаусдорфово топологическое. 

    Как выглядит шар? Как конус. 
\end{to_exm}