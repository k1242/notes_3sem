\subsection{Обращение с обратным образом (?)}

На данный момент у нас есть отображения для $U \in \mathbb{R}^n$ и $V \in \mathbb{R}^k$, считая $U \overset{F}{\longrightarrow} V$

\vspace{-5mm}

\begin{minipage}[t]{0.45\textwidth}
\begin{align*}
    C^{\infty} (U) &\overset{F^*}{\longleftarrow} C^{\infty} (V) \\
    T_P U &\overset{d_P F}{\longrightarrow} T_{F(P)} V
\end{align*}
\end{minipage}
\hfill
\begin{minipage}[t]{0.45\textwidth}
\begin{align*}
    U &\overset{F}{\longrightarrow} V \overset{\varphi}{\to} \mathbb{R} \\
    U &\overset{F^* \varphi}{\longrightarrow} \mathbb{R},
    \hspace{0.5cm} 
    \text{где }
    \hspace{0.5cm} 
    F^* \varphi  \overset{\mathrm{def}}{=} \varphi \circ F,
\end{align*}
\end{minipage}

\vspace{-3mm}

\begin{equation}
    \overbrace{
        d_P F
            \underbrace{X}_{\in T_P U}
    }^{\in T_{F(P)} V}
    \underbrace{\varphi}_{\in C^{\infty} (V)} 
    =
    X 
    \underbrace{F^* \varphi}_{\in C^\infty (U)}.
\end{equation}

С формами ситуация схожая с функциями, то есть
$$
    C^\infty (V) = \Gamma^0 (V),
$$
получается
\begin{align*}
    \Omega^k (U) \overset{F^*}{\longleftarrow} \Omega^k (V), \\
    T_U U \overset{d_P F}{\longrightarrow} T_{F(P)} (V).
\end{align*}
Теперь пусть $X_1, \ldots, X_k$ -- векторное поле на $U$, тогда
$$
    (F^* \omega) (X_1, \ldots, X_k) = \omega\left(
        d F(X_1), \ldots, d F(X_k)
    \right).
$$
Собственно, факт:
\begin{equation}
    d F^* \omega = F^* \d \omega.
\end{equation}
И ещё факт
\begin{equation}
    F^* (\sigma \wedge \tau) = F^* \sigma \wedge F^* \tau.
\end{equation}



%%%%%%%%%%%%%%%%%%%%%%%%%%%%%%%%%%%%%%%%%%%%%%%%%%%%%%%%%%%%%%%%%%%%%%%%%%%%%%%%%%%
\subsection{Кривые}
Кривые должны быть гладкими, но этого недостаточно. Поэтому требуем и \textit{регулярность}:
\begin{equation}
    \forall x, y \colon F(x, y) = 0
    \hspace{0.5cm} 
    \left(
        \frac{\partial F}{\partial x}, \frac{\partial F}{\partial y} 
    \right) \neq (0, 0),
\end{equation}
а в параметрическом задание
\begin{equation}
    \forall t \in (a, b)
    \hspace{0.5cm} 
    \dot{\vc{r}} (t) = (\dot{x} (t), \dot{y}(t)) \neq (0, 0).
\end{equation}

Пусть $F(x, y) = 0$ -- регулярная гладкая неявно заданная кривая. Тогда в окрестности любой своей точки её можно задать как регулярную гладкую параметрическую кривую. В самом деле,
$$
    F(x_0, y_0) = 0, \hspace{0.5cm} \left(
        \frac{\partial F}{\partial x}, \frac{\partial F}{\partial y} \bigg|_{(x_0, y_0)} \neq 0
    \right),
    \hspace{0.5cm} \Rightarrow \hspace{0.5cm} 
    \exists \varphi \in U(x_0, y_0) \colon F(x, y) \Leftrightarrow x = \varphi(y).
$$

А вот пусть теперь есть гладкая регулярная параметризованная регулярная кривая $(x, y)(t) \colon (\dot{x}, \dot{y})\neq 0$. Пусть $\dot{x}\neq0$, тогда по \underline{тереме об обратной функции} $t = t(x)$. 


\subsection{Явно заданные поверхности}


Регулярная (не особая) гладкая $k$-мерная поверхность в $n$-мерном аффинном пространстве, заданная параметрически. 


\incfig{1}

\noindent
Формально, 
$$
    r \colon D \in \mathbb{R}^k \to \mathbb{R}^n,
    \hspace{0.5cm} \text{при чем} \hspace{0.5cm} 
    \left\{\begin{aligned}
        &1) \text{ гладкость: } & \vc{r} \equiv [x^1 (\set{u}{k}), \ldots, x^n(\set{u}{k})] \\
        &2) \text{ регуярность: } & \rg (\partial x^i / \partial u^j) = k.
    \end{aligned}\right.
$$
где подразумевается $r \in C^\infty (D, \mathbb{R}^n)$. Регулярность же, по сути, это утверждение о том что в $J$ существует невырожденный минор $k \times k$. 

Пусть это $(\partial x^i / \partial u^j)$, где $i, j = 1, \ldots, k$. Тогда,  по теореме об обратной функции, в окрестности этой точки 
\begin{align*}
    u^1 &= u^1 (\set{x}{k}) \\
    &\ldots \\
    u^k &= u^k (\set{x}{k}).
\end{align*}
Тогда, это просто график отображения 
\begin{align*}
    x^{k+1} &= x^{k+1} (u^1 (\set{x}{k}), \ldots, u^k\left(x_1, \ldots, x^k\right)) \\
    \ldots\\
    x^{n} &= x^{n} (u^1 (\set{x}{k}), \ldots, u^k\left(x_1, \ldots, x^k\right)) \\
\end{align*}
такого, что
$$
    \mathbb{R}^k \to \mathbb{R}^{n-k}
$$
Так, например, для сферы, можно выразить $z = \sqrt{1 - x^2 - y^2}$.




\subsection{Неявно заданные поверхности}
Гладкая регулярная $k$-мерная поверхность в $n$-мерном афинном пространстве, заданная неявно.  Тогда есть $n-k$ уравнений
\begin{align*}
    \left\{\begin{aligned}
        F^1 (\set{x}{n})&=0\\
        &\ldots\\
        F^{n-k} (\set{x}{n}) &= 0
    \end{aligned}\right.
    \hspace{0.5cm} \Leftrightarrow \hspace{0.5cm} 
    \vc{F} \colon \mathbb{R}^n \to \mathbb{R}^{n-k}, \hspace{0.25cm} \vc{F}=0.
\end{align*}
Аналогично мы требуем гладкость: $F^i \in C^\infty \left(\mathbb{R}^n\right)$, и регулярность в тех точках, где $\vc{F}=0$. Условие регулярности в таком случае
\begin{equation}
    \rg \left(\frac{\partial F^i}{\partial x^j} \right) = n-k.
\end{equation}

\begin{to_lem} 
     Гладкая регулярная неявно заданная поверхность, может рассматриваться, как параметрическая.
\end{to_lem}

\begin{proof}[$\triangle$]
    \begin{minipage}[t]{0.9\textwidth}
        \begin{enumerate}[label = \Roman*.]
            \item Пусть в точке $P$ 
            $$
                \rg \left(\frac{\partial F^i}{\partial x^j} \right) = n-k.
            $$
            \item Тогда можем считать, что есть невырожденный минор $\rg \left(\frac{\partial F^i}{\partial x^j} \right) (P)$, где $i = 1, \ldots, n-k$ и $j = k+1, \ldots, n$.
            \item По теореме о неявной функции 
            $$
                \left\{\begin{aligned}
                    x^{k+1} &= x^{k+1} (\set{x}{k})\\
                    &\ldots \\
                    x^{n} &= x^n (\set{x}{k})
                \end{aligned}\right.
                \hspace{0.5cm} \text{---} \hspace{0.5cm} \text{гладкие}.
            $$
            \item Тогда понятно, как утроен параметрический вид:
            $$
                \left.\begin{aligned}
                    x^1 &= u^1 \\
                    \ldots \\
                    x^k &= u^k \\
                    x^{k+1} &= x^{k+1} (\set{u}{k}) \\
                    \ldots \\
                    x^{n} &= x^n (u^1, \ldots, u^k)
                \end{aligned}\right\}
                \hspace{0.5cm} \Rightarrow \hspace{0.5cm} 
                J = \begin{pmatrix}
                    1 & \ldots & 0 \\
                    \vdots & \ddots & \vdots \\
                    0 & \ldots & 1 \\
                    & * & 
                \end{pmatrix},
                \hspace{0.5cm} 
                \rg J = k.
            $$
        \end{enumerate}
    \end{minipage}

\phantom{42}
\end{proof}

\begin{to_def} 
    Назовем $\set{x}{n}$ \textit{координатами объемлющего пространства}, а $\set{u}{k}$ \textit{локальными координатами}.
\end{to_def}


%%%%%%%%%%%%%%%%%%%%%%%%%%%%%%%%%%%%%%%%%%%%%%%%%%%%%%%%%%%%%%%%%%%%%%%%%%%%%%%%%%%

\subsection{Гладкие функции и пути на поверхности}


\subsubsection*{Функции}


\begin{to_def} 
     Пусть есть гладкая функция $F(\set{x}{n})$ -- гладкая в окрестности $\Sigma$, тогда $F\big|_\Sigma$ -- \textit{гладкая} на поверхности $\Sigma$.
\end{to_def}

\begin{to_def} 
    Пусть  $f(\set{u}{k})$ -- гладкая, тогда $f$ -- гладкая функция на $\Sigma$. 
\end{to_def}

Докажем равносильность двух следующих определений. 


\begin{proof}[$\triangle$]
    \begin{minipage}[t]{0.9\textwidth}
        \begin{enumerate}[label = \Roman*.]
            \item[$\Rightarrow$] Пусть $F(\set{x}{n})$ -- гладкая, тогда и $F(x^1(\set{u}{k}),\ldots,x^n(\set{u}{k}))$ -- гладкая.
            \item[$\Leftarrow$] Пусть есть $f(\set{u}{k})$ -- гладкая, тогда и $f(u^1(\set{x}{k}), \ldots, u^k(\set{x}{k}))$ -- тоже гладкая. 
        \end{enumerate}
    \end{minipage}

\phantom{42}
\end{proof}

\subsubsection*{Пути}

\begin{to_def} 
     Путь $\vc{r}(u^1(t), \ldots, u^k (t))$ гладкий, если $u^i$ -- гладкие.
\end{to_def}

\begin{to_def} 
     Если $x^1 (t), \ldots, x^n(t)$ -- гладкие, такие что $[x^1(t), \ldots, x^n(t)] \in \Sigma$, то и путь $\vc{r}$ гладкий.
\end{to_def}

Эти определения равносильны. Получается, что пути можно описывать как в глобальных, так и в локальных координатах. Далее ограничимся рассмотрением путей в локальных координатах, \texttt{ничего при этом не потеряв}.


%%%%%%%%%%%%%%%%%%%%%%%%%%%%%%%%%%%%%%%%%%%%%%%%%%%%%%%%%%%%%%%%%%%%%%%%%%%%%%%%%%%

\subsection{Векторы на поверхности}

Точка $A$ имеет локальные координаты $\set{u_0}{k}$, то есть $A = \vc{r}(\set{u_0}{k})$.

\phantom{42}    

\incfig{2}
    

\noindent
Если мы посмотрим на путь точки $A$, то увидим (считая, что в $t=0$ $\vc{r}=A$).
$$
    \vc{r}(t) = \vc{r}(\set{u}{k})(t)
    \hspace{0.5cm} \Rightarrow \hspace{0.5cm} 
    \frac{dr}{dt} = r_{u^i} (u^1(t), \ldots, u^k(t)) \cdot \dot{u}^i(t).
$$
где подразумевается, что
$$
    r_{u^i} = \frac{\partial \vc{r}}{\partial u^i} 
    = \left(\frac{\partial x^1}{\partial u^i} , \ldots, \frac{\partial x^n}{\partial u^i} \right).
$$
При $t=0$, увидим
$$
    \frac{dr}{dt} \bigg|_{t=0} = 
    \underbrace{r_{u^i} (\set{u_0}{k}) }_{
    \text{векторы}
    }
    \underbrace{
     \dot{u}^i(0)
    }_{
    \text{числа}
    }
    ,
$$
где векторы зависят только от точки $A$, а числа зависят от конкретной кривой.
Получается, что есть некоторые пространство, порожденное этими векторами.

\begin{to_def} 
    Назовём \textit{касательным пространством} к $\Sigma$ в точке $A$
$$
    T_A \Sigma = \sp \left(
        r_{u^1} (A), \ldots, r_{u^k} (A)
    \right).
$$
\end{to_def}

Пусть есть некоторый вектор $\vc{V}$
$$
    \vc{V} = \alpha^i r_{u^i} (A).
$$
Он может быть получен кривой $u^i = u^1_0 + \alpha^1 t$. Получается, что $T_A \Sigma$ состоит в точности из векторов скорости кривых в точке $A$. 

\begin{to_lem} 
    Размерность $\dim T_A \Sigma = k$.  
\end{to_lem}

\begin{proof}[$\triangle$]
    Действительно, по условию регулярности
    $$
        \rg (r_{u^i}) = \rg \left(\frac{\partial x^i}{\partial u^j} \right) 
        \overset{\mathrm{reg}}{=} 
         k.
    $$
    \texttt{В этом и состоит геометрический смысл условия регулярности.}
\end{proof}


%%%%%%%%%%%%%%%%%%%%%%%%%%%%%%%%%%%%%%%%%%%%%%%%%%%%%%%%%%%%%%%%%%%%%%%%%%%%%%%%%%%
\subsection{Замена локальных координат}

С одной стороны понятно, что множество всех кривых на поверхности $D$ инвариантно.  С другой стороны интересно посмотреть, что же происходит с векторами.
$$
    \left\{\begin{aligned}
        v^1 &= v^1 (x^1(\set{u}{k}), \ldots, x^k (\set{u}{k})) \\
        &\ldots \\
        v^k &= v^k (x^1(\set{u}{k}), \ldots, x^k (\set{u}{k}))
    \end{aligned}\right.
    \hspace{0.5cm} \text{---} \hspace{0.5cm} 
    \text{диффеоморфизм.}
$$
Действительно, Якобиан композиции равен произведению Якобианов, получается композия двух невырожденных преобразований будет невырождена.
$$
    r_{u^i} = \frac{\partial r}{\partial v^j} \frac{\partial v^j}{\partial u^i} =
    \underbrace{\frac{\partial v^j}{\partial u^i} }_{J}
    r_{v^i},
$$
получается, что матрица перехода от базиса $r_{v^j}$ к $r_{u^i}$ -- матрица Якоби $J$ замены координат.
$$
    \forall V \in T_A \Sigma
    \hspace{0.5cm} 
    V = V^i r_{v^i} = 
    \underbrace{V^i \frac{\partial r^j}{\partial u^i} }_{\widetilde{V}^j}
    r_{v^j}.
$$
Тогда
\begin{equation}
    V = \widetilde{V}^j r_{v^j}
    \hspace{0.5cm} \Rightarrow \hspace{0.5cm} 
    \widetilde{V}^j = \frac{\partial v^j}{\partial u^i} V^i.
\end{equation}
Оказывается, что если 
$$
\begin{pmatrix}
    \widetilde{V}^1 \\
    \ldots \\
    \widetilde{V}^k \\ 
\end{pmatrix} = 
    \left(\frac{\partial v^j}{\partial u^i}\right)_{(A)} \begin{pmatrix}
        V^1 \\
        \ldots \\
        V^k
    \end{pmatrix}.
$$
