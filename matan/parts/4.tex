\subsection{Обращение с обратным образом (?)}

На данный момент у нас есть отображения для $U \in \mathbb{R}^n$ и $V \in \mathbb{R}^k$, считая $U \overset{F}{\longrightarrow} V$

\vspace{-5mm}

\begin{minipage}[t]{0.45\textwidth}
\begin{align*}
    C^{\infty} (U) &\overset{F^*}{\longleftarrow} C^{\infty} (V) \\
    T_P U &\overset{d_P F}{\longrightarrow} T_{F(P)} V
\end{align*}
\end{minipage}
\hfill
\begin{minipage}[t]{0.45\textwidth}
\begin{align*}
    U &\overset{F}{\longrightarrow} V \overset{\varphi}{\to} \mathbb{R} \\
    U &\overset{F^* \varphi}{\longrightarrow} \mathbb{R},
    \hspace{0.5cm} 
    \text{где }
    \hspace{0.5cm} 
    F^* \varphi  \overset{\mathrm{def}}{=} \varphi \circ F,
\end{align*}
\end{minipage}

\vspace{-3mm}

\begin{equation}
    \overbrace{
        d_P F(
            \underbrace{X}_{\in T_P U}
            )
    }^{\in T_{F(P)} V}
    \underbrace{\varphi}_{\in C^{\infty} (V)} 
    =
    X 
    \underbrace{F^* \varphi}_{\in C^\infty (U)}.
\end{equation}

С формами ситуация схожая с функциями, то есть
$$
    C^\infty (V) = \Gamma^0 (V),
$$
получается
\begin{align*}
    \Omega^k (U) \overset{F^*}{\longleftarrow} \Omega^k (V), \\
    T_U U \overset{d_P F}{\longrightarrow} T_{F(P)} (V).
\end{align*}
Теперь пусть $X_1, \ldots, X_k$ -- векторное поле на $U$, тогда
$$
    (F^* \omega) (X_1, \ldots, X_k) = \omega\left(
        d F(X_1), \ldots, d F(X_k)
    \right).
$$
Собственно, факт:
\begin{equation}
    d F^* \omega = F^* \d \omega.
\end{equation}
И ещё факт
\begin{equation}
    F^* (\sigma \wedge \tau) = F^* \sigma \wedge F^* \tau.
\end{equation}



%%%%%%%%%%%%%%%%%%%%%%%%%%%%%%%%%%%%%%%%%%%%%%%%%%%%%%%%%%%%%%%%%%%%%%%%%%%%%%%%%%%
\subsection{Плоские кривые}
Кривые должны быть гладкими, но этого недостаточно. Поэтому требуем и \textit{регулярность}:
\begin{equation}
    \forall x, y \colon F(x, y) = 0
    \hspace{0.5cm} 
    \left(
        \frac{\partial F}{\partial x}, \frac{\partial F}{\partial y} 
    \right) \neq (0, 0),
\end{equation}
а в параметрическом задание
\begin{equation}
    \forall t \in (a, b)
    \hspace{0.5cm} 
    \dot{\vc{r}} (t) = (\dot{x} (t), \dot{y}(t)) \neq (0, 0).
\end{equation}

Пусть $F(x, y) = 0$ -- регулярная гладкая неявно заданная кривая. Тогда в окрестности любой своей точки её можно задать как регулярную гладкую параметрическую кривую. 




