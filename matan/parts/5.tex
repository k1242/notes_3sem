\subsection{Производная по направлению}

Раньше определили
$$
    \partial_V f(A) = \lim_{\varepsilon \to 0} \frac{f(A + \varepsilon V) - f(A)}{\varepsilon},
$$
но сложность в том, что $A + \varepsilon V \notin \Sigma$. 
Но гладкую функцию с поверхности может всегда продлить в некоторую окрестность поверхности. Это продолжение $F$ не единственно. 

\begin{to_def} 
    Определим
    $$
         \partial_V f(A) \overset{\mathrm{def}}{=} \partial_V F(A),
    $$ 
    при чём def инвариантно к выбору $F$.
\end{to_def}

\begin{proof}[$\triangle$]
    \begin{minipage}[t]{0.9\textwidth}
        \begin{enumerate}[label = \Roman*.]
            \item Мы дифференцируем только вдоль касательных векторов к $\Sigma$, следовательно существует кривая $\gamma$ на $\Sigma$ такая, что
            \begin{align*}
                &1) \ \forall t \; \gamma(t) \in \Sigma \\
                &2) \ \gamma(0) = A \\
                &3) \ \ddot{\gamma}(0) = V.
            \end{align*}
            \item Тогда
            $$
                \underbrace{\frac{d}{dt} f(\gamma(t)) \bigg|_{t=0}}_{
                *
                } = 
                \frac{d}{dt} F(\gamma(t)) \bigg|_{t=0} =
                \frac{\partial F}{\partial x^i} (\set{x}{n}) \dot{x}^i (t) \bigg|_{t=0} =
                \frac{\partial F}{\partial x^i} (A) V^i = 
                \underbrace{\partial_V F(A)}_{
                **
                }
                ,
            $$
            считая $\gamma(t) = [x^1(t),\ldots, x^n(t)]$.
            \item Но, т.к. $*$ не зависит от выбора $F$, то и $**$ не зависит от выбора $F$. Тогда $\partial_V f(A)$ определена корректно.
            \item К слову, $**$ не зависит от выбора пути, тогда и $*$ не зависит от выбора пути.
        \end{enumerate}
    \end{minipage}

\phantom{42}
\end{proof}

Получается мы можем определить понятие  дифференцирования гладкой функции на поверхности в точке.

\begin{to_def} 
     Для $f\colon \Sigma \to \mathbb{R}$ достаточно быть определенной в некоторой окрестности точки $A$. Скажем, что $D$ -- дифференцирование на $\Sigma$ в точке$A$, если
    \begin{align*}
        &1) \ Df \in \mathbb{R} \\
        &2) \ D(f+g) = Df + Dg \\
        &3) \ D(fg) = (Df) \cdot g(A) + f(A) \cdot (Dg).
    \end{align*}
    Пусть $\set{u}{k}$ -- локальные координаты в окрестности точки $A$.
\end{to_def}


\begin{to_lem} 
     Для $\forall D \ \exists \set{V}{K}$ такой, что
     $$
         Df = \frac{\partial f}{\partial u^i} (A) V^i.
     $$
\end{to_lem}

Пусть есть некоторый касательный вектор $W \in T_A \Sigma$
$$
    W = W^i r_{u^i} (A).
$$
Тогда можно рассматривать путь $\gamma(t)$ в локальных координатах такой, что
$
    \gamma(0) = A, \ \dot{\gamma}(0) = W,
$
то есть для $A = (\set{u_0}{k})$ и $\gamma(t) = \left[
    u^1(t), \ldots, u^k(t)
\right]$ верно, что
$$
    u^i(0) = u^i_0, \hspace{0.5cm} \dot{u}^i(0) = W^i.
$$
Тогда
$$
    \partial_W f(A) = \frac{d}{dt} f(\gamma(t)) \bigg|_{t=0}
    =
    \frac{\partial f}{\partial u^i} (A) \ W^i.
$$

Получается, что \textbf{каждый} касательный вектор $W$ даёт дифференцирование $\partial_W \big|_A$, и \textbf{каждое} дифференцирование в $A$ получается из касательного вектора. Поэтому будем писать просто
\begin{equation}
    \boxed{
    W = \partial_W = W^i \frac{\partial }{\partial u^i} .
    }
\end{equation}


\subsection{Двойственность}

Раз есть касательные векторы, то есть и кососимметрические полилинейные функии на них. 
Так приходим к следующей двойственной структуре:

$\cdot$ $T_P \Sigma$ -- \textit{касательное пространство} к $\Sigma$ в $P$,

$\cdot$ $T_P^* \Sigma \overset{\mathrm{def}}{=}  \left(T_P \Sigma\right)^*$ -- \textit{кокасательное пространство} к $\Sigma$ в $P$.

\noindent
Получаются векторное поле $X$: $X(P) \in T_P \Sigma$, и ковекторное поле $\xi$:
$\xi (P) \in T_P^* \Sigma$.

Если $\set{u}{k}$ -- локальные координаты на $\Sigma$, то
$$
    \frac{\partial }{\partial u^i} = r_{u^i}
    \hspace{0.5cm} \text{---} \hspace{0.5cm} 
    \text{базис в } T_P \Sigma.
$$
Соответственно,
$$
    du^1, \ldots, du^k
    \hspace{0.5cm} \text{---} \hspace{0.5cm} 
    \text{базис в } T_P^* \Sigma.
$$
А вот
$$
    \underset{ i_1 < \ldots < i_q}{ du^{i_1} \wedge \ldots \wedge du^{i_q}}
    \hspace{0.5cm} \text{---} \hspace{0.5cm} 
    \text{базис в } \Lambda^q T_P^* \Sigma,
$$
где $\Lambda^q T_P^* \Sigma$ -- пространство $q$-форм.


