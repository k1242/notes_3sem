\subsection{Дифференциальная форма}

В линейной алгебре есть ковекторы, а вот в дифференциальной геометрии ковекторные поля суть дифференциальные 1-формы.

\begin{to_def} 
    \textit{Дифференциальная 1-форма} -- это  ковекторное поле.
\end{to_def}

\begin{to_def} 
    \textit{Дифференциал} функции $f$ от векторного поля $X$ это
    $
        d f (X) \overset{\mathrm{def}}{=}  X f.
    $
\end{to_def}

Что это нам даёт? Ну, во-первых, пусть $\set{x}{n}$ -- некоторые координаты. 
$$
    X = X^i \frac{\partial }{\partial x^i}.
$$
Тогда
$$
    df(X) = Xf = X^i \frac{\partial f}{\partial x^i}.
$$
Но, заметим, что $\frac{\partial }{\partial x^1}, \ldots, \frac{\partial }{\partial x^n}$ -- базис в каждой точке. Рассмотрим теперь $f = x^i$ и $X = \frac{\partial }{\partial x^j} $, тогда
\begin{equation}
    d x^i \left(\frac{\partial }{\partial x^j} \right)
    =
    \frac{\partial x^i}{\partial x^j} = \delta^i_j.
\end{equation}
Из этого следует, что $\set{dx}{n}$ -- двойственный к $\frac{\partial }{\partial x^1}, \ldots, \frac{\partial }{\partial x^n}$ базис в $V^*$.
Тогда в этом базисе
$$
    d f = \omega_i \d x^i.
$$
Заметим, что
$$
\underbrace{\omega_i \d x^i }_{df}
    \left(
        \frac{\partial }{\partial x^j} 
    \right) = \omega_i \delta^i_j = \omega_j,
    \hspace{0.5cm} \Rightarrow \hspace{0.5cm} 
    \omega_j = df \left(\frac{\partial }{\partial x^j} \right) =
    \frac{\partial f}{\partial x^j}.
$$
Тогда
\begin{equation}
    df = \frac{\partial f}{\partial x^i} \d x^i.
\end{equation}
Получается ковектор $df$ расписывается по базису
$dx^i$
 двойственного пространства с координатами $\partial f / \partial x^i$.

А для общей 1-формы
$$
    \omega = \omega_i \d x^i,
$$
где $\set{\omega}{n}$ -- координаты $\omega$ в локальной системе координат.

\begin{to_def} 
    $\omega$  гладкая, если $\forall X$, где $X$ -- гладкое поле, верно, что
    $\omega(X)$ -- гладкая функция.
\end{to_def}

\begin{to_lem} 
    $\omega = \omega_i \d x^i$  -- гладкая $\Leftrightarrow$ $\omega_i$ -- гладкая форма $\forall i$.
\end{to_lem}


\subsection{Билинейные формы}

Пространство билинейных форм на $V$ -- $V^* \otimes V^* = S^2 V^* \oplus \Lambda^2 V^*$. Что ж, в $V*$ базис $\set{\vc{e}}{n}$, в $S^2 V^*$ базис 
$$
    e^i \cdot e^j (X, Y) = \frac{1}{2} \left(
        X^i Y^j + X^j Y^i
    \right),
$$
а скалярное произведение
$$
    g = g_{ij} dx^i \cdot dx^j.
$$
В кососимметрических же $\Lambda^2 V^*$ базис
\begin{equation}
    e^i \wedge e^j (X, Y) = X^i Y^j - X^j Y^i,
    \hspace{0.5cm} 1 \leq i \leq j \leq n.
\end{equation}
В таком случае, если есть некоторая кососимметрическая $\omega$, то
$$
    \omega = \sum_{i < j} \omega_{ij} \d x^i \wedge dx^j.
$$

\begin{to_def} 
     Поле кососимметрических билинейных форм -- дифференциальные 2-формы.
\end{to_def}

Возьмём два поля и засунем в 2-форму, получим функцию. 

\subsection{Полилинейные формы}

Пусть $V$ -- векторное пространство, $\Lambda^k V^k$ -- векторное пространство кососимметрических полилинейных функций от $k$ векторов. 
$$
    \omega\left(
        X_1, \ldots, X_k
    \right) \in \mathbb{R}.
$$
Введём некоторое внешнее умножение
$$
    \wedge \colon
    \Lambda^k V^* \times \Lambda^l V^* \to \Lambda^{k+l} V^*.
$$
Пусть $\sigma \in \Lambda^k V^*$, $\tau \in \Lambda^l V^*$, тогда
$$
    \sigma \wedge \tau \left(X_1, \ldots, X_{k+l}\right)
    =
    \frac{1}{k! l!} \sum_{\pi \in S_{k+l}} \sign (\pi) 
    \
    \sigma
    \left(
        X_{\pi(1)}, \ldots, X_{\pi(k)} 
    \right)
    \
    \tau
    \left(
        X_{\pi(k+1)},\ldots,X_{\pi(k+l)}
    \right).
$$

Если в $V$ базис $\vc{e}_1, \ldots, \vc{e}_k$, то в $\Lambda^k V$ в качестве базиса можно взять
$$
    e^{i_1} \wedge \ldots \wedge e^{i_k},
    \hspace{0.5cm} 
    i_1 < \ldots < i_k.
$$

\begin{to_def} 
    Дифференциальная $k$-форма -- поле полилинейных кососимметрических форм от $k$ векторов, при чем
    \begin{equation}
        \omega = \sum_{i_1 < \ldots < i_k} \omega_{i_1,\ldots,i_k}
        e^{i_1} \wedge \ldots \wedge e^{i_k},
    \end{equation}
    где $\omega_{i_1,\ldots,i_k} = \omega\left(\vc{e}_{i_1}, \ldots, \vc{e}_{i_k} \right)$ -- гладкие функции.. 
\end{to_def}


%%%%%%%%%%%%%%%%%%%%%%%%%%%%%%%%%%%%%%%%%%%%%%%%%%%%%%%%%%%%%%%%%%%%%%%%%%%%%%%%%%
\subsection{Внешний дифференциал}

Обозначим $\Omega^k (U)$ -- пространство дифференциальных $k$-форм на некоторой $U \in \mathbb{A}^n$. Также будем говорить, что $X^\infty (U) = \Omega^0 (u)$ -- 0-формы. У нас уже есть такое отображение
$$
    \Omega^0 (U) \overset{d}{\longrightarrow} \Omega^1 (U) \overset{?}{\longrightarrow} \ldots
$$
Ну и введём тогда операцию внешнего дифференцирования
\begin{equation}
    d \colon
    \Omega^k (U) \to \Omega^{k+1} (U).
\end{equation}
Введём её аксиоматически\footnote{
    \texttt{
        Формы образуют градуированную алгебру. Это такой эмпирический факт: в градуированной алгебре дифференциал должен быть с таким знаком и счастье будет.
    }
}
\begin{align*}
    &1) \quad d \left(\omega_1 + \omega_2\right) = d \omega_1 + d \omega_2; \\
    &2) \quad d(\sigma \wedge \tau) = (d \sigma) \wedge \tau + (-1)^{|\sigma|} 
    \sigma \wedge (d \tau); \\
    &3) \quad d^2 = 0,  \text{ т.е. } d(d\omega) = 0; \\
    &4) \quad f \in \Omega^0 (U) = C^{\infty} (U)
    \hspace{0.15cm} \Rightarrow \hspace{0.15cm} 
        d f(X) = Xf.
\end{align*} 


\begin{to_thr} 
    Внешний дифференциал $d$ существует и единственнен. 
\end{to_thr}

\begin{proof}[$\triangle$]
    \begin{minipage}[t]{0.9\textwidth}
        \begin{enumerate}[label = \Roman*.]
            \item Пусть существует внешний дифференциал. Тогда получим, что
            \begin{equation}
            \label{d}
                    d \omega = d \left(
                    \sum_{i_1 < \ldots < i_k} \omega_{i_1, \ldots, i_k} dx^{i_1} \wedge \ldots \wedge dx^{i_k}
                    \right) =
                    \sum_{i_1 < \ldots < i_k} 
                    \frac{\partial \omega_{i_1, \ldots, i_k}}{\partial x^i} 
                    dx^i \wedge dx^{i_1} \wedge \ldots \wedge dx^{i_k}.
            \end{equation}
                Собственно, подобный ответ является единственным. 
                \item Докажем теперь существование. Пусть $\set{x}{n}$ -- координаты, тогда определим $d$, как \eqref{d}. Легко показать, что такое определение удоволетворяет всем свойствам.
        \end{enumerate}
    \end{minipage}

\phantom{42}
\end{proof}


% \subsection{Векторозначная форма}

