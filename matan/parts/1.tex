\subsection{Векторы как дифференцирование функций}

Что такое вектор? С одной стороны можем посмотреть на производную функции по направлению
\begin{equation}
    \partial_{X} f(A) = \lim_{\varepsilon \to 0} \frac{1}{\varepsilon} 
    \left(
        f(A + \varepsilon X) - f(A) \vphantom{\frac{1}{2}}
    \right).
\end{equation}
Что очень просто выглядит в декартовых координатах
$$
    \partial_X f(A) =
    \lim_{\varepsilon \to 0} \ldots =
    \frac{d}{d\varepsilon} f\left(
        A^1 + \varepsilon X^1, \ldots, A^n + \varepsilon XN
    \right) \bigg|_{\varepsilon=0} =
    \frac{\partial f}{\partial x^1} \left(A^1, \ldots, A^n\right) X^1 +
    \ldots +
    \frac{\partial f}{\partial x^n} \left(A^1, \ldots, A^n\right).
$$
Таким образом
\begin{equation}
    \partial_X f(A) = X^i \frac{\partial f}{\partial x^i} (A).
\end{equation}
Таким образом построили отображение
\begin{equation*}
    X \mapsto \partial_X \big|_A.
\end{equation*}
Выпишем несколько свойств такого оператора
\begin{align*}
    \partial_X (f+g)(A) &= \partial_X f(A) + \partial_X g(A) \\
    \partial_X (fg) (A) &=   (\partial_X f(A)) g(A) + f(A) (\partial_X g(A)).
\end{align*}
Что соответсвует правилу Лейбница.

\subsection{Дифференцирование как вектор}

Теперь зайдём с другой стороны. Рассмотрим $C^{\infty} (\mathbb{R}^n)$. Рассмотрим отображение $D$
\begin{equation*}
    D \colon C^\infty (U) \to \mathbb{R},
\end{equation*}
удоволетворяющее свойствам
\begin{align*}
    D(f+g) &= Df + Dg \\
    D(fg) &= (Df) \cdot g(A) + f(A) \cdot (Dg).
\end{align*}
Что и назовём дифференцированием в точке $A$. 

Легко показать, что $D (\const )= 0$, $D \lambda f = \lambda D f$ и $f(A)=g(A)=0 \Rightarrow D(fg)=0$. Вспомним теперь формулу Тейлора в координатах $u^1, \ldots, u^n$.
$$
    f(u^1, \ldots, u^n) = f(A^1, \ldots, A^n)
    +
    \frac{\partial f}{\partial u^i} (A^1, \ldots, A^n) \cdot (u^i - A^i)
    +
    h_{ij} (u^1, \ldots, u^n) (u^i - A^i) \cdot (u^j - A^j).
$$
Тогда
\begin{equation*}
    D(f) = 0 + \underbrace{D(u^i - A^i) }_{X^i}
    \frac{\partial f}{\partial u^i}(A^1, \ldots, A^n).
\end{equation*}
Таким образом
\begin{equation}
    Df = X^i \frac{\partial f}{\partial u^i} \left(A^1, \ldots, A^n\right).
\end{equation}
Итого
\begin{enumerate}
    \item В ДСК $X \mapsto \partial_X \big|_A$.
    \item В ДСК $D$ имеет вид $\partial_X \big|_A$ для некоторого $X$.
    \item Получили взаимно-однозначное соответствие векторы -- дифференцирование.
    \item Определим векторы, как дифференцирование. Это определение 
    \texttt{инвариантно}.
    \begin{equation}
        X = X^i \frac{\partial }{\partial u^i},
    \end{equation}
    где $(X^1, \ldots, X^n)$ -- координаты вектора в координатах $(u^1, \ldots, u^n)$.
\end{enumerate}


\subsection{Замена координат}
Допустим выбрали некоторые $(u^1, \ldots, u^n)$ и $(v^1, \ldots, v^n)$.
Тогда
$$
    D = X^i \frac{\partial }{\partial u^i} = Y^j \frac{\partial }{\partial v^j}.
$$
По правилу дифференцирования сложной функции
$$
    \frac{\partial f}{\partial u^i} = \frac{\partial v^j}{\partial u^i} \frac{\partial f}{\partial v^j},
    \hspace{0.5cm} \Rightarrow \hspace{0.5cm} 
    X^i \frac{\partial }{\partial u^i} = 
    \underbrace{X^i \frac{\partial v^j}{\partial u^i} }_{Y^j}
    \frac{\partial }{\partial v^j} .
$$
Получили формулу изменения координат вектора при смене системы\footnote{
    \texttt{<<В Царство небесное войдут только те кто думают про вектор, как про дифференцирование, потому что там нет координат.>>}
} координат
\begin{equation}
    Y^j = \frac{\partial v^j}{\partial u^i} X^i
    \hspace{0.5cm} \Leftrightarrow \hspace{0.5cm} 
    Y = J X.
\end{equation}



\subsection{Коммутатор}

Для матриц известен коммутатор вида
$$
    \left[A, B\right] = AB - BA.
$$
Аналогично для дифференцирования
\begin{align*}
    \left[\partial_X, \partial_Y \right] f
    =
    \partial_X \partial_Y f - \partial_Y \partial_X f 
    =
    X^i \frac{\partial }{\partial u^i}
     \left(Y^j \frac{\partial f}{\partial u^j} \right)
    -
    Y^j \frac{\partial }{\partial u^j} 
    \left(X^i \frac{\partial f}{\partial u^i} \right) 
    = X^i \frac{\partial Y^j}{\partial u^i} \frac{\partial f}{\partial u^i} 
    -
    Y^j \frac{\partial X^i}{\partial u^j} \frac{\partial f}{\partial u^j}
\end{align*}
Таким образом
\begin{equation}
    \left[\partial_X, \partial_Y \right] f
    =
    \left[
        X^i \frac{\partial Y^j}{\partial u^i} - Y^i \frac{\partial x^j}{\partial u^i} 
    \right] \frac{\partial f}{\partial u^i} .
\end{equation}
Это, как ни странно, дифференциальный оператор первого порядка. Это значит что есть такое векторное поле $\left[X, Y\right]$, что
$$
    \partial_{\left[X, Y\right]} = \left[\partial_X, \partial_Y\right] f.
$$
Таким образом $\left[X, Y\right]$ существует и равен
\begin{equation}
    \left[X, Y\right] =   X^i \frac{\partial Y^j}{\partial u^i} - Y^i \frac{\partial x^j}{\partial u^i}.
\end{equation}
