\subsection{Обратный образ}

Пусть
$$
    X^n \overset{F}{\longrightarrow} X^k \overset{\varphi}{\to} \mathbb{R}.
$$
Или можем рассмотреть отображение
$$
    X^n \overset{F^* \varphi}{\longrightarrow} \mathbb{R},
    \hspace{0.5cm} 
    \text{где }
    \hspace{0.5cm} 
    F^* \varphi  \overset{\mathrm{def}}{=} \varphi \circ F,
$$
что и является обратным образом.

Пусть теперь $P \in X^n$ отображается в $F(P) \in X^k$. Пусть $W(P) \in X^n$, постороим $d_p F(W)$ -- вектор $F(P) \in X^k$. Пусть $\varphi \in C^\infty (X^k)$, тогда
\begin{equation}
    \underbrace{d_P F(W)}_{\text{вектор}}
     \ \varphi \overset{\mathrm{def}}{=} W(F^* \varphi).
\end{equation}

\begin{to_def} 
    $d_P F$  -- \textit{дифференциал} $F$  в точке $P$.
\end{to_def}

Пусть $\varphi \circ \Psi = \varphi(v^1, \ldots, v^k)$ в координатах $v^1, \ldots, v^k$. Тогда
$$
    F^* \varphi = \varphi(F)
    \hspace{0.5cm} \Rightarrow \hspace{0.5cm} 
    F^* \varphi (\set{u}{k}) = 
    \underbrace{
        \varphi
        (v^1 (\set{u}{n}), \ldots, v^k (\set{u}{n}))
    }_{F^* \varphi \text{ в координатах } \set{u}{n}}
    = \varphi \circ F \circ \Phi,
$$
где $\Phi$ -- координатное отображение. Теперь вектор $W$ 
$$
    W = W^1 \frac{\partial }{\partial u^1} + \ldots + W^n \frac{\partial }{\partial u^n} = W^i \frac{\partial }{\partial u^i} .
$$
Соответсвенно, по определению
\begin{equation}
    d_P F(W) \ \varphi \overset{\mathrm{def}}{=} W F^* \varphi,
\end{equation}
расписывая, получим
\begin{equation*}
    W F^* \varphi
    =
    W^i \frac{\partial }{\partial u^i} \varphi(v^1(\set{u}{n}), \ldots)
    =
    W^i \frac{\partial \varphi}{\partial v^j} \frac{\partial v^j}{\partial u^i} 
    =
    \underbrace{
        \frac{\partial v^j}{\partial u^i} W^i \frac{\partial }{\partial v^j} 
    }_{
        d_p F (W)
    } \ \varphi.
\end{equation*}
А это кто? А вот матрица Якоби $F$, записанного в координатах $\set{v}{k}$
\begin{equation}
    \begin{bmatrix}
        d_P F(W) \, ^1 
        \\
        \raisebox{4pt}{\vdots}
        \\
        d_P F(W) \, ^k
    \end{bmatrix} =
    \begin{pmatrix}
        &&\\
        &\dfrac{\partial v^j}{\partial u^i} &\\
        &&
    \end{pmatrix}
    \begin{pmatrix}
        W^1 \\ 
        \raisebox{4pt}{\vdots}
        \\ W^n
    \end{pmatrix}
\end{equation}
Тогда выясняется, что $d_P F$ -- линейное отображение. Действительно,
$$
    d_P F(W_1 + W_2) \varphi = (W_1 + W_2) F^* \varphi =
    W_1 D^* \varphi + W_2 F^* \varphi =
    \left(
        d_P F(W_1) + d_P F(W_2)
    \right) \varphi. 
$$

\subsection{Тензор}

Есть пространство $V$ с векторами и двойственное $V^*$ с ковекторами, пространство линейных функций.
Тогда $\vc{e}_1, \ldots, \vc{e}_n$ -- базис в $V$, $\vc{e}^1, \ldots, \vc{e}^n$ -- двойственный базис в $V^*$, т.е. $\vc{e}^i \vc{e}_j = \delta^i_j$. 

Для начала скажем, что $W$ -- вектор и он же линейная функция на ковекторах. 
$$
    W(\xi) = \xi(W) = \langle W, \xi \rangle,
$$
что называется спариванием вектора и ковектора.

Пусть есть некоторая $B(W, Y)$ -- билинейная функция от двух векторов. А теперь посмотрим на линейный оператор $A \colon V \to V$, билинейную функцию от вектора и ковектора.
$$
    A(W, \xi) = \langle A(W), \xi \rangle
$$
Обобщим до понятия тензора:
$$
    T \colon 
    \underbrace{V^* \otimes \ldots \otimes V^* }_{p}
    \otimes 
    \underbrace{V \otimes \ldots \otimes V }_{q}
    \to \mathbb{R},
$$
где $T$ полилинейная функция от $p$ ковекторов и $q$ векторов, тензор типа $p, q$. Они образуют линейное пространство
$$
    T \in     
    \underbrace{V \otimes \ldots \otimes V}_{p}
    \otimes 
    \underbrace{V^* \otimes \ldots \otimes V^*}_{q} = \mathbb{T}^p_q (V).
$$
