\begin{proof}[\ref{thr_6.32}]   

\begin{enumerate*}
    \item По условию $df_1, \ldots, df_k, \ dx_{k+1}, \ldots, dx_n$ -- линейно независимы. Тогда $f_1, \ldots, f_k, x_{k+1, \ldots, x_n}$ дают криволинейную систему координат. 
    \item Тогда старые координаты (НД) выражаются через новые: $x_i = \varphi_i(f_1, \ldots, f_k, x_{k+1}, \ldots, x_n)$, при чём выберем $\vc{x}\colon$ $f_i = y_i$. $\mapsto$ Sol СУ содержится в графике отображения $\varphi \colon V \mapsto  U$, при достаточно малых $V, \ U$: $\varphi(V) \subseteq U$. 
    \item Но график отображения содержится в Sol(СУ), т.к. в точке $p = (y_1, \ldots, y_k, x_{k+1},\ldots,x_n)$ значения $f_i = y_i$, т.к. $\varphi_i (p)$ даст такие $x_1, \ldots, x_k$, что $f_i(x_i)=y_i$. Q. E. D.
\end{enumerate*}
    
\end{proof}
