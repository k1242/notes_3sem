\begin{proof}[\ref{thr_6.21}]
	
	1) $f_k(x) - f(x) = \int_\mathbb{R}^n (f(x-t) - f(x)) \varphi_k(t) \d t$;

	2) Пусть $f$ р-но непр. в $U_\delta(K\subset \mathbb{R}^n) $ и пусть $|f(x) - f(y)|<\varepsilon$ при $|x-y|<\delta$ там же;

	3) Выбирая $k\colon 1/k <\delta$, тогда $\varphi_k(t) \neq 0 $ при $|t|<\delta$ и тогда $|f(x-t) - f(x)|<\varepsilon$ при $x \in K$.

	4) при $x \in K$ верна р-ная сходимость: $|f_k(x) - f(x)| \leq \varepsilon \int_{\mathbb{R}^n}\varphi_k(x) \d x = \varepsilon$.

	5) продифференцируем по параметру $\int_{\mathbb{R}^n} f(t) \varphi_k (x-t)\d t $;

	6) производная (5) при $x \in K$ будет зависеть только значений $f$ в $U_{1/k}(K)$, то есть $f$ можно считать интегрируемой при дифференцировании по параметру, что позволяет применять теорему.
\end{proof}

\begin{proof}[\ref{thr_6.22}]
	 По различным $\partial_{x_i} f*\varphi_k(x)$ получим по лемме \ref{lem_6.19}, для производных свёрток схожее равенство, с самой $f$, а значит и р-ную сходимость.
	\begin{equation*}
		\frac{\partial^m (f*\varphi_k)}{\partial x_{i_1}\ldots \partial x_{i_m}} = \frac{\partial^m f}{\partial x_{i_1}\ldots \partial x_{i_m}}*\varphi_k.
	\end{equation*}
\end{proof}

\begin{proof}[\ref{thr_6.23}]
	1) по thr(\ref{thr_5.75}) $f = h + g$, где $g$ -- эл. ступ., $\int_{\mathbb{R}^n} |h|\d x <\varepsilon$;

	2) по thr(\ref{thr_6.18}): $\int_{\mathbb{R}^n}|h * \varphi_k|\d x < \varepsilon$. То есть, если окажется: $\int_{\mathbb{R}^n}|g - g*`f_k|\d x < \varepsilon$, то
	\begin{equation*}
		\int_{\mathbb{R}^n}|f - f*\varphi_k| \d x \leq \int_{\mathbb{R}^n} |g - g* \varphi_k|\d x + \int_{\mathbb{R}^n}|h|\d x + \int_{\mathbb{R}^n} |h*\varphi_k| \d x < 3 \varepsilon.
	\end{equation*}

	3) Раскладывая $g$ в сумму х-их $\chi_P$, останется доказать для одной $\chi_P$;

	4) $\chi_P - \chi_P - \varphi_k \neq 0$ только в $U_{1/k}(\partial P) $ и по модулю $\leq 1$;

	5) То есть после интегрирования получим не более $\mu(U_{1/k}(\partial P))$.

	6) Напрямую можно убедиться, что эта $\mu \mapsto 0$ при $k\mapsto 0$.
\end{proof}