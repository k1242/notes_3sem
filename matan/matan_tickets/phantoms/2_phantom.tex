\begin{to_tas}[Замена координат в интеграле для собственных отображений вообще]
    \label{task_6.108}
    Пусть гладкое отображение $\varphi \colon \mathbb{R}^n \mapsto \mathbb{R}^n$ является собственным. Тогда
    \begin{equation*}
        \int_{\mathbb{R}^n}\varphi^* \nu = C_\varphi \int_{\mathbb{R}^n} \nu, \hspace{0.5cm} 
        C_\varphi \in \mathbb{Z}.
    \end{equation*}
\end{to_tas}

\subsubsection*{Формула Стокса}


\begin{to_lem}[формула Стокса в узком смысле]
     Для компактной двумерной поверхности с краем (то есть вложенного двумерного многообразия с краем) $S \subset \mathbb{R}^3$ верна
\begin{equation*}
    \int_{\partial S} P \d x + Q \d y + R \d z =
    \int_S \left(\frac{\partial R}{\partial y} - \frac{\partial Q}{\partial z} \right) \d y \wedge d z + 
    \left(\frac{\partial P}{\partial z} - \frac{\partial R}{\partial x} \right) \d z \wedge dx + 
    \left(\frac{\partial Q}{\partial x} - \frac{\partial P}{\partial y} \right) \d x \wedge \d y.
\end{equation*}
\end{to_lem}


\begin{to_tas} 
    Площадь области, ограниченной замкнутой гладкой кривой без самопересечений $C \subset \mathbb{R}^2$, можно посчитать по формуле:
    \begin{equation*}
         A = \pm \int_C x \d y,
     \end{equation*} 
    где знак выбирается в зависимости от ориентации кривой.
\end{to_tas}

\begin{to_tas} 
    Объём области в $\mathbb{R}^3$, ограниченной связной вложенной компактной поверхностью без края $S \subset \mathbb{R}^3$, можно посчитать по формуле:
    \begin{equation*}
         A = \pm \int_S x \d y \wedge d z,
     \end{equation*} 
    где знак выбирается в зависимости от ориентации поверхности.
\end{to_tas}




\begin{to_tas}[Порядок точки относительно кривой]
Для замкнутой кусочно-гладкой $\gamma \in \mathbb{R}^2$, не проходящей через начало координат определим порядок начала координат относительно кривой:
\begin{equation*}
    w (\gamma, 0) - \frac{1}{2 \pi} \int_{\gamma} \frac{x \d y - y \d x}{x^2 + y^2},
\end{equation*}
и он не меняется при непрерывных деформациях кривой, при которых она не проходит через начало координат.
\end{to_tas}

\begin{to_tas}
    Порядок начала координат относительно кривой является целым.
\end{to_tas}

\begin{to_tas}
    Порядок начала координат относительно не проходящей через него нечётной кривой является нечётным числом. ($\gamma \colon \mathbb{S}^1 \mapsto \mathbb{R}^2, \; \gamma(-u) = - \gamma(u)$).
\end{to_tas}

\begin{to_tas}
    Для замкнутой кривой на плоскости с всюду не нулевой скоростью $\int k(s) \d s = 2 \pi N, \; N \in \mathbb{Z}$.
\end{to_tas}

\begin{to_tas}[Лемма Жордана]
    Замкнутая кусочно-гладкая кривая $\gamma \subset \mathbb{R}^2$ без самопересечений делит плоскость на две связные части внутреннюю и внешнюю (можно усложнить и сформулировать для непрерывных кривых).
\end{to_tas}



\subsubsection*{Коммутатор}

Для матриц известен коммутатор вида
$$
    \left[A, B\right] = AB - BA.
$$
Аналогично для дифференцирования
\begin{align*}
    \left[\partial_X, \partial_Y \right] f
    =
    \partial_X \partial_Y f - \partial_Y \partial_X f 
    =
    X^i \frac{\partial }{\partial u^i}
     \left(Y^j \frac{\partial f}{\partial u^j} \right)
    -
    Y^j \frac{\partial }{\partial u^j} 
    \left(X^i \frac{\partial f}{\partial u^i} \right) 
    = X^i \frac{\partial Y^j}{\partial u^i} \frac{\partial f}{\partial u^i} 
    -
    Y^j \frac{\partial X^i}{\partial u^j} \frac{\partial f}{\partial u^j}
\end{align*}
Таким образом
\begin{equation}
    \left[\partial_X, \partial_Y \right] f
    =
    \left(
    X^i \partial_i Y^j - Y^i \partial_i X^j
    \right) \partial_j f
    .
\end{equation}
Это, как ни странно, дифференциальный оператор первого порядка. Это значит что есть такое векторное поле $\left[X, Y\right]$, что
$$
    \partial_{\left[X, Y\right]} = \left[\partial_X, \partial_Y\right] f.
$$
Таким образом $\left[X, Y\right]$ существует и равен
\begin{equation}
    \left[X, Y\right] =   X^i \partial_i Y^j - Y^i \partial_i X^j.
\end{equation}

\subsubsection*{Уравнения Эйнштейна}
Выпишем уравнение Эйнштейна для гравитационного поля (полуримановой структуры), которое с точностью до единиц измерения в 4-мерном пространстве времени выглядит как
\begin{equation*}
    \text{Ric}\, - \frac{1}{2} S g + \Lambda g = T,
\end{equation*}
где $S = \delta^{ij} \text{Ric}\, (e_i, f_j )$ -- свёртка (след) тензора Риччи ($g(e_i, f_j)=\delta_{ij}$), \textit{скалярная кривизна}, отвечающая за искажение объёма окрестности точки по сравнению с объемом из экспоненциального отображения, $\Lambda$ -- космологическая постоянная (<<темная энергия>>), а $T$ -- тензор энергии импульса.