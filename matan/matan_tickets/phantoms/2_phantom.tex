\begin{to_tas}[Замена координат в интеграле для собственных отображений вообще]
    \label{task_6.108}
    Пусть гладкое отображение $\varphi \colon \mathbb{R}^n \mapsto \mathbb{R}^n$ является собственным. Тогда
    \begin{equation*}
        \int_{\mathbb{R}^n}\varphi^* \nu = C_\varphi \int_{\mathbb{R}^n} \nu, \hspace{0.5cm} 
        C_\varphi \in \mathbb{Z}.
    \end{equation*}
\end{to_tas}

\subsubsection*{Формула Стокса}


\begin{to_lem}[формула Стокса в узком смысле]
     Для компактной двумерной поверхности с краем (то есть вложенного двумерного многообразия с краем) $S \subset \mathbb{R}^3$ верна
\begin{equation*}
    \int_{\partial S} P \d x + Q \d y + R \d z =
    \int_S \left(\frac{\partial R}{\partial y} - \frac{\partial Q}{\partial z} \right) \d y \wedge d z + 
    \left(\frac{\partial P}{\partial z} - \frac{\partial R}{\partial x} \right) \d z \wedge dx + 
    \left(\frac{\partial Q}{\partial x} - \frac{\partial P}{\partial y} \right) \d x \wedge \d y.
\end{equation*}
\end{to_lem}


\begin{to_tas} 
    Площадь области, ограниченной замкнутой гладкой кривой без самопересечений $C \subset \mathbb{R}^2$, можно посчитать по формуле:
    \begin{equation*}
         A = \pm \int_C x \d y,
     \end{equation*} 
    где знак выбирается в зависимости от ориентации кривой.
\end{to_tas}

\begin{to_tas} 
    Объём области в $\mathbb{R}^3$, ограниченной связной вложенной компактной поверхностью без края $S \subset \mathbb{R}^3$, можно посчитать по формуле:
    \begin{equation*}
         A = \pm \int_S x \d y \wedge d z,
     \end{equation*} 
    где знак выбирается в зависимости от ориентации поверхности.
\end{to_tas}

