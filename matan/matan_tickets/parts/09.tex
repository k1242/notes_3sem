\begin{to_def}
	Точка $p$ называется локальным экстремумом функции $f$, если она является точкой экстремума (максимума или минимума) ограничения $f$ на некоторую окрестность $p$.
	\label{def_6.36}
\end{to_def}

\begin{to_thr}[Необходимое условие экстремума]
	\begin{equation*}
		\left\{\begin{aligned}
	    	&f \in C^1(U(p))\\
	    	&p \text{ --- def(\ref{def_6.36})}
		\end{aligned} \right.
		\hspace*{0.5 cm} \Rightarrow \hspace*{0.5 cm}
		\d f_p = 0.
	\end{equation*}
	\label{thr_6.37}
\end{to_thr}

\begin{to_lem}
	Если квадратичная форма $Q > 0$, \textbf{то} $\exists \varepsilon>0 \colon $ $Q(v) \geq \varepsilon|v|^2 \;(\forall v)$.
	\label{lem_6.38}
\end{to_lem}
