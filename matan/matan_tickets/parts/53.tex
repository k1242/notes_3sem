\begin{to_def}[аксиоматическое определение для ковариантной производной от вектора]
    \begin{enumerate*}
        \item $\nabla_{fX} Y = f \nabla_X Y$ (линейность по перовому аргументу)
        \item $\nabla_X f Y = X(f) Y + f \nabla_X Y$ (правило Лейбница для второго аргумента)
        \item $\nabla_X Y - \nabla_Y X = [X, Y]$ (отсутствие кручения)
        \item $X(g(Y, Z)) = g(\nabla_X Y, Z) + g(Y, \nabla_X Z)$ (совместимость с (полу)римановой структурой $g$)
    \end{enumerate*}     
\end{to_def}



\begin{to_thr}[Формула Козюля]
     Из условий на ковариантное дифференцирование следует формула
\begin{equation*}
    2 g(\nabla_X Y, Z) = 
    X(g(Y, Z)) + Y(g(Z, X)) - Z(g(X, Y)) + g([X, Y], Z) - g([Y, Z], X) + g([Z, X], Y).
\end{equation*}
\end{to_thr}


\begin{to_def}[определение ковариантной производной от форм]
     Требуем выполнение правила Лейбница для умножения векторных форм первой степени на векторные поля
    \begin{equation*}
        (\nabla_X \alpha)(Y) = \nabla_X (\alpha(Y)) - \alpha(\nabla_X Y) = X(\alpha(Y)) - \alpha(\nabla_X Y).
    \end{equation*}
\end{to_def}

\begin{to_tas} 
    Условие отсутствия кручения эквивалентно тому, что для форм первой степени выполняется
    \begin{equation*}
         (\nabla_X \alpha)(Y) - (\nabla_Y \alpha)(X) = d \alpha(X, Y).
     \end{equation*} 
     Действительно, $d \alpha(X, Y) = X(\alpha(Y)) - Y(\alpha(X)) - \alpha([X, Y])$.
\end{to_tas}

\begin{to_tas} 
    Пусть $\nabla_X Y =Z$, тогда
    \begin{equation*}
         Z^k = X^i \partial_i Y^k + \Gamma_{ij}^k X^i Y^j,
         \hspace{0.5cm} \text{где} \hspace{0.5cm} 
         \Gamma_{lij} = \frac{1}{2} \left( 
         \vphantom{\frac{1}{2}} 
            \partial_j g_{li} + \partial_i g_{lj} - \partial_l g_{ij}
         \right).
    \end{equation*}        
\end{to_tas}



