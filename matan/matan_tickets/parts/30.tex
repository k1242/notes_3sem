\begin{to_def} 
    Гладкое многообразие $M$ называется \textit{ориентируемым}, если можно выбрать покрывающий атлас так, что якобианы замен координат между любыми двумя картами атласа будут положительными. 
\end{to_def}

Если в исходном атласе был задан некоторый объект, например векторное поле $X$, то во всякой новой карте $\psi$ мы тоже будем иметь векторное поле, собранное из прямых образов $(\psi  \circ \varphi^{-1})_* X_\varphi$ полученных с имеющихся карт $\varphi$ и образов $X_\varphi$ в них.

\begin{to_def} 
    \textit{Ориентацией гладкого многообразия} $M$ называется атлас с положительными якобианами перехода между картами, максимальный по включению среди всех таких атласов. 
\end{to_def}

\begin{to_lem} 
    Связное многообразие либо неориентируемо, либо допускает два класса ориентации. 
\end{to_lem}

\begin{to_lem} 
    Многообразие $M$ размерности $n$ ориентируемо тогда и только тогда, когда существует дифференциальная форма $\nu \in \Omega^n (M)$, которая ни в одной точку не равна нулю.
\end{to_lem}

\begin{to_lem} 
\label{zorich_cXV.p2.lem4} % см. страницу 308
    Многообразие ориентируемо тогда и только тогда, когда на нём не существует противоречивой (дезориентирующей) цепочки карт.
\end{to_lem}

\begin{to_def} 
    Для $n$-мерного ориентированного многообразия с краем $M$ введём ориентацию на его крае $\partial M$ следующим образом. Пусть карта $M$ с координатами $x_1, \ldots, x_n$ соответсвует ориентации $M$, причём образ отображения карты удовлетворяет неравенству $x_1 \leq 0$, а образ края соответствует равенству $x_1 = 0$. Тогда карта на соответствующей части $\partial M$ из координат $x_2, \ldots, x_n$ по определению объявляется положительной. Если же многообразие в этой карте задано неравенством в другую сторону, $x_1 \geq 0$, то карта $x_2, \ldots, x_n$ на его краю по определению объявляется отрицательной. 
\end{to_def}

\begin{to_lem} 
    Предыдущее определение корректно задаёт ориентацию на $\partial M$. 
\end{to_lem}






