\begin{to_lem} 
\label{lem_6.100}
    Пусть $U = \prod_{i=1}^n (a_i, b_i)$, где $(a_i, b_i) \ni 0$. Пусть $\varphi \colon \mathbb{R} \mapsto \mathbb{R}^+$ -- гладкая функция с компактным носителем, содержащимся в каждом $(a_i, b_i)$, и с единичным интегралом. Для всякой $\nu \in \Omega_{\text{c}}^n (U)$ найдётся число $I$ и форма $\lambda \in \Omega_{\text{c}}^{n-1}(U)$, такие что
    $\nu = I \varphi(x_1) \ldots \varphi(x_n) dx_1 \wedge \cdots \wedge dx_n + d \lambda$.
\end{to_lem}


\begin{to_con} 
\label{con_6.101}
    Пусть $U = \prod_{i=1}^{n} (a_i, b_i)$ -- произведение интервалов. Факторпространство $\Omega_{\text{c}}^n(U) / d \Omega_{\text{c}}^{n-1} (U)$ одномерно. Получается, что всевозможные способы определить интеграл формы $\nu \in \Omega_{\text{c}}^n (U)$ так, чтобы интеграл от $d \lambda$ равнялся нулю, могут отличаться только умножением на константу. \red{Ещё раз.} 
\end{to_con}