\begin{to_thr}[Теорема о неявной функции]
     Пусть функции $f_1, \ldots, f_k$ непрерывно дифференцируемы в окрестности $p \in \mathbb{R}^n$ и 
    \begin{equation*}
        \det \left(
            \frac{\partial f_i}{\partial x_j} 
        \right) \neq 0
    \end{equation*}
    в этой окрестности (поверхность является \textit{регулярной}). Пусть $f_i(p) = y_i$. Тогда найдётся окрестности точки $p$ вида $U \times V$, $U \subset \mathbb{R}^k$, $V \subset \mathbb{R}^{n-k}$, такая что в этой окрестности множество решений системы уравнений
    \begin{equation*}
        \left\{\begin{aligned}
            f_1(x) &= y_1, \\
            &\ldots \\
            f_k(x) &= y_k,
        \end{aligned}\right.
    \end{equation*}
    совпадает с графиком непрерывно дифференцируемого отображения $\varphi \colon V \to U$, заданного в координатах как
    \begin{equation*}
        \left\{\begin{aligned}
            x_1 &= \varphi_1 (y_1, \ldots, y_k,\ x_{k+1}, \ldots, x_n),\\
            &\ldots\\
            x_k &= \varphi_k (y_1, \ldots, y_k,\ x_{k+1}, \ldots, x_n),
        \end{aligned}\right.
    \end{equation*}
    то есть отображения $\mathbb{R}^{n-k} \mapsto \mathbb{R}^k$.
\end{to_thr}
