
\begin{to_def} 
    Функция $f \colon M \mapsto \mathbb{R}$ называется \textit{гладкой функцией на многообразии}, $f \in C^{\infty}(M)$, если в каждой координатной карте $\varphi \colon U \mapsto \mathbb{R}^n$ эта функция ($f \circ \varphi^{-1}$) является гладкой функцией на образе $\varphi(U)$.
\end{to_def}



\begin{to_def} 
    \textit{Гладкой структурой} на топологическом пространстве называется максимальный по включению атлас, с которым пространство становится многообразием. 
\end{to_def}


\begin{to_def} 
    \textit{Гладким отображением} между многообразиями $f \colon M \mapsto N$ размерностей $m$ и $n$ называется непрерывное отображение, которое в окрестности каждой точки, в достаточно малых координатных картах, выглядит как гладкое отображение из $\mathbb{R}^m$ в $\mathbb{R}^n$.
\end{to_def}


\begin{to_def} 
    Гладкое обратимое отображение $f \colon M \mapsto N$ с обратным гладким назовётся \textit{диффеоморфизмом многообразий}. 
\end{to_def}

\begin{to_tas} 
\label{task_6.131} % + текст с 227 страницы.
    Если взять некоторое \textit{компактное} гладкое многообразие $M$ (область параметров) и гладкое отображение $f \colon M \mapsto \mathbb{R}^n$, такое, что $\rg Df_p = \dim M$ $\forall p$, то $f(M)$ будет вложенным многообразием.     
\end{to_tas}

\begin{to_lem} 
    Для гладкого отображения $f \colon M \mapsto \mathbb{R}^n$ с $\rg Df \equiv m = \dim M$, для всякой $p \in M$ найдётся окрестность $U \ni p$ такая, что $f(U)$ в некоторой криволинейной системе координат в окрестности $f(p)$ является открытым подмножеством стандартно вложенного $\mathbb{R}^m \subseteq \mathbb{R}^n$. 
\end{to_lem}
