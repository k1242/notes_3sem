 \begin{to_def}
 	Риманова метрика на гладком $M$ --- симметричное положительно определённое невырождение тензорное поле $(g_{i j}) \in \mathbb{T}_2^0(M^n)$. 
 \end{to_def}

 \begin{to_def}
 	$\sharp \colon \mathbb{T}_1^0(M^n) \rightarrow \mathbb{T}_0^1(M^n)$ (диез) --- операция поднятия индекса: $\alpha_i \rightarrow g^{i j} \alpha_j$.
 \end{to_def}

 \begin{to_def}
 	$\flat \colon \mathbb{T}^1_0(M^n) \rightarrow \mathbb{T}^0_1(M^n)$ (бемоль) --- операция опускания индекса: $v^i \rightarrow g_{i j}v^j$.
 \end{to_def}

Таким образом форма (полу)римановой структуры может рассматриваться как отображение $g\colon T_pM \otimes T_pM \rightarrow \mathbb{R}$.
Композиция $g$ и двух поднятий индексов на её аргументов даёт билинейное отображение $\tilde{g}\colon T_p^*M \otimes T_p^*M \rightarrow \mathbb{R}$. 

\begin{to_tas}[Коши-Буняковский]
	$\alpha(X)^2 \leq g(X,X) \cdot \tilde{g} (\alpha,\alpha)$.
\end{to_tas}

\begin{to_def}
	В присутствии (полу)римановой структуры $g$ на ориентированном многообразии $M^n$ (чтобы $\text{vol}_{g}$) можно было считать элементом $\Omega^n(M)$) формула 
	\begin{equation*}
		\alpha \wedge *\beta = \tilde{g}(\alpha,\beta) \text{vol}_g, \; \forall \alpha \in \Omega^k(M),	
	\end{equation*}
	корректно определяет линейный оператор $*\colon \Omega^k(M) \rightarrow \Omega^{n-k}(M)$ --- звёздочку Ходжа.
\end{to_def}