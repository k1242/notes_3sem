Проверить, что всё работает, когда у нас $N$ не компактно, и мы имеем дело с собственным отображением (прообраз компакта --- компакт).

Определим степень для таких отображений:
\begin{to_def}
	$f \colon M \rightarrow N$ --- собственное и гладкое, $M,N$ --- ориентированные и без края, и одной и той же размерности, причём $N$ связно. Тогда для $\omega \in \Omega_c^n (M)$ выполняется:
\begin{equation*}
	\int_{M} f* \omega = \deg f \cdot \int_{N} \omega.
\end{equation*}	
И вот оно эквивалентно, всё как всегда в качестве упражнения, мужайтесь.
\end{to_def}

\begin{to_lem}
	$f \colon M\rightarrow N$ гладкое отображение ориентированных многообразий с карем одной и той же размерности, причём $f(\partial M) \subseteq \partial N$ и $N$ связно. В этом случае степень $f$ корректно определена: $\deg f = \deg f |_{\partial N} $.
\end{to_lem}