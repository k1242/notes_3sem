Основным компонентом системы IRE является интеллектуальный механизм рекомендаций, основанный на AI-Quantum Matching ©. «В естественных науках природа дала нам мир, а мы только что открыли его законы. В компьютерах мы можем запихивать в них законы и создавать мир »(Алан Кей, н.о.), формируя искусственную систему, подчиняющуюся законам, которых нет в Природе. 

Одним из примеров такой искусственной системы является квантовый компьютер, который следует тем же эффектам квантовой механики, как квантовый параллелизм и квантовая интерференция. Традиционный компьютер обрабатывает данные, кодируя их в 0 и 1, создавая большое количество возможных значений. Однако он может находиться только в одном из этих состояний одновременно. Точно так же квантовый компьютер может присутствовать в комбинации всех этих состояний одновременно, что позволяет расширить вычислительную мощность с помощью квантового параллелизма. Задача состоит в том, чтобы закодировать результаты потенциально большого объема вычислений.

 В традиционном мире, если есть два возможных пути к одному результату и каждый путь выбирается с вероятностью 0,25, общая вероятность получения этого результата составляет 0,5. 

 В квантовом мире два пути могут мешать, увеличивая вероятность успеха до 1. 

 Другими словами, квантовый параллелизм дает преимущества большого объема вычислений,
  в то время как интерференция помогает объединить выходные данные в значимые результаты.

  . Недавние исследования показывают, что квантовая интерференция может использоваться для моделирования реальных финансовых сценариев с высоким уровнем неопределенности, показывая многообещающий подход для систем поддержки принятия решений и искусственного интеллекта (Moreira et al., 2018, Moreira & Wichert, 2018).


Предлагаемый DDT AI-Quantum Matching © основан на концепции квантовых искусственных систем с точки зрения информации, обеспечивающей наилучшие разумные рекомендации. В рамках подхода AI-Quantum Matching © рекомендации могут быть представлены в виде многомерных векторов, отражающих потенциалы всех событий. Матч можно назвать суперпозицией в квантовой механике. Векторное представление суперпозиции не подчиняется ни булевой логике, ни закону полной вероятности. В результате это позволяет строить более общие модели и повышает доступность информации, которую невозможно уловить в традиционных моделях. Это создает возможность моделировать потенциально различные сценарии принятия решений. Эти отличительные особенности квантовой теории обеспечивают несколько преимуществ и дают более точные рекомендации в ситуациях, когда одна только классическая теория вероятностей приводит к загадочным и противоречивым предсказаниям (смещениям и ошибкам).

Давайте исследуем возможность поиска. В реальной жизни мы могли бы использовать несколько поисковых систем, чтобы найти все множество информации по определенной теме. Хотя Google считается лучшей поисковой системой для этой работы, мы все же можем упустить некоторые важные данные, которые могут быть предложены другими поисковыми системами, такими как Yahoo и Bing. Как найти наиболее полезную для нас информацию? AI-Quantum Matching может помочь найти наилучшую возможную рекомендацию, используя квантовую структуру в поиске информации. Структура AI-Quantum Matching представлена ​​на рисунке 2. Например, при принятии решения о выключении света модель AI одновременно находится в неопределенном состоянии, соответствующем «да», и в состоянии соответствует «нет». Каждая модель может рассуждать в соответствии со своим набором правил. Различные модели получаются при высоком уровне неопределенности, информация непрерывно поступает в систему базы знаний.


Мы предлагаем унифицированную структуру для интеллектуальных рекомендаций, основанных на квантовой механике, которые предоставляют более обобщенные модели принятия решений, способные представить больше информации, чем классические модели. Он может учесть несколько парадоксальных выводов и когнитивных предубеждений и привести к альтернативным и мощным выводам, сфокусированным на перспективной зависимости лица, принимающего решения.


К истории про интереференцию никаких вопросов, звучит адекватно. 

Есть множество различных способов работать с распределением вероятности по ответам, то есть ничего нам не мешает (при наличии достаточного количества данных) засунуть это всё в нейросеть и научить её делать то, что нам нужно. Как вариант, мы можем делать что-то похожее на квантмех, введя в вероятности возможность интерференции. То есть у нашего механизма принятия решений есть некоторая интепретация. Наверное, хотелось бы понять чем такой вариант лучше (других нелинейных механизмов), кроме наличия естественной интерпретации.



