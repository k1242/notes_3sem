\begin{to_thr}[Необходимые условия экстремума]
	\begin{equation*}
	\left\{\begin{aligned}
	    	&f \in C^2(U(p))\\
	    	&p \text{ --- def(\ref{def_6.36})} \\
	    	& \d f_p = 0
		\end{aligned} \right.
		\hspace*{0.5 cm} 
		\Rightarrow
		 \hspace*{0.5 cm}
		\left[
		\begin{aligned}
			\d_2 f_p \geq 0 \text{ --- min}\\
			\d_2 f_p \leq 0	\text{ --- max}
		\end{aligned}\right.
\end{equation*}	
\end{to_thr}

\begin{to_lem}
	Если квадратичная форма $Q > 0$, \textbf{то} $\exists \varepsilon>0 \colon $ $Q(v) \geq \varepsilon|v|^2 \;(\forall v)$.
	\label{lem_6.38}
\end{to_lem}


\begin{to_thr}[Достаточные условия экстремума]
\begin{equation*}
	\left\{\begin{aligned}
	    	&f \in C^2(U(p))\\
	    	&p \text{ --- def(\ref{def_6.36})}\\
	    	&\d f_p = 0 \text{ и } \d_2 f_p <0 
		\end{aligned} \right.
	\hspace*{0.5 cm} 
	 \Rightarrow
	  \hspace*{0.5 cm}
	p \text{ --- точка строгого локального максимума}
	\label{thr_6.39}
\end{equation*}
\end{to_thr}