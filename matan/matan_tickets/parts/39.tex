\begin{to_def}
	Если $f \colon M \rightarrow N$ --- гладкое отображение ориентированных многообразий одной и той же размерности, $M$ --- компактно, и $y$ --- регулярное значение $f$, степенью отображения в точке $y$ называется:
	\begin{equation*}
		\sum_{f(x_i) = y} \sign J f_{x_i}.
	\end{equation*}
	(число точек в прообразе $f^{-1}(y)$, для которых якобиан $J_{f_x}>0$ за вычетом числа точек в прообразе $f^{-1}(y)$, у которых $J_{f_x}<0$)

	В случае отсутствия ориентации хотя бы одного многообразия степень определена по модулю 2 как чётность количества точек в прообразе $f^{-1}(y)$.
\end{to_def}

Из условия того, что $y$ --- регулярное значение, следует, что множество $f^{-1}(y)$ состоит из изолированных точек, то есть это дискретное множество. В случае компактного $M$ число точек $f^{-1}(y)$ должно оказаться конечным, так как дискретное компактное множество конечно. То же будет верно для отображения, для которого прообраз любого компакта компактен, но использовать мы этого уже не будем.

\begin{to_lem}
	Если $f\colon M \rightarrow N$ --- гладкое отображение ориентированных многообразий без края одной и той же размерности, многообразие $M$ компактно, и $y$ --- регулярное значение $f$, \textbf{то}$ \exists U \ni y$ (окр-ть), такая что $\forall y' \in U$ регулярны и $\deg_y f = \deg_{y'} f$. 

	В случае отсутствия ориентации в $M$ мы просто утверждаем независимость количества точек в $f^{-1} (y')$ от $y'\in U$.
\end{to_lem}

\begin{to_thr}[Гомотопическая инвариантность степени отображения]
	Пусть многообразие $M$ компактное без края, $N$ --- не обязательно компактное без края, $h \colon M \times [0,1] \rightarrow N$ --- гладкая гомотопия, а $y \in N$ такова, что она является регулярным значением для $h_{0}$ и $h_{1}$. 

	Тогда степени отображения $h_0$ и $h_1$ в точке $y$ равны. Если оба многообразия ориентированы, то степень считается со знаком, иначе она считается как чётность.
\end{to_thr}

\begin{to_def}
	Семейство диффеоморфизмов $h_t \colon M \rightarrow N$ назовём изотопией, если оно гладко зависит от параметра $t$, то есть даёт гладкое $h \colon M \times [0,1] \rightarrow N$.
\end{to_def}

\begin{to_lem}
	Если многообразие $M$ связное, без края и $x,y \in M$, то существует изотопия $h_t \colon M \rightarrow M $, такая что $h_0 = id$ и $h_1(x)  = y$.
\end{to_lem}

\begin{to_con}
	При связном $N$ степень $f \colon M \rightarrow$ не зависит от выбора $y \in N$.
\end{to_con}

\begin{to_thr}[Корректность определения степени отображения]
	Степень отображения $f \colon M \rightarrow N$ для связного многообразия без края $N$ и компактного многообразия без края $M$ не зависит от выбора регулярного значения в $M$. Если оба многообразия ориентированы, то степень считается со знаком, иначе она считается как чётность.
\end{to_thr}

\begin{to_con}
	Пусть $M$ --- компактное многообразие без края положительной размерности. Тогда тождественное отображение $id\colon M \rightarrow M$ не гомотопно постоянному отображению $M \rightarrow M$ в одну точку.
\end{to_con}