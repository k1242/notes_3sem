

% 6.6. Векторы и векторные поля

\begin{to_lem} 
    \label{lem_6.44}
    Всякую гладкую функцию, определенную в некоторой окрестности $x_0 \in \mathbb{R}^n$, в возможно меньшей окрестности $x_0$, можно представить в виде
    \begin{equation*}
         f(x) = f(x_0) + \sum_{k=1}^n \partial_k f|_{x_0}
         \
         (x^k  - x_0^k) , 
     \end{equation*} 
     с гладкими $\partial_k f$.
\end{to_lem}

% Начнём наше удивительное приключение в определение вектора. Для всякого $v \in \mathbb{R}^n$ мы можем определеить производную по направлению
% \begin{equation*}
%     \frac{\partial f}{\partial v}  = \lim_{t \to 0} 
%     \left(
%         \frac{f(p+tv)-f(p)}{t} 
%     \right),
%     \hspace{0.5cm} 
%     \frac{\partial (fg)}{\partial v} = \frac{\partial f}{\partial v} g + f \frac{\partial g}{\partial v}.
% \end{equation*}
% Проблема -- $p + tv$ определено не инвариантно. Решение -- определим вектор, как \textit{дифференцирование алгебры гладких функций в точке}, удовлетворяющее формуле Лейбница.

\begin{to_def} 
    Определим \textit{касательный вектор} в точке $p \in U$ открытого множества $U \subseteq \mathbb{R}^n$ как $\mathbb{R}$-линейное отоборражение $X \colon C^{\infty}(U) \mapsto \mathbb{R}$, удовлетворяющее 
    \begin{equation*}
        X(fg) = X(f) g(p) + f(p) X(g).
    \end{equation*}
    \textit{Касательное пространство} $T_p U$ к $U$ в точке $p$ состоит из всех касательных векторов в точке $p$.
\end{to_def}

\begin{to_lem} 
    Если $X$ -- касательный вектор в точке $p \in U$, то для любой окресности $V \ni p$, $V \subseteq U$, выражение $X(f)$ может зависеть только от значений $f$ в $V$, а не на всём $U$. 
\end{to_lem}

В силу предыдущих лем мы можем перейти в окрестность, где $f$ представима в виде \eqref{eq_lem_6.44}, тогда
\begin{equation*}
    X(f) = X(f(p)) + \sum_{i=1}^n X(x_i) \partial_i f|_p + \sum_{i=1}^n x_i(p) X(\partial_i f|_p) = 
    \sum_{i=1}^n X(x_i) \ \partial_i f|_p.
\end{equation*}
Числа $X_i = X(x_i)$ называются координатами касательного вектора в данной криволинейной системе координат, тогда весь вектор в точке $p$ записывается, как $X = X^i \partial_i$.
