\begin{to_def}
	Диф-форма $\alpha \in \Omega^k (M)$ --- замкнутая, если $\d \alpha = 0$; $\alpha$ --- точная, если $\exists \beta \in \Omega^{k-1} (M) \colon \d \beta = \alpha$.
	\label{def_7.9}
\end{to_def}

\begin{to_tas}
	Если $\alpha$ --- точная, а $\beta$ --- замкнутая, \textbf{то} $\alpha \wedge \beta$ --- точная.
	\label{tas_7.10}
\end{to_tas}

\begin{to_def}
	Отображения $h_0, h_1 \colon M \rightarrow N$ гладко гомотопны, если $\exists h \colon M \times [0,1] \rightarrow N$, такое что $h_0(x) = h(x,0)$ и $h_1(x) = h(x,1)$.
	\label{def_7.12}
\end{to_def}

Многообразие $M \times [0,1]$ (цилиндр) является топологическим пространством, которое локально устроено как произведение открытого $U\subset \mathbb{R}^n$ на отрезок.
Интересно заметить, что если $\partial M = \varnothing$, то $\partial(M \times [0,1]) = M \times {0,1}$. 

\begin{to_def}
	М называется стягиваемым в точку $x_0 \in M$ или гомотопным точке, если $\exists h \colon M \times [0,1] \rightarrow M$ такое, что $h(x,1) = x$ и $h(x,0) = x_0$. ($\mathbb{R}^n$ --- $h(x,t) = t x$)
\end{to_def}

\begin{to_thr}[Теорема Пуанкаре]
	$\forall \omega \in \Omega^{k+1}(M)$, замкнутая на стягиваемом в точку $M$, является точной.
	\label{thr_poin}
\end{to_thr}

В умных книжках говорят чаще про непрерывные гомотопии, но так как мы работаем с гладкими $M$, мы будем использовать гладкие гомотопии. Обладая определенной сноровкой можно показать, что если два отображения между многообразиями непрерывно гомотопны, то они и гладко гомотопны.

Станет легче, если свести вопрос к случаю, когда для $\varepsilon > 0$: $h(x,t) = h(x,0)$ при $t < \varepsilon$ и $h(x,t) = h(x,1)$ при $t > 1-\varepsilon$.
Определим новую гомотопию:
\begin{equation*}
\varphi(\in C^\infty) \colon \mathbb{R} \rightarrow [0,1]:
\left\{
\begin{aligned}
	&\varphi = 0,\; t \leq \varepsilon\\
	&\varphi = 1,\; t \geq 1-\varepsilon
\end{aligned}
\right.
\hspace*{0.5 cm}  \leadsto \hspace*{0.5 cm}
	h'(x,t)=h(x,\varphi(t)).
\end{equation*} 

\begin{to_thr}[Цепная гомотопия]
	Если отображения $f_0, f_1 \colon M \rightarrow N$ гладко гомотопны, \textbf{то} $\forall \alpha \in \Omega^k (N)$ выполняется для некоторой $H \colon \Omega^k (N) \rightarrow \Omega^{k-1}(M)$:
	\begin{equation*}
		f_1^* \alpha - f_0^*\alpha - H(\delta \alpha) + \delta(H(\alpha)).
	\end{equation*}
\end{to_thr}