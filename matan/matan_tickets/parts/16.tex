\begin{to_def} 
    Определим \textit{дифференциальную форму степени} $k$ на открытом $U \subseteq \mathbb{R}^n$ как кососимметричное отображение наборов из $k$ гладких векторных полей $X_1, \ldots, X_k$ на $U$ в $C^{\infty}(U)$, линейное по каждому аргументу и относительно умножения на бесконечно гладкие функции.
\end{to_def}

\begin{to_lem} 
    Значение выражения $\alpha(X_1, \ldots, X_k)$ в точке $p$ зависит только от значений векторных полей $X_i$ в точке $p$.     
\end{to_lem}

Пространство диф-форм степени $k$ на $U \subseteq \mathbb{R}^n$ обозначим $\Omega^k(U)$. Интересно, что $\Omega^n(U)$ в фиксированной системе координат выглядит как $C^{\infty}(U)$, но при замене координат ведёт себя иначе.

% Для $ \forall f \in C^{\infty}(U)$ её дифференциал, как отображение $U \mapsto \mathbb{R}$ можно считать формой первой степени $df(X) = X(f)$. 


\red{Свойства диф-форм?}




