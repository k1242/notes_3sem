\begin{to_tas}[теорема о дивергенции]
     Пусть на многообразии $M$ фиксирована нигде не нулевая форма $\nu \in \Omega^n(M)$ при $n = \dim M$. Интегрирование этой формы задаёт некоторое понятие объема (меры) на многообразии. Тогда дивергенцию векторного поля $X$ относительно этого объема можно определить как
     \begin{equation*}
        \text{Vol}\, U = \int_U \nu, 
        \hspace{0.5cm} 
        U \subseteq M,
        \hspace{1cm}
         L_X \nu = (\div_\nu X) \nu = d\, (i_X \nu).
     \end{equation*}
     Таким образом построили отображение $X \mapsto \div_\nu X$ (скаляр). Получается, что
    \begin{equation*}
        \int_M (\div_\nu X) \nu = \int_{\partial M} i_X \nu.
    \end{equation*}
    Что позволяет формализовать понятие потока векторного поля через некоторую поверхность.
\end{to_tas}

\begin{to_thr}[Геометрический смысл дивергенции]
     Пусть на ориентированном многообразии $M$ мера определена как интеграл от некоторой всюду ненулевой соответствующей ориентации формы $\nu \in \Omega^n (M)$. Тогда для дивергенции векторного поля $X$ относительно объёма $\nu$ и всякого компактного $K \subseteq M$ имеет место формула
     \begin{equation*}
         \int_K (\div_\nu X) \nu = \frac{d}{dt} \text{Vol}_\nu\, \varphi_t (K) \bigg|_{t=01,}
     \end{equation*}
     где $\varphi_t$ -- соответствующая однопараметрическая группа диффеоморфизмов.
\end{to_thr}

\subsubsection*{Физическая интерпретация векторных операторов}

\begin{itemize}
    \item[$\div \vc{B}$.] 
    Для некоторой точки $x$ области $V$ ($V_x$ -- также объём области, $r$ --её диаметр) с заданным полем $\vc{B}$, по формуле Стокса и теореме о среднем ($\exists x' \in V(x) $ такая, что)
\begin{equation*}
    \int_{\partial V} \vc{B} \cdot \d \vc{\sigma} = 
    \int_V \div \vc{B} \d V = \div \vc{B}(x')V_x,
    \hspace{1cm} \Rightarrow \hspace{1cm} 
    \div \vc{B}(x) \overset{\mathrm{def}}{=} 
    \lim_{r \to 0} \left(\frac{
                \iint_{\partial V(x)} \vc{B} \cdot \d \vc{\sigma}
            }{
                V_x
            }\right).
\end{equation*}
\item[$\rot \vc{A}$.] 
    Возьмём круг $S_i (x)$ с центром в точке $x$, лежащей в плоскости, $\bot$ к $\partial_i$, для $i = 1, 2, 3$. Ориентируем $S_i(x)$ с помощью нормали, в качестве которой возьмём орт $\partial_i$, пусть $r$ -- диаметр $S_i(x)$, тогда по формуле Стокса
    \begin{equation*}
        \oint_{\partial S} \vc{A} \cdot \d \vc{s} = \iint_S (\rot \vc{A}) \cdot \d \vc{\sigma},
        \hspace{0.5cm} \Rightarrow \hspace{0.5cm}   
        (\rot \vc{A})^i = 
        \lim_{r \to 0}
        \left(
            \frac{
            \oint_{\partial S_i(x)} \vc{A} \cdot \d \vc{s}
            }{
            S_i(x)
            } 
        \right)
    \end{equation*}
\item[$\grad f$.] Посколько $\omega^1_{\grad f} (\vc{\xi}) = (\grad f \cdot \vc{\xi}) = d f(\vc{\xi}) = D_{{\xi}}f$, где $D_{{\xi}}f$ -- производная функции $f$ по вектору $\vc{\xi}$,то вектор $\grad f$ ортогонален поверхностям уровня функции $f$, указывает в каждой точке направление наиболее быстрого роста значений функции.
\end{itemize}

