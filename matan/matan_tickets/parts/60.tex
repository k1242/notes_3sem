Можно развить пример со сферой чуть более интересным образом с помощью полуримановой метрики в $\mathbb{R}^{n+1}$:
\begin{equation*}
	g = - \d x^0 \otimes \d x^0 + \d x^1 \otimes \d x^1 + \ldots + \d x^n. \otimes \d x^n.
\end{equation*}
Зададим ещё гиперповерхность $\mathbb{H}^n$: $-x_0^2 + x^2_1 + \ldots + x_n^2 = -1 $, $x_0 >0$.

Заметим, что получили ситуацию аналогичную сфере. Группа линейных изометрий $\mathbb{R}^{n+1}$ $O(1,n)$ может перевести любую точку $\mathbb{H}^n $ в любую другую и любой ортогональный базис $T_p \mathbb{H}^n$ в любой другой.

\begin{to_tas}
	Группа $O(1,n)$ не меняющая местами связные компоненты гиперболоида --- даёт всё изометрии $\mathbb{H}^n$.
\end{to_tas}

Также можно построить изометричные вложения $\mathbb{H}^2 \subset \mathbb{H}^n$ через любую точку и пару непараллельных касательных векторов в ней и убедиться, что $\mathbb{H}^n$ можно произвольно вращать, оставляя на месте $\mathbb{H}^2$ как было в случае сферы.

\begin{to_thr}
	В любой точке гиперболического пространства $\mathbb{H}^n$ для ортогонального базиса касательных векторов выполняется:
	\begin{equation*}
		R_{e_i,e_j}e_k = e_l R_{e_i, e_j} e_l = -e_k,
	\end{equation*}
	при условии, что $i = k$, $j = l$ в остальных случаях компоненты римановой кривизны равны нулю. Это означает, что гиперболическое пространство имеет постоянную кривизну $-1$.
\end{to_thr}

Аналогично случаю сферы:
\begin{equation*}
	R(X,Y,Z,T) = g(R_{X,Y}Z,T) = - g (X,T) g(Y,Z) g (Y,T).
\end{equation*}

\begin{to_tas}[Координаты Пуанкаре]
	Центральная проекция из точки $(-1,0,\ldots,0)$ на гиперплоскость $\{x_0=0\}$ задаёт систему координат на гиперболическом пространстве (со значениями в открытом шаре), в котором метрика имеет вид:
	\begin{equation*}
		g = \frac{4 \d x^1 \otimes \d x^1 + 4 \d x^2 \otimes \d x^2 + \ldots+ 4 \d x^n \otimes \d x^n}{1 - x_1^2 - \ldots - x_n^2}.
	\end{equation*}
\end{to_tas}
