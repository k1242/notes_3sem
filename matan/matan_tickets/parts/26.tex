\begin{to_def} 
    Замкнутое подмножество $M \subseteq \mathbb{R}^N$ называется \textit{вложенным многообразием размерности} $n$, если $\forall \ p \in M$ $\exists U_{\varepsilon}(p)$ и криволинейная система координат в ней, в которой включение $M \subset \mathbb{R}^N$ в пересечении с некоторой окрестностью нуля.
\end{to_def}

Яркий пример\footnote{
    Так, например, любая сфера в $\mathbb{R}^n$ является вложенным многообразием размерности $n-1$.
} -- работа с условными экстремумами. Если $M$ задаётся гладкими уравнениями $f_1 = \ldots = f_{N_n} = 0$ и дифференциалы этих уравнений линейно независимы в каждой точке $M$, то $M$ будет вложенным многообразием размерности $n$, так как определяющие его функции можно считать частью системы координат $y_{n+1}=f_1,\ldots,y_N=f_{N-n}$ в окрестности каждой точки $p \in M$, и $M$ в такой окрестности выглядит в точности как $\mathbb{R}^n \subset \mathbb{R}^N$ около нуля, а функции $y_1,\ldots,y_n$ задают систему координат в $M$, пересеченном с окрестностью $p$.


\begin{to_def} 
    Замкнутое подмножество $M \subseteq \mathbb{R}^N$ называется \textit{вложенным многообразием с краем\footnote{
        Край $\partial M$ многообразия с краем $M$ сам по себе является $(n-1)$-мерным многообразием без края.
    } размерности} $n$, если для $\forall \ p \in M \ \exists U_{\varepsilon}(p)$ и криволинейная система координат в ней, в которой включение $M \subseteq \mathbb{R}^N$ \textbf{либо} превращается в стандартное вложение $\mathbb{R}^n \subset \mathbb{R}^N$, \textbf{либо} превращается в стандартное вложение $(-\infty, 0] \times \mathbb{R}^{n-1} \subset \mathbb{R}^N$, пересеченное с окрестностью 0.
\end{to_def}


\begin{to_def} 
    Из определения $M$ понятно, что $\forall p \in M$  есть окрестность\footnote{
        Относительно открытое подмножество многообразия. 
    } в многообразие $M \cap U$ и отображение $\varphi \colon M \cap U \mapsto \mathbb{R}^n$, являющееся диффеоморфизмом между $M \cap U$ и $\varphi(M \cap U)$, которое называется \textit{координатной картой} многообразия $M$.
\end{to_def}


\begin{to_def} 
    Набор карт, районы действия которых в совокупности покрывают всё многообразие, называется \textit{атласом многообразия}. 
\end{to_def}
