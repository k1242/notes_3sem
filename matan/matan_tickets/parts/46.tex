Решая задачу Коши, можно сопоставить векторному полю с компактным носителем семейства диффеоморфизмов $\varphi_{t, t_0} \colon M \mapsto M$, удовлетворяющее соотношению
\begin{equation*}
    \frac{d }{d t} \varphi_{t,t_0} (x) = X_{\varphi_{t, t_0}, t}
    \hspace{0.5cm} \text{и} \hspace{0.5cm} 
    \varphi_{t_0, t_0} = \id_M.
\end{equation*}
Если векторное поле зависит от времени гладко, то и $\varphi_{t, t_0}$ будет зависеть от времени гладко. Смысл $\varphi_{t, t_0} (x)$ можно иначе объяснить как нахождение интегральной кривой $\gamma(t)$ векторного поля $X$ с начальным условием $\gamma(t_0) = x$ и определение $\varphi_{t, t_0} (x) = \gamma(t)$.



\begin{to_thr} 
    Для возможно зависящего от времени векторного поля $X$ на многообразии без края $M$ и соответствующих ему диффеоморфизмов $\varphi_{t, t_0}$ выполняется 
    $\varphi_{t_2, t_1} \circ \varphi_{t_1, t_0} = \varphi_{t_2, t_0}$. 
\end{to_thr}

Получается, что $\varphi_{t_1, t_0}$  имеет гладкое обратное отображение, соотвественно является диффеоморфизмом.

Рассмотрим отдельно $X$ не зависящее от времени. Тогда если $\gamma(t)$ является решением, то и $\gamma(t+s)$ тоже является решением, как функция от $t$, тогда
\begin{equation*}
    \varphi_{t_1, t_0} = \varphi_{t_1 + s, t_0+s}, \ \forall s,
\end{equation*}
то есть диффеоморфизм зависит только от разности $t_1-t_0$, тогда удобно положить $\varphi_t = \varphi_{t, 0}$, тогда $\varphi_t \circ \varphi_s = \varphi_{t+s}$. В таком случае говорят, что векторное поле порождает \textit{однопараметрическую группу диффеоморфизмов}.


\begin{to_thr}[геометрический смысл производной Ли]
    Производная Ли может быть определена с помощью соответствующей полю $X$ однопараметрической группой $\{\varphi_t\}$ диффеоморфизмов для дифференциальной форма $\alpha$ или другого векторного поля $Y$ как поточечный предел
    \begin{equation*}
        L_X \alpha = 
        \lim_{t \to 0} \left(
            \frac{\varphi^*_t \alpha - \alpha}{t} 
        \right) = \frac{d }{d t} \varphi^*_t \alpha\bigg|_{t=0}, \hspace{0.5cm}
        L_X Y = \lim_{t \to 0} 
        \frac{\varphi_t^* Y - Y}{t} = \frac{d }{d t} \varphi^*_t Y \bigg|_{t=0},{}
    \end{equation*}
    где обратный образ векторного поля при диффеоморфизме определяется как обратное отображение к прямому образу.
\end{to_thr}









