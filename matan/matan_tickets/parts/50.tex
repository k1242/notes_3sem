Рассмотрим частный случай риманова объёма --- площадь двумерной поверхности в евклидовом пространстве, то есть интеграл от $\text{vol}_g$ по этой поверхности. Заметим, что если поверхность задана параметрически и положительно ориентированные параметры на поверхности --- $(u,v)$, то для индуцированной с $\mathbb{R}^n$ римановой структуры
\begin{equation*}
	\text{vol}_g = \sqrt{|r_u'|^2 |r_v'|^2 - (r_u' \cdot r_v')^2} \d u \wedge \d v.
\end{equation*}
В трёхмерном случае эту формулу можно продолжить как
\begin{equation*}
	\text{vol}_g = \sqrt{|r_u'|^2 |r_v'|^2 - (r_u' \cdot r_v')^2} \d u \wedge \d v = \big\|[r_u' \times r_v']\big\| \d u \wedge \d v.
\end{equation*}