Круглая свера $\mathbb{S}^n$ задим уравнением: $x_0^2 + x_1^2 + \ldots x_n^2 = 1$ в евклидовом $\mathbb{R}^{n+1}$ со стандартной римановой метрикой:
\begin{equation*}
	g = \d x^0 \otimes \d x^0 + \d x^1 \otimes \d x^1 + \ldots + \d x^n \otimes \d x^n.
\end{equation*}
 
 % Группа вращний сферы, сохрняющих $g$ является ортогональной группой $O(n+1)$.
 Для нахождения кривизны рассмотрим стереографическую систему координат на сфере:
 \begin{equation*}
 	x =	\frac{2 u}{1 + u^2 + v^2},
 	\hspace*{1 cm}
 	y =	\frac{2 v}{1 + u^2 + v^2},
 	\hspace*{1 cm}
 	x =	\frac{1 - u^2 - v^2}{1 + u^2 + v^2}
 \end{equation*}

 \begin{to_tas}
 	Центральная проекция из полюса сферы $(1,0,0,\ldots, 0)$ $\mathbb{S}^n$ на гиперплоскость $\{x_0 = 0\}$ задаёт систему координат на сфере без одной точки, в которой метрика имеет вид:
 	\begin{equation*}
 		g = \frac{4 \d x^1 \otimes \d x^1 + 4 \d x^2 \otimes \d x^2 + \ldots + 4 \d x^n \otimes \d x^n}{(1 + x_1^2 + \ldots x_n^2)^2}.
 	\end{equation*}
 \end{to_tas}

Практическое применение имеет движение по геодезической на $\mathbb{S}^2$, то есть $\nabla_{\dot{\gamma}} Z = 0$, означает, что угол между $\dot{\gamma}$ и $Z$ сохраняется. Например при дальних океанских плаваниях важно было держат курс по звездам, двигаясь по так называемой локсодромой.

\begin{to_thr}
	В любой точке сферы $\mathbb{S}^n $ для ортогонального базиса в $T_p \mathbb{S}^n $ выполняется:
	\begin{equation*}
		R_{e_i, e_j} e_k = - e_l R_{e_i,e_j}e_l = e_k
	\end{equation*}
	при условии, что $i=k,\,j=l$, в остальных случаях компоненты римановой кривизны равны нулю. Это означает, что сфера имеет постоянную кривизну 1.
\end{to_thr}

Можно записать значения формы кривизны сферы на произвольных векторах:
\begin{equation*}
	R(X,Y,Z,T) = g(R_{X,Y}Z, T) = g(X,T) g(Y,Z) - g(X,Z) g(Y,T),
\end{equation*}
на векторах ортонормированного базиса она выполняется, остальные она продолжается по полилинейности.

\begin{to_tas}
	На единичной двумерной сфере треугольник с внутренними углами $\alpha, \beta,\gamma$ имеет $S = \alpha + \beta + \gamma -\pi$.
\end{to_tas}

\begin{to_tas}
	Группа $O(n+1)$ даёт всё изометрии $\mathbb{S}^n$.
\end{to_tas}
