Физический потенциал силового поля в математический терминах означает поиск $f \in C^\infty (M) \colon \d f = \alpha$ для заданной силы $\alpha \in \Omega^1(M)$

\begin{to_def} 
	Пусть $\vc{A}$ -- векторное поле в области $D \subset \mathbb{R}^n$. Функция $U \colon D \mapsto \mathbb{R}$ называется \textit{потенциалом поля} $\vc{A}$ в области $D$, если в этой области $\vc{A} = \grad U$. Поле, обладающее потенциалом, называется \textit{потенциальным полем}. 
\end{to_def}

\begin{to_thr}
	Необходимым и достаточным условием наличия потенциала у непрерывной $\alpha \in \Omega^1(M)$, для гладкого $M$, является независимость $\int_\gamma \alpha$ от выбора  кусочно-гладкой кривой $\gamma$ между двумя точками.

	Эквивалентно можно потребовать равенства нулю интегралов по всем замкнутым кусочно-гладким кривым.
	\label{thr_7.1}
\end{to_thr}

\begin{to_lem}[Необходимое условие потенциальности]
	 Необходимым условием существования потенциала у $\alpha \in \Omega^1(M)$ является $\d \alpha = 0$ (т.к. $\d (\d u) = 0$).
\end{to_lem}

\begin{to_lem} 
	В случае $\mathbb{R}^3$ по определению $d \omega^1_{\vv{A}} = \omega^2_{\rot \vv{A}}$, поэтому необходимое условие потенциальности поля $\vc{A}$ переписывается в виде $\rot \vc{A} = 0$.
\end{to_lem}

Однако этого не достаточно, так например в открытой $U = \mathbb{R}^2 \backslash \{ 0 \}$:
\begin{equation*}
	\alpha = \frac{x \d y - y \d x}{x^2 + y^2}
	\hspace*{0.5 cm} \leadsto \hspace*{0.5 cm}
	\d \alpha = 0,
	\hspace*{0.5 cm} \text{ но } \hspace*{0.5 cm}
	\oint_{S^1} \alpha = 2 \pi.
\end{equation*}


\begin{to_def} 
	Поле $\vc{A}$ называется \textit{векторным потенциалом} поля $\vc{B}$ в области $D \subset \mathbb{R}^3$, если в этой области выполняется соотношение $\vc{B} = \rot \vc{A}$. 
\end{to_def}

Это можно переписать в виде $\omega^2_{\vv{B}} = d \omega^1_{\vv{A}}$, тогда $\omega^3_{\div \vv{B}} = d \omega_{\vv{B}}^2 = d^2 \omega^1_{\vv{A}} = 0$, то есть необходимое условие $\div \vc{B} = 0$, принято такое поле называть \textit{соленоидальным}. 






