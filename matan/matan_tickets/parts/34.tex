Физический потенциал силового поля в математический терминах означает поиск $f \in C^\infty (M) \colon \d f = \alpha$ для заданной силы $\alpha \in \Omega^1(M)$

\begin{to_thr}
	Необходимым и достаточным условием наличия потенциала у непрерывной $\alpha \in \Omega^1(M)$, для гладкого $M$, является независимость $\int_\gamma \alpha$ от выбора между двумя точками кусочно-гладкой кривой $\gamma$.

	Эквивалентно можно потребовать равенства нулю интегралов по всем замкнутым кусочно-гладким кривым.
	\label{thr_7.1}
\end{to_thr}

Удобным на практике необходимым условием существования потенциала у $\alpha \in \Omega^1(M)$ является $\d \alpha = 0$ (т.к. $\d (\d u) = 0$).
Однако этого не достаточно, так например в открытой $U = \mathbb{R}^2 \backslash \{ 0 \}$:
\begin{equation*}
	\alpha = \frac{x \d y - y \d x}{x^2 + y^2}
	\hspace*{0.5 cm} \leadsto \hspace*{0.5 cm}
	\d \alpha = 0,
	\hspace*{0.5 cm} \text{ но } \hspace*{0.5 cm}
	\oint_{S^1} \alpha = 2 \pi.
\end{equation*}

Далее в качестве упражнений оставлены следующие важные замечания:

\begin{to_tas}[Порядок точки относительно кривой]
Для замкнутой кусочно-гладкой $\gamma \in \mathbb{R}^2$, не проходящей через начало координат определим порядок начала координат относительно кривой:
\begin{equation*}
	w (\gamma, 0) - \frac{1}{2 \pi} \int_{\gamma} \frac{x \d y - y \d x}{x^2 + y^2},
\end{equation*}
и он не меняется при непрерывных деформациях кривой, при которых она не проходит через начало координат.
\end{to_tas}

\begin{to_tas}
	Порядок начала координат относительно кривой является целым.
\end{to_tas}

\begin{to_tas}
	Порядок начала координат относительно не проходящей через него нечётной кривой является нечётным числом. ($\gamma \colon \mathbb{S}^1 \rightarrow \mathbb{R}^2, \; \gamma(-u) = - \gamma(u)$).
\end{to_tas}

\begin{to_tas}
	Для замкнутой кривой на плоскости с всюду не нулевой скоростью $\int k(s) \d s = 2 \pi N, \; N \in \mathbb{Z}$.
\end{to_tas}

\begin{to_tas}[Лемма Жордана]
	Замкнутая кусочно-гладкая кривая $\gamma \subset \mathbb{R}^2$ без самопересечений делит плоскость на две связные части внутреннюю и внешнюю (можно усложнить и сформулировать для непрерывных кривых).
\end{to_tas}