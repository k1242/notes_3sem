
\begin{to_def} 
    \textit{Векторным полем} на открытом множестве $U \subseteq \mathbb{R}^n$ называется выбор касательного вектора $X(p) \in T_p U$ для каждой точки $p \in U$, гладко\footnote{
        Гладкая зависимость понимается в смысле гладкой зависимости координат векторного поля $X_i(p)$ в точке $p$.
    } зависящий от $p$.  
\end{to_def}


\begin{to_lem} 
    Для открытого $U \subseteq \mathbb{R}^n$ всякое $\mathbb{R}$-линейное отображение $X \colon C^{\infty} \mapsto C^{\infty}(U)$, удовлетворяющее правилу Лейбница $X(fg) = X(f) g + f x(g)$ задаётся векторным полем на $U$. 
\end{to_lem}



\begin{to_def} 
    Пусть есть вектор $X \in T_pU,$ $q = \varphi(p)$, тогда \textit{прямой\footnote{
        Производную отображения $\varphi$ в точке $p$ можно определить как $\varphi_*\colon T_p U \mapsto  T_q V$ при $q = \varphi(p)$. Иначе можем обозначать, как $F \varphi_p$.
    } образ вектора} $\varphi_*(X)$ определяется по формуле
    \begin{equation*}
        \varphi_* (X) f = X(f \circ \varphi),
        \hspace{1cm}\mapsto\  \left(\varphi_*X\right)^i = \frac{\partial \varphi^i}{\partial x^j} X^j 
        \hspace{0.25cm} \Leftrightarrow \hspace{0.25cm} 
        \text{\red{переписать в матричном виде}}
        .
    \end{equation*}
\end{to_def}

\begin{to_def} 
    Отображение $\varphi \colon U \mapsto  V$ задаёт \textit{гомоморфизм алгебр} (операция, сохраняющая умножение, сложение, и переводящая $\const$ в $\const$): 
    $\varphi^* \colon C^{\infty}(V) \mapsto C^{\infty}(U)$ по формуле
    \begin{equation*}
        \varphi^* (f) = f \circ \varphi.
    \end{equation*}
    вектор даёт дифференцирование алгебры $X \colon C^{\infty}(U) \mapsto \mathbb{R}$, и тогда $\varphi_* X = X \circ \varphi^*$ тоже дифференцирование алгебры. 
\end{to_def}

