\begin{to_def}[Обратный образ] 
    Для всякого гладкого отображения $\varphi \colon U \mapsto V$ между открытими подмножествами евклидовых пространств определено отображение пространств дифференциальных форм $\varphi^* \colon \Omega^k(V) \mapsto \Omega^k (U)$, действующее по формуле\footnote{
        Важно заметить, что если левая часть вычисляется в точке $p \in U$, то правая в $\varphi(p)$.
    }
\begin{equation*}
    \varphi^* \alpha (X_1, \ldots, X_k) = \alpha(\varphi_* X_1,\ldots,\varphi_* X_k).
\end{equation*}
Для функции $f \in C^{\infty}(V) = \Omega^0 (V)$ оказывается $\varphi^* f = f \circ \varphi$, что совпадает с замены переменных в функции. Для форм первой степени $\alpha \circ \varphi_*$, где $\alpha|_{f(p)}$, а $\varphi_*|_p$. \red{Чего?}
\end{to_def}

\begin{to_lem} 
    Взятие обратного образа диф-форм коммутирует с внешним умножением и внешним дифференцированием.  
\end{to_lem}

% \begin{proof}[$\triangle$]
%     \begin{equation*}
%         d(\varphi^* f) = d (f \circ \varphi) = df \circ D\varphi = df \circ \varphi_* = \varphi^*(df).
%     \end{equation*}
% \end{proof}

Таким образом взятие обратного образа происходит формально подстановкой\footnote{
    \red{Было бы здорово посмотреть на задачи 6.96 и 6.97.}
} выражений новых переменных через старые в коэффициенты формы и в дифференциалы новых переменных.


