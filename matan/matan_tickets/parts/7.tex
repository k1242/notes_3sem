
\begin{to_thr}[Теорема о расщеплении отображения на элементарные]
     Если отображение $\varphi$ непрерывно дифференцируемо в окрестности точки $p \in \mathbb{R}^n$ и имеет обратимый $D \varphi_x$, то его можно представить в виде композиции перестановки координат, отображений координат и элементарных отображений, непрерывно дифференцируемо и возрастающим образом меняющих только одну координату $y_i = \psi_i (x_1, \ldots, x_n)$.
\end{to_thr}

\begin{to_thr} 
    Теоремы об обратном отображении, о неявной функции и о расщеплении отображения дают отображения класса $C^k$ при $k \geq 1$, если исходные отображени были класса $C^k$. 
\end{to_thr}