Пусть $\varphi\colon \mathbb{R}^n \rightarrow \mathbb{R} $, неотрицательная $\varphi\in C^\infty $, $\varphi \neq 0$ при $|x| \leq 1$ и пусть $\int_{\mathbb{R}^n} \varphi(x) \d x = 1$.
Положим  $\varphi_k (x) = k^n \varphi(k x) $, у которых так же будут $\int = 1$ и которые $ \varphi_k \neq 0$ при $|x| \leq 1/k$.

\begin{to_thr}
	Для непрерывной $f \colon \mathbb{R}^n \rightarrow \mathbb{R} $ определим свёртки:
	\begin{equation*}
		f_k(x) = \int_{\mathbb{R}^n} f(x-t)\varphi_k \d t = \int_{\mathbb{R}^n} f(t) \varphi_k (x -t) \d t \hspace*{0.5 cm} \leadsto \hspace*{0.5 cm} f_k \in C^\infty, \; f_k \rightarrow f \text{ равномерно на компактах в } \mathbb{R}^n.
	\end{equation*}
	\label{thr_6.21}
\end{to_thr}

\begin{to_thr}
	Если $f$ имеет непр. производные до $m$-го порядка, то производные $f_k$ до $m$-го порядка равномерно сходятся на компактах к соответствующим $f'$.
	\label{thr_6.22}
\end{to_thr}

\begin{to_thr}
	Пусть $f\colon \mathbb{R}^n \rightarrow \mathbb{R}$ и $f \in \L_{c}$. Тогда свёртки $f*\varphi_k$ c функциями из теоремы \ref{thr_6.21} сколь угодно близко приближают $f$ в среднем.
	\label{thr_6.23}
\end{to_thr}