% 6.4. Непрерывно дифференцируемые отображения и криволинейные системы координат

\begin{to_def} 
    \textit{Криволинейная замена координат} --- бесконечно гладкое отображение $\varphi \colon U \mapsto  V$ такое, что $\varphi^{-1}$ определено и тоже бесконечно гладко. 
\end{to_def}

\begin{to_lem} 
    Пусть открытое множество $U \subset \mathbb{R}^n $ выпукло. Для непрерывно дифференцируемого отображения $\varphi \colon U \to \mathbb{R}^m $ найдётся непрерывное отображение $A \colon U \times U \mapsto  \mathcal L (\mathbb{R}^n, \mathbb{R}^m)$, такое что $\forall x', \ x'' \in U$ 
\begin{equation*}
     \varphi(x'') - \varphi(x') = A(x', x'')(x'' - x')
 \end{equation*} 
 и $A(x, x) = D \varphi_x$.  
\end{to_lem}

% \begin{to_lem} 
%     Для всякого линейного отображения $A \colon \mathbb{R}^n \mapsto \mathbb{R}^m$ найдётся число $\|A\|$ такое что для все $x \in \mathbb{R}^n$ $|Ax| \leq \|A\| \cdot |x|$. 
% \end{to_lem}

\begin{to_thr}[Теорема об обратном отображении]
     \textbf{Если} отображение $\varphi \colon U \mapsto \mathbb{R}^n$ непрерывно дифференцируемо в окрестности точки $x$ \textbf{и} его дифференциал $D\varphi_x$ являетсяя невырожденным линейным преобразованием, \textbf{то} это отображение взаимно однозначно отображает некоторую окрестность $V \ni x$ на окрестность $W \ni y$, где $y = \varphi(x)$. Обратное отображение $\varphi^{-1} \colon W \to V$ тоже непрерывно дифференцируемо. 
\end{to_thr}


\begin{to_def} 
    \textit{Криволинейной системой координат} в окрестности точки $p \in \mathbb{R}^n$ называется набор таких функций, которые явяются координатами гладкого отображения окрестности $p$ на некоторое открытое множество в $\mathbb{R}^n$ с гладким обратным\footnote{
        По теореме об обратном отображении для проверки системы преобразования достаточно проверить невырожденность $\left(
    \partial y_i / \partial x_j
    \right)$ в точке $p$, или линейную независимость $dy^i$ в точке $p$.
    } отображением.
\end{to_def}

