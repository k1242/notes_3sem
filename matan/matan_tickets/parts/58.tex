<<Дело может происходить в полуримановом многообразии с сигнатурой $1+3$.>> В добавок зададим ещё диф-форму $\alpha \in \Omega^1 (M)$ и рассмотрим вопрос поиска экстремальных кривых естественного функционала
\begin{equation*}
    A(\gamma) = \int_{t_0}^{t_1} \left(
        \frac{1}{2} g(\dot{\gamma}, \dot{\gamma}) + \alpha(\dot{\gamma})
    \right) \d t.
\end{equation*}
Здесь $\alpha$ -- \textit{1-форма потенциала}, $d \alpha$ -- \textit{2-форма электромагнитного поля}.

Можно проверить, что уравнение предположительно экстремальной кривой будет отличаться от уравнения геодезической слагаемым, зависящим от $\alpha$, как $\nabla_{\dot{\gamma}} \dot{\gamma} = \ddot{\gamma} =
    (i_{\dot{\gamma} \d \alpha})^\sharp,
$.
Это уравнение, с точностью до единиц измерения массы и заряда, в полуримановой метрике СТО соотвествует движению заряда в ЭМ поле $F = d \alpha$.

Само по себе поле $F$ удовлетворяет достаточно простой системе уравнений. Если добавить понятие электрического тока как формы $j \in \Omega^3(M)$, описывающее усредненное движение большого количества зарядов, то можно написать уравнения
\begin{equation*}
    d F = 0, \hspace{1cm} 
    d(*F) = j,
\end{equation*}
где первое выражает замкнутость $F$, а второе связь $F$ с током.

Условие $j = d(*F)$ и формула Стокса гарантируют, что интеграл тока по компактным трёхмерным многообразиям без края равен нуля, что называется сохранением заряда.




