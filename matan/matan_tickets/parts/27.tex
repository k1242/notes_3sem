\begin{to_def}[Абстрактное определение многообразия]
     \textit{Гладкое $n$-мерное многообразие} $M$ -- хаусдорфово топологическое пространство со счётной базой, покрытое открытыми картами $U_i$ так, что для каждой карты задано отображение $\varphi_i \colon U_i \mapsto \mathbb{R}^n$ являющееся гомеоморфизм на открытое подмножество $\mathbb{R}^n$, и для пары таких отображений (\textit{карт}) $\varphi_i$ и $\varphi_j$ композиция $\varphi_i \circ \varphi_j^{-1}$ является диффеоморфизмом на своей естественной области определения.
\end{to_def}

\begin{to_def} 
     \textit{Гладкое $n$-мерное многообразие с краем} $M$ отличается тем, что некоторые из карт являются не такими, как описано выше, а являются гомеоморфизмами на относительно открытое подмножество полупространства $(-\infty, 0] \times \mathbb{R}^{n-1})$, в котором точки из $\{0\}\times \mathbb{R}^{n-1}$ образуют \textit{край} в этой карте, а замены координат $\varphi_i \circ \varphi_j^{-1}$ переводят край в одной карте в край в другой карте.
\end{to_def}

