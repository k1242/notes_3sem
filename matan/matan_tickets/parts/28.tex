\begin{to_def} 
    \textit{Дифференциальной формой} $\alpha \in \Omega^k (M)$ \textit{на многообразии} мы будем называть набор диф-форм $\alpha_i$ на образах карт $\varphi_i \colon U_i \mapsto \mathbb{R}^n$, которые обладают свойством
    \begin{equation*}
        (\varphi_i \circ \varphi_j^{-1})^* \alpha_i = \alpha_j
    \end{equation*}
    на естественной области определения $\varphi_i \circ \varphi_j^{-1}$ в $\mathbb{R}^n$, для всяких двух карт $\varphi_i$, $\varphi_j$. 
\end{to_def}

Можно неформально сказать, что глобальная форма собирается из локальных форм, если одна локальная форма переходит в другую при замене одной карты на другую, причём делает это именно так, как это происходит в раннее изученном случае, когда многообразие является областью в $\mathbb{R}^n$.

Достаточно рассматривать набор карт, покрывающих многообразие $M$. Для в любой другой координатной карте $\varphi$ соответствующее представление $\alpha \in \Omega^k (\varphi(U))$ будет выглядеть как $\alpha = (\varphi_i \circ \varphi^{-1})^* \alpha_i$ на множестве $\varphi(U \cap U_i)$. 
По определению диф-формы и \eqref{task_6.97} оказывается 
\begin{equation*}
    (\varphi_j \circ \varphi^{-1})^* \alpha_j = 
    (\varphi_j \circ \varphi^{-1})^* (\varphi_i \circ \varphi_j)^* \alpha_i = (\varphi_i \circ \varphi^{-1})^* \alpha_i.
\end{equation*}
на $\varphi(U \cap U_i \cap U_j)$. 


В силу установленной ранее независимости от выбор системы криволинейных координат операции $\wedge$ и $d$ верно определены для форм на многообразиях.

Простой и естественный способ получить диф-форму на $M \subset \mathbb{R}^N$ -- ограничить какую-то диф-форму из евклидова пространства, или из окрестности $M$, или положить $\alpha_i = (\varphi_i^{-1})^* \alpha$ для $\alpha \in \Omega^k (\mathbb{R}^n)$. 

Касательный вектор к вложенному многообразию $M \subset \mathbb{R}^N$ также можно рассматривать как касательный вектор к $\mathbb{R}^N$, так как любой вектор $X$ в некоторой точке образа карты $\varphi_i$ можно перенести в $\mathbb{R}^n$ отображением $(\varphi_i^{-1})_*$.



Для гладкого отображения многообразий корректно определена производная $Df_p \colon T_p M \mapsto T_{f(p)}N$ в каждой точке $p \in M$, которую мы также называли прямым образом вектора $\varphi_*$, что и является линейным отображением касательных пространств в точке.

Отображение обратного образа диф-форм $f^* \colon \Omega^k(N) \mapsto \Omega^k(M)$ как
\begin{equation*}
    f^* \alpha|_p (X_1, \ldots, X_k) = \alpha|_{f(p)} (f_* X_1, \ldots, f_* X_k)
\end{equation*}
Для векторных полей $f_* = (f^{-1})^*$.
