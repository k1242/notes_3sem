\subsubsection*{Кусочек курса диф-уров}


Для диф-уров, при непрерывных первых производных $f$, в области $U \subseteq \mathbb{R}^n$, 
\begin{equation*}
    \dot{x} = f(x(t), t), \hspace{0.5cm}  
    f \colon U \times (t_0 - \varepsilon, t_0 + \varepsilon) \mapsto U \subseteq \mathbb{R}^n
\end{equation*}
\begin{to_thr}[Существование и единственность решений диф-уравнений]
     Если $f$ непрерывно по всем аргументам и удовлетворяет условию Липшица по $x$ в окрестности $x(t_0)$, то решение с данным начальным условием существует и единственно в некотором диапазоне $t \in (t_0 - \varepsilon, t_0 + \varepsilon)$.
\end{to_thr}


\begin{to_thr}[Существование и единственность решений линейного уравнение]
     Решение линейного уравнения $\dot{x} = A(t) x(t) + b(t)$ с непрерывно зависящими от времени линейным оператором $A(t)$ и вектором $b(t)$, при любом начальном условии $x(t_0)$ существует и единственно на любом промежутку времени, на котором $A$ и $b$ непрерывны.
\end{to_thr}


\begin{to_thr}[Непрерывная зависимость решений диф-уравнений от параметров и н.у.] 
    Решим задачу Коши с н.у. $x(t_0)=a \in U$.
    \textbf{Если} $f(x, t, p) (= \dot{x})$ непрерывна по всем аргументам, удовлетворяет условию Липшица по $x$ в окрестности $x(t_0)$ 
    \underline{равномерно} по $t \in (t_0 - \varepsilon_0, t_0 + \varepsilon_0)$ и $p \in P$ (некоторое метрическое пространство параметров), а также $f$ равномерно ограничена $\forall p \in P$, \textbf{то} решение существует и единственно в $\forall t \in (t_0 - \varepsilon, t_0 + \varepsilon)$, при значениях $a$ в некоторой окрестности $U(a_0)$ и $\forall p \in P$. Решение \underline{непрерывно зависит} от $a \in U(a_0)$ и $p \in P$.
\end{to_thr}


\begin{to_thr}[Дифференцируемая зависимость решений дифференциальных уравнений от параметров и н.у.]
     \textbf{Пусть} правая часть диф-уравнения $f(x, t, p)$ непрерывна по времени в $(t_0 - \varepsilon_0, t_0 + \varepsilon_0)$, а её производные по $x \in U$ и параметру $p$ непрерывно зависят от $x, t, p$ в некотором открытом множестве $U \times (t_0 - \varepsilon_0, t_0 + \varepsilon_0) \times P$.
     \textbf{Тогда} решение задачи Коши непрерывно дифференцируемым образом зависит от начальных условий $x_0$ и параметра $p$ при значениях времени в некотором диапазоне $(t_0 - \varepsilon, t_0 + \varepsilon), \ \varepsilon > 0$.
\end{to_thr}


\begin{to_con} 
    \textbf{Если} правая часть диф-уравнения непрерывна по времени и $m$ раз непрерывно дифференцируемо зависит от $x$ и параметров, а также её производные порядка не более $m$ по $x$ и параметрам непрерывно зависят от времени, \textbf{то} решение уравнения $m$ раз непрерывно дифференцируемо зависит от параметров и н.у.  
\end{to_con}

\subsubsection*{Собственно, сам билет}


\begin{to_def} 
    Дифференциальное уравнение на многообразии $M$ и точки $p \in M$, нахождение такой \textit{интегральной кривой} $\gamma \colon (a, b) \mapsto M$, для которой
    \begin{equation*}
        \gamma'(t) = X_{\gamma(t)},
    \end{equation*}
    и $\gamma(t_0) = p$ при данном $t_0 \in (a, b)$.
\end{to_def}


\begin{to_thr}[Выпрямление векторного поля]
     Если векторное поле $X$ в точке $p \in M$ не равно нулю, то в некоторой криволинейной системе координат $x_1, \ldots, x_n$ в окрестности точки $p$ оно может быть приведено к виду $X = \partial_1$. 
\end{to_thr}

\begin{to_lem} 
    Пусть $X$ -- возможно зависящее от времени векторное поле на многообразии без края $M$. Тогда для всякого момента времени $t_0$ и точки $p \in M$ существует интегральная кривая $\gamma$, определенная на некотором интервале (не обязательно конечном) $(T_1, T_2) \ni t_0$ и удовлетворяющая условию $\gamma(t_0) = x_0$, максимальная в том смысле, что любая другая интегральная кривая $\tilde \gamma$ векторного поля $X$, удовлетворяющая тому же условию $\tilde \gamma(t_0) = p$, является ограничением $\gamma$ на некоторый интервал $(\tilde T_1, \tilde T_2) \subseteq (T_1, T_2)$.
\end{to_lem}


\begin{to_thr}
     Пусть $X$ -- возможно зависящее от времени векторное поле на многообразии без края $M$, а $\gamma \colon (T_1, T_2) \mapsto M$ -- его максимальная интегральная кривая, не продолжающаяся за пределы интервала $(T_1, T_2)$. Без ограничения общности, если $T_2$ конечно, то кривая $\gamma$ покидает любой компакт при $t \mapsto T_2 - 0$ в следующем смысле: для всякого компактного $K \subseteq M$ найдётся $T_K \in (T_1, T_2)$, такое что $\gamma(t) \notin K$ при $t > T_K$.
\end{to_thr}

\begin{to_con} 
    Для возможно зависящего от времени векторного поля $X$ с компактным носителем на многообразии без края $M$ все интегральные кривые продолжаются по времени неограниченно в обе стороны. 
\end{to_con}