Для диф-уров, при непрерывных первых производных $f$, в области $U \subseteq \mathbb{R}^n$, 
\begin{equation*}
    \dot{x} = f(x(t), t), \hspace{0.5cm}  
    f \colon U \times (t_0 - \varepsilon, t_0 + \varepsilon) \mapsto U \subseteq \mathbb{R}^n
\end{equation*}
\begin{to_thr}[Существование и единственность решений диф-уравнений]
     Если $f$ непрерывно по всем аргументам и удовлетворяет условию Липшица по $x$ в окрестности $x(t_0)$, то решение с данным начальным условием существует и единственно в некотором диапазное $t \in (t_0 - \varepsilon, t_0 + \varepsilon)$.
\end{to_thr}

\begin{to_def} 
    Дифференциальое уравнение на многообразии $M$ и точки $p \in M$, нахождение такой \textit{интегральной кривой} $\gamma \colon (a, b) \mapsto M$, для которой
    \begin{equation*}
        \gamma'(t) = X_{\gamma(t)},
    \end{equation*}
    и $\gamma(t_0) = p$ при данном $t_0 \in (a, b)$.
\end{to_def}


\begin{to_thr}[Существование и единственность решений линейного уравнение]
     Решение линейного уравнения $\dot{x} = A(t) x(t) + b(t)$ с непрерывно зависящими от времени линейным оператором $A(t)$ и вектором $b(t)$, при любом начальном условии $x(t_0)$ существует и единственно на любом промежутку времени, на котором $A$ и $b$ непрерывны.
\end{to_thr}


\begin{to_thr}[Непрерывная зависимость решений диф-уравнений от параметров и н.у.] 
    Решим задачу Коши с н.у. $x(t_0)=a \in U$.
    \textbf{Если} $f(x, t, p) (= \dot{x})$ непрерывна по всем аргументам, удовлетворяет условию Липшица по $x$ в окрестности $x(t_0)$ 
    \underline{равномерно} по $t \in (t_0 - \varepsilon_0, t_0 + \varepsilon_0)$ и $p \in P$ (некоторое метрическое пространство параметров), а также $f$ равномерно ограничена $\forall p \in P$, \textbf{то} решение существует и единственно в $\forall t \in (t_0 - \varepsilon, t_0 + \varepsilon)$, при значениях $a$ в некоторой окрестности $U(a_0)$ и $\forall p \in P$. Решение \underline{непрерывно зависит} от $a \in U(a_0)$ и $p \in P$.
\end{to_thr}


f


