\begin{to_def} 
    \textit{Длиной кривой} в римановом многообразии назовём
\begin{equation*}
     l(\gamma) = \int_{t_0}^{t_1} \sqrt{g(\dot{\gamma}, \dot{\gamma})}
     \d t,
 \end{equation*} 
  что является частным случаем риманова объёма, если мы рассматриваем кривую как вложенное в $M$ одномерное многообразие, с индуцированной римановой структурой.
\end{to_def}

\begin{to_def} 
    Определим \textit{внутреннюю метрику многообразия}, как нижнюю грань длин кривых, соединяющих две данные точки. Таким образом оправдано использование термина \textit{риманова метрика}, хотя $g$ метрикой не является.
\end{to_def}

\begin{to_def} 
    Так как выбор параметризации не меняет длины кривой, а зафиксировать параметризацию хочется, то естественно ввести \textit{функционал действия}
    \begin{equation*}
         A(\gamma) = \frac{1}{2} \int_{t_0}^{t_1} g(\dot{\gamma}, \dot{\gamma}) \d t.
     \end{equation*} 
\end{to_def}



\begin{to_thr} 
    На римановом многообразии, среди всех параметризацией одной и той же кривой отрезком $[t_0, t_1]$ минимальное действие достигается на параметризации с постоянной скоростью и в этом случае выполняется равенство
    \begin{equation*}
         A(\gamma) = \frac{1}{2} \frac{l(\gamma)^2}{(t_1 - t_0)}.
     \end{equation*} 
\end{to_thr}
