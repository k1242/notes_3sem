\begin{to_def}
	Точка $p \in M$ называется регулярной точкой гладкого отображения $f \colon M \rightarrow N$, если $D f_{p}$ сюръективно. Иначе называется критической точкой.

	Если для точки $q \in N$ найдётся критическая $p \in M$, такая что $q = f(p)$, то $q$ называется критическим значением отображения. Некритическое значение $q \in N$ назвается регулярным значением.
\end{to_def}

\begin{to_def}
	Множество лебеговой меры нуль в многообразии $M$ --- это множество, пересечение которого с областью определения любой координатной карты $M$ в образе карты имеет меру нуль.
\end{to_def}

Формула меры образа (\ref{con_6.109}) при гладкой замене координат тогда показывает, что множество меры нуль в евкл. пр-ве переходит в множество меры нуль. Значит свойство подмножества $X$ многообразия $M$ иметь лебегову меру нуль достаточно проверить не во всех возможных картах, а лишь в любом атласе, который покрывает $M$.

\begin{to_thr}[Теорема Сарда]
	Для бесконечно гладкого отображения $f \colon M \rightarrow N$ множество критических значений $f$ имеет меру нуль в $N$, то есть почти все точки $N$ являются значениями регулярными значениями $f$.
\end{to_thr}