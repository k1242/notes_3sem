\begin{to_thr}[Теорема Брауэра о неподвижной токе]
	Пусть $B \subset \mathbb{R}^n$ --- некторый замкрнутый шар. Всякое непрерывное отображения в себя $f \colon B \rightarrow B$ имеет неподвижную точку, то есть точку, в которой $f(x) = x$.
\end{to_thr}

\begin{to_thr}[Отсутствие ретракции шара на его границу]
	Пусть сфера $S^{n-1}$ рассматривается как край шара $B^n$. Не существует непрерывного отображения $f \colon B^n \rightarrow S^{n-1}$, такого что $f |_{S^{n-1}} = id_{S^{n-1}}$.
\end{to_thr}