
Исследование вариации геодезических, то есть производных семейств геодезических по параметру, естественным образом приводит к понятию \textit{кривизны Римана}. 

\begin{to_def} 
    Опрелим \textit{тензор кривизны Римана} ($R \colon X, Y, Z \mapsto T_p M$), как
\begin{equation*}
    R_{X, Y} Z = \nabla_X \nabla_Y Z - \nabla_Y \nabla_X Z - \nabla_{[X, Y]} Z,
    \hspace{1cm} 
    R_{\partial_i, \partial_j} \partial_k = R_{ijk}^l \partial_l
    \hspace{0.25cm} \Rightarrow \hspace{0.25cm} 
    (R_{X,Y} Z)^l = R^l_{ijk} X^i Y^j Z^k.
\end{equation*}
\end{to_def}

\begin{to_thr} 
    Выражение тензора кривизны Римана является тензором в том смысле, что оно линейно $X, \ Y, \ Z$ и при умножении на функцию $f$ векторного поля $X, \ Y,\ Z$ всё выражение просто умножается на $f$.
\end{to_thr}

Важно, что тензор кривизны Римана является поточечной операцией, гладко зависящей от точки. Геометрический смысл тензора кривизны можно понимать так: если два векторных поля $X$ и $Y$ коммутируют, то ковариантная производная третьего векторного поля по этим двум зависит от порядка дифференцирования, и это зависимость как раз выражается тензором кривизны Римана. \red{Написатьь про перенос вектора вдоль параллелограма.}

Другая интепретация -- пусть есть замкнутая $\gamma$, край двумерной поверхности $S$ в многообразии $M$ с координатами $u, v$. Тогда с помощью формулы Грина, считая $X = \partial_u, \ Y = \partial_v$
\begin{equation*}
    \int_\gamma g(\nabla_{\dot{\gamma}} Z, T) \d t = 
    \int_S g(R_{X, Y} Z, T) \d u \wedge d v + \int_S
    (
        g(\nabla_Y Z, \nabla_X T) - g(\nabla_X Z, \nabla_Y T)
    ) \d u \wedge d v.
\end{equation*}
\begin{to_def} 
    \textit{Формой кривизны Римана} называется $R(X, Y, Z, T) = g(R_{X, Y} Z, T)$.
\end{to_def}

\begin{to_def} 
    Важна билинейная форма (тензор ранга $(0, 2)$), которая получается в качестве свёртки тензора кривизны по паре переменных. Выбрав взаимные базисы в касательном пространсвте $\{e_i\}$ и $\{f_i\}$, так что $g(e_i, f_i) = \delta_{ij}$, опредлим \textit{тензор кривизны Риччи}
    \begin{equation*}
        \text{Ric}\, (Y, Z) = \delta^{ij} R(e_i, Y, Z, f_j) = \delta^{ij} g(R_{e_i, Y} Z, f_j),
    \end{equation*}
    при чем тезнор Риччи не зависит от выбора $\{e_i\}$ и $\{f_i\}$ (свертка тензора инвариантна).
\end{to_def}

Геометрический смысл тензора Риччи в том, что отношение риманова объема в окрестности точки $p$ к объёму, пришедшему из экспоненциального (\red{?}) отображения с касательного пространства, в квадратичном приближении в точке $x \in U(p)$ равно
\begin{equation*}
    1 - \frac{1}{6} \text{Ric}\, \left(
        \exp^{-1}_p x, \exp^{-1}_p x
    \right) + o(\rho(x,p)^2).
\end{equation*}
