\begin{to_def}
	Римановой структурой на гладком $M$ называется задание квадратичной формы $g_p >0$ на касательном пространстве $T_p M$, гладко зависящее от точки $p$. Полуриманова структура --- это задание невырожденной, но не обязательно положительно определённой квадратичной формы.
\end{to_def}

Будем отождествлять квадратичную форму с симметричным скалярным произведением:
\begin{equation*}
	g_{i,j} = g \left(\frac{\partial}{\partial x_i}, \frac{\partial}{\partial x_j}\right),
	\hspace*{1 cm}
	g = g_{i j} \d x^i \otimes \d x^j.
\end{equation*}
На $\forall M$ (гладком) $\exists g$. Достаточно взять локальное конечное разбиение единицы $\{\rho_i\}$, подчинённое картам $\{U_i\}$, в каждой карте построить $g_i$ (например: стандартная риманова структура $\delta_{i j} \d x^i \otimes \d x^j $) и положить $g = \sum_i \rho_i g_i$.
Эта сумма будет локально конечна и в любой точке будет давать $g>0$, так как сумма неотрицательно определённых форм, хотя бы одна из которых положительно определена, будет положительно определена.

На вложенном $M \subset \mathbb{R}^N$ можно просто ограничить стандартную риманову структуру с евклидова пространства на его подмногообразие $M$:
\begin{equation*}
	 i\colon M \mapsto \mathbb{R}^n, \hspace*{1 cm}  
	 g = i^* g_0.
\end{equation*}

В таком случае, если локальные координаты $M$ --- это $u_{1}, \ldots, u_n $, то риманова структура задаётся в координатах как
\begin{equation}
	g_{i j} = g_0\left(\frac{\partial r}{\partial u^i} \cdot \frac{\partial r}{\partial u^j}\right) = \left(\frac{\partial r}{\partial u^i} \cdot \frac{\partial r}{\partial u^j}\right),	
\end{equation}
где $g_{0}$, $(\cdot)$ --- евклидово скалярное произведение.
