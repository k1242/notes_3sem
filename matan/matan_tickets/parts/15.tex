\begin{to_def} 
    \textit{Дифференциальная 1-форма} -- это ковекторное поле. Иначе, элемент двойственного пространства $(T_p U)^* \equiv T_p^* U$, линейная форма на касательнм пространстве, гладко зависящая от $p$. 
    \textit{Дифференциал} функции $f$ от векторного поля $X$ это $df (X) \overset{\mathrm{def}}{=} X f$.
\end{to_def}

Дифференциалы $dx_1, \ldots, dx_n$ дают базис $T_p^* U$, двойственный к $\partial_1, \ldots, \partial_n$, в смысле $dx^i \partial_j = \delta_j^i$. По этому базису можно разложить любую форму в точке, а применяя это $\forall p \in U \subseteq \mathbb{R}^n$ видим, что $\omega^1 = \alpha_i dx^i$.


При замене координат компоненты $\omega^1$ преобразуются как дифференциалы функции, то есть 
\begin{equation*}
    \alpha = \alpha_j dx^j = \tilde \alpha_i dy^i = 
    \underbrace{\tilde \alpha_i \partial_j y^i }_{\alpha_j}
    dx^j, \hspace{0.5cm} \Rightarrow \hspace{0.5cm} 
    \alpha_j = \tilde \alpha_i \partial_j y^i
    \hspace{0.25cm} \Leftrightarrow \hspace{0.25cm} 
    \text{\red{переписать в матричном виде}}.
\end{equation*}



% Дописать что происходит при замене базиса

% \red{Далее некоторые рассуждения из НМУ}. Заметим, что $\frac{\partial }{\partial x^1}, \ldots, \frac{\partial }{\partial x^n}$ -- базис в каждой точке. Рассмотрим теперь $f = x^i$ и $X = \frac{\partial }{\partial x^j} $, тогда
% \begin{equation}
%     d x^i \left(\frac{\partial }{\partial x^j} \right)
%     =
%     \frac{\partial x^i}{\partial x^j} = \delta^i_j.
% \end{equation}
% Из этого следует, что $dx_1,\ldots,dx_n$ -- двойственный к $\frac{\partial }{\partial x^1}, \ldots, \frac{\partial }{\partial x^n}$ базис в $V^*$.
% Тогда в этом базисе
% $$
%     d f = \omega_i \d x^i.
% $$
% Заметим, что
% $$
% \underbrace{\omega_i \d x^i }_{df}
%     \left(
%         \frac{\partial }{\partial x^j} 
%     \right) = \omega_i \delta^i_j = \omega_j,
%     \hspace{0.5cm} \Rightarrow \hspace{0.5cm} 
%     \omega_j = df \left(\frac{\partial }{\partial x^j} \right) =
%     \frac{\partial f}{\partial x^j}.
% $$
% Тогда
% \begin{equation}
%     df = \frac{\partial f}{\partial x^i} \d x^i.
% \end{equation}
% Получается ковектор $df$ расписывается по базису
% $dx^i$
%  двойственного пространства с координатами $\partial f / \partial x^i$.

% А для общей 1-формы
% $$
%     \omega = \omega_i \d x^i,
% $$
% где $\omega_1, \ldots, \omega_n$ -- координаты $\omega$ в локальной системе координат.

% \begin{to_def} 
%     $\omega$  гладкая, если $\forall X$, где $X$ -- гладкое поле, верно, что
%     $\omega(X)$ -- гладкая функция.
% \end{to_def}

% \begin{to_lem} 
%     $\omega = \omega_i \d x^i$  -- гладкая $\Leftrightarrow$ $\omega_i$ -- гладкая форма $\forall i$.
% \end{to_lem}

% \red{Конец рассуждений из НМУ}.