Оператор звёздочки является поточечным, то есть линейным относительно умножения на функцию $*(f \alpha) = f(*\alpha)$. Например на $\mathbb{R}^3$ со стандартной римановой структурой:
\begin{gather*}
   	*1 = \d x \wedge \d y \wedge \d z;
   	\\
   	*\d x = \d y \wedge \d z, 
   	\hspace*{0.5 cm}
   	*\d y = \d z \wedge \d x, 
   	\hspace*{0.5 cm}
   	*\d z = \d x \wedge \d y;
   	\\
   	*(\d x \wedge \d y) = \d z,
   	\hspace*{0.5 cm}
   	*(\d y \wedge \d z) = \d x,
   	\hspace*{0.5 cm}
   	*(\d z \wedge \d x) = \d y;
   	\\
   	*(\d x \wedge \d y \wedge \d z) = 1.
\end{gather*}

Тогда можно определить векторные операторы с помощью введённых терминов:
\begin{align*}
   &\grad \colon C^\infty(\mathbb{R}^3) \mapsto \mathbb{T}_0^1 (\mathbb{R}^3)
   \hspace*{0.5 cm} 
   \leadsto 
   \hspace*{0.5 cm}
   \grad f = (\d f)^\sharp 
   \hspace*{0.5 cm} 
   \leadsto 
   \hspace*{0.5 cm}
   \grad = \sharp \d;
   \\
   &\rot \colon \mathbb{T}_0^1(\mathbb{R}^3) \mapsto \mathbb{T}_0^1(\mathbb{R}^3)
   \hspace*{0.5 cm} 
   \leadsto 
   \hspace*{0.5 cm}
   \rot \vc{v} = (*\d(\vc{v}^\flat))^\sharp
   \hspace*{0.5 cm} 
   \leadsto 
   \hspace*{0.5 cm}
   \rot = \sharp * \d \, \flat;
   \\
   &\div \colon \mathbb{T}_0^1(\mathbb{R}^3) \mapsto C^\infty(\mathbb{R}^3)
   \hspace*{0.5 cm} 
   \leadsto 
   \hspace*{0.5 cm}
   \div \vc{v} = *\d(*\vc{v}^\flat)
   \hspace*{0.5 cm} 
   \leadsto 
   \hspace*{0.5 cm}
   \div = * \d * \sharp.
\end{align*}

Таким образом можем записать лапласиан: $\Delta f = *\d * \d f$. В координатах можем проверить, что это совпадает с операциями, определенными в билете №18.
