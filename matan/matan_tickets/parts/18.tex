\begin{to_lem} 
    На гладких диф-формах на $U$ существует единсвтенный $\mathbb{R}$-линейный оператор $\delta \colon \Omega^k (U) \mapsto \Omega^{k+1} (U)$, удовлетворяющий условиям: 
    1) $d(f) = df$;
    2) $d^2 = 0$;
    3) $d(\alpha \wedge \beta) = d \alpha \wedge \beta + (-1)^{\deg \alpha} \alpha \wedge d \beta$ (а-ля правило Лейбница).
    Более того, операция $d$ определена инвариантно.
\end{to_lem}


\begin{to_def} 
    Внешнему дифференцированию 0,1,2-форм в ориентированном $\mathbb{R}^3$ отвечают соответственнооперации \textit{градиента} скалярного поля, \textit{ротора} и \textit{дивергенции} векторного поля, определнные соотношениями
\begin{equation*}
     d \omega^0_f \overset{\mathrm{def}}{=} \omega^1_{\grad f}, \hspace{1cm} 
     d \omega^1_{\vv{A}} \overset{\mathrm{def}}{=} \omega^2_{\rot \vv{A}}, \hspace{1cm} 
     d \omega^2_{\vv{B}} \overset{\mathrm{def}}{=} \omega^3_{\div \vv{B}}.
 \end{equation*} 
    Также можно определить $\Delta \vc{A}$, как
\begin{equation*}
    \Delta \vc{A} = \grad \div \vc{A} - \rot \rot A.
\end{equation*}
\end{to_def}





