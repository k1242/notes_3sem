\begin{to_def} 
    \textit{Пространство-время  СТО} $\mathbb{R}^{1+3}$ с координатами $t, x, y, z$ и полуримановой структурой 
    \begin{equation*}
        g = dt \otimes dt - dx \otimes dx - dy \otimes dy - dz \otimes dz.
    \end{equation*}
    Векторы $v$ с условием $g(v, v) = 0$ соответствуют движению световых лучей, векторы $g(v, v) < 0$ называются \textit{пространственно-подобными}, а $g(v, v) > 0$ -- \textit{времениподобными}.
\end{to_def}

\begin{to_lem}[обратное неравенство Коши-Буняковского]
     В $\mathbb{R}^{1+3}$ для двух времениподобных векторов $X, Y$ выполняется
     \begin{equation*}
         g(X, Y)^2 \geq g(X, X) \cdot g(Y, Y).
     \end{equation*}
\end{to_lem}

\begin{to_lem} 
    Времениподобные прямые максимизируют $\int_\gamma \sqrt{|g(\dot{\gamma}, \dot{\gamma}|} \d t$ (собственное время частицы) среди всех времениподобных кривых, соединяющих две данные точки. 
\end{to_lem}

\begin{to_lem} 
    Геодезические в пространстве-времени $\mathbb{R}^{1+3}$ -- прямые линии, параметризованные с постоянной скоростью в данной системе координат.
\end{to_lem}

\begin{to_def} 
    Если рассмотреть афинные преобразования по модулю сдвигов (изометрии), то есть линейные преобразования, сохраняющие $g$ как квадратичную форму, то получится \textit{группа Лоренца}, обозначаемая $O(1, 3)$. 
\end{to_def}

\red{Тут может быть много забавных свойств группы Лоренца и фактики из ОТО.}