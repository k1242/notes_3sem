\begin{to_lem}[Разбиение единицы в окрестности компакта на многообразии]
     Пусть $M$ -- гладкое многообразие, а $K \subseteq M$ -- его компактное подмножество. Для любого покрытия $\{U_\alpha\}_\alpha$ компакта $K$ открытыми множествами найдётся набор неотрицательных гладких функций $\{\rho_\alpha\}_\alpha$ с компактными носителями $\mathrm{supp}\, \rho_\alpha$ таких, что
\begin{equation*}
    \forall \alpha \ \textnormal{supp}\, \rho_\alpha \subset U_\alpha,
\end{equation*}
    только конечное число из них отлично от нуля и
    $\sum_\alpha \rho_\alpha (x) \equiv 1$
    в некоторой окрестности $K$.
\end{to_lem}

\begin{to_def} 
    Интеграл дифференциальной формы $\nu \in \Omega_{\text{c}}^n (M)$ с компактным носителем по ориентированному $n$-мерному многообразию $M$ определяется с помощью разбиения единицы в окрестности носителя $\nu$ 
    \begin{equation*}
         \rho_1 + \ldots + \rho_m = 1,
     \end{equation*} 
     подчиненного некоторому набору положительно ориентрированных карт как
     \begin{equation*}
         \int_M \nu = \sum_i \int_M \rho_i \nu_i,
     \end{equation*}
     где интегралы справа рассматриваются в координатных картах, содержащих носители соответствующих $\rho_i$.
\end{to_def}

\begin{to_lem} 
    Определение интеграла не зависит от выбора системы положительных карт в данной ориентации и подчиненного им разбиения единциы. 
\end{to_lem}



