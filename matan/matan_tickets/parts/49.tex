\begin{to_def}
	\textit{Плотность меры} --- это функция в каждых локальных координатах, которая при заменах координат преобразуется почти как диф-форма высшего ранга, но умножается при замене координат на модуль якобиана обратной замены, а не на якобиан без модуля. Её интеграл уже не зависит от ориентации.
\end{to_def}

\begin{to_lem}[Формула риманова объёма]
	Для (полу)римановой структуры $g$ формула
	\begin{equation*}
		\textnormal{vol}_g = \sqrt{|\det g|} \d x^1 \wedge \ldots \wedge \d x^n,
	\end{equation*}
	где $\det g$ подразумевает $\det (g_{i j})$, корректно определяет плотность меры.
\end{to_lem}

В случае ориентированного многообразия $\textnormal{vol}_{g}$ можно считать формой высшей степени, положительной  относительно выбранной ориентации.

Для двух римановых многообразий $M$ и $N$ их произведение $M \times N$ можно  тоже считать римановым многообразием по формуле
\begin{equation*}
	g_{M \times N}(X,Y) = g_{M}(p_*X, p_*Y) + g_{N}(q_*X, q_*Y),
\end{equation*}
где $p \colon M \times N \mapsto M$ и $q \colon M \times N \mapsto N$ --- естественные проекции.

В матричном виде на произведении координат $g_{M\times N}$ будет $\oplus$ матриц $g_M $ и $g_N$. Так как детерминант прямой суммы матриц равен произведению детерминантов исходных, для риманова произведения (напр. борелевских) подмножеств:
\begin{equation*}
	X \subseteq M, Y \subseteq N \colon
	\textnormal{vol}_{M \times N} (X \times Y) = \textnormal{vol}_M X \cdot  \textnormal{vol}_N Y.
\end{equation*}
Свойство произведения в некотором смысле обосновывает естественность выбора риманова объёма.

\begin{to_tas}
	Евклидова структура на $\mathbb{R}^n$ является произведением $n$ римановых структур прямой $\mathbb{R}^1$.
\end{to_tas}