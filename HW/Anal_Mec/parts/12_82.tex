\subsubsection*{12.82}
Знаем, что символ Кристофеля
\begin{equation*}
    \Gamma_{i,kl} = \frac{1}{2} \left(
        \frac{\partial g_{li}}{\partial q^k} + \frac{\partial g_{ki}}{\partial q^l} -
        \frac{\partial g_{kl}}{\partial q^i} 
    \right).
\end{equation*}
Кинетическая энергия склерономной системы в обобщенных координатах запишется как
\begin{equation*}
    T  = \frac{1}{2} g_{ij} \dot{q}^i \dot{q}^j.
\end{equation*}
Хотелось бы в терминах сивола Кристофеля записать уравнения Лагранжа
\begin{equation*}
    \frac{d }{d t} \frac{\partial T}{\partial \dot{q}^i} - \frac{\partial T}{\partial q^i} = Q_i.
\end{equation*}

Для начала найдём
\begin{equation*}
    \frac{\partial T}{\partial q^k} = \frac{1}{2} \dot{q}^i \dot{q}^j \frac{\partial g_{ij}}{\partial q^k}.
\end{equation*}
Теперь
\begin{equation*}
    \frac{\partial T}{\partial \dot{q}^k} = 
    \frac{1}{2} g_{ij} \left(
        \dot{q}^j \delta^i_k + q\delta^i \delta^j_k
    \right) = g_{kj} \dot{q}^j.
\end{equation*}
Дифференцируя по времени, получим
\begin{equation*}
    \frac{d }{d t} \frac{\partial T}{\partial \dot{q}^k} =
    g_{kj} \ddot{q}^j + 
    \dot{q}^j \left(
        \frac{\partial g_{kj}}{\partial q^i} \frac{d q^i}{d t} 
    \right) = 
    g_{kj} \ddot{q}^j + \frac{\partial g_{kj}}{\partial q^i} \dot{q}^j \dot{q}^i.
\end{equation*}
Теперь заметим, что
\begin{equation*}
    \dot{q}^i \dot{q}^j \frac{\partial g_{kj}}{\partial q^i} =
     \dot{q}^j \dot{q}^i \frac{\partial g_{ki}}{\partial q^j} .
\end{equation*}
Тогда
\begin{equation}
    Q_k =
    g_{kj} \ddot{q}^j + \frac{1}{2} \dot{q}^j \dot{q}^i
    \left(
        \frac{\partial g_{jk}}{\partial q^i} + \frac{\partial g_{ik}}{\partial q^j} 
        - \frac{\partial g_{ij}}{\partial q^k} 
    \right) , \hspace{0.5cm} \Rightarrow \hspace{0.5cm} 
    \boxed{
        g_{kj} \ddot{q}^j + \Gamma_{k, ij} \dot{q}^j \dot{q}^i = Q_k
    }.
\end{equation}