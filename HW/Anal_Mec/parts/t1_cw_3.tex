\subsection*{Задача №3}

Искать центр вращения -- дело гиблое, лучше посмотрим с точки зрения $A$ на точку $C$ -- центр масс, расположенный в ${A} + \vv{AB}/2$. С учётом того, что в начальный момент времени все скорости равны 0, получим
\begin{equation}
    \vc{\mathrm{w}}_C = \vc{\mathrm{w}}^e + \vc{\mathrm{w}}^r_C,
\end{equation}
где $\vc{\mathrm{w}}^e = \vc{\mathrm{w}}_A$, а $\vc{\mathrm{w}}^r_C$ -- вращение с угловым ускорением $\vc{\varepsilon}$ точки $C$ относительно точки $A$, другими словами
\begin{equation}
    \vc{\mathrm{w}}^r_C = \vc{\varepsilon} \times \vv{AC}.
\end{equation}
По теореме об изменение кинетического момента
\begin{equation}
\label{cat}
    J_C \vc{\varepsilon} = \vc{M}_C^e + 0 = \vv{CA} \times \vc{T} = \vc{T} \times \vv{AC}.
\end{equation}
По теореме об изменение количества движения
\begin{equation}
\label{IIN}
    m \vc{\mathrm{w}}_C = \vc{R}^{e} = \vc{T} + m \vc{g}.
\end{equation}
Осталось выбрать хорошие оси и покоординатно это записать.

Так как не очень хочется задумываться об ускорении точки $A$, выберем ось $OX \bot \vc{\mathrm{w}}_A$, получается повернутую на $\alpha$ от $\vv{AB}$ в начальный момент времени, $OY$ выберем так, чтобы $\omega_z > 0$, тогда
$$
    \vc{\varepsilon} = \begin{pmatrix}
        0 \\ 0 \\ \varepsilon
    \end{pmatrix},
    \hspace{0.5cm} 
    \vv{AC} = \begin{pmatrix}
        L \cos \alpha \\ L \sin \alpha \\ 0
    \end{pmatrix},
    \hspace{0.5cm} 
    \vc{\mathrm{w}}_A = \begin{pmatrix}
        0 \\ {\mathrm{w}}_a \\ 0
    \end{pmatrix},
    \hspace{0.5cm} 
    \vc{\mathrm{w}}_C^r = \vc{\varepsilon} \times \vv{AC} = \begin{pmatrix}
        - \varepsilon L \sin \alpha \\
        \varepsilon L \cos \alpha \\
        0
    \end{pmatrix},
$$
Момент инерции для однородного стержня $J_c = m L^2 / 3$, в таком случае из проекции \eqref{cat} на ось $OZ$ найдём
\begin{equation}
    J_C \varepsilon = T L \sin \alpha
    \hspace{0.5cm} \Rightarrow \hspace{0.5cm} 
    \varepsilon = \frac{3T\sin\alpha}{mL}.
\end{equation}
Перепишем \eqref{IIN} в проекции на ось $OX$ и подставим $\varepsilon$:
$$
    - \varepsilon L \sin \alpha = \frac{T}{m} - g \cos \alpha,
    \hspace{0.5cm} \Rightarrow \hspace{0.5cm} 
    \boxed{
        T = mg \frac{\sin \alpha}{1 + 3 \sin^2 \alpha} 
    }.
$$