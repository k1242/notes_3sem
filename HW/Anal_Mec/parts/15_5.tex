\subsubsection*{15.5}


Перейдём в СО, вращающуюся вместе с телом. В таком случае в уравнениях <<возникнут>>  силы инерции. Ввиду того что $\vc{\omega} \bot \vc{r}$ запишем 
\begin{equation*}
    \vc{F}_{\text{и}} = m \vc{\omega} \times \vc{\omega} \times (\vc{r} + \vc{l}) + m \vc{\omega} \times \vc{\omega} \times \left( \vc{r} - \vc{l}\right) =
    2 m\vc{\omega} \times \vc{\omega} \times \vc{r} = 2m\omega^2 r.
\end{equation*}
Тогда добавка к потенциалу системы будет
\begin{equation*}
    \Pi_{\text{и}} = - \omega^2 r^2 m.
\end{equation*}
Силы между гантелями и стержнем аналогичны потенциалу
\begin{equation*}
    \Pi_{\text{g}} = 
    - \frac{\alpha m}{\rho} \times 2, \hspace{0.5cm} 
    \rho = \sqrt{r^2 + l^2},
\end{equation*}
где $r$ -- расстояние от центра до стержня. 
 
Запишем теперь потенциал системы
\begin{equation*}
    \Pi =  - \frac{2\alpha m}{\rho} - \omega^2 r^2 m.
\end{equation*}
Найдём стационарные точки потенциала
\begin{equation*}
    \frac{\partial \Pi}{\partial r} = \frac{2\alpha m r}{(r^2+l^2)^{3/2}} 
    -
    2 \omega^2 r m = 2mr \left(
        \frac{\alpha}{(r^2+l^2)^{3/2}} - \omega^2
    \right) = 0
    \hspace{0.5cm} \Rightarrow \hspace{0.5cm} 
    \left\{\begin{aligned}
        r_1^* &= \sqrt{\left({\alpha}/{\omega^2} \right)^{2/3} - l^2}
        , &\omega^2 l^3 < \alpha. \\
        r_2^* &= 0.
    \end{aligned}\right.
\end{equation*}
И определим локальные экстремумы потенциала
\begin{equation*}
    \frac{\partial^2 \Pi}{\partial r^2} (r) =
    2 m
    \left(
    \frac{\alpha}{(r^2+l^2)^{3/2}} -
    \frac{\alpha r^2}{(r^2+l^2)^{5/2}} - \omega^2
    \right).
    .
\end{equation*}
При $r = r^*_2$ верно, что
\begin{equation*}
    \frac{\partial^2 \Pi}{\partial r^2} (0) = 
    \frac{2\alpha}{l^3} - 2 \omega^2,
    \hspace{0.5cm} \Rightarrow \hspace{0.5cm} 
    \left\{\begin{aligned}
        r^* = 0 &\text{\ -- устойчиво при } 
        &\omega^2 l^3 < \alpha, \\
        r^* = 0 &\text{\ -- неустойчиво при } 
        &\omega^2 l^3 > \alpha.
    \end{aligned}\right.    
\end{equation*}
Чуть сложнее для $r = r^*_1$, заметим, что случай реализуется только при $\omega^2 l^3 < \alpha$:
\begin{equation*}
    \frac{\partial^2 \Pi}{\partial r^2} (r^*_1)= 
    \underbrace{\left(\ \ldots\ \right)}_{>0}
    \left(
        \alpha^{-2/3} \omega^{4/3} l^2 - 1
    \right),
    \hspace{0.5cm} a^{-2/3} < \omega^{-4/3} l^{-2}
    \hspace{0.5cm} \Rightarrow \hspace{0.5cm} 
    \frac{\partial^2 \Pi}{\partial r^2} (r^*_1) < 0.
\end{equation*}
Таким образом $r = r^*_1$ -- неустойчивое положение равновесия.

