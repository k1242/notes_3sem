\subsubsection*{Т11.}

Движение среды происходит по закону ($\tau = \const > 0$),
$$
    \vc{r} = \begin{pmatrix}
        x & y & z
    \end{pmatrix}\T, \hspace{0.5cm} 
    x = \xi_1 \left(1 +   \frac{t}{\tau} \right), \hspace{0.5cm} 
    y = \xi_2 \left(1 + 2 \frac{t}{\tau} \right), \hspace{0.5cm} 
    z = \xi_3 \left(1 +\frac{t^2}{\tau^2}\right).
$$
Тогда поля скорости и ускорения в лагранжевом описании
$$
    \vc{v} = \dot{\vc{r}} =
    \frac{1}{\tau} 
    \begin{pmatrix}
        \xi_1 & 2 \xi_2 & 2 \dfrac{t}{\tau} \xi_3      
    \end{pmatrix}\T,
    \hspace{0.5cm} 
    \vc{\mathrm{w}} = \ddot{\vc{r}} = \dfrac{1}{\tau^2} \begin{pmatrix}
        0 & 0 & 2 \xi_3
    \end{pmatrix}\T.
$$

Пусть деформация произвола через малый промежуток времени $dt$, тогда $\vc{u} = \vc{v} \d t$. Представим $\partial \vc{u} / \partial \vc{r}$, как сумму симметричного и косо-симметричного
$$
    \frac{\partial \vc{u}}{\partial \vc{r}} = u_{ij} + \varphi_{ij},
$$
где
$$
    u_{ij} = \frac{1}{2} \left(
    \frac{\partial u_i}{\partial x^j} + \frac{\partial u_j}{\partial x^i} 
    \right),
    \hspace{0.5cm} 
    \varphi_{ij} = \frac{1}{2} \left(
        \frac{\partial u_i}{\partial x^j} - \frac{\partial u_j}{\partial x^i} 
    \right).
$$
Подставим $\vc{v}$ в эйлеровом описании
$$
    \vc{v} = \begin{pmatrix}
        \dfrac{x}{\tau+t} &
        \dfrac{2y}{\tau+t} &
        \dfrac{2tz}{\tau^2+t^2} 
    \end{pmatrix}\T,
    \hspace{0.5cm} \Rightarrow \hspace{0.5cm} 
    u_{ij} =
    \diag\left(
        \frac{1}{t+\tau}, \ \frac{2}{\tau+t}, \ \frac{2t}{\tau^2+t^2} 
    \right)_{ij}\d t,
    \hspace{0.5cm} 
    \varphi_{ij} = 0.
$$
Вращательное движение отсутствует.


Как можно заметить из выражения для $\vc{v}$ неподвижными будут частицы с $\xi_1 = 0, \ \xi_2 = 0, \xi_3 = 0$, в начальный момент времени неподвижными будут все частицы с $\xi_3 = 0$.