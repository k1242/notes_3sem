\subsection*{Задача №1}
Введём координаты так, чтобы $OX \parallel \omega$, и $OY \parallel \vv{OO}_1 \equiv \vc{l}$. Запишем скорость точки $O_1$ двумя способами
\begin{equation}
    \vc{v}_{O_1} = \vc{\omega} \times \vc{l} =
    \vc{\omega}_0 \times \vc{r}.
\end{equation}
Скорость точки $B$
$$
    \vc{v}_B^r = 2 \vc{\omega}_0 \times \vc{r},
$$
Ускорение точки $B$
\begin{equation}
    \vc{\mathrm{w}}_B^a
    =
    \vc{\mathrm{w}}_B^r 
    +
    \vc{\mathrm{w}}_B^e
    +
    \vc{\mathrm{w}}_B^c. 
\end{equation}
Что ж, по порядку
\begin{align*}
    \vc{\mathrm{w}}_B^r  
    &=
    2 \vc{\omega}_0 \times \left(
        \vc{\omega}_0 \times \vc{r}
    \right)
    = 
    \begin{pmatrix}
        - \omega_0^2 r& 0& 0
    \end{pmatrix}\T,
    \\
    \vc{\mathrm{w}}_B^e 
    &=
    \vc{\omega} \times \left(\vc{\omega} \times \vc{l}\right) 
    = 
    \begin{pmatrix}
        0& -\omega^2 l& 0
    \end{pmatrix}\T,
    \\
    \vc{\mathrm{w}}_B^c
    &= 2 \vc{\omega} \times \vc{v}_B^r =
    4 \vc{\omega} \times \left(\vc{\omega}_0 \times \vc{r}\right)
    = 
    \begin{pmatrix}
        0& -4\omega\omega_0 r& 0
    \end{pmatrix}\T,
\end{align*}
где подразумевается, что
$$
    \omega = \begin{pmatrix}
        \omega \\ 0 \\ 0
    \end{pmatrix}, \hspace{0.5cm} 
    \omega_0 = \begin{pmatrix}
        0 \\ -\omega_0 \\ 0
    \end{pmatrix},\hspace{0.5cm} 
    \vc{l} = \begin{pmatrix}
        0 \\ l \\ 0
    \end{pmatrix},\hspace{0.5cm} 
    \vc{r} = \begin{pmatrix}
        r \\ 0 \\ 0
    \end{pmatrix}.
$$
Также имеет смысл найти из первого уравнения ${\omega}_0$ и собрать всё вместе
$$
    \omega_0 = \frac{l}{r} \omega,
    \hspace{1cm} \Rightarrow \hspace{1cm} 
    \vc{\mathrm{w}}_B^a = 
    -
    \begin{pmatrix}
     \omega^2 l^2 / r \\
     5 \omega l \\
     0    
    \end{pmatrix}.
$$
