\subsubsection*{12.61}

Составим уравнения движения в форме Лагранжа для системы, представленно на рис. \ref{t12n61}. Хотелось бы найти уравнения относительного движения стержня в форме Лагранжа.

Потенциальная энергия стержня
\begin{equation*}
    \Pi = mgy.
\end{equation*}
Записав кинетическую энергию,, рассматривая движение центра масс и вращение относительно него, найдём 
\begin{equation}
    L = T - \Pi = 
    \frac{1}{2} m \left(
        \dot{y}^2 + \omega^2 x^2 + \dot{x}^2
    \right) + 
    \frac{1}{12} m l^2 \left(
        \dot{\varphi}^2 + \omega^2 \cos^2 \varphi
    \right) - mgy.
\end{equation}
В угловой скорости появляется добавка $\omega \cos \varphi$ как проекции $\vc{\omega}$ на нормаль к стержню. 

Уравнения движения системы найдём, как
\begin{equation}
    \left\{\begin{aligned}
        \EqL{x},  \\
        \EqL{y}, \\
        \EqL{\varphi}.
    \end{aligned}\right.
    \hspace{0.5cm} \Rightarrow \hspace{0.5cm} 
    \left\{\begin{aligned}
       \vp  \ddot{x} &= \omega^2 x,\\
       \vp  \ddot{y} &= g, \\
       \vp  \ddot{\varphi} &= -  \omega^2 \sin (2 \varphi) / 2.
    \end{aligned}\right.
\end{equation}