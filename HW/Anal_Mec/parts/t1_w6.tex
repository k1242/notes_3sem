\subsection{Движение точки в центральном поле сил}

\subsubsection*{8.36}
Для начала выразим $a$ и $b$ через $h$ и $H$:
\begin{equation}
    r_1 + r_2 = H+h 
    \hspace{0.5cm} \Rightarrow \hspace{0.5cm} 
    b = \frac{H+h}{2}, \hspace{0.5cm} a = \sqrt{Hh}.
\end{equation}
Запишем теперь уравнение Бине
\begin{equation}
    u'' + u = \frac{F}{mc^2u^2} = \frac{\alpha}{mc^2},
\end{equation}
т.к. $F(u) = \alpha u^2$. Тогда
\begin{equation}
    u'' = (u - \frac{\alpha}{mc^2} )
    \hspace{0.5cm} \Rightarrow \hspace{0.5cm} 
    u - \frac{\alpha}{mc^2} = A \cos \varphi + B \sin \varphi
    \hspace{0.5cm} \Rightarrow \hspace{0.5cm} 
    r = \frac{1}{\frac{\alpha}{mc^2} - A\cos \varphi} 
    \overset{\mathrm{def}}{=} \frac{p}{1-e\cos \varphi} .
\end{equation}
Из граничных условия находим, что
\begin{equation}
\left\{\begin{aligned}
    r(\varphi = 0) &= H = \frac{p}{1-e}.
    r(\varphi = \pi) &= h = \frac{p}{1+e}.
\end{aligned}\right.
\hspace{0.5cm} \Rightarrow \hspace{0.5cm} 
e = \frac{H-h}{H+h}; \hspace{0.5cm} p = \frac{2Hh}{H+h}.
\end{equation}



\subsubsection*{8.48}

Рассмотрим движение в поле с силой
\begin{equation}
    F = - \frac{\alpha}{m^2} \psi (\varphi) - \frac{\beta}{r^3} .
\end{equation}
Запишем уравнение Бине:
\begin{equation}
    u'' + u=
    \frac{F(u)}{mc^2u^2} = \frac{-\alpha u^2 \psi - \beta u^3}{mc^2 u^2} 
    \hspace{0.5cm} \Rightarrow \hspace{0.5cm} 
    u'' + 
    \underbrace{\left(1 + \frac{\beta}{mc^2} \right)}_{\omega^2}
    u = 
    \underbrace{-\frac{\alpha}{mc^2}  }_{-B}
    \psi(\varphi).
\end{equation}
Методом вариации постоянных, получм
\begin{equation}
    u(\varphi) = C_1 (\varphi) \cos (\omega \varphi) + C_2 (\varphi) \sin (\omega \varphi).
\end{equation}
Тогда
\begin{equation}
    u' = C_1' \cos \omega \varphi - \omega C_1 \sin \omega \varphi + C_2 \sin \omega \varphi + \omega C_2 \cos \omega \varphi.
\end{equation}
В силу предоставленной нам свободы, потреуем для простоты и адекватности выкладок
\begin{equation}
    C_1 \cos \omega \varphi + C_2 \sin \omega \varphi \equiv 0
    \hspace{0.5cm} \Rightarrow \hspace{0.5cm} 
    u' = - \omega C_1 \sin \omega \varphi + \omega_2 C_2 \cos \omega \varphi.
\end{equation}
Найдём из нашего условия и условия диф. уравнения $C_1$ и $C_2$:
\begin{equation}
    \left\{\begin{aligned}
        C_1' &= - \frac{1}{\omega} \psi(\varphi) \sin \omega t \\
        C_2' &=  \frac{1}{\omega} \psi (\varphi) \cos \omega t
    \end{aligned}\right.
    \hspace{0.5cm} \Rightarrow \hspace{0.5cm} 
    \left\{\begin{aligned}
        C_1 &= - \frac{1}{\omega} \int_0^\varphi \psi(\varphi) \sin \omega \tau \d \tau + \tilde C_1 \\
        C_2 &=  \frac{1}{\omega} \int_0^\varphi \psi(\varphi) \cos \omega \tau \d \tau + \tilde C_2
    \end{aligned}\right.
\end{equation}
Из  формулы синуса суммы, получим
\begin{equation}
    u(\varphi) = C_1 \sin \omega \varphi + C_2 \cos \omega \varphi +
    \frac{1}{\omega} \int_0^t \sin \omega(\varphi-\tau) \psi(\varphi) \d \tau,
\end{equation}
где $\omega(\varphi) = \sqrt{1 + \beta/mc^2}$.

\input{parts/t1_w6_821}
\subsubsection*{Т10*.}

В ОТО движение в центральном поле тяжести описывается как движение в метрике Шварцшильда:
$$
    d s^2 = \left(1 - \frac{a}{r} \right) \d \tau^2
    -
    \left(1 - \frac{a}{r} \right)^{-1} \d r^2
    -
    (r \sin \theta)^2 \d \varphi^2 - r^2 \d \theta^2.
$$
Здесь 4 независимых переменных $(\tau, r,  \varphi, \theta)$, где три из сферических координат, а $\tau$ -- физическое время,
также введен радиус Шварцшильда $a = 2 GM$.

Движение точек рассматриваем, как движение по геодезическим, то есть $\vc{\mathrm{w}}_i = 0$, где $i \in \{\tau, r, \varphi, \theta\}$. Движение будет в некотором смысле происходить в одной плоскости, так что пусть $\theta (t) = \pi / 2$. Так мы получим следующую систему уравнений:
\begin{equation}
    \left\{
        \begin{aligned}
            v^2 &= 
            \left(
            \frac{ds}{dt} \right)^2 = \left(1 - \frac{a}{r} \right) \dot{\tau}^2
            - \left(1 - \frac{a}{r} \right)^{-1} \dot{r}^2 - r^2 \dot{\varphi}^2 \\
            \vc{\mathrm{w}}_\tau &=
            \frac{d}{dt} \frac{\partial (v^2/2)}{\partial \dot{\tau}} -
            \underbrace{\frac{\partial (v^2/2)}{\partial \tau}}_{0} =
            \frac{d}{dt} \left[\left(
                       1 -  \frac{a}{r} 
                    \right) \dot{\tau}\right] = 0 \\
            \vc{\mathrm{w}}_\varphi &= 
            - \frac{d}{dt} \left[
                r^2 \dot{\varphi}
            \right] = 0
        \end{aligned}
    \right.
\end{equation}
Таким образом получим пару первых интегралов системы, в частности
\begin{align*}
    \left(1 - \frac{a}{r} \right) \dot{\tau} &= \mathcal D,
    \\ r^2 \dot{\varphi} &= \mathcal C.
\end{align*}
Подставляя их в выражение для скорости, получим, что
\begin{equation*}
\left(
1 - \frac{a}{r} 
\right)^{-1} \mathcal D^2 -
\left(1 - \frac{a}{r} \right)^{-1} \dot{\vc{r}}^2 - \frac{c^2}{r^2}  = v^2.
\end{equation*}
По замене Бине 
$$
    r = \frac{1}{u}, \hspace{0.5cm} 
    \dot{r} = r' \dot{\varphi} = \left(\frac{1}{u} \right)' c u^2 =
    - u' c,
$$
перейдём к функции $u(\varphi)$:
\begin{equation}
    2 c^2 u'' + c^2 (2u - 3 au^2) = v^2.
\end{equation}
Преобразуя, получим
\begin{equation}
\boxed{
    u'' + u = \frac{a}{2c^2} v^2 + \frac{3}{2} au^2
}.
\end{equation}


Найдём теперь видимый радиус черной дыры -- минимальное значение прицельного параметра, при котором луч, проходящий через окрестность черной дыры не падает на центр. Для светового луча верно, что $\dot{s}^2=v^2=0$, тогда
$$
    u'' + u = \frac{3}{2} a u^2.
$$
Интегрируя, получим
$$
    \frac{u'^2}{2} + \frac{u^2}{2} = \frac{au^3}{2} + c'.
$$
Посмотрим теперь на поведение света при $u \to 0$ верно, что $r\varphi \to b$, тогда 
$$
    \frac{dr}{d\varphi} = - \frac{r}{\varphi} 
    \hspace{0.5cm} \Rightarrow \hspace{0.5cm} 
    \frac{du}{d\varphi} = - \frac{dr}{d\varphi} \frac{1}{r^2} = \frac{r}{\varphi} \frac{1}{r^2} = \frac{1}{r \varphi}
    \hspace{0.5cm} \Rightarrow \hspace{0.5cm} 
    u'\big|_{t=0} = \frac{1}{b} 
    \hspace{0.5cm} \Rightarrow \hspace{0.5cm} 
    c' = \frac{1}{2b^2} 
$$
Переписав, получим
$$
    u'^2 = au^3 - u^2 + \frac{1}{b^2}.
$$
Вблизи точки с критическим $u$ верно, что $\dot{r} \sim 0$, тогда нас интересует экстремум функции $au^3 - u^2 + b^{-2}$, тогда
$$
    3 a u^2 - 2 u = 0
    \hspace{0.5cm} \Rightarrow \hspace{0.5cm} 
    \frac{1}{r_{\text{min}}} = \left(\frac{3}{2} a\right)^{-1}.
$$
Условие падения -- уменьшение радиуса (увеличиение $u$) при $r=r_{\text{min}}$:
$$
    u'^2_{\text{min}} = \frac{1}{a^2} \left(
        \frac{8}{27} - \frac{4}{9} 
    \right) + \frac{1}{b^2} =
    -\frac{4}{27a^2}  + \frac{1}{b^2}  \geq 0
    \hspace{0.5cm} \Rightarrow \hspace{0.5cm} 
    b^2 \leq \frac{27}{4} a^2
$$
Тогда минимальное значение прицельного параметра, при котором луч, проходящий через окрестность черной дыры не падает на центр
\begin{equation}
    \boxed{b_{\text{min}} = \frac{3\sqrt{3}}{2} a},
\end{equation}
что и является видимым радиусом черной дыры.