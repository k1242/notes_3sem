\subsubsection*{12.73}
Выберем начало координат в положение равновесия. Запишем второй закон Ньютона для системы:
\begin{equation*}
    m \ddot{x} = -cx-\beta v.
\end{equation*}
Формально, мы хотим найти такой $L(x, \dot{x}, t)$, что 
\begin{equation*}
    \frac{d }{d t} \frac{\partial L}{\partial \dot{q}} - \frac{\partial L}{\partial q}  = m \ddot{x} + \beta \dot{x}  + cx = 0.
\end{equation*}
При отсутствии вязкого трения $L$ имел бы вид
\begin{equation*}
    L^* = T - \Pi = \frac{1}{2} m \dot{x}^2 + \frac{1}{2} cx^2.
\end{equation*}
Как мы видим $L^* \neq L(t)$, соответсвенно энергия такой системы сохраняется. Мы же рассматриваем систему с вязким трением, которая в пределе с $\beta \to 0$ приходила бы к $L = L^*$ так что будем искать $L$ вида 
\begin{equation*}
    L = f(t) \cdot L^*.
\end{equation*}
В таком случае
\begin{equation*}
    \frac{d }{d t} \frac{\partial L}{\partial \dot{q}} - \frac{\partial L}{\partial q}  =    
    f(t) m \ddot{x} + \underbrace{\dot{f} (t) m}_{= f(t) \beta}
     \dot{x}  + f(t) cx = 0
     \hspace{0.5cm} \Leftrightarrow \hspace{0.5cm} 
     m \ddot{x} + \beta \dot{x} +   c x = 0.
\end{equation*}
Воплощая в жизнь стремление сократить уравнение на $f(t)$ находим, что
\begin{equation*}
    \frac{d }{d t} f(t) = \frac{\beta}{m} f(t),
    \hspace{0.5cm} \Rightarrow \hspace{0.5cm} 
    f(t) = \exp\left(\frac{\beta}{m} t\right).
\end{equation*}
Тогда уравнене движения осциллятора с вязким трением можно записать, как уравнение лагранжа второго рода, для лагранжаиана
\begin{equation}
    L(x, t) = \exp \left(\frac{\beta}{m} t\right) \cdot \frac{1}{2} 
    \left(
        m \dot{x}^2 + c x^2
    \right).
\end{equation}