\subsubsection*{Т14*. (решение для криволинейных координат, образующих ортогональный базис) }


Хотелось бы выразить лапласиан $\Delta p$ через частные производные в произвольной криволинейной системе координат. Легко показать, что
\begin{equation}
    \Delta p = \div \grad p.
\end{equation}
Так что начнём с вида $\div \vc{v}$ и $\grad f$ в криволинейной системе координат. Понадеемся, что достаточно рассмотреть случай криволинейных координат, образующих ортогональный базис в каждой точке пространства. 

В криволинейных координатах базисные направления сформированы векторами $\vc{g}_i (\vc{r}) \overset{\mathrm{def}}{=} \partial \vc{r} / \partial q^i$. Для удобства введём единичные орты координатных направлений для ортогональной системы
\begin{equation}
\label{b1}
    \vc{e}_1 (q) = \left(
        \frac{1}{\sqrt{g_{11}}}, 0, 0
    \right), \hspace{0.5cm} 
    \vc{e}_2 (q) = \left(
        0, \frac{1}{\sqrt{g_{22}}}, 0
    \right), \hspace{0.5cm} 
    \vc{e}_3 (q) = \left(
        0, 0, \frac{1}{\sqrt{g_{33}}}
    \right).
\end{equation}
Тогда
\begin{align}
\label{b2}
    d q^j (\vc{e}_i) = \frac{1}{\sqrt{g_{ii}}} \delta^i_j,
    \hspace{0.5cm} 
    dq^i \wedge d q^j (\vc{e}_k, \vc{e}_l) = \frac{1}{\sqrt{g_{ii} g_{jj}}} \delta^{i}_{k} \delta^j_l.
\end{align}


Известно, что градиент функции соответствует дифференциальной 1-форме. 
Её (по вектору $\vc{A}$) можно записать как $\omega^1_{A} = a_i \d q^i$.
С учётом введеного базиса можно записать, что 
$ \vc{A} = A^i \vc{e}_i, \; \forall \vc{A} \in T \mathbb{R}_q^3. $
Из \eqref{b2} получим, что
$$
    \omega^1_A (\vc{e}_i) = (\vc{A} \cdot \vc{e}_i) = A^i = \frac{a_i}{\sqrt{g_{ii}}},
$$
следовательно $a_i = A^i \sqrt{E_i}$, и, соответственно
\begin{equation}
    \omega^1_A = A^i \sqrt{g_{ii}} dq^i.
\end{equation}
Аналогично, пусть теперь $\grad f = A^i \vc{e}_i$. По определению
$$
    \omega^1_{\grad f} = d \omega_f^0 = d f = \frac{\partial f}{\partial q^i} d q^i.
    \hspace{0.5cm} \Rightarrow \hspace{0.5cm} 
    \boxed{
        \grad f = \frac{1}{\sqrt{g_{ii}}}  \frac{\partial f}{\partial q^i} \vc{e}_i.    
    }
$$

Теперь найдём $\div \vc{B}$, как дифференциальную 3-форму. Для начала поймём, что для вектора $\vc{B} (q) = (B^i \vc{e}_i) (q)$ форма
\begin{equation}
\label{wb2}
    \omega^2_{B} = b_1 dq^2 \wedge dt^3 + b_2 dq^3 \wedge dt^1 + b_3 dq^1 \wedge dt^2
\end{equation}
имеет следующий вид:
$$
    \omega^2_B (\vc{e}_2, \vc{e}_3) = (\vc{B}, \vc{e}_2, \vc{e}_3) = B^1.
$$
С другой стороны, из \eqref{b2} и \eqref{wb2},
$$
    \omega^2_B (\vc{e}_2, \vc{e}_3) =
    b_1 dq^2 \wedge dq^3 (\vc{e}_2, \vc{e}_3) = \frac{b_1}{\sqrt{g_{22} g_{33}}}.
$$
Получаем, что $b_1 = B^1 \sqrt{g_{22} g_{33}}$, аналогично можем получить, что
$b_2 = B^2 \sqrt{g_{11} g_{33}}$, $b_3 = B^3 \sqrt{g_{11} g_{22}}$.

Теперь, из определения, получаем
$$
    \omega^3_{\div B} = d \omega^2_B = 
    \underset{P(i,g,k)=1}{\sum_{i \neq j \neq k}^3}
    % \left(
    \frac{\partial \sqrt{g_{jj} g_{kk}} B^i}{\partial q^i}
    % \right)
    dq^i \wedge dq^j \wedge dq^k.
$$
Тогда
\begin{equation}
\label{div}
    \div \vc{B} = \frac{1}{\sqrt{\det g}} 
    \left(
        \frac{\partial \sqrt{g_{22} g_{33}} B^1}{\partial q^1} +
        \frac{\partial \sqrt{g_{33} g_{11}} B^1}{\partial q^2} +
        \frac{\partial \sqrt{g_{11} g_{22}} B^1}{\partial q^3} 
    \right)
\end{equation}
Собирая всё вместе получаем, что
\begin{equation}
    \Delta f = \div
    \left(
        \frac{1}{\sqrt{g_{ii}}}  \frac{\partial f}{\partial q^i} \vc{e}_i
    \right) =
    \frac{1}{\det g} 
    \underset{P(i,g,k)=1}{\sum_{i \neq j \neq k}^3}
    \left(
        \frac{\partial }{\partial q^i} \left[
            \sqrt{\frac{g_{jj} g_{kk}}{g_{ii}} } \frac{\partial f}{\partial q^i} 
        \right] 
    \right).
\end{equation}

В частности, для полярных
$$
    g_{ij} = \diag\left(1, r^2, 1\right)
    \hspace{0.5cm} \Rightarrow \hspace{0.5cm} 
        \Delta f = \frac{1}{r} \frac{\partial }{\partial r} 
    \left(r \frac{\partial f}{\partial r} \right) +
    \frac{1}{r^2} \frac{\partial^2 f}{\partial \varphi^2}  + \frac{\partial^2 f}{\partial z^2} .
$$




