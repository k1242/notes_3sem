\subsubsection*{14.41}

Материальная точка может двигаться по линии пересечения двух плоскостей:
\begin{equation*}
    \left\{\begin{aligned}
        &\text{I}: &-2x_1 + x_2 + x_3  &= t \\
        &\text{II}: &x1 - 2x_2 + x_3 &= -t^2.
    \end{aligned}\right.
\end{equation*}
Найдём систему бесконечно малых возможных перемещений.
Знаем, что направляющая прямой,
\begin{equation*}
    \vc{n}_\text{I} \times \vc{n}_\text{II} = 
    \begin{pmatrix}
        -2 \\ 1 \\ 1
    \end{pmatrix} \times
    \begin{pmatrix}
        1 \\ -2 \\ 1
    \end{pmatrix} = 3 \begin{pmatrix}
        1 \\ 1 \\ 1
    \end{pmatrix},
    \hspace{0.5cm} \Rightarrow \hspace{0.5cm} 
    \vc{a} = \begin{pmatrix}
        1 \\ 1 \\ 1
    \end{pmatrix}.
\end{equation*}
Хотелось бы найти уравнения прямой, в виде
\begin{equation*}
    \vc{r} = \vc{r}_0 + \vc{a} k.
\end{equation*}
Подставляя $\vc{a}$ в уравнения плоскости, найдём, что
\begin{equation*}
    x_0 = \frac{1}{3} t (t-2), \hspace{0.5cm} 
    y_0 = \frac{1}{3} t (2t - 1), \hspace{0.5cm} 
    z_0 = 0.
\end{equation*}
Тогда
\begin{equation*}
    \vc{r} = \frac{t}{3} \begin{pmatrix}
        t - 2 \\ 2t -1 \\ 0
    \end{pmatrix} + 
    k \begin{pmatrix}
        1 \\ 1 \\ 1
    \end{pmatrix},
    \hspace{0.5cm} \Rightarrow \hspace{0.5cm} 
    \frac{\partial \vc{r}}{\partial t} =
    \frac{1}{3} \begin{pmatrix}
        2t-2 \\ 4t-1 \\ 0
    \end{pmatrix},
    \hspace{0.5cm} 
    \frac{\partial \vc{r}}{\partial k} = \begin{pmatrix}
        1 \\ 1 \\ 1
    \end{pmatrix}.
\end{equation*}
В таком случае возможные перемещения:
\begin{equation*}
    \delta \vc{r} = \begin{pmatrix}
        1 \\ 1 \\ 1
    \end{pmatrix} \delta k,
    \hspace{1cm} 
    d \vc{r} = \delta \vc{r} + \frac{\partial \vc{r}}{\partial t} \d t = 
    \begin{pmatrix}
        1 \\ 1 \\ 1
    \end{pmatrix} \delta k + 
    \frac{1}{3} \begin{pmatrix}
        2t-2 \\ 4t-1 \\ 0
    \end{pmatrix} dt.
\end{equation*}














