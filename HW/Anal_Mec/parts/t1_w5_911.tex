\subsubsection*{9.11}

Точка $A$ подвеса математического маятника длины $l$ совершает вертикальные колебания по закону
$$
    \vv{OA} = \vc{a} \sin (\omega t) = \vc{r}_A,
    \hspace{0.5cm} 
    \vc{a} \omega \cos (\omega t) = \vc{v}_A,
    \hspace{0.5cm} 
    - \vc{a} \omega^2 \sin (\omega t) = -\vc{\mathrm{w}}^e.
$$
Тогда по II закону Ньютона для неИСО
\begin{equation}
    m \vc{\mathrm{w}}^r = \vc{F} + m \vc{g} - m \vc{\mathrm{w}}^e.
\end{equation}
Пусть ось $OX$ противонаправлена силе натяжения нити $\vc{F}$, $OY$ в плоскости движения, тогда, введём $\vc{g}' = \vc{g}-\vc{\mathrm{w}}^e$ и получим
\begin{align*}
    &OX: \ \  m \mathrm{w}_x = - F + m g' \cos \varphi  = 0,\\
    &OY: \ \  m \mathrm{w}_y = mg' \sin \varphi,
\end{align*}
Так приходим к уравнению вида
\begin{equation}
    \ddot{\varphi} +  \frac{1}{L} \left(
    \vphantom{\frac{1}{2} }
        g - a \omega^2 \sin \left(\omega t\right)
    \right) \sin (\varphi)= 0.
\end{equation}
В частности, заметим, что $\varphi(t)\equiv0$ и $\varphi(t) \equiv \pi$ являются частными решениями этого уравнения.