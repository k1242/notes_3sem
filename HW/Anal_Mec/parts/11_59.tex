\subsubsection*{11.59}

Есть твёрдое тело в отсутствие внешних сил с $\vc{K}_O = \const$ и $A = B \neq C$. Выберем в качестве оси динамической симметрии ось $O\zeta$. Запишем динамические уравнения Эйлера
\begin{equation*}
    \left\{\begin{aligned}
        A \dot{p} + (C-A) qr &= 0 \\
        A \dot{q} - (C-A) pr &= 0 \\
        C \dot{r} &= 0
    \end{aligned}\right.
    \hspace{0.5cm} \Rightarrow \hspace{0.5cm} 
    C \dot{r} = 0, \hspace{0.5cm} Cr_0 = K_O \cos \theta = \const
    \hspace{0.5cm} \Rightarrow \hspace{0.5cm} 
    \left\{\begin{aligned}
        r(t) &= r_0 &= \const\\
        \theta(t) &= \theta &= \const
    \end{aligned}\right.
\end{equation*}
Посмотрим теперь на $\|\vc{K}_O\|$
\begin{equation*}
    K_O^2 = A^2 (p^2 + q^2) + (K_O \cos \theta)^2
    \hspace{0.5cm} \Rightarrow \hspace{0.5cm} 
    p^2 + q^2 = \left(\frac{K_0 \sin \theta}{A} \right)^2.
\end{equation*}
Теперь посмотрим на $\vc{\omega}$
\begin{equation*}
    \vc{\omega} = \dot{\vc{\varphi}} + \dot{\vc{\psi}} + \cancel{\dot{\vc{\theta}}},
\end{equation*}
проецируя всё на базис $O\xi\eta\zeta$ находим, что
\begin{equation*}
    \left\{\begin{aligned}
        r = \dot{\varphi} + \dot{\psi} \cos \theta \\
        \sqrt{p^2 + q^2} = \dot{\psi} \sin \theta       
    \end{aligned}\right.
    \hspace{0.5cm} \Rightarrow \hspace{0.5cm} 
   \left\{\begin{aligned}
       \dot{\psi} &= {K_O}/{A} &= \const \\
       \dot{\varphi} &= r_0 \left(1 - {C}/{A} \right) &=\const
   \end{aligned}\right.
\end{equation*}
Теперь мы готов записать \textit{параметры регулярной прецессии в случае Эйлера}:
\begin{equation}
\label{regp}
    \cos \theta = \frac{Cr_0}{K_0}, \hspace{0.5cm}
    \dot{\psi} = \frac{K_O}{A}, \hspace{0.5cm} 
    \dot{\varphi} = r_0 \left( 1- \frac{C}{A}  \right),
    \hspace{1cm} 
    K_O = \sqrt{C^2 r_0^2 + A^2 (\omega^2_0 - r_0^2)}.
\end{equation}
