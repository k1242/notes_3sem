\subsubsection*{12.46}

Составим уравнения движения в форме Лагранжа для системы, представленно на рис. \ref{t12n46} Для начала запишем потенциальную энергию системы, как
\begin{equation*}
    \Pi = - (\rho - r) \cos (\varphi  + \theta).
\end{equation*}
Момент инерции полого цилиндра:
\begin{equation*}
    I_1 = \int_\rho^R \sigma r^2 \d V 
    \hspace{0.2cm} 
    \overset{dV = h 2 \pi r \d r}{\longrightarrow} 
    \hspace{0.2cm} 
    I_1 = 2 \pi \sigma h \int_\rho^R r^3 \d r = 
    \frac{1}{2} \left(R^2 - \rho^2  \right) (R^2 + \rho^2) \pi  h \sigma =
    \frac{1}{2} M \left(R^2 + \rho^2\right).
\end{equation*}
Тогда его кинетическая энергия 
\begin{equation*}
    T_1 = \frac{1}{4} M \left(R^2 + \rho^2\right) \dot{\theta}^2.
\end{equation*}
Скорость центра масс сплошного цилиндра:
\begin{equation*}
    v_2 = \dot{\varphi} (\rho  - r).
\end{equation*}
Пусть цилиндр катится с угловой скоростью $\omega$, тогда запишем условие того, что он не проскальзывает
\begin{equation*}
    (\rho - r) \dot{\varphi} = \rho \dot{\theta} + \omega r,
    \hspace{0.5cm} \Rightarrow \hspace{0.5cm} 
    \omega = (\rho - r) \dot{\varphi} - \rho \dot{\theta}.
\end{equation*}
Тогда кинетическая энергия сплошного цилиндра
\begin{equation*}
    T_2 = 
    \frac{1}{2} m \left(
        \dot{\varphi} (\rho - r) \vp
    \right)^2 + 
    \frac{1}{2} \left(\frac{1}{2} m r^2 \right) \omega^2
    .
\end{equation*}
Лагранжиан системы:
\begin{equation}
    L = 
    mg (\rho - r) \cos \varphi +
     m \left(
        \frac{1}{2}  \dot{\varphi}^2 (\rho - r)^2 + 
        \frac{1}{4} \left(
            \dot{\theta} \rho - \dot{\varphi} (\rho - r)
        \right)^2
    \right) + \frac{1}{4} M \left(R^2  + \rho^2\right)\dot{\theta}^2.
\end{equation}
Соответсвенно, уравнения движения системы
\begin{equation}
    \left\{\begin{aligned}
        \frac{d }{d t} \frac{\partial L}{\partial \dot{\varphi}} - \frac{\partial L}{\partial \varphi} &= 0, \\
        \frac{d }{d t} \frac{\partial L}{\partial \dot{\theta}} - \frac{\partial L}{\partial \theta} &= 0.
    \end{aligned}\right.
    \hspace{0.5cm} \Rightarrow \hspace{0.5cm} 
    \left\{\begin{aligned}
    - \ddot{\theta} \rho + 3 \ddot{\varphi} \left(\rho - r\right) + 2 g \sin{\left(\varphi \right)} &= 0, \\
    M \ddot{\theta} \left(R^{2} + \rho^{2}\right) + \rho m \left(\ddot{\theta} \rho - \ddot{\varphi} \left(\rho - r\right)\right) &= 0.
    \end{aligned}\right.
\end{equation}