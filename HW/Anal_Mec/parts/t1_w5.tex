\subsection{Основные теоремы динамики}

\subsubsection*{5.10}

Парметризуем систему движением по оси $z \parallel \vc{g}$, тогда
\begin{equation}
    m \dot{v} = \beta v^2 - mg \frac{R^2}{(R+z)^2}.
\end{equation}
Что аналогично диф. уравнению
\begin{equation}
    \ddot{z} = \frac{\beta}{m} \dot{z}^2 - g \frac{R^2}{(R+z)^2} .
\end{equation}
Решим диф. уравнение заменой $\dot{z} = p(v) \equiv v$, тогда $\ddot{z} = p'p$.
Пусть теперь $y = v^2$, $x'/2 = p'p$, тогда приходим к однородному диф. уравнению
\begin{equation}
    \frac{1}{2} y' - \frac{\beta}{m} x = \frac{-gR^2}{(R+z)^2}.
\end{equation}
Решая, получим, что
\begin{equation}
    C(z) = -2g \int \left(1 + \frac{z}{R} \right)^{-2} \exp\left(-\frac{2\beta}{m} z\right) \d z + C_0.
\end{equation}
Из начальных условия  находим $C_0$, получая
\begin{equation}
    v^2 = v_0^2 \exp\left(-\frac{2\beta}{m} \left(H-h\right)\right)
    -2gR^2 \exp\left(\frac{2\beta}{m} h\right)
    \int_{H}^{h} \left(R+z\right)^{-2} \exp\left(-\frac{2\beta}{m} z\right)\d z.
\end{equation}

%%%%%%%%%%%%%%%%%%%%%%%%%%%%%%%%%%%%%%%%%%%%%%%%%%%%%%%%%%%%%%%%%%%%%%%%%%%%%%%%%%%

\subsubsection*{6.13}

Из теоремы об изменение количества движения, ц.м. системы покоится. Т.к. на систему не действуют внешние силы с ненулевым относительно вертикальной оси моментом, то по теореме об изменение кин. момента, он сохраняется $K_0 = K_1$. 

Далее всё запишем  в проекции на вертикальную ось. В начальный момент времени
\begin{equation} 
     K_0 = I \omega_0 = \frac{3}{10} k m r^2 \omega_0.
 \end{equation} 
При достижении шариком пола
\begin{equation}
    K_1 = I'\omega' + m (kl^2) \omega'.
\end{equation}
По т. Штейнера
\begin{equation}
    I'\omega' = I \omega' + kml^2 = \frac{3}{10} kmr^2 \omega' _ kml^2\omega'.
\end{equation}
Собирая всё вместе получаем, что
\begin{equation}
    3k (k+1) \omega_0 = \left(
        3k(k+1) + 10k
    \right) \omega'
    \hspace{0.5cm} \Rightarrow \hspace{0.5cm} 
    \boxed{
        \omega'= \frac{3(k+1)}{3k + 13} \omega_0
    }
\end{equation}


%%%%%%%%%%%%%%%%%%%%%%%%%%%%%%%%%%%%%%%%%%%%%%%%%%%%%%%%%%%%%%%%%%%%%%%%%%%%%%%%%%%

\subsubsection*{6.25}

Кинетический момент 
\begin{equation}
    \frac{d}{dt} \vc{K}_A = \vc{M}_A^e + \vc{Q} \times \vc{v}_A,
\end{equation}
где $A$ -- мгновенный центр скоростей, $\vc{v}_A = 0$. Тогда, в проекции на вертикальную ось, получим, что
\begin{equation}
    \frac{d}{dt} \vc{K}_A = \vc{M}_A^e = \vc{l} \times \vc{F}_{\text{тр}}
    \hspace{0.5cm} \Rightarrow \hspace{0.5cm} 
    \vc{K}_A\big|_z = \const.
\end{equation}


%%%%%%%%%%%%%%%%%%%%%%%%%%%%%%%%%%%%%%%%%%%%%%%%%%%%%%%%%%%%%%%%%%%%%%%%%%%%%%%%%%%

\subsubsection*{6.35}
Для точек пластины $\vc{r}_{AC} = \vc{r}_C - \vc{r}_A$, $\vc{v}_i = \vc{v}_A + \vc{\omega} \times \vc{r}_{Ai}$. Соответственно
\begin{equation}
    \vc{K}_A = \sum_i \vc{r}_{Ai} \times m_i \left(
        \vc{v}_A + \vc{\omega} \times \vc{r}_Ai
    \right) =
    \vc{v}_A \times \sum_i \left(\vc{r}_A - \vc{r}_i\right) m_i
    +
    \sum_i \vc{r}_{Ai} \times (m_i \vc{\omega} \times \vc{r}_{Ai}).
\end{equation}
Раскрывая по правилу Лагранжа, получим, что
\begin{equation}
    \vc{K}_A = m \left(
        \vv{AC} \times \vc{v}_A
    \right) + I_A \vc{\omega},
    \hspace{0.5cm} 
    \text{Q.E.D.}
\end{equation}


%%%%%%%%%%%%%%%%%%%%%%%%%%%%%%%%%%%%%%%%%%%%%%%%%%%%%%%%%%%%%%%%%%%%%%%%%%%%%%%%%%%
\subsubsection*{7.4}
Во-первых запишем кинетическую энергию системы, как
\begin{equation}
    T_{\text{сист}} = T_{\text{диска}} + T_{\text{стержня}}.
\end{equation}
Начнем с простого,
\begin{equation}
    T_{\text{ст}} = \frac{1}{2} \left(
        M \frac{l^2}{3} \omega^2
    \right).
\end{equation}
Точка $K$ -- мгновенный центр скоростей, то
\begin{equation}
    \vc{v}_d = \vc{\omega} \times \vc{l} = \vc{\omega}_d \times \vc{r}
    \hspace{0.5cm} \Rightarrow \hspace{0.5cm} 
    \omega_d = \omega \frac{l}{r}.
\end{equation}
Аналогично для обруча
\begin{equation} 
    \vc{v}_{\text{об}} = \vc{\omega}_{\text{об}} \times \vc{\rho} = \vc{\omega} \times \vv{OB}
    \hspace{0.5cm} \Rightarrow \hspace{0.5cm} 
    \omega_{\text{об}} = \omega \frac{R-\rho}{\rho} .
\end{equation}
Теперь можем записать для диска 
\begin{equation}
    T_{\text{д}} = \frac{1}{2} I_d \omega_d^2 + \frac{mv_0^2}{2} =
    \frac{3}{4} m \omega^2 l^2.
\end{equation}
И, наконец, для обруча,
\begin{equation}
    T_{\text{об}} = \frac{1}{2} \mu \omega^2 (R-\rho) + \frac{1}{2} \mu \rho^2 \omega^2 \frac{(R-\rho)^2}{\rho^2} = \mu (R-\rho)^2 \omega^2.
\end{equation}
Собирая всё вместе, получим
\begin{equation}
    \boxed{
        T = \omega^2
        \left(
            \mu \left(l+r-\rho\right)^2 + \frac{3}{4} m l^2 + \frac{1}{6} M l^2
        \right)
    }
\end{equation}


\subsubsection*{7.12}

Изначально, известно, что
\begin{equation}
    \left\{\begin{aligned}
        F_x &=  yz \sin \omega t\\
        F_y &= x z \sin \omega t\\
        F_z &= xy \sin \omega t\\
    \end{aligned}\right.
\end{equation}
Проверим, что поле потенциально
\begin{equation}
    \rot \vc{F} = 0.
\end{equation}
Да, действитеьно потенциально. Тогда выбрав в качетсве $0$ потенциальной энергии $U$ нулевой момент времени, получим
\begin{equation}
    U = \int_{x_0}^{x} F_x \d x +
    \int_{y_0}^{y} F_y \d y +
    \int_{z_0}^{z} F_z \d z
    \hspace{0.5cm} \Rightarrow \hspace{0.5cm} 
    \boxed{
        U = xyz \sin \omega t
    }
\end{equation}

\subsubsection*{9.11}

Точка $A$ подвеса математического маятника длины $l$ совершает вертикальные колебания по закону
$$
    \vv{OA} = \vc{a} \sin (\omega t) = \vc{r}_A,
    \hspace{0.5cm} 
    \vc{a} \omega \cos (\omega t) = \vc{v}_A,
    \hspace{0.5cm} 
    - \vc{a} \omega^2 \sin (\omega t) = -\vc{\mathrm{w}}^e.
$$
Тогда по II закону Ньютона для неИСО
\begin{equation}
    m \vc{\mathrm{w}}^r = \vc{F} + m \vc{g} - m \vc{\mathrm{w}}^e.
\end{equation}
Пусть ось $OX$ противонаправлена силе натяжения нити $\vc{F}$, $OY$ в плоскости движения, тогда, введём $\vc{g}' = \vc{g}-\vc{\mathrm{w}}^e$ и получим
\begin{align*}
    &OX: \ \  m \mathrm{w}_x = - F + m g' \cos \varphi  = 0,\\
    &OY: \ \  m \mathrm{w}_y = mg' \sin \varphi,
\end{align*}
Так приходим к уравнению вида
\begin{equation}
    \ddot{\varphi} +  \frac{1}{L} \left(
    \vphantom{\frac{1}{2} }
        g - a \omega^2 \sin \left(\omega t\right)
    \right) \sin (\varphi)= 0.
\end{equation}
В частности, заметим, что $\varphi(t)\equiv0$ и $\varphi(t) \equiv \pi$ являются частными решениями этого уравнения.

\subsubsection*{9.16}

Невесомый стержень вращается с постоянной угловой скоростью $\vc{\omega}$ вокруг оси $Oz$, перпендикулярной плоскости рисунка. По диску катится диск радиуса $r$ и массы $m$, в начальный момент точки $O$ и $A$ совпадали, а диск покоился.

Перейдём в СО стержня, тогда 
\begin{equation}
    m\vc{\mathrm{w}}_d^r =
     m\vc{\mathrm{w}}_d^a - m\vc{\mathrm{w}}_d^c - m\vc{\mathrm{w}}_d^e.
\end{equation}
Так как движение происходит без проскальзывания, сила трения не совершает работу. С учётом II закона Ньютона в неИСО, и тем что сила Кориолиса не изменяет кинетическую энергию системы, получим, что внешний момент
\begin{equation}
    \vc{M}_i = \vc{R}_i \times \left(
        m_i \vc{g} 
        - 
        m_i \vc{\omega} \times \vc{\omega} \times \vc{r}_i
    \right),
    \hspace{0.5cm} 
    \vc{r}_i = \vv{OA} + \vc{R}_i,
\end{equation}
Тогда
$$
    \vc{M}_i =
    m_i \vc{R}_i \times \vv{OA} \omega^2 -
    m_i \vc{R}_i \times \vc{\omega} \left(\vc{\omega} \cdot \vv{OA}\right) + m_i \vc{R}_i \times \vc{g},
$$
Суммируя, по теореме об изменение кинетического момент, получим, что
$$
    I \vc{\varepsilon}_d = \sum \vc{M}_i = m \omega^2  \ \vv{AB} \times \vv{OA} + m \ \vv{AB} \times \vc{g}.
$$
Пусть $L$ -- пройденное расстояние, в проекции на ось, сонаправленную с $\vc{\omega}_d^r$,
$$
    \frac{d \omega_d^r}{dt} =
    \frac{2}{3} \frac{L}{R} \omega^2 + \frac{2}{3} \frac{g}{R} \cos \varphi,
$$
интегрируя, с учётом начальных условий,
\begin{equation}
    (\omega_d^{r})^2 (L) =
    \frac{2}{3} \frac{L^2}{R^2} \omega^2 + \frac{4}{3} \frac{L}{R^2} \, g \cos \varphi.
\end{equation}

Запишем теперь Кориолисову силу, как
$$
    \vc{F}^c_i = - 2 \omega \times (\vc{\omega} \times \vc{r}_i) m_i
    \hspace{0.5cm} \Rightarrow \hspace{0.5cm} 
    \vc{F}^c = - 2 \vc{\omega} \times
    \left(
        \vc{\omega}_d^r \times \vv{AB} 
    \right) \ m.
$$
Догда записав II закон Ньютона на ось, сонаправленную с $\vc{N}$, получим
\begin{equation}
    0 = N - mg\sin\varphi - R m \omega^2  + 2 m \omega \omega_d^r R
    \hspace{0.5cm} \Rightarrow \hspace{0.5cm} 
    {
        N = mg \sin \varphi + Rm \omega^2-
        2 m\omega \sqrt{
            \frac{2}{3} \omega^2 L^2 + \frac{4}{3} g L \cos \varphi 
        }
    }    
\end{equation}
Аналогично записав уравнение в проекции на ось, сонаправленную с $\vc{F}_\text{тр}$, найдём
\begin{equation}
    m {\mathrm{w}} = m \varepsilon R = F_{\text{тр}} + L m \omega^2 + mg \cos \varphi,
    \hspace{0.5cm} \Rightarrow \hspace{0.5cm} 
    {
        F_{\text{тр}} = \frac{m}{3} \left(
            \vphantom{\frac{1}{2}}
            L\omega^2 + g \cos \varphi
        \right)
    }.
\end{equation}



\subsubsection*{9.27}
Посмотрим на систему с точки зрения вращающейся с угловой скоростью $\vc{\omega}_0$ плоскости, тогда по теореме об изменение количества движения в неИСО
\begin{equation}
    m \vc{\mathrm{w}}^r = \vc{R} - m \vc{\mathrm{w}}^e - m \vc{\mathrm{w}}^c.
\end{equation}
Выберем в качестве полюса тела центр масс $A$, тело вращается относительно него с $\vc{\omega}$, тогда 
$$
    - \frac{1}{2} \vc{F}^C  = 
    \sum_i m_i \left(
        \vc{\omega}_0 \times (\vc{v}_A + \vc{\omega} \times \vv{Ci})
    \right) = m \vc{\omega}_0 \times \vc{v}_A +
    \vc{\omega}_0 \times \left(
        \vc{\omega} \times \big(
        \cancel{\sum m_i \vc{r}_i}
        \big)
    \right) = m \vc{\omega}_0 \times \vc{v}_A.
$$
Аналогично для переносной
$$
    -\vc{F}^e = \sum m_i \vc{\omega} \times \vc{v}_i^r =
    \sum_i m_i \vc{\omega} \times \vc{\omega} \times \vc{r}_i =
    \vc{\omega} \times \vc{\omega} \times \left(
        \sum_i m_i \vc{r}_i
    \right) = m \vc{\omega} \times \left(\vc{\omega} \times \vc{r}_A\right).
$$
Таким образом переносные и кориолисовы силы приводятся к равнодействующим, проходящим через центр масс фигуры.
% \subsubsection*{9.32}

% Шарик движется так, что скорость всех его точек параллельны плоскости, которая вращается с угловой скоростью $\omega(t)$ вокруг неподвижной оси, лежащей вэтой плоскости. 

% В плоскости введем координаты так, что
% $$
%     \vc{\omega} = \begin{pmatrix}
%         0 \\ 0 \\ \omega
%     \end{pmatrix},
%     \hspace{0.5cm} 
%     \vc{a} = \begin{pmatrix}
%         \sin \omega t \\
%         \cos \omega t \\
%         0
%     \end{pmatrix},
%     \vc{v} = \begin{pmatrix}
        
%     \end{pmatrix}
% $$
 %-