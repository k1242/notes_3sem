\subsubsection*{14.37}

Переёдём в СО, вращающуюся с $\vc{\omega}$, соотвественно хочется ввести потенциальное поле для сил инерции и гравитационных.
\begin{align*}
    \d F_{\text{ц. б.}} &= \omega^2 x \d m, 
    \hspace{0.5cm} \Rightarrow \hspace{0.5cm} 
    \d \Pi_{\text{и}} = - \frac{1}{2} \omega^2 x^2 \d m.
\end{align*}
Тогда потенциал
\begin{align*}
    \Pi_{\text{g}, 1} &= - \frac{m}{l} \int_0^{l \sin \varphi}
    \frac{1}{2} \omega^2 x^2 \d x  = 
    - \frac{m_1 l^2}{6} \omega^2 \sin^2 \varphi. \\
    \Pi_{\text{g}, 2} &= - \frac{m_2 l^2}{6} \omega^2 \sin^2 \varphi.
\end{align*}
Полная энергия системы:
\begin{equation*}
    \Pi = -gl \cos \varphi \left(m_1 + \frac{3}{2} m_2\right) - \frac{1}{6} \omega^2 l^2 \sin^2 \varphi (m_1 + m_2).
\end{equation*}
Найдём стационарные точки потенциала
\begin{equation*}
    \frac{\partial E}{\partial \varphi} = \frac{1}{2} gl \sin \varphi  (m_1 + 3 m_2)
    - \frac{1}{3} \omega^2 l^2 \sin \varphi \cos \varphi (m_1 + m_2) = 0.
\end{equation*}
Находим положения равновесия:
\begin{equation*}
    \sin \varphi^* = 0, \hspace{0.5cm} \Rightarrow \hspace{0.5cm} \varphi^* = 0, \ \pi.
\end{equation*}
При условии, что rhs следующего уравнения $\leq 1$, найдём также
\begin{equation*}
    \cos \varphi^* = \frac{2g (m_1 + 3 m_2)}{2 \omega^2 (m_1 + m_2)},
    \hspace{0.5cm} 
    \omega^2 \geq \frac{3g (m_1 + 3m_2)}{2l(m_1+m_2)}.
\end{equation*}

