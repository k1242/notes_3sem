\subsubsection*{11.118}

Как и в решение к №11.72 у нас симметричный волчок. Требуется определить начальную угловую скорость прецессии $\dot{\varphi}_0$, чтобы $\dot{\theta}=0$. 
Формально можем поставить задачу несколько иначе, какой должен быть момент внешних сил $\vc{M}_O$ чтобы происходила регулярная прецессия $\dot{\theta}=0$?

Для начала введём отдельно $\vc{\omega}_1 \parallel O\zeta$ и $\vc{\omega}_2 \parallel OZ$.
 По раннее проделанной работе с регулярной прецессией, мы знаем, что $K_z$ и $K_3$ постоянны, соответсвенно $\omega_1, \omega_2, \omega = \const$.
 Аналогично случаю Эйлера (см. №11.59)
 \begin{equation*}
     (K_O)_\zeta = Cr, \hspace{0.5cm} (K_O)_Z = A \sqrt{p^2 + q^2}.
 \end{equation*}
 То есть $\vc{K}_O \in O\zeta Z$ и $K_O=\const$. Но, т.к. плоскость $O\zeta Z$ вращаеся с угловой скорсотью $\vc{\omega}_2$ то и вектор $\vc{K}_O$ аналогично. Тогда для $\vc{M}_O$ верно, что
 \begin{equation}
     \frac{d \vc{K}_O}{d t} = \vc{\omega}_2 \times \vc{K}_O = \vc{M}_O.
 \end{equation}
 Нетрудно показать, что
 \begin{equation*}
     \vc{\omega}_2 \times \vc{K}_O = 
     \frac{\vc{\omega}_2\times \vc{\omega}_1}{\|\vc{\omega}_2\times \vc{\omega}_1\|} \omega_2 \sin \theta
     \left(
        C (\omega_1 + \omega_2 \cos \theta) - A \omega_2 \cos \theta
     \right) 
 \end{equation*}
 Т.к. $\|\vc{\omega}_2 \times \vc{\omega}_1\| = \omega_1 \omega_2 \sin \theta$, то
 \begin{equation}
 \label{ggg}
    \vc{M}_O = 
     (\vc{\omega}_2 \times \vc{\omega}_1) \left[
        C + (C-A) \frac{\omega_2}{\omega_1} \cos \theta
     \right].
 \end{equation}
 Это \textit{основная формула гироскопии}, так что, наверное, можно было принять её на веру. В частном случае, когда $\omega_1 \gg \omega_2$ можно приближенно записать эту формулу, как
 \begin{equation}
     \vc{M}_O = C \left(\vc{\omega}_2 \times \vc{\omega}_1\right).
 \end{equation}

 Конкретно для нашей задачи \eqref{ggg} перепишется как
 \begin{equation*}
     \dot{\varphi} \omega \sin \theta 
     \left(
        C + (C-A) \frac{\dot{\varphi}}{\omega} \cos \theta
     \right) = mgl \sin \theta,
 \end{equation*}
 т.к. мы действительно требуем регулярной прецессии. Так получаем квадратное уравнение вида
 \begin{equation}
     (C-A) \dot{\varphi}^2 \cos \theta + C \omega \vc{\varphi} - mgl = 0
     \hspace{0.5cm} \Rightarrow \hspace{0.5cm} 
     \dot{\varphi} = 
     \frac{
        - C \omega \pm \sqrt{
            C^2 \omega^2 + (C-A) mgl \cos \theta
        }
     }{
        2(C-A) \cos \theta
     }  .
 \end{equation}
Стоит заметить, что при $C^2 \omega^2 + (C-A) mgl \cos \theta < 0$ регулярная прецессия, по всей видимости, невозможна. При $\omega >> \dot{\varphi}$ угловая прецессия будет равна
\begin{equation}
    \dot{\varphi} = \frac{mgl}{C\omega},
\end{equation}
и, как видно, не зависит от угла нутации. 

Теперь про силы. Запишем II закон Ньтона в проекции на вертикаль и нормаль к вертикали, повернутую на $+\varphi$ от $X$, получим
\begin{equation}
    \left\{\begin{aligned}
        N_x = m \dot{\varphi}^2 l \sin \theta \\
        N_z - mg = 0        
    \end{aligned}\right.
    \hspace{0.5cm} \Rightarrow \hspace{0.5cm} 
    N = m\sqrt{g^2 + \dot{\varphi}^2 l^2 \sin^2 \theta}.
\end{equation}