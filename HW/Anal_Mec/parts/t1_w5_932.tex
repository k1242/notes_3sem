% \subsubsection*{9.32}

% Шарик движется так, что скорость всех его точек параллельны плоскости, которая вращается с угловой скоростью $\omega(t)$ вокруг неподвижной оси, лежащей вэтой плоскости. 

% В плоскости введем координаты так, что
% $$
%     \vc{\omega} = \begin{pmatrix}
%         0 \\ 0 \\ \omega
%     \end{pmatrix},
%     \hspace{0.5cm} 
%     \vc{a} = \begin{pmatrix}
%         \sin \omega t \\
%         \cos \omega t \\
%         0
%     \end{pmatrix},
%     \vc{v} = \begin{pmatrix}
        
%     \end{pmatrix}
% $$
