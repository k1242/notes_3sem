\subsubsection*{9.32}

Шарик движется так, что скорость всех его точек параллельны плоскости, которая вращается с угловой скоростью $\omega(t)$ вокруг неподвижной оси, лежащей вэтой плоскости. 

В плоскости введем координаты так, что
$$
    \vc{\omega} = \begin{pmatrix}
        0 \\ 0 \\ \omega
    \end{pmatrix},
    \hspace{0.5cm} 
    \vc{a} = \begin{pmatrix}
        \sin \omega t \\
        \cos \omega t \\
        0
    \end{pmatrix}, \hspace{0.5cm} 
    \vc{v} = \begin{pmatrix}
        v_y (t) \tg \omega t \\
        v_y(t) \\
        v_z (t) \\
    \end{pmatrix}.
$$
По условию 
$$\vc{v} \cdot (\vc{n}) = 0,
\hspace{0.5cm} \Rightarrow \hspace{0.5cm} 
v_x \cos \omega t = v_y \sin \omega t,
$$
где $\vc{n}$ -- нормаль к плоскости, равная, например $
(-\cos \omega t, \ \sin \omega t, \ 0)\T
$.
Знаем, что
\begin{equation}
    \vc{\mathrm{w}}^r = \vc{\mathrm{w}}^a - \vc{\mathrm{w}}^e - \vc{\mathrm{w}}^c,
    \hspace{0.5cm} 
    \vc{v}^r = \vc{v}^a - \vc{v}^c = \vc{v}^a - \vc{\omega}\times \vc{r}
\end{equation}
Тогда
\begin{equation}
    \ddot{\vc{r}} = \vc{\mathrm{w}}^a - 2 \vc{\omega}\times
    (\vc{v}^a - \vc{\omega}\times \vc{r}) + \vc{\omega} \times \left(\vc{\omega} \times \vc{r}\right) + \vc{\varepsilon} \times \vc{r}.
\end{equation}