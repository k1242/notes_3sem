\subsubsection*{11.45}
Твердое тело с неподвижной точкой движется под действием момента 
\begin{equation*}
    \vc{M}_O = \vc{a} \times \vc{\omega},
\end{equation*}
где вектор $\vc{a}$ вращается вместе с твёрдым телом.
Хотим перейти к динамическим уравнениям Эйлера, так что
\begin{equation*}
    \hat{J}_O = \dmat{3}{A}{B}{C}, \hspace{0.5cm} 
    \vc{\omega} = \begin{pmatrix}
        p \\ q \\ r
    \end{pmatrix},
    \hspace{0.5cm} 
    \vc{a} = \begin{pmatrix}
        a_\xi \\ a_\eta \\ a_\zeta
    \end{pmatrix}, 
    \hspace{0.5cm} 
    \vc{M}_O = \begin{pmatrix}
        a_\eta r - a_\zeta q \\
        a_\zeta p - a_\xi r \\
        a_\xi r - a_\eta p
    \end{pmatrix}.
\end{equation*}
Для начала попробуем в лоб, домножив динамические уравнения эйлера на $p, q, r$ соответсвенно
\begin{equation*}
    \left\{\begin{aligned}
        A \dot{p} + (C-B) q r &= - M_{\xi} \\
        B \dot{q} + (A-C) p r &= - M_{\eta} \\
        C \dot{r} + (B-A) p q &= - M_{\zeta} \\
    \end{aligned}\right.,
    \hspace{0.5cm} \Rightarrow \hspace{0.5cm}
    A \dot{p} p + B \dot{q} q + C \dot{r} r = 0.
\end{equation*}
Не густо. 

Пойдём в чуть более низкоуровневую запись
\begin{equation*}
    \frac{d \vc{K}_O}{d t} = \dot{K}_{Oi} \vc{e}_i + \vc{\omega} \times \vc{K}_O = \vc{a} \times \vc{\omega}.
\end{equation*}
Т.к. $\dot{a}_{i} \vc{e}_i = 0$, то
\vspace{-1mm}
\begin{equation*}
    (\dot{K}_{Oi} + \dot{a}_{i}) \vc{e}_i + \vc{\omega} \times (\vc{K}_O+\vc{a}) = 0
    \hspace{0.25cm} \Rightarrow \hspace{0.25cm} 
    \frac{d }{d t} \left(\vc{K}_O + \vc{a}\right) = 0
    \hspace{0.25cm} \Rightarrow \hspace{0.25cm} 
    \boxed{
       \vc{K}_O + \vc{a}   = \const
    } \text{ --- первый I интеграл}.
\end{equation*}
Теперь, т.к. $\vc{\omega} \bot \vc{M}_O$ предположим, что $T = \const$. Действительно
\begin{equation*}
    d T = \partial A = \cancel{\left(
        \vc{R} \cdot \vc{v}_O
    \right)} \d t +
    \cancel{\left(\vc{M})_O \cdot \vc{\omega}\right)} \d t = 0
    \hspace{0.5cm} \Rightarrow \hspace{0.5cm} 
    \boxed{
        T = \const
    }  \text{  -- второй I интеграл.}
\end{equation*}