\subsubsection*{Задача №4 (I)}

Запишем кинетический момент относительно точки $A$,
\begin{equation}
    \vc{K}_A = \sum_i \vc{r}_{Ai} \times (m_i \vc{v}_i),
\end{equation}
и выразим скорость, как
$$
    \vc{v}_i = \vc{\omega}_0 \times  \vc{r}_{Ai} + \vc{\omega} \times \left(
    \vc{l} + \vc{r}_{Ai}
    \right),
$$
где $\vc{l}$ -- радиус вектор от $O$ до $A$, $\vc{\omega}_0 \colon \vc{\omega}_0 \times \vc{r} = \vc{v}$, $\vc{r}$ -- вектор от $A$ до центра масс $B$.
Тогда
$$
    \vc{K}_A = 
    \underbrace{
        \sum_i m_i \vc{r}_{Ai} \times \vc{\omega}_0 \times \vc{r}_{Ai}
    }_{K_1}
    \ + \
    \underbrace{
        \left(\sum_i \vc{r}_{Ai} m_i\right) \times \vc{\omega} \times \vc{l}
    }_{K_2}
    \ + \
    \underbrace{
        \sum_i m_i \vc{r}_{Ai} \times \vc{\omega} \times \vc{r}_{Ai}
    }_{K_3}.
$$
В частности, раскрывая двойное векторное,
$$
    \vc{K}_1 = \vc{\omega}_0 \sum_i m_i r_{Ai}^2 = \frac{3}{2} m r^2 \vc{\omega}_0,
    \hspace{0.5cm} 
    \vc{K}_2 = \vc{r} \times \vc{\omega} \times \vc{l}.
$$
Третье слагаемое, аналогично первому,
$$
    \vc{K}_3 = \vc{\omega} \frac{3}{2} m r^2 -
    \underbrace{
    \sum_i m_i \vc{r}_{Ai} \left(\vc{r}_{Ai} \cdot \vc{\omega}\right)
    }_{K_4}
    ,
$$
где последнее слагаемое сохранит только $y$ компонену $\parallel \vc{i}$ (где $\vc{i}$ -- единичный вектор), соответсвенно
$$
    \vc{K}_4 = \vc{i} \omega \int_{0}^{2r} \rho(h) h^2 \d h,
$$
где
$$
    \rho(h) \colon \hspace{0.5cm} 
    \int_{0}^{2r} 
    \underbrace{
    \alpha \sqrt{2 Rh - h^2} 
    }_{\rho(h)}
    \d h 
    = m.
$$



\subsubsection*{Задача №4 (II)}

Теперь относительно точки $O$:
По всей видимости речь о 
$$
    \vc{K}_O^r = \sum_i \vc{r}_i \times (m_i \vc{v}^r_i) =
    \sum_i \left(\vc{l} + \vc{r}_{Ai}\right) \times (m_i \vc{v}^r_i)
    ,
$$
где
$$
    \vc{v}^r_i = \vc{\omega}_0 \times \vc{r}_{Ai}.
$$
Тогда
$$
    \vc{K}_O^r = 
    \vc{l} \times \vc{\omega}_0 \times \left(\sum_i m_i \vc{r}_i \right)
    +
    \sum_i \vc{r}_{Ai} m_i \times \vc{\omega}_0 \times \vc{r}_{Ai}
    =
    \cancel{
        \vc{l} \times \vc{\omega}_0 \times \vc{r}
    } +
    \vc{\omega}_0 \frac{3}{2} mr^2 - 
    \sum_i m_i
    \vc{r}_{Ai} \cancel{\left(\vc{r}_{Ai} \cdot \vc{\omega}_0\right)}
    .
$$
Тогда
$$
    \vc{K}_O^r = 
    \vc{\omega}_0 \frac{3}{2} mr^2.
$$