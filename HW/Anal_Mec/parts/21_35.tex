\subsubsection*{21.35}

Хотелось бы от действия $S$ вида
\begin{equation*}
    S = \int_A^B L \d t, \hspace{0.5cm} 
    L = T - \Pi = \frac{1}{2} p_i \dot{q}^i - \Pi
\end{equation*}
к действию (или \textit{укороченному действию}) $\delta S^* = 0$, где $S^*$ вида
\begin{equation}
\label{sample}
    S^* = \int_A^B n \d s, \hspace{0.5cm} 
    \phantom{
    L = T - \Pi = \frac{1}{2} p_i \dot{q}^i - \Pi
    }
\end{equation}
где под интегрирование от $A$ до $B$ подразумевается интегрирование интегрирование уравнение от состояния в точке $A$ до состояния в точке $B$. Можно было бы сразу получить ответ из принципа Мопертюи, так что давайте его выведем. 

Перейдём к энергии системы, как функции $p$ и $q$, где $p_i = \partial L / \partial \dot{q}^i$ -- обобщенный импульс. Тогда
\begin{equation}
\label{I2135}
    dS = L \d t = \left(p_i  \dot{q}^i - H\right) \d t
    \hspace{0.5cm} \Rightarrow \hspace{0.5cm} 
    S  = S_0 - H \cdot(t_B-t_A), 
\end{equation}
так как мы рассматриваем аналогию с консервативной системой, то есть $\dot{H} = 0$. Величина $S_0$ -- \textit{укороченное действие},
\begin{equation*}
    S_0 = \int_A^B p_i \dot{q}^i \d t = 
    2 \int_A^B (H - \Pi) \d t.
\end{equation*}
Найдём $dt$, как
\begin{equation*}
    dt = \frac{ds}{v},
    \hspace{0.25cm} 
    v^2 = 2 (H-\Pi) / m
    \hspace{0.5cm} \Rightarrow \hspace{0.5cm} 
    dt = \frac{ds}{
    \sqrt{2(H-\Pi) / m}}.
\end{equation*}
Собирая всё вместе, получаем
\begin{equation*}
    S_0 = \int_A^B\sqrt{2m(H-\Pi)} \d s.
\end{equation*}
Вернёмся к варьированию. Если допускать варьирование конечного момента времени, то
\begin{equation}
\label{II2135}
    \delta S = \frac{\partial S}{\partial t}  \delta t = -H \delta t, \hspace{0.5cm} \Rightarrow \hspace{0.5cm} 
    \delta S + H \delta t = 0.
\end{equation}
Подставляя \eqref{I2135} в \eqref{II2135}, получим, что
\begin{equation}
    \delta S_0 = 0, 
    \hspace{0.5cm} \Rightarrow \hspace{0.5cm} 
    \delta \left(
        \int_A^B\sqrt{2m(H-\Pi)} \d s
    \right) = 0.
\end{equation}
Сравнивая полученное выражение с \eqref{sample}, полагая $m =1$, находим
\begin{equation}
    \Pi = -\frac{n^2}{2} + H.
\end{equation}

