\subsubsection*{11.72}

Решим чуть более общую задачу о движении тяжелого симметричного волчка с неподвижней нижней точкой. Начало координат $O$ совпадает с неподвижной точкой волчка, расстояние до центра масс равно $l$. 

Запишем кинематические и динамические уравнения Эйлера
\begin{equation*}
    \left\{\begin{aligned}
        p &= \dot{\varphi} \sin \theta \sin \psi + \dot{\theta} \cos \psi,\\
        q &= \dot{\varphi} \sin \theta \cos \psi - \dot{\theta} \sin \psi,\\
        r &= \dot{\varphi} \cos \theta + \dot{\psi}.
    \end{aligned}\right.,
    \hspace{0.6cm} 
    \left\{\begin{aligned}
        I_1 \dot{p} + (I_3-I_2) q r &= - M_{\xi} \\
        I_2 \dot{q} + (I_1-I_3) p r &= - M_{\eta} \\
        I_3 \dot{r} + (I_2-I_1) p q &= - M_{\zeta} \\
    \end{aligned}\right.,   
    \hspace{0.6cm} 
    \hat{J}_O = \dmat{3}{I_1}{I_1}{I_3}, \hspace{0.6cm} 
    \vc{\omega} = \begin{pmatrix}
        p \\ q \\ r
    \end{pmatrix}.
\end{equation*}
Кинетическая энергия волчка (с учетом параллельного переноса тензора инерции с центра масс к точке $O$)
\begin{equation*}
    T = \vc{\omega}\T \hat{J}_O \vc{\omega} = 
    \frac{I_1+ml^2}{2} \left(\dot{\theta}^2+\dot{\varphi}^2\sin^2 \theta\right) + \frac{1}{2} I_3 \left(\dot{\psi}+ \dot{\varphi} \cos \theta \right)^2.
\end{equation*}
Потенциальная энергия, соответсвенно, равна
\begin{equation*}
    \Pi = mgl \cos \theta.
\end{equation*}
Собирая вместе, находим
\begin{equation*}
    L = T - \Pi.
\end{equation*}
Понятно, что $K_3 = \const$, докажем также что $K_z = \const$. Действительно,
\begin{equation*}
    \frac{d K_z}{d t} = {\vc{M}_A}\bigg|_Z + \cancel{\vc{Q} \times \vc{v}_O} = 0
    \hspace{0.5cm} \Rightarrow \hspace{0.5cm} 
    K_z = \const.
\end{equation*}
Явно выпишем их
\begin{equation}
\label{ll1}
    \left\{\begin{aligned}
        K_3 &= {\partial L}/{\partial \dot{\psi}} = I_3\left(\dot{\psi} + \dot{\varphi} \cos \theta\right) \\
        K_z &= {\partial L}/{\partial \dot{\varphi}} = 
        \left((I_1+ml^2) \sin^2 \theta + I_3 \cos^2 \theta\right) \dot{\varphi} + I_3 \dot{\psi} \cos \theta.
    \end{aligned}\right.
\end{equation}
Кроме того, в системе сохраняется энергия
\begin{equation*}
    E = T + \Pi = \frac{1}{2} (I_1 + ml^2) \left(\dot{\theta}^2+\dot{\varphi}^2 \sin^2 \theta\right) 
    + \frac{1}{2} I_3 \left(\dot{\psi} + \dot{\varphi} \cos \theta\right)^2 + mgl\cos \theta.
\end{equation*}
Из \eqref{ll1} находим явные выражения для $\dot{\varphi}$ и $\dot{\theta}$ 
\begin{align*}
    \dot{\varphi} &= \frac{K_z - K_3 \cos \theta}{(I_1 + ml^2)\sin^2\theta} ,\\
    \dot{\psi} &= \frac{K_3}{I_3}  - \cos \theta \frac{K_z - K_3 \cos \theta}{(I+ml^2)\sin^2 \theta} .
\end{align*}
Подставляя это в выражения для энергии $E$ получим
\begin{equation}
    E  = \frac{1}{2} (I_1 + ml^2) \dot{\theta}^2 +
    \frac{
        (K_z - K_3 \cos \theta)^2
    }{
        2 (I_1 + ml^2) \sin^2 \theta
    } + \frac{K_3^2}{2I_3} + mgl \cos \theta.
\end{equation}
Таким образом мы находим 
\begin{equation}
    \dot{\theta} = \frac{d \theta}{d t} = f_{\dot{\theta}}(E, K_z, K_3)
    \hspace{0.5cm} \Rightarrow \hspace{0.5cm} 
    t = \int_{\theta_0}^\theta \frac{d \theta}{f_{\dot{\theta}}(E, K_z, K_3)},
\end{equation}
что и является нашим искомым решением в квадратурах. Конкретно для №11.72 следует положить $I_3=0$ и, в силу доступного для стержня произволя, $\dot{\psi}=0$. Слагаемые вида $K_3/I_3$ в таком случае просто не возникнут, решение сохранится.















