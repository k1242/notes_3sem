\subsubsection*{21.14 и 20.15}

Точка массы $m$ може двигаться по гладкой вертикальной плоскости $xz$, вращающейчя вокруг векртикальной оси $Oz$ с постоянной угловой скоростью $\omega$. 

Лагранжиан системы
\begin{equation}
    L = \frac{1}{2} m \left(
        \dot{x}^2 + \dot{z}^2
    \right) + 
    \frac{1}{2} m x^2 \omega^2.
\end{equation}
Вариация действия
\begin{equation*}
    \delta S = m \int_{A}^{B} \left(
        \dot{x} \delta x + \dot{z} \delta z + \omega^2 x \delta x - g \delta z
    \right) \d t.
\end{equation*}
Посмотрим на действие
\begin{align*}
    S 
    &= \int_{A}^{B} 
    L(x+\delta x, z + \delta z, t) \d t = \\
    &= \int_{A}^{B}
    L(x, z, t) \d t + 
    \underbrace{
    m \int_{A}^{B} \dot{x} \delta \dot{x} + \dot{z} \delta \dot{z} + \omega^2 x \delta x - g \delta z \d t
    }_{\delta S (L(x, z, t)) = 0} +
    \frac{1}{2} m \int_{A}^{B}
     (\delta \dot{x})^2 + \left(\delta \dot{z}\right)^2 + \omega^2
     (\delta x)^2 \d t, \hspace{0.5cm} \text{Q.E.D.}
\end{align*}
