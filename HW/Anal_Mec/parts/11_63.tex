\subsubsection*{11.63}

Для начала поймём куда диск движется, точнее найдем (или хотя бы сделаем шаги в эту сторону) мгновенную ось вращения проходящую через точку $A$ и некоторую точку $C$.

Для начала посмотрим на геометрию системы (введя неизвестные $a, b, c$):
\begin{equation*}
    \vv{AB} = \begin{pmatrix}
        -r \\ -r \\ 0
    \end{pmatrix},
    \hspace{0.25cm} 
    \vv{AD} = \begin{pmatrix}
        r \\ -r \\ 0
    \end{pmatrix},
    \hspace{0.25cm} 
        \vv{CA} = \begin{pmatrix}
        a \\ b \\ c
    \end{pmatrix},
    \hspace{0.5cm} 
    \left\{\begin{aligned}
    \vc{v}_A &= \vc{\omega} \times \vv{CA} = 0\\
    \vc{v}_B &= \vc{\omega} \times \vv{CB} \\
    \vc{v}_A &= \vc{\omega} \times \vv{CD}
    \end{aligned}\right.,
    \hspace{0.5cm} 
    \left\{\begin{aligned}
        \vv{CD} &= \vv{CB} + \vv{BD} \\
        \vv{CD} &= \vv{CA} + \vv{AD} \\
        \vv{CB} &= \vv{CA} + \vv{AB}
    \end{aligned}\right.,
\end{equation*}
Для удобства далее будем считать $\vc{\omega} = {k} \vv{CA}$. Посчитаем векторы скоростей в нашеих обозначениях
\begin{equation*}
    \vc{v}_D = kr \begin{pmatrix}
        c \\ c \\ -b - a
    \end{pmatrix}
    \vc{v}_B = k \vv{CA} \times \vv{AB} = k r
    \begin{pmatrix}
        c \\ - c \\ b-a
    \end{pmatrix}
    \hspace{0.5cm} \Rightarrow \hspace{0.5cm} 
    a = b
\end{equation*}
Так как мы знаем абсолютные значения скоростей точек, то запишем
\begin{equation*}
    v_D^2 - v_B^2 = v_0^2 = 4 a^2 k^2
    \hspace{0.5cm} \Rightarrow \hspace{0.5cm} 
    a = \frac{v_0}{2rk}.
\end{equation*}
Подставив теперь значения $a$ в $v_B^2$ получим
\begin{equation*}
    v_B^2 = 2 k^2 c^2 r^2 = v_0^2
    \hspace{0.5cm} \Rightarrow \hspace{0.5cm} 
    c = \frac{v_0 \sqrt{2}}{2kr},
    \hspace{0.5cm} \Rightarrow \hspace{0.5cm} 
    \vc{\omega} = k \begin{pmatrix}
        a \\ b \\ c
    \end{pmatrix} = \frac{v_0}{2r} \begin{pmatrix}
        1 \\ 1 \\ \sqrt{2}
    \end{pmatrix}
\end{equation*}
Теперь найдём скорость центра масс
\begin{equation*}
    \vc{v}_O = \vc{\omega} \times \vc{r}_{CO} = 
    \vc{\omega} \times \left(
        \vc{r}_{CA} + \vv{AO}
    \right) = \vc{\omega} \times \begin{pmatrix}
        0 \\ - r \\ 0
    \end{pmatrix} = 
    \frac{v_0}{2} \begin{pmatrix}
        -\sqrt{2} \\ 0 \\ 1
    \end{pmatrix},
    \hspace{0.5cm} \Rightarrow \hspace{0.5cm} 
    \vc{r}_O(t) = 
    \begin{pmatrix}
        v_0 t / \sqrt{2} \\
        0 \\
        -gt^2/2+v_0 t / 2
    \end{pmatrix}.
\end{equation*}
Теперь мы знаем как будет двигаться в условиях гравитации наш диск (его центр масс)! 

Теперь посмотрим на вращение диска относительно центра масс. Для этого пересядем в СО падающую с $\vc{g}$, теперь $\vc{M}_O = \vc{0}$ и мы пришли к случаю Эйлера (который подробно был рассмотрен в задаче №11.59).

Для начала вспомним, что для диска кинетический момент
\begin{equation*}
    \vc{K}_O = \hat{J}_O \vc{\omega} = 
    \frac{m r^2}{4} \dmat{3}{1}{1}{2} 
    \begin{pmatrix}
        1 \\ 1  \\ \sqrt{2} 
    \end{pmatrix}
    \frac{v_0}{2r}  = \frac{mrv_0}{8} \begin{pmatrix}
        1 \\ 1 \\ 2 \sqrt{2}
    \end{pmatrix} = \const,
    \hspace{0.5cm} \Rightarrow \hspace{0.5cm} 
    K_O = \frac{\sqrt{10}}{8} mrv_0.
\end{equation*}
Зная $\vc{K}_O$ можем найти ось прецессии $\vc{e} \parallel \vc{K}_O$
\begin{equation*}
    \vc{e} = \frac{1}{\sqrt{10}} \left(1, \ 1, \ 2 \sqrt{2}\right).
\end{equation*}
Подставляя параметры системы в уравнения \eqref{regp}, найдём
\begin{equation*}
    \dot{\psi} = \frac{K_O}{A} = \frac{\sqrt{10}}{2} \frac{v_0}{r} ,
    \hspace{1cm} 
    \dot{\varphi} = r_0 \left(1 - \frac{C}{A} \right) =  
    -\frac{v_0 \sqrt{2}}{2r},
    \hspace{1cm} 
    \cos \theta = \frac{Cr_0}{K_O} = \frac{2\sqrt{5}}{5}.
\end{equation*}
