\subsubsection*{Т17.}

Рассмотрим движение точки по цилиндру радиуса $r_0$. Тогда $L$ 
\begin{equation*}
    L  = T - \Pi = T = \frac{1}{2} m v^2 = \frac{1}{2} m \dot{q}_i \dot{q}^i
    = \frac{1}{2} m g_{ij} \dot{q}^i \dot{q}^j
    = \frac{1}{2} \left(r_0^2 \dot{\varphi}^2 + \dot{z}^2\right)
    .
\end{equation*}
Тогда вариация действия для системы (свободной материальной точки)
\begin{equation*}
    \frac{d }{d t} (\dot{x} \delta x) = \dot{x} \delta x + \dot{x} \delta \dot{x}
    \hspace{0.25cm} \Rightarrow \hspace{0.25cm} 
    \delta S =
    m \int_A^B 
    \left( 
        r_0^2 \dot{\varphi} \delta \dot{\varphi} + \dot{z} \delta \dot{z}
     \right)
    \d t = 
    m (r_0^2 \dot{\varphi} \delta \varphi + \dot{z} \delta z)\bigg|_A^B
    + 
    \int_A^B 
    \left(
        -r_0^2 \ddot{\varphi} \delta \varphi - \ddot{z} \delta z
    \right) \d t = 0.
\end{equation*}
Вариация на $A$ и $B$ тождественно равна 0, в силу прозвольности $\delta z$ и $\delta \varphi$ получаем, что
\begin{equation*}
    \left\{\begin{aligned}
        \ddot{\varphi} &= 0, \\
        \ddot{z} &= 0
    \end{aligned}\right.
    \hspace{0.5cm} \Rightarrow \hspace{0.5cm} 
    \left\{\begin{aligned}
        \varphi &= C_1 t + C_2 \mod 2 \pi, \\
        z &= C_3 t + C_4.
    \end{aligned}\right.
\end{equation*}
так как в силу выбора $\varphi$ верно, что $\varphi + 2 \pi k = \varphi \ \  \forall k \in \mathbb{Z}$. В таком случае
\begin{equation*}
    C_1 = \frac{\varphi_B - \varphi_A}{t_B - t_A} + \frac{2\pi}{t_B - t_A} k,
    \hspace{0.5cm} \forall k \in \mathbb{Z},
\end{equation*}
таким образом для свободной материальной точки существует счётное количество истинных путей для перемещения из $A$ в $B$ за фиксированное время $t_B - t_A$.
