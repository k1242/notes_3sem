\subsection*{Задача №3}

Аналогично предыдущей задачи, найдём уравнения движения системы (рис. \ref{cw3}) в форме Лагранжа. Выберем в качестве начала координат правый верхний угол системы,выбрав оси, как показано на рисунке. По всей видимости будем считать в угла стержни <<прикрепленными>> к опорам так, чтобы не было отрыва -- иначе задача не имеет смысла. 

\begin{figure}[h]
    \centering
    \incfig{cw3}
    \caption{К задаче №3}
    \label{cw3}
\end{figure}

Так как на платформу действует постоянная сила $\vv{F}$, потенциальная энергия системы
\begin{equation*}
    \vc{\tilde g} = \vc{g} - \frac{\vv{F}}{m_1+m_2}  = \vc{g} + 3 \vc{g} = 4 \vc{g},
    \hspace{0.5cm} 
    \Pi = - (m_1 + m_2) \vc{\tilde g} y_1.
\end{equation*}
Координаты диска и платформы, соотвественно
\begin{equation*}
    \vc{r}_{\text{д}} = \vc{r}_1 = \begin{pmatrix}
        x_1 \\ y_1
    \end{pmatrix} = 
    \begin{pmatrix}
        l_2 \cos \varphi \\
        l_2 \sin \varphi.
    \end{pmatrix},
    \hspace{0.5cm} 
    \vc{r}_{\text{п}} = \vc{r}_1 = \begin{pmatrix}
        0 \\
        l_2 \sin \varphi + R
    \end{pmatrix}.
\end{equation*}
Из геометрии системы, можем понять, что
\begin{equation*}
    l_2 / 2a = \cos \varphi, \hspace{0.5cm} \Rightarrow \hspace{0.5cm} 
    \left\{\begin{aligned}
        x_1 = 2 a \cos^2 \varphi \\
        y_1 = a \sin (2\varphi).
    \end{aligned}\right.
\end{equation*}
Дифференцируя по времени, найдём скорость диска:
\begin{equation*}
    \dot{\vc{r}}_1 = 2 a \dot{\varphi} \begin{pmatrix}
        - \sin 2 \varphi \\
        \phantom{-} \cos 2 \varphi
    \end{pmatrix}, \hspace{0.25cm} 
    \omega R = \dot{x}_1
    \hspace{0.5cm} \Rightarrow \hspace{0.5cm} 
    \omega^2 = \frac{1}{R^2} 4 a^2 \dot{\varphi}^2 \sin^2 2 \varphi.
\end{equation*}
Кинетическая энергия диска, как сумма поступательной и вращательной
\begin{equation*}
    T_1 = \frac{1}{2} \left(\frac{m_1 R^2}{2} \right) \frac{1}{R^2} 4 a^2 \dot{\varphi}^2 \sin^2 2 \varphi + 2 a^2 \dot{\varphi}^2 m_1.
\end{equation*}
Для платформу кинетическая энергия, соотвественно
\begin{equation*}
    T_2 = \frac{m_2}{2} \dot{y}_1^2.
\end{equation*}
К огромному счастью трудящихся полная кинетическая энергия системы оказывается равна
\begin{equation*}
    T = T_1 + T_2 = 6 m_2 a^2 \dot{\varphi}^2.
\end{equation*}
Лагранжиан системы
\begin{equation}
    L = T - \Pi = 6m_2 a\left(
        a \dot{\varphi}^2 + 2 g \sin 2 \varphi
    \right).
\end{equation}
Запишем уравнение Лагранжа второго рода:
\begin{equation}
    \frac{d}{dt} \frac{\partial L}{\partial \dot{\varphi}} - \frac{\partial L}{\partial \varphi} = 0, \hspace{0.5cm} \Rightarrow \hspace{0.5cm} 
    \ddot{\varphi} = \frac{2g}{a} \cos (2 \varphi).
\end{equation}
Решение этого дифференциального уравнения в терминах аналитических функций не представляется возможным, однако можем посмотреть на происходящее вблизи положения равновесия:
\begin{equation*}
    \ddot{\varphi} = 0, \hspace{0.5cm} \Leftrightarrow \hspace{0.5cm} 
    \varphi = \pi / 4.
\end{equation*}
В таком случа рассмотрим $\alpha \colon 2 \varphi = \pi/2 + \alpha$,
\begin{equation}
\boxed{
     \ddot{\alpha} = - \omega^2 \sin \alpha,   
}
    \hspace{0.5cm} \omega^2 = \frac{4g}{a}.
\end{equation}
И, при малых $\alpha$, получили гармонический осциллятор:
\begin{equation*}
    \alpha = C_1 \sin \omega t + C_2 \cos \omega t.
\end{equation*}
Из начальных условий (при $t=0$ $\varphi = \varphi_0$), находим, что
\begin{equation*}
    \varphi(t) = \frac{\pi}{4} + \left(\varphi_0 - \frac{\pi}{4} \right) \cos \omega t,
    \hspace{0.5cm} 
    \dot{\varphi} = \left(\frac{\pi}{4} - \varphi_0\right) \omega \sin \omega t.
\end{equation*}
Если нас интересует угловая скорость $\dot{\varphi}$ в момент, когла $\varphi = \varphi_1$, то
\begin{equation}
    \dot{\varphi}(t_1) = \pm 2 \sqrt{\frac{g}{a} } \cdot \sqrt{
        \left(\varphi_1 + \varphi_0 - \frac{\pi}{2}\right) (\varphi_1 - \varphi_0)
    },
    \hspace{0.5cm} 
    0 < \varphi_0 - \pi/4 < \varphi_1 - \pi/4 \ll 1.
\end{equation}