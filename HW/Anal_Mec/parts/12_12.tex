\subsubsection*{12.12}

Найдём уравнения движения для двух материальных точек, массами $m_1$ и $m_2$, притягивающихся по закону Ньютона. В качестве обобщенных координат выберем $x, y, z$ центра масс системы, расстояние между точками $r$ и углы $\varphi, \theta$, определяющие направление прямой. 

Потенциальная энергия системы $\Pi$ 
\begin{equation*}
    \Pi = - \gamma \frac{m_1 m_2}{r}.
\end{equation*}
Для каждого из тел можем записать расстояние до центра масс и абсолютное положение:
\begin{equation*}
    r_1 = \frac{m_2}{m_1+m_2} r, \hspace{0.5cm} 
    r_2 = \frac{m_1}{m_1+m_2} r,
    \hspace{0.5cm} 
    \left\{\begin{aligned}
        x_1 &= x + r_1 \sin \theta \cos \varphi, \\
        y_1 &= y + r_1 \sin \theta \cos \varphi, \\
        z_1 &= z + r_1 \cos \theta.
    \end{aligned}\right.
    \hspace{0.5cm} 
    \left\{\begin{aligned}
        x_2 &= x + r_2 \sin \theta \cos \varphi, \\
        y_2 &= y + r_2 \sin \theta \cos \varphi, \\
        z_2 &= z + r_2 \cos \theta.
    \end{aligned}\right.
\end{equation*}
Вспомнив, что для сферических координат$(r,  \theta, \varphi)$ метрический тензор $g_{ij} = \diag \left({1}, \ {r^2}, \ {r^2 \sin^2 \theta} \right)$, найдём  квадрат относительной скорости
\begin{equation*}
    v_1^2(r_1) = g_{ij} \dot{q}^i \dot{q}^j = 
    r_1^2 \sin^2 \theta \dot{\varphi}^2 + r_1^2 \dot{\theta}^2 + \dot{r_1}^2 
    \hspace{0.5cm} \Rightarrow \hspace{0.5cm} 
    v_1^2(r_1) = \left(
    \frac{m_2}{m_1+m_2} 
    \right)^2 \cdot v_1^2 (r).
\end{equation*}
Теперь можем записать кинетическую энергию движения ($T_1, T_2$ -- кинетические энергии движения тел относительно центра масс) :
\begin{equation*}
    T_1 + T_2 + \frac{1}{2} (m_1+m_2) \left(\frac{d}{dt} (x, \ y, \ z)\right)^2 =
    \frac{1}{2} \frac{m_1 m_2}{m_1 + m_2} 
    \left(
        r^2 \sin^2 \theta \dot{\varphi}^2 + r^2 \dot{\theta}^2 + \dot{r}^2
    \right) + 
    \frac{1}{2} (m_1+m_2) \left(\dot{x}^2 + \dot{y}^2 + \dot{z}^2\right).
\end{equation*}
И, наконец, лагранжиан системы
\begin{equation}
    L = T - \Pi = \frac{1}{2} \frac{m_1 m_2}{m_1 + m_2} 
    \left(
        r^2 \sin^2 \theta \dot{\varphi}^2 + r^2 \dot{\theta}^2 + \dot{r}^2
    \right) + 
    \frac{1}{2} (m_1+m_2) \left(\dot{x}^2 + \dot{y}^2 + \dot{z}^2\right)
    + \gamma \frac{m_1 m_2}{r}.
\end{equation}
Найдём уравнения движения системы относительно центра масс:
\begin{equation}
    \left\{\begin{aligned}
        \frac{d }{d t} \frac{\partial L}{\partial \dot{\varphi}} - \frac{\partial L}{\partial \varphi} &= 0 \\
        \frac{d }{d t} \frac{\partial L}{\partial \dot{r}} - \frac{\partial L}{\partial r} &= 0 \\
        \frac{d }{d t} \frac{\partial L}{\partial \dot{\theta}} - \frac{\partial L}{\partial \theta} &= 0.
    \end{aligned}\right.
    \hspace{0.5cm} \Rightarrow \hspace{0.5cm} 
    \left\{\begin{aligned}
        \dot{\varphi} \left(
        r  \dot{\theta} \sin (2 \theta) + \dot{r} \dot{\varphi} (1 - \cos 2 \theta)
    \right) 
    + \frac{1}{2} r \ddot{\varphi} \left(
         1 - \cos 2 \theta
    \right) = 0, 
    \\
        \gamma (m_1 + m_2) - r^3
    \left(
        \sin^2 \theta \dot{\varphi}^2 + \dot{\theta}^2
    \right) + r^2 \ddot{r} = 0,
    \\
    2 \dot{\theta} \dot{r} + r \ddot{\theta} - \frac{1}{2} r \sin (2 \theta) \dot{\varphi}^2 = 0.
    \end{aligned}\right.
\end{equation}
И для центра масс:
\begin{equation}
    \ddot{x} = 0, \hspace{0.5cm} 
    \ddot{y} = 0, \hspace{0.5cm} 
    \ddot{z} = 0.
\end{equation}
Что логично, на центр масс не действует никаких сил.

Теперь к интегралам системы. Пусть $\frac{d }{d t} (x_1, y_1, z_1)\T = \vc{v}_1$, аналогично для второго тела. Во-первых сохраняется количество движения системы
($x, \ y, \ z$ не входят явно в $L$), также не входят $t, \ \varphi$, тогда
\begin{equation}
    \left\{\begin{aligned}
        \frac{d }{d t} \frac{\partial L}{\partial x} &= 0 \\
        \frac{d }{d t} \frac{\partial L}{\partial \varphi} &= 0 \\
        L &\neq L(t)
    \end{aligned}\right.    
    \hspace{0.5cm} \Rightarrow \hspace{0.5cm} 
    \left\{\begin{aligned}
        m_1 \vc{v}_1 + m_2 \vc{v}_2 &= \const, \\
        r^2 \dot{\varphi} \sin^2 \theta  &= \const, \\
        E = \Pi + T &= \const.
    \end{aligned}\right.
\end{equation}
Вообще, в силу отсутсвия внешних сил на систему, сохраняется кинетический момент,
\begin{equation}
    \vc{K} = m_1 \vc{r}_1 \times \vc{v}_1 + 
    m_2 \vc{r}_2 \times \vc{v}_2 = \const.
\end{equation}
