\subsection*{Задача №2}

Для системы, представленной на рис. \ref{cw_t2_2} найдём лагранжиан. Потенциальная энергия системы (считая, что в $(0,0)$ находится тело массы $M \gg m$),
\begin{equation*}
    \Pi = - \frac{2GMm}{r} \overset{\mathrm{def} \varkappa}{=} - \varkappa \frac{m}{r} .
\end{equation*}
Положение массивных тел, из геометрии системы, можем записать, как
\begin{equation*}
    r_{1,\,2} = r \begin{pmatrix}
        \cos \varphi \\
        \sin \varphi 
    \end{pmatrix} 
    \pm l \begin{pmatrix}
        - \cos (\varphi + \psi) \\
        \phantom{-} \sin (\varphi + \psi)
    \end{pmatrix}.
\end{equation*}
Кинетиечская энергия системы теперь может быть записана как
\begin{equation*}
    T = \frac{m}{2} \left(
        \dot{r}^2_1 + \dot{r}^2_2
    \right),
\end{equation*}
где
\begin{equation}
\left\{\begin{aligned}
    \dot{r}_1 = \dot{r} \begin{pmatrix}
        \cos \varphi \\
        \sin \varphi  
    \end{pmatrix} + r \begin{pmatrix}
        - \sin \varphi \\
        \phantom{-}  \cos \varphi
    \end{pmatrix}
    \dot{\varphi} + 
    l \begin{pmatrix}
        \sin (\varphi + \psi) \\
        \cos (\varphi + \psi) \\
    \end{pmatrix} (\dot{\varphi} + \dot{\psi}), \\
    \dot{r}_2 = \dot{r} \begin{pmatrix}
        \cos \varphi \\
        \sin \varphi  
    \end{pmatrix} + r \begin{pmatrix}
        - \sin \varphi \\
        \phantom{-}  \cos \varphi
    \end{pmatrix}
    \dot{\varphi} - 
    l \begin{pmatrix}
        \sin (\varphi + \psi) \\
        \cos (\varphi + \psi) \\
    \end{pmatrix} (\dot{\varphi} + \dot{\psi}), \\
\end{aligned}\right.
\end{equation}
что позволяет, наконец, записать лагранжиан системы
\begin{equation}
    L = T - \Pi,
    \hspace{0.5cm} \Rightarrow \hspace{0.5cm} 
    L / m =  \dot{r}^2 + r^2 \dot{\varphi}^2 + l^2 \left(\dot{\varphi}+ \dot{\psi}\right)^2 + \frac{\varkappa}{r} .
\end{equation}
Так как на систему действуют только радиальные силы, то
\begin{equation}
    \frac{d \vc{K}_O}{d t} = \vc{M}_O = 0, 
    \hspace{0.5cm} \Rightarrow \hspace{0.5cm} 
    \vc{r}_1 \times \dot{\vc{r}}_1 + \vc{r}_2 \times \dot{\vc{r}}_2 = \const.
\end{equation}
Также сохраняется энергия системы,
\begin{equation}
    T + \Pi = \const, 
    \hspace{0.5cm} \Rightarrow \hspace{0.5cm} 
     \dot{r}^2 + r^2 \dot{\varphi}^2 + l^2 \left(\dot{\varphi}+ \dot{\psi}\right)^2 + \frac{\varkappa}{r} = \const.
\end{equation}