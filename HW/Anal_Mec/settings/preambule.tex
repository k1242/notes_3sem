% file's preambule


% connect packages
%%%%%%%%%%%%%%%%%%%%%%%%%%%%%%%%%%%%%%%%%%%%%%%%%%%%%%%%%%%%%%%%%%%%%%
\usepackage[T2A]{fontenc}                   %!? закрепляет внутреннюю кодировку LaTeX
\usepackage[utf8]{inputenc}                 %!  закрепляет кодировку utf8
\usepackage[english,russian]{babel}         %!  подключает русский и английский
\usepackage{amsmath}                        %!  |
\usepackage{amssymb,textcomp, esvect,esint} %!  |важно для формул 
\usepackage[margin=2cm]{geometry}           %!  фиксирует оступ на 2cm
\usepackage{amsfonts}                       %!  математические шрифты
\usepackage{amsthm}                         %!  newtheorem и их сквозная нумерация
\usepackage{graphicx}                       %?  графическое изменение текста
\usepackage{indentfirst}                    %   добавить indent перед первым параграфом
\usepackage{xcolor}                         %   добавляет цвета
\usepackage{enumitem}                       %!  задание макета перечня.
\usepackage[unicode, pdftex]{hyperref}      %!  оглавление для панели навигации по PDF-документу + гиперссылки
\usepackage{booktabs}                       %!  добавляет книжные линии в таблицы
\usepackage{hypcap}                         %?  адресация на картинку, а не на подпись к ней
\usepackage{abraces}                        %?  фигурные скобки сверху или снизу текста
\usepackage{caption}                        %-  позволяет корректировать caption 
\usepackage{multirow}                       %   объединение ячеек в таблицах
\usepackage{pifont}                         %!  нужен для крестика
\usepackage{cancel}                         %!  аутентичное перечеркивание текста
\usepackage{ulem}                           %!  перечеркивание текста
\usepackage{tikz}                           %!  высокоуровневые рисунки (кружочек)
\usepackage{titling}                        %-  автоматическое заглавие 
\usepackage{blindtext}                      %-  слепой текст
\usepackage{fancyhdr}                       %   добавить верхний и нижний колонтитул
\usepackage{wrapfig}                        %! обтекание таблиц и рисунков

\usepackage{esvect}                         % векторы над буквами

\usepackage{import}                         % рисунки в Inkscape
\usepackage{xifthen}
\usepackage{pdfpages}
\usepackage{transparent}
%%%%%%%%%%%%%%%%%%%%%%%%%%%%%%%%%%%%%%%%%%%%%%%%%%%%%%%%%%%%%%%%%%%%%%


% add (renew) commands
%%%%%%%%%%%%%%%%%%%%%%%%%%%%%%%%%%%%%%%%%%%%%%%%%%%%%%%%%%%%%%%%%%%%%%
\renewcommand{\Im}{\mathop{\mathrm{Im}}\nolimits}
\renewcommand{\Re}{\mathop{\mathrm{Re}}\nolimits}
\renewcommand{\d}{\, d}
\renewcommand{\leq}{\leqslant}
\renewcommand{\geq}{\geqslant}

\newcommand{\vc}[1]{\mbox{\boldmath $#1$}}
\newcommand{\T}{^{\text{T}}}
\newcommand{\vp}{\vphantom{\frac{1}{2}}}
\newcommand{\rot}{\mathop{\mathrm{rot}}\nolimits}
\renewcommand{\div}{\mathop{\mathrm{div}}\nolimits}
\newcommand{\grad}{\mathop{\mathrm{grad}}\nolimits}

\newcommand{\diag}{\mathop{\mathrm{diag}}\nolimits}
\newcommand{\Ker}{\mathop{\mathrm{Ker}}\nolimits}
\newcommand{\Spec}{\mathop{\mathrm{Spec}}\nolimits}
\newcommand{\sign}{\mathop{\mathrm{sign}}\nolimits}
\newcommand{\tr}{\mathop{\mathrm{tr}}\nolimits}
\newcommand{\rg}{\mathop{\mathrm{rg}}\nolimits}

\newcommand{\const}{\text{const}}
\newcommand{\red}[1]{\textcolor{red}{#1}}
\newcommand{\xmark}{\ding{55}}

\newcommand{\dmat}[4]{
  \ifthenelse{
    \equal{#1}{3}
  }{
\begin{pmatrix}
    #2 & 0 & 0 \\
    0 & #3 & 0 \\
    0 & 0 & #4 \\
\end{pmatrix}
  }{
  \ifthenelse{
      \equal{#1}{2}
    }{
  \begin{pmatrix}
      #2 & 0 \\
      0 & #3 \\
  \end{pmatrix}
    }{
      \text{\textcolor{red}{error}}
    }
  }
}

\newcommand{\skmat}[4]{
  \ifthenelse{
    \equal{#1}{3}
  }{
\begin{pmatrix}
    0 & -#4 & #3 \\
    #4 & 0 & -#2 \\
    -#3 & #2 & 0 \\
\end{pmatrix}
  }{
  \ifthenelse{
      \equal{#1}{2}
    }{
  \begin{pmatrix}
      0 & #2 \\
      -#2 & 0 \\
  \end{pmatrix}
    }{
      \text{\textcolor{red}{error}}
    }
  }
}
%%%%%%%%%%%%%%%%%%%%%%%%%%%%%%%%%%%%%%%%%%%%%%%%%%%%%%%%%%%%%%%%%%%%%%


\newcommand{\incfig}[1]{%
    % \def\svgwidth{\columnwidth}
    \import{./figures/}{#1.pdf_tex}
}

% some tikz commands
%%%%%%%%%%%%%%%%%%%%%%%%%%%%%%%%%%%%%%%%%%%%%%%%%%%%%%%%%%%%%%%%%%%%%%
\makeatletter %%%%%%%%%%%%%%% КРУЖОЧЕК %%%%%%%%%%%%%%%%%%%%%%%%%%%%%%%
\newcommand*{\encircled}[1]{\relax\ifmmode\mathpalette
\@encircled@math{#1}\else\@encircled{#1}\fi}
\newcommand*{\@encircled@math}[2]{\@encircled{$\m@th#1#2$}}
\newcommand*{\@encircled}[1]{%
  \tikz[baseline,anchor=base]{\node[draw,circle,outer sep=0pt,
                                        inner sep=.2ex] {#1};}}
\makeatother

\usepackage{arydshln} %%%%%%%%%%%%%%% ЛИНИИ В МАТРИЧКЕ %%%%%%%%%%%%%%%
\makeatletter
  \renewcommand*\env@matrix[1][*\c@MaxMatrixCols c]{%
    \hskip -\arraycolsep
    \let\@ifnextchar\new@ifnextchar
  \array{#1}}
\makeatother

\def\letuscom{%%%%%%%%%%%%%%%%%%%%%% ПУСТЬ %%%%%%%%%%%%%%%%%%%%%%%%%%
\mathord{\setbox0=\hbox{$\exists$}%
     \hbox{\kern 0.125\wd0%
           \vbox to \ht0{%
              \hrule width 0.75\wd0%
              \vfill%
              \hrule width 0.75\wd0}%
           \vrule height \ht0%
           \kern 0.125\wd0}%
   }%
}
\newcommand{\letus}{\raisebox{-1.2pt}{$\letuscom$}}
%%%%%%%%%%%%%%%%%%%%%%%%%%%%%%%%%%%%%%%%%%%%%%%%%%%%%%%%%%%%%%%%%%%%%%


% create environment
%%%%%%%%%%%%%%%%%%%%%%%%%%%%%%%%%%%%%%%%%%%%%%%%%%%%%%%%%%%%%%%%%%%%%%
\newtheorem{to_thr}{Thr}[section]
\newtheorem{to_suj}[to_thr]{Suj}
\newtheorem{to_lem}[to_thr]{Lem}
\newtheorem{to_com}[to_thr]{Com}
\newtheorem{to_con}[to_thr]{Con}
\theoremstyle{definition}
\newtheorem{to_def}[to_thr]{Def}
%%%%%%%%%%%%%%%%%%%%%%%%%%%%%%%%%%%%%%%%%%%%%%%%%%%%%%%%%%%%%%%%%%%%%%


% add some colors
%%%%%%%%%%%%%%%%%%%%%%%%%%%%%%%%%%%%%%%%%%%%%%%%%%%%%%%%%%%%%%%%%%%%%%
\definecolor{grey}{HTML}{666666}
\definecolor{linkcolor}{HTML}{0000CC}
\definecolor{urlcolor}{HTML}{006600}
\hypersetup{
    pdfstartview=FitH,  
    linkcolor=linkcolor,
    urlcolor=urlcolor, 
    colorlinks=true,
    citecolor=blue}
%%%%%%%%%%%%%%%%%%%%%%%%%%%%%%%%%%%%%%%%%%%%%%%%%%%%%%%%%%%%%%%%%%%%%%


% add page header
%%%%%%%%%%%%%%%%%%%%%%%%%%%%%%%%%%%%%%%%%%%%%%%%%%%%%%%%%%%%%%%%%%%%%%
\pagestyle{fancy}
\fancyhf{}
\fancyhead[RE,LO]{\textsc{Ф\raisebox{-1.5pt}{и}з\TeX}}
\fancyhead[LE,RO]{Хоружий К.А.}
\fancyhead[CO,CE]{\leftmark}
\fancyfoot[LE,RO]{\textcolor{grey}{\texttt{\thepage}}}
%%%%%%%%%%%%%%%%%%%%%%%%%%%%%%%%%%%%%%%%%%%%%%%%%%%%%%%%%%%%%%%%%%%%%%
