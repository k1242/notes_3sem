\section{Движение точки в ценральном поле}
\subsection{Уравнение Бине}

\begin{to_def} 
    Полем центральных сил называется поле, в котором сила действующая на точку такая, что 
    $$
    \vc{F}(\vc{r}) = \vc{F}(r) \frac{\vc{r}}{r}.
    $$
\end{to_def}

Логично перейти к $(r, \varphi, \theta)$. Тогда
$$
    m \vc{\mathrm{w}} = F(r) \frac{\vc{r}}{r} 
    \hspace{0.5cm} \Rightarrow \hspace{0.5cm} 
    \left\{\begin{aligned}
        m \vc{\mathrm{w}}_r &= F(r) \\
        m \vc{\mathrm{w}}_\varphi &= 0 \\
        m \vc{\mathrm{w}}_\theta &= 0
    \end{aligned}\right.
$$
Тперье, получсется, знаем, что
$$
    \vc{\mathrm{w}}_\theta = - \ddot{r\theta} + 2 \dot{r} \theta + r \sin \theta \cos \theta \dot{\varphi}^2 = 0,
$$
и, учитывая, что $\theta(0) = \pi / 2$, $\dot{\theta}(0)=0$, тогда $\theta(t)=\pi/2$, это с точки зрения диффуров. А с точки зрения физики кинетический момент сохраняется, то есть
$$
    \vc{\mathrm{K}}_0 = m \vc{r} \times \vc{v} = \const.
    \hspace{0.5cm} \Rightarrow \hspace{0.5cm} 
    \vc{r}, \vc{v} \in \text{постоянной плоскости.}
$$
Тогда $\theta$ мы можем просто выбросить. 

Приходим к системе уравнений
\begin{equation}
    \left\{\begin{aligned}
        m \left( \ddot{r} - r \dot{\varphi}^2 \right) &= F(r) \\
        m \frac{d}{dt} (r^2 \dot{\varphi}) &= 0
    \end{aligned}\right.
    \hspace{0.5cm} \Leftrightarrow \hspace{0.5cm} 
    \left\{\begin{aligned}
        m \left( \ddot{r} - r \dot{\varphi}^2 \right) &= F(r) \\
        r^2 \dot{\varphi} &= \frac{\mathrm{K}_0}{m} = \const.
    \end{aligned}\right. 
\end{equation}
Это, собственно, соответсвует закону Кеплера о сохранение секториальной скорости.

Первое уравнение как-то не очень, перейдём от $d/dt$ к $d/d\varphi$. Тогда $r(t) \to u(\varphi) = \frac{1}{r}$ -- переменная Бине.
$$
    \dot{r} = \frac{d(1/u)}{dt} = - c u',
$$
а для второй производной
$$
    \ddot{r} = - c^2 u^2 u''.
$$
Тогда уравнение перепишем, как
$$
    -c^2 u^2 u'' - c^2 u^3 = \frac{F(u)}{m},
$$
получая диффур вида
\begin{equation}
\boxed{
    u'' + u = \frac{-F(u)}{m c^2 u^2} 
}
    \hspace{0.5cm} \text{-- уравнение Бине}.
\end{equation}
Так мы свели всё к гармоническому осцилятору.


%%%%%%%%%%%%%%%%%%%%%%%%%%%%%%%%%%%%%%%%%%%%%%%%%%%%%%%%%%%%%%%%%%%%%%%%%%%%%%%%%%%
\subsection{Метрика Шварцшильда}

Заметим (Эйнштейн заметил), что

\incfig{4}

\noindent
тогда 
$$
    d s^2 = \left(1 - \frac{a}{r} \right) \d \tau^2
    -
    \left(1 - \frac{a}{r} \right)^{-1} \d r^2
    -
    (r \sin \theta)^2 \d \varphi^2 - r^2 \d \theta^2.
$$
Здесь 4 независимых переменных $(\tau, r,  \varphi, \theta)$, где три из сферических координат, а $\tau$ -- физическое время. 

Также введен радиус Шварцшильда $a = 2 GM$.

Движение точек рассматриваем, как движение по геодезическим, то есть $\vc{\mathrm{w}}_i = 0$, где $i \in \{\tau, r, \varphi, \theta\}$, и положим $v^2=1$. Из раннее полученного, $\theta (t) = \pi / 2$, то есть в некотором смысле движение плоское.
\begin{align}
    v^2 &= 
    \left(
    \frac{ds}{dt} \right)^2 = \left(1 - \frac{a}{r} \right) \dot{\tau}^2
    - \left(1 - \frac{a}{r} \right)^{-1} \dot{r}^2 - r^2 \dot{\varphi}^2 = 1 \\
    \vc{\mathrm{w}}_\tau &=
    \frac{d}{dt} \frac{\partial (v^2/2)}{\partial \dot{\tau}} -
    \underbrace{\frac{\partial (v^2/2)}{\partial \tau}}_{0} =
    \frac{d}{dt} \left[\left(
               1 -  \frac{a}{r} 
            \right) \dot{\tau}\right] = 0 \\
    \vc{\mathrm{w}}_\varphi &= 
    - \frac{d}{dt} \left[
        r^2 \dot{\varphi}
    \right] = 0
\end{align}
Тогда у нас есть первый интеграл
\begin{equation}
    \left(1 - \frac{a}{r} \right) \dot{\tau} = \mathcal D.
\end{equation}
И другой первый интеграл
\begin{equation}
    r^2 \dot{\varphi} = \mathcal C.
\end{equation}
Подставляя, получим, что
\begin{equation}
    \mathcal D^2 - \mathcal C^2 u'^2 = 1 - au + \mathcal C^2 u^3 (1 - au).
\end{equation}
Применяя $d/d\varphi$ и полагая $\mathcal C = c$, получим 
\begin{equation}
     u'' + u = \frac{3}{2} a u^2 + \frac{a}{2c^2} = - \frac{F}{mc^2u^2}.
 \end{equation} 

Получается, что мы можем или говорить про движение по геодезическим в метрике Шварцшильда, или движение в центральном поле с силой
\begin{equation}
    F =  - m \left(\frac{3}{2} a c^2 u^4 + \frac{a}{2} u^2\right).
\end{equation}



