\subsection{Задачи с II семинара}

\subsubsection*{О геодезических на гиперболоиде.}

Уравнение гиперболоида -- $x^2 + y^2 - z^2 = 1$.
\begin{to_def} 
    \textit{Геодезическая} -- линия в пространстве, по которой движется точка при нулевых компонентах ускорения в локальном базисе, задающем касательное пространство. 
    $$
        w_i = (\vc{w} \cdot \vc{g}_i) = 0.
    $$
\end{to_def}

Пусть $q = \{\varphi, h=z\}$, тогда $z^y + y^2 = 1 + h^2$. Тогда
$$
    \vc{r} = 
    \left(\begin{aligned}
        x &= \sqrt{1 + h^2} \cos \varphi\\
        y &= \sqrt{1+h^2} \sin \varphi \\
        z &= h
    \end{aligned}\right), \hspace{0.5cm} 
    \vc{g}_\varphi = \frac{\partial \vc{r}}{\partial \varphi} = \sqrt{1 + h^2}
    \begin{pmatrix}
        -\sin\varphi \\ \cos \varphi \\ 0
    \end{pmatrix}, \hspace{0.5cm} 
    \vc{g}_h = \frac{\partial \vc{r}}{\partial h} = \begin{pmatrix}
        h  / \sqrt{1 + h^2} \cdot  \cos\varphi \\
        h  / \sqrt{1 + h^2}  \cdot \sin\varphi \\
        1
    \end{pmatrix}
$$
Метрический тензор
$$
    g_{ij} = \begin{pmatrix}
        1 + h^2 & 0 \\
        0 & \frac{1+2h^2}{1+h^2} 
    \end{pmatrix}.
$$
Найдём $v$:
$$
    v^2 = g_{ji} v^i v^j = (1+h^2) \dot{\varphi}^2 + \frac{1+2h^2}{1+h^2} \dot{h}^2.
$$
Теперь можно домножить на $dt^2$ и найти коэффициенты, с которыми учитываем расстояния. То есть
\begin{equation}
    ds^2 = g_{ij} \d q^i \d q^j
\end{equation}
% см. Т.19 -- покажите, что пи


Для ускорений:
\begin{align*}
    w_\varphi &= \frac{d}{dt} \left[(1+h^2) \dot{\varphi}\right] = 0; \\
    w_h &= \frac{d}{dt} \left[\frac{1+2h^2}{1+h^2} h \right] - h \dot{\varphi}^2 -
    \frac{\partial}{\partial h} \left(\frac{1+2h^2}{1+h^2} \right) \frac{\dot{h}^2}{2} = 0.
\end{align*}
Заметим, что
$$
    \left.\begin{aligned}
        w_\varphi = 0 \\
        w_h = 0
    \end{aligned}\right\} \hspace{0.5cm} \Rightarrow \hspace{0.5cm} 
    w_\tau = 0 \hspace{0.5cm} \Rightarrow \hspace{0.5cm} 
    v^2 = \const, \hspace{0.5cm} \letus v^2  = 1.
$$
Ну, тогда перейдём к
$$
    \left.\begin{aligned}
        w_\varphi = 0 \\
        w_h = 0
    \end{aligned}\right\}
    \hspace{0.5cm} \Leftrightarrow \hspace{0.5cm} 
    \left.\begin{aligned}
        w_\varphi = 0 \\
        v^2 = 1
    \end{aligned}\right\} \text{ т.к. } 
    w_\tau v = 0 = (\vc{w} \cdot \vc{v}) = w_i v^i = (w_\varphi \dot{\varphi} + w_h \dot{h} ) = w_h \dot{h}
    \Rightarrow
    \left[\begin{aligned}
        \dot{h} = 0 \\
        w_h = 0
    \end{aligned}\right.
$$
Так перейдём к уравнению
$$
    \frac{c^2}{1+h^2} + \frac{1+2h^2}{1+h^2}  \dot{h}^2 = 1
    \hspace{0.5cm} \Rightarrow \hspace{0.5cm} 
    \dot{h}^2 = \frac{1 + h^2 - c^2}{1 + 2h^2} .
$$
И сменим параметризацию $h(t) \to h(\varphi)$.
$$
    \frac{dh}{d\varphi} = \frac{\dot{h}}{\dot{\varphi}} =
     \frac{1+h^2}{c} \sqrt{\frac{1 + h^2 - c^2}{1 + 2h^2}}.
$$
Получили двухпараметрическое\footnote{
    Потому что константа интегрирования.
} семейство геодезических. 

Посмотрим на частные случаи. Например, $h == h_0$. Тогда\footnote{
    \red{Проверить!}
} $w_h = 0 \Leftrightarrow h_0 = 0$, $c = \pm 1$.
Или, $c =1 / \sqrt{2} \Rightarrow dh/d\varphi = 1 + h^2$. 
Тогда $h = \tg \varphi$.
