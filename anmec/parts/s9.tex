\section{Геометрия масс твёрдого тела}

\subsection{Тензор инерции}

Движение тела может быть разбито на поступательное плюс вращательное. 
Есть три классические величины: $\vc{Q} = m \vc{v}_C, \ T=\frac{mv^2}{2} + T_{...}^{...}, \ \vc{K}$. Мгновенная ось вращения проходит через точку $O$. 
$$
    \vc{v}_i = \vc{\omega} \times \vc{r}_i \equiv \widetilde r_i \vc{\omega},
    \hspace{0.5cm} \widetilde{r_i} = \begin{pmatrix}
        0 &z_i &- y_i \\
        -z_i & 0 & x_i \\
        y_i & -x_i & 0\\
    \end{pmatrix}.
$$
Известно, что
$$
    v_i^2 = \left(
        \widetilde{r_i} \vc{\omega}
    \right)\T \left(
        \widetilde{r_i} \vc{\omega}
    \right) = \vc{\omega}\T \widetilde{r_i}\T \widetilde{r_i} \vc{\omega}.
$$
Так приходим к
\begin{to_def} 
Тензором величину назовём величину 
\begin{equation}
    \hat J_0 = \sum m_i \widetilde{r_i}\T \widetilde{r_i}.
\end{equation}
\end{to_def}
Тогда кинетическую энергию запишем, как
\begin{equation}
    T = \frac{1}{2} \vc{\omega}\T 
        \hat J_0 \vc{\omega}.
\end{equation}
Но опыт кричит о том, что там момент инерции, действительно
\begin{equation}
    J_e = \vc{e}\T \hat J_0 \vc{e}.
\end{equation}
Найдём его элементы:
\begin{equation}
    \widetilde{r_i}\T \widetilde{r_i} = \begin{pmatrix}
        y_i^2 + z_i^2 & -x_i y_i & - x_i z_i \\
        -x_i y_i & x_i^2 + y_i^2 & - y_i z_i \\
        -x_i z_i & -y_i z_i & x_i^2 + y_i^2 \\
    \end{pmatrix} = \hat j_i,
\end{equation}
суммируя, получим
\begin{equation}
    \hat J_0 = \begin{pmatrix}
        J_x & - J_{xy} & -J_{xz} \\
        -J_{xy} & J_y & -J_{yz} \\
        -J_{xz} & -J_{yz} & J_z \\
    \end{pmatrix},
\end{equation}
где $J_x$ -- \textit{осевые моменты инерции}, а $J_{xy}$ -- центробежные момент инерции. 

Но, в силу симметричности тензора, существуют такие оси, что
\begin{equation}
    \hat J_0 = \begin{pmatrix}
        J_x & 0 & 0 \\
        0 & J_y & 0 \\
        0 & 0 & J_z \\
    \end{pmatrix}.
\end{equation}

\subsection{Кинетический момент}


Кинетический момент найдём из
\begin{equation}
    T = \frac{1}{2} \sum m_i \vc{v}_i \left[\vc{\omega} \times \vc{r}_i\right]
    =
    \frac{\vc{\omega}}{2} \sum m_i \vc{r}_i \times \vc{v}_i = 
    \frac{1}{2} \vc{\omega} \cdot \vc{K}_O,
\end{equation}
тогда
\begin{equation}
    \boxed{
        \vc{K}_O = \hat J_O \vc{\omega}
    }.
\end{equation}

% \subsubsection*{Интерпретация №1}
На самом деле
\begin{align*}
    \hat J_0 \colon \vc{\omega} \in \mathbb{R}^3 \to \vc{K}_O \in \mathbb{R^3}, \\
    \vc{\omega}, \Omega \colon \vc{r} \in \mathbb{R}^3 \to \vc{v} \in \mathbb{R}^3.
\end{align*}
% \subsubsection*{Интерпретация №2}
Вообще, получается $\vc{K}_O \nparallel \vc{\omega}$.


Введём оси $\xi\eta\zeta$, тогда в них
\begin{equation}
    \hat{J}_0 = \diag (A, B, C),
    \hspace{0.5cm} 
    \vc{\omega} = \begin{pmatrix}
        p \\ q \\ r
    \end{pmatrix},
    \hspace{0.5cm} \Rightarrow \hspace{0.5cm} 
    T = \frac{1}{2} \left(
        Ap^2 + B q^2 + C r^2
    \right),
    \hspace{0.5cm} 
    \vc{K}_O = \begin{pmatrix}
        Ap \\ Bq \\ Cr
    \end{pmatrix}.
\end{equation}


%%%%%%%%%%%%%%%%%%%%%%%%%%%%%%%%%%%%%%%%%%%%%%%%%%%%%%%%%%%%%%%%%%%%%%%%%%%%%%%%%%%
\subsection{Компоненты тензора инерции в других СО}


\subsubsection{Поворот}


Во-первых, посмотрим на поворот
$$
    T = \frac{1}{2}  \vc{\omega}_{1}\T \hat{J}_{O1} \vc{\omega}_1 =
    \frac{1}{2} \vc{\omega}\T \hat J_{O2} \vc{\omega}_2,
    \hspace{0.5cm} 
    \vc{\omega}_1 = R \vc{\omega}_2,
$$
Тогда
\begin{equation}
    \boxed{
        \hat J_{O1} = R^{-1} \hat J_{O2} R
    }.
\end{equation}



\subsubsection{Параллельный перенос (Т. Гюйгенса-Штейнера)}

Запишем
\begin{equation}
    \hat J_O = \hat J_C + m \ \hat j_{CO},
\end{equation}
если $\vv CO = (\xi \eta \zeta)$, то
\begin{equation}
    \hat j_{CO} =
    \begin{pmatrix}
        \eta^2 + \zeta^2 & - \xi \eta & - \xi \zeta \\
        - \xi \eta & \xi^2 + \zeta^2 & - \eta \zeta \\
        - \xi \zeta & - \eta \zeta & \xi^2 + \eta^2 \\
    \end{pmatrix}
\end{equation}

\subsection{Цилиндр}

Перейдём к переменным $r, \varphi, z$, тогда, например
\begin{equation}
    J_z=
    \int
    \left(x^2 + y^2\right) \rho \d V=
    \frac{M}{\pi R^2 H} 
    \iiint r^2 r \d r \d \varphi \d z.
\end{equation}
Считая, получим
\begin{equation}
    \hat{J}_C = \diag\left(
        \frac{MR^2}{4} +\frac{MH^2}{12}, \
        \frac{MR^2}{4} + \frac{MH^2}{12}, \
        \frac{MR^2}{2}  
    \right).
\end{equation}

В частности, при $\vv{CA} = \begin{pmatrix}
    R & 0 & -H/2
\end{pmatrix}\T$, получим
$$
    \hat J_A = \hat J_C + m \hat j_{CA} =
    \begin{pmatrix}
        A & 0 & \frac{1}{2} MRH \\
        0 & B & 0 \\
        \frac{1}{2} MRH & 0 & C
    \end{pmatrix}.
$$
Теперь приведем к главным осям, поворотом относительно оси $z$:
\begin{equation*}
    \hat J_A' = \diag\left(A', B', C'\right) =
    R\T \hat J_A R,
    \hspace{0.5cm} 
    R = \begin{pmatrix}
        \cos \alpha & 0 & \sin \alpha \\
        0 & 1 & 0 \\
        -\sin\alpha & 0 & \cos \alpha \\
    \end{pmatrix}.
\end{equation*}
Решая, получим
$$
    \tg 2 \alpha = 4 \sqrt{3}.
$$
Подставляя, найдём
$$
    \hat J_A' = \frac{mR^2}{4}  \diag\left(2, 9, 9\right).
$$
Ну или просто к главным осям привести можно, через собственные числа.

\subsection{Диск}

Есть некоторая квадратная рама (полное условие см. дополнение). Для простоты положим $\omega_1 = \omega_2 = \omega$.
 Найдём $T, \vc{N}_A, \vc{N}_B$. 

 Во-первых,
 $$
     T = \frac{1}{2} m V_O^2 + \frac{1}{2} \vc{\Omega}\T \hat{J}_O \vc{\Omega},
 $$
где $v_O = \omega a / 2$.
Выберем такие оси, что
$$
    \hat J_O = \frac{1}{4} mR^2 \diag\left(1,1,2\right) .
$$
Посчитаем теперь $\vc{\Omega}$:
$$
    \vc{\Omega} = \begin{pmatrix}
        O & \omega \sqrt{2}/2 & \omega \sqrt{2}/2 + \omega
    \end{pmatrix}\T.
$$
Из теоремы об изменение импульса
$$
    m \vc{\mathrm{w}}_O = \vc{N}_A + \vc{N}_B,
    \hspace{0.5cm} 
    m \frac{\omega^2 \ a^2/4}{a/2}  = N_A + N_B.
$$
А ещё знаем, что
\begin{align*}
    \frac{d}{dt} \vc{K}_A =
    \vc{\omega}_1 \times \left[
    \left(
        \hat J_O + m \hat{j}_{OA}
    \right) \vc{\Omega}
    \right]
    = \vv{AB} \times \vc{N}_B. \\
    \frac{d}{dt} \vc{\mathrm K}_A =
    \vc{\omega}_1 \times \left[
    \left(
        \hat J_O + m \hat{j}_{OA}
    \right) \vc{\Omega}
    \right]
    = \vv{AB} \times \vc{\mathrm N}_B. 
\end{align*}
