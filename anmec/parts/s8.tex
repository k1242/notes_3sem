\section{Элементы механики сплошных сред (МСС)}

\subsection{Переменные Лагранжа и Эйлера}

Пусть каждой точке среды соответсвует $\xi^1, \xi^2, \xi^3$, собственно $(\xi, t)$ -- \textit{лгранжевы переменные}. \textit{Закон движения среды} в таком случае это
\begin{equation}
    \vc{r} (\xi, t),
\end{equation}
скорость же
$$
    \vc{v} = \frac{\partial \vc{r}(\xi, t)}{\partial t},
    \hspace{0.5cm} 
    \vc{\mathrm{w}} = \frac{\partial \vc{v} (\xi, t)}{\partial t},
$$
и так далее.

Альтернативно можеем задать $(x, t)$ -- эйлерово описание. Тогда
$$
    \vc{v}(x, t), \vc{\mathrm{w}}(x, t) 
    \hspace{0.5cm} \text{-- поля скоростей и ускорений.}
$$

В частности, представляя движение по шоссе, полоса 1,2,3 и участок трассы -- эйлерово описание среды.
Если же мы будем следить за каждой машиной, то это будет лагранжево описание.


\subsubsection*{Задача 1}

Пусть 
$$
    v_1 = \frac{-x_2}{\sqrt{x_1^2 + x_2^2}}, \hspace{0.5cm} 
    v_2 = \frac{x_1}{\sqrt{x_1^2 + x_2^2}}.
$$
Найти $\vc{r} (\xi, t)$. Легко получить, что
$$
    \dot{x}_1^2 + \dot{x}_2^2 = 1.
$$
Тогда
$$
    x_1 = \frac{1}{\omega} \cos \left(\omega t + \alpha\right),
    \hspace{0.5cm} 
    x_2 = \frac{1}{\omega} \sin \left(\omega t + \alpha\right).
$$
Переменные запишутся как
$$
    \xi_1 = \frac{\cos \alpha}{\omega}, \hspace{0.5cm} 
    \xi_2 = \frac{\sin \alpha}{\omega} 
    \hspace{0.5cm} \Rightarrow \hspace{0.5cm} 
    \omega = (\xi_1^2 + \xi_2^2)^{-1/2},
    \alpha = \arcsin (\omega \xi_2).
$$
Получается, что
$$
    \vc{r}(\xi, t) = \frac{1}{\omega}  \begin{pmatrix}
        \cos (\omega t + \alpha) \\
        \sin (\omega t + \alpha)
    \end{pmatrix}.
$$


%%%%%%%%%%%%%%%%%%%%%%%%%%%%%%%%%%%%%%%%%%%%%%%%%%%%%%%%%%%%%%%%%%%%%%%%%%%%%%%%%%%


\subsection{Деформации}

Пусть
$$
    \vc{r} = \begin{pmatrix}
        x_1 \\ x_2 \\ x_3
    \end{pmatrix},
    \hspace{0.5cm} 
    \vc{r}' = \begin{pmatrix}
        x_1' \\ x_2' \\ x_3'
    \end{pmatrix},
    \hspace{0.5cm} 
    \vc{u} = \vc{r}' - \vc{r},
$$
где $\vc{u}$ -- вектор дифформации. Тогда
$$
    d x_i' = dx_i + du_i = dx_i + \frac{\partial u_i}{\partial x_k} \d x_k.
$$
Введём некоторый $ds'$
$$
    (ds')^2 = dx_i' \d x_i'; \hspace{0.5cm} ds^2 = \d x_i \d x_i.
$$
Подставляя, получим, что
$$
    (ds')^2 = ds^2 + 2 \frac{\partial u_i}{\partial x_k} \d x_i \d x_k +
    \left(
        \frac{\partial u_i}{\partial x_k} \d x_k
    \right) \left(
        \frac{\partial u_i}{\partial x_k} \d x_k
    \right) 
    \approx
    ds^2 + 2 \varepsilon_{ik} \d x_i \d x_k,
$$
где $\varepsilon_{ik}$ -- \textit{тензор малых дефформаций}:
$$
    \varepsilon_{ik} = \frac{1}{2} \left(
        \frac{\partial u_i}{\partial x_k} + \frac{\partial u_k}{\partial x_i} 
    \right),
$$
который в главных осях диагонален.

Тогда
$$
    ds' = \sqrt{
        1 + 2 \varepsilon_{ii} 
    } dx_i.
$$
В общем смысл в том, что
$$
    \frac{dx_i' - dx_i}{dx_i} \approx 1 + \frac{1}{2} (2\varepsilon_{ii}) - 1 = \varepsilon_{ii}.
$$
Или
$$
    \frac{dV' - dV}{dV} = \tr \varepsilon = \sum_{i=1}^3 \frac{\partial u_i}{\partial x_i} = \mathrm{div}\, \vc{u}.
$$
Тогда, в частности,
$$
    \mathrm{div}\, \vc{u} = 0
    \hspace{0.5cm} 
    \text{--} \hspace{0.5cm}  \text{несжимаемая среда}.  
$$


%%%%%%%%%%%%%%%%%%%%%%%%%%%%%%%%%%%%%%%%%%%%%%%%%%%%%%%%%%%%%%%%%%%%%%%%%%%%%%%%%%%


\subsection{Напряжение}

\begin{minipage}[c]{0.5\textwidth}
    
Ну, собственно,
$$
    \vc{F} = \begin{pmatrix}
        F_1 \\ F_2 \\ F_3
    \end{pmatrix}, \hspace{0.5cm} 
    \int F_i \d V = \int \frac{\partial \sigma_{ik}}{\partial x} \d V =
    \oint \sigma_{ik} \d A_k
    .
$$
Кстати, 
\begin{equation}
    \sigma_{ik} = \sigma_{ki}.
\end{equation}


\end{minipage}
\hfill
\begin{minipage}[c]{0.4\textwidth}
    \begin{center}
        \incfig{5}
    \end{center}
\end{minipage}



\subsection{Обобщенный закон Гука}

Пусть $E$ -- модуль Юнга, $\mu$ -- коэффициент Пуассона. Тогда
$$
    \varepsilon_{11} = \frac{\sigma_{11}}{E}, \hspace{0.5cm} 
    \varepsilon_{22} = \varepsilon_{33} = - \frac{\mu}{E} \sigma_{11}.
$$
Перепишем это в виду
$$
    \varepsilon_{11} = \frac{\sigma_{11}}{E}  - \frac{\mu}{E} \sigma_{22} -
    \frac{\mu}{E} \sigma_{33} = \frac{1+\mu}{E} \sigma_{11} - \frac{\mu}{E} \tr \sigma.
$$
Или, в матричном виде
$$
    \begin{pmatrix}
        \varepsilon_{11} & 0 & 0\\
        0 & \varepsilon_{22} & 0 \\
        0 & 0 & \varepsilon_{33}
    \end{pmatrix} =
    \frac{1+\mu}{E} 
    \begin{pmatrix}
        \sigma_{11} & 0 & 0\\
        0 & \sigma_{22} & 0 \\
        0 & 0 & \sigma_{33}
    \end{pmatrix} -
    \frac{\mu}{E} \tr \sigma 
    \begin{pmatrix}
        1 & 0 & 0\\
        0 & 1 & 0 \\
        0 & 0 & 1
    \end{pmatrix}.
$$
В тензорном виде
$$
    \varepsilon_{ik} = \frac{1+\mu}{E} \sigma_{ik} - \frac{\mu}{E} \delta_{ik} \tr \sigma .
$$
Выразим $\varepsilon$:
$$
    \tr \varepsilon = \frac{1+\mu}{E} \tr \sigma - \frac{3\mu}{E} \tr \sigma
    \hspace{0.5cm} \Rightarrow \hspace{0.5cm} 
    \tr \sigma = \frac{E}{1-2\mu} \tr \varepsilon.
$$
Так и получаем \textit{обобщенный закон гука}:
\begin{equation}
    \sigma_{ik} = \frac{E}{1+\mu} \left[
        \varepsilon_{ik} + \frac{\mu}{1 - 2\mu} \delta_{ik} \tr \varepsilon 
    \right]
\end{equation}

\subsubsection*{Задача: самосжимающийся шар}
Запишем
$$
    \frac{\partial \sigma_{ik}}{\partial x_k} = - f_i.
$$
Тогда, после некоторых преобразований, получим, что
$$
    \frac{\partial \sigma_{ik}}{\partial x_k} =
    \frac{E}{1+\mu}  \left[
        \frac{1}{2} \Delta \vc{u} + \frac{1}{2(1-2\mu)} \grad \div \vc{u}
    \right].
$$

Вспомним, что
$$
    \Delta \vc{u} = \grad \div \vc{u} - \cancel{\rot \rot \vc{u}} = \grad \div \vc{u}.
$$
Перейдём к уравнению
\begin{equation}
    \frac{d}{dr} \left(
        \frac{1}{r^2} \frac{d (r^2 u)}{r} = kr.
    \right)
\end{equation}
Верно, что
$$
    \frac{d (r^2 u)}{dr} = \frac{kr^4}{2} + c_1 r^2.
$$

Логично, что на границе $\sigma_{11} = 0$. То есть
$$
    \sigma_{11} = \frac{E}{1+\mu} \left[\varepsilon_{11} + \frac{\mu}{1-2\mu} \tr \varepsilon \right] = 0.
$$
Но $\varepsilon_{rr} = du / dr = \varepsilon_{11}$, $\tr \varepsilon = \div \vc{u}$, из этого можем найти $c_1$.