\section{Кинематика точки}

Пусть $\vc{r}(t), t \in \mathbb{R}$ -- движение точки и траектория движения.
\begin{to_def} 
    
    \begin{equation}
        \text{\textit{Скорость}:}  \hspace{0.5cm}  \vc{v} = \frac{d \vc{r}}{d t};
        \hspace{1cm} 
        \text{\textit{ускорение}:} \hspace{0.5cm}  \vc{w} = \frac{d \vc{v}}{d t} = \frac{d^2 \vc{r}}{d t^2}.
     \end{equation} 
\end{to_def}

\subsection{Естественный трёхгранник}
Из геометрии $\letus s(t)$ -- длина кривой. Тогда 
\begin{equation}
\label{dl}
    \vc{v} = \underbrace{\frac{d \vc{r}}{d s}}_{\vc{\tau}} \frac{ds}{dt} = v \vc{\tau}. 
\end{equation}
Дифференцируя \eqref{dl}
\begin{equation}
    \vc{w} = \frac{d \vc{v}}{d t} \vc{\tau} + v \underbrace{\frac{d \vc{\tau}}{ds}}_{\vc{n} / \rho}  \frac{ds}{dt} = 
   \underbrace{ \frac{dv}{dt} \vc{\tau}}_{\vc{w_{\tau}}} + \underbrace{\frac{v^2}{\rho} \vc{n}}_{\vc{w_{n}}}.
\end{equation}
где $(\vc{\tau}, \vc{n}, \vc{b})$ -- базис, преследующий точку.

%%%%%%%%%%%%%%%%%%%%%%%%%%%%%%%%%%%%%%%%%%%%%%%%%%%%%%%%%%%%%%%%%%%%%%%%%%%%%%%%%%%

\subsection{Компоненты скорости и ускорения}

Есть локальный базис. Тогда компоненты скорости
\begin{equation}
    \vc{v} = \frac{d \vc{r}}{dt} = \frac{\partial \vc{r}}{\partial q^i} \frac{d q^i}{dt} = \dot{q}^i \vc{g}_i = v^i \vc{g}_i 
    \hspace{0.5cm} \Rightarrow \hspace{0.5cm} v^i = \dot{q}^i.
\end{equation}
Для компоненты ускорения:
\begin{equation*}
    w_i = (\vc{w} \cdot \vc{g}_i) = \frac{d}{dt} (\vc{v} \cdot \vc{g}_i) - (\vc{v} \cdot \frac{d \vc{g}_i}{dt}).
\end{equation*}
Но, во-первых:
\begin{equation*}
\label{I}
    \frac{d \vc{g}_i}{dt} = \frac{d}{dt} \frac{\partial \vc{r}}{\partial q^i} =
    \frac{\partial}{\partial q^i} \frac{d \vc{r}}{dt} = \frac{\partial \vc{v}}{\partial q^i} .
\end{equation*}
Во-вторых:
\begin{equation}
\label{II}
    \vc{v} = \dot{g}^i \vc{g}_i \hspace{0.25cm}  \bigg|\frac{\partial}{\partial \dot{q}^k} 
    \hspace{0.5cm} \Rightarrow \hspace{0.5cm} 
    \frac{\partial \vc{v}}{\partial \dot{q}^k} = \frac{\partial}{\partial \dot{q}^k}
    (\underbrace{\dot{g}^1 \vc{g}_1}_{0} + \underbrace{\dot{g}^2 \vc{g}_2}_{\vc{g}_2} + \underbrace{\dot{g}^3 \vc{g}_3}_{0})  = \vc{g}_k
\end{equation}
Тогда
\begin{equation}
    w_i = \frac{d}{dt} (\vc{v} \cdot \frac{\partial \vc{v}}{\partial \dot{q}^i} ) - 
    (\vc{v} \cdot \frac{\partial \vc{v}}{\partial \dot{q}^i}) =
    \frac{d}{dt} \frac{\partial (\vc{v} \cdot \vc{v})}{\partial \dot{q}^i} \frac{1}{2} - \frac{\partial (\vc{v} \cdot \vc{v})}{\partial \dot{q}^i} \frac{1}{2} =
    {
    \frac{d}{dt} \frac{\partial (v^2/2)}{\partial \dot{q}^i} - \frac{\partial (v^2/2)}{\partial q^i}
    }  \hspace{0.15cm} \Rightarrow \hspace{0.15cm} 
    \boxed{
    mw_i = \frac{d}{dt} \frac{\partial T}{\partial \dot{q}^i} - \frac{\partial T}{\partial q^i}}
\end{equation}