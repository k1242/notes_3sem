\section{Основные теоремы динамики}

\subsection{Аксиоматика}

...


\subsection{Основные теоремы динамики}

Пусть $\vc{Q}$ -- количество движения, $\vc{\mathrm{K}}_{A}$ -- кинематический момент относительно полюса $A$. Далее, $T$ -- механическая энергия, $\delta A^{\text{всех}} = \sum \left(\vc{F}_i \cdot d \vc{r}_i \right)$ и $- d \Pi = \delta A$, где $\Pi$ -- потенциальная энергия. 

\begin{table}[h]
    \centering
    % \caption*{}
        \begin{tabular}{lcr}
    \toprule
            Величина & Thr об изменение & Первые интегралы системы \\
    \midrule
            $\vc{Q} = \sum m_i \vc{v}_i$ & 
            $d \vc{Q} / dt = \vc{R}^{\text{внеш}}$ &
            $\vc{Q} = \const$
            \\
            $\vc{\mathrm{K}}_{A} = \sum \vc{r}_{A_i} \times (m_i \vc{v}_i)$ &
            $d \vc{\mathrm{K}}_A / dt = \vc{M}_A^{\text{внеш}} + \vc{Q} \times \vc{v}_A$ &
            $\vc{K}_A = \const$
            \\
            $T = \sum m_i v_i^2 / 2$ &
            $dT = \delta A^{\text{всех}}$ &
            $T + \Pi = \const$ \\
    \bottomrule
        \end{tabular}
    \label{tab:}
\end{table}


\subsection{Вычисление динамических величин}

\subsubsection*{Формула переноса полюса}
$$
    \vc{R}_A = \sum \left(\vv{AB} + \vc{r}_{Bi}\right) \times m_i \vc{v}_i =
    \vv{AB} \times \sum m_i \vc{v}_i + \sum \vc{r}_{Bi} \times m_i \vc{v}_i
    \hspace{0.5cm} \Rightarrow \hspace{0.5cm} 
    \boxed{
        \vc{K}_A = \vc{K}_B + \vc{Q} \times \vv{BA}
    } 
$$

\subsubsection*{Теорема Кёнига}
Выберем некотору СК (книгову СК), движущуюся поступательно.
$$
    T = \frac{1}{2} m v_C^2 + T_{c \xi \eta \zeta}^r.
$$
В частности, для твёрдого тела
$$
    T_{c \xi \eta \zeta}^r = \sum \frac{1}{2} m_i (v_i^r)^2 = 
    \sum \frac{1}{2} m_i \omega^2 \rho_{Ci}^2 = \frac{\omega^2}{2} \sum m_i \rho^2_{Ci} = \frac{1}{2} J_\omega \omega^2.
$$
Тогда, для твёрдого тела,
$$
    T = \frac{1}{2} m v_C^2 + \frac{1}{2} J_\omega \omega^2
$$