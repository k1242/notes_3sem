\subsection{Задачи с VII семинара}
\subsubsection*{8.27}
Потенциальная энергия частицы
$$
    \Pi(r) = -\frac{\alpha}{r} + \frac{\beta}{r^2},
$$
где $\alpha, \beta > 0$.
Найдём силу, как
$$
    \vc{F} = - \grad \Pi 
    \hspace{0.5cm} \Rightarrow \hspace{0.5cm} 
     F = - \frac{d \Pi}{dr}  = -\frac{\alpha}{r^2} + 2 \frac{\beta}{r^3} = -\alpha u^2 + 2\beta u^3.
$$
Уравнение движения в центральном поле запишется, как
$$
    u'' + \underbrace{
    \left(1 + \frac{2\beta}{mc^2} \right)
    }_{\omega^2}
    u = \underbrace{\frac{\alpha}{mc^2}}_{a}.
    \hspace{0.5cm} \Rightarrow \hspace{0.5cm} 
    u'' + \omega^2 u = a.
$$
Тогда
$$
    u = u_{\text{общ}} + u_{\text{частн}} = A \cos \left(
        \omega \varphi + \psi
    \right) + \frac{a}{\omega^2} .
$$
Наконец, выражая $r$, получим
$$
    r = \left(
        \frac{a}{\omega^2} + A \cos \left(\omega \varphi + \psi\right)
    \right)^{-1} =
    \bigg/
    \begin{aligned}
        p &= \omega^2 / a \\
        e &= A \omega^2 / a
    \end{aligned}
    \bigg/ =
    \frac{p}{1 + e \cos \left(\omega \varphi + \psi\right)}
    =
    \frac{p}{1 + e \cos \left(\omega \varphi\right)}
$$
