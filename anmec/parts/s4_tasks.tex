\subsection{Задачи с III семинара}
\subsubsection*{Задача 2.12}
Знаем, что $\omega_1, \omega_2 = \const$.
Во-первых свяжем подвижную систему координат с пушкой: <<$\vc{\omega}_1 + \vc{\omega}_2$>>
Во-вторых,
$$
    \vc{v}^a = \vc{v}^e + \vc{v}^r,
    \hspace{0.5cm} 
    \vc{v}^r = \begin{pmatrix}
        0 \\ v_0 \cos \alpha \\ v_0 \sin \alpha
    \end{pmatrix}
$$. 
Тогда
$$
    \vc{v}^e = (\vc{\omega}_1 + \vc{\omega}_2) \times \vv{O'M} = \begin{pmatrix}
        \omega_1 \\ 0 \\ \omega_2
    \end{pmatrix} \times
    \begin{pmatrix}
        0 \\ l \cos \alpha \\ l \sin \alpha
    \end{pmatrix}.
$$
Теперь про ускорения, 
$$
    \vc{\mathrm{w}}^a = \vc{\mathrm{w}}^e + \vc{\mathrm{w}}^r + \vc{\mathrm{w}}^c.
$$
В частности верно, что
$$
    \vc{\mathrm{w}}^c = 2 \vc{\omega} \times \vc{v}^r = 2 \begin{pmatrix}
        \omega_1 \\ 0 \\ \omega_2
    \end{pmatrix} \times
    \begin{pmatrix}
        0 \\ v_0 \cos \alpha \\ v_0 \sin \alpha
    \end{pmatrix},
    \hspace{0.5cm} 
    \vc{\mathrm{w}}^r = \vc{\mathrm{w}}_O = \begin{pmatrix}
        0 \\ \mathrm{w}_0 \cos \alpha \\ \mathrm{w}_0 \sin \alpha
    \end{pmatrix}.
$$
Для углового ускорения (можно ввести новую подвижную систему координат или ...), верно что
$$
    \vc{\varepsilon} = 
    \frac{d}{dt} \left(\vc{\omega}_1 + \vc{\omega}_2 \right) = \vc{\omega}_2 \times \vc{\omega}_1,
$$
а для переносного ускорения, по формуле Ривальса, 
$$
    \vc{\mathrm{w}}^e = \vc{\varepsilon} \times \vv{O'M} + (\vc{\omega}_1 + \vc{\omega}_2) \times \left(
        (\vc{\omega}_1 + \vc{\omega}_2) \times \vv{O'M}
    \right).
$$

%%%%%%%%%%%%%%%%%%%%%%%%%%%%%%%%%%%%%%%%%%%%%%%%%%%%%%%%%%%%%%%%%%%%%%%%%%%%%%%%%%%


\subsubsection*{Другая задача}

Введём систему координат, связанную со стержнем $\eta\xi$. Верно, что
$$
    \vc{v}^a_{M'} = \vc{v}^e_{M'} + \vc{v}^r_{M'}, \hspace{0.5cm} \text{ где } \vc{v}^e_{M'} = \vc{v}_{M'},
    \hspace{0.5cm} \vc{v}_{M'}^r = \dot{\xi} \vc{e}_{\xi},
    \hspace{0.5cm} \vc{v}^a_{M'} = 0.
$$
Но $\vc{v}_M \| \vc{e}_{\xi} \| AB$.

\subsubsection*{Задача 3.29}

Найдём ускорение точки $M$, 
$$
    \vc{\mathrm{w}}_M = \vc{\mathrm{w}}_O + \underbrace{\vc{\varepsilon} \times \vv{OM}}_{0} + \vc{\omega} \times \left(\vc{\omega} \times \vv{OM}\right) = \vc{\mathrm{w}}_O - \omega^2 \vv{OM}.
$$
Введём систему координат, связанную с диском $\eta\xi$. Верно, что
$$
    \vc{\mathrm{w}}_{O'}^a = \vc{\mathrm{w}}_{O'}^e + \vc{\mathrm{w}}_{O'}^r + \cancel{\vc{\mathrm{w}}_{O'}^c}
    \hspace{0.5cm} \Rightarrow \hspace{0.5cm} 
    0 = \vc{\mathrm{w}}_O + \omega^2 R \vc{e}_\eta
$$
Собирая всё вместе, получим
$$
    \vc{\mathrm{w}}_M = - \omega^2 R \vc{e}_\eta - \omega^2 |OM| \vc{e}_\xi
$$

% \resizebox{\textwidth}{!}{
%     \section{Закон Кулона и теорема Гаусса}

Здесь попробуем индуктивно построить содержательную теорию, \textbf{начнём с двух эксперементальных фактов}, положенных в основу теории. Закона Кулона (сгсэ)
\begin{equation}
    \vc{F} = \frac{q_1 q_2 }{r^2} \frac{\vc{r}}{r},
\end{equation}
и, введя вектор напряженности электростатического поля $\vc{E} = \vc{F} / q$, принцип суперпозиции:
\begin{equation}
    \vc{E} = \sum \vc{E}_i.  
\end{equation}

\subsubsection*{Дипольный момент}
Простейшим примером системы зарядов является диполь $q_1 + q_2 = 0$, для которого введём $\vc{p}=q \vc{l}$:
$$
    \vc{E} = \frac{q}{r_1^2} \frac{\vc{r}_1}{r_1} - \frac{q}{r_1^2} \frac{\vc{r}_2}{r_2} 
    \hspace{0.5cm} \overset{l \ll r_2, r_1}{\underset{\Longrightarrow}{}} \hspace{0.5cm} 
    \vc{E} = \frac{3 (\vc{p} \cdot \vc{n}) \vc{n}}{r^3}  - \frac{\vc{p}}{r^3}
$$

Для заряженной нити верно, что
$$
    E = 2 \frac{\varkappa}{r}. 
$$



Теперь дойдём до двух теорем (кусочки уравнений Максвелла), описывающих электростатическое поле. 
\begin{to_thr}[теорема Гаусса]
    Для потока $\vc{E}$ через замкнутую поверхность $S$ верно, что
    \begin{equation}
        \oint_S E_n \d S =
        \oint_S (\vc{E} \d \vc{S}) 
        = 4 \pi q_{\textnormal{вн}}.
    \end{equation}
\end{to_thr}


\begin{proof}[$\triangle$]
\textcolor{grey}{
    \begin{minipage}[t]{0.9\textwidth}
        \begin{enumerate}[label = \Roman*.]
            \item Доказательство (из закона Кулона) для сферы вокруг точечного заряда очевидно. 
            \item Рассмотрим произвольную поверхность $\Omega$, содержащую заряд, и телесный угол в онной:
            $$
                E_n \d S = E \cos \alpha \d S = E \d S' 
            $$
            То есть поток через наклонную площадку равен потоку через тот же телесный угол через некоторую вспомогательную сферу. Так как $s_1 / s_2 = r_1^2 / r_2^2$ и $E_1 / E_2 = r_2^2/r_1^2$, получается интегрировать по $\Omega$ то же самое, что и интегрировать по выбранной хорошей сфере. 
            \item Рассмотрим теперь некоторую $\Omega$, не содержащую заряд. Посмотрим на телесный угол от $q$. По модулю потоки через них одинаковые, а знаки противоположны, следовательно вклада в поток через $\Omega$ нет.
            \item Для сложного распределения зарядов, по принципу суперпозиции верно, что
            $$
                \vc{E} = \sum_i \vc{E}_i
                \hspace{0.5cm} \Rightarrow \hspace{0.5cm} 
                \oint_S E_n \d S = \sum_i \oint_S \vc{E}_i \d S.
            $$
        \end{enumerate}
    \end{minipage}
}

\phantom{42}
\end{proof}



% }