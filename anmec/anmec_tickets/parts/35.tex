\subsubsection*{Преобразование Лежандра}

\begin{to_def} 
    В уравнениях Лагранжа второго рода движения голономной системы в потенциальном поле сил, функция Лагранжа зависит от $q, \ \dot{q}, \ t$ -- \textit{переменные Лагранжа}.
    Если в качестве параметров взять $q, \ p, \ t$, где $p_i$ -- \textit{обобщенные импульсы}\footnote{
        Обобщенный импульс $p_i$ -- ковектор, а  не вектор!
    }, определяемые как
    $
        p_i = {\partial L}/{\partial \dot{q}^i}.
    $
    То получим набор $q, \ p, \ t$ -- \textit{переменные Гамильтона}. 
\end{to_def}

В силу невырожденности $\partial L / (\partial \dot{q}^i \partial \dot{q}^j) = J_{p}$, то есть по \textit{теореме о неявной функции} эти равенства разрешимы относительно переменных $\dot{q}^i$. Через преобразование Лежандра естестественно ввести функцию
\begin{equation*}
    H(q, p, t) = p_i \dot{q}^i - L(q, \dot{q}, t),  \hspace{0.5cm} \dot{q} \equiv \dot{q}(q, p, t).
\end{equation*}

\subsubsection*{Уравнения Гамильтона}
% Маркеев, п.149.

Полный дифференциал функции Гамильтона можем выразиь двумя способами:
\begin{equation*}
    \left.\begin{aligned}
        d H &= \frac{\partial H}{\partial q^i} \d q^i + \frac{\partial H}{\partial p_i} \d p_i + \frac{\partial H}{\partial t} \d t,
        \\
        dH &= \dot{q}^i \d p_i - \frac{\partial L}{\partial q^i} \d q^i - \frac{\partial L}{\partial t} \d t.
    \end{aligned}\right\}
    \hspace{0.7cm} \Rightarrow \hspace{0.7cm} 
    \left.\begin{gathered}
        \frac{\partial H}{\partial t} = - \frac{\partial L}{\partial t} \\
        \frac{\partial H}{\partial p_i} = \dot{q}^i, \ \ 
        \frac{\partial H}{\partial q^i} = - \frac{\partial L}{\partial q^i}
    \end{gathered}\right.
    \hspace{0.7cm} \Rightarrow \hspace{0.7cm} 
    \left\{\begin{aligned}
        \frac{d q^i}{d t} &= \frac{\partial H}{\partial p_i}, \\
        \frac{d p_i}{d t} &= - \frac{\partial H}{\partial q^i}.    
    \end{aligned}\right.
\end{equation*}
Эти уравнения называются \textit{уравнениями Гамильтона}, или \textit{каноническими уравнениями}.


\subsubsection*{Физический смысл функции Гамильтона}

Пусть система натуральна, тогда $L = L_2 + L_1 + L_0$, и, соответсвенно,
\begin{equation*}
    H = \frac{\partial L}{\partial \dot{q}^i} \dot{q}^i - L.
\end{equation*}
По теореме Эйлера об однородных функциях
\begin{equation*}
    \frac{\partial L_2}{\partial \dot{q}^i} \dot{q}^i = 2 L_2,
    \hspace{1cm} 
    \frac{\partial L_1}{\partial \dot{q}^i} \dot{q}^i = L_1,
    \hspace{0.5cm} \Rightarrow \hspace{0.5cm} 
    H = L_2 - L_0.
\end{equation*}
пусть $T = T_2 + T_1 + T_0$, если силы имеют обычный потенциал $\Pi$, то $L_0 = T_0 - \Pi$, 
\begin{equation*}
    H = T_2 - T_0 + \Pi.
\end{equation*}
Если же силы имеют обобщенный потенциал $V = V_1 + V_0$, то $L_0 = T_0 - V_0$, и
\begin{equation*}
    H = T_2 - T_0 + V_0.
\end{equation*}
В случае натуральных и склерономных систем $T_1 = T_0 = 0$ и $T = T_2$, тогда $H = T + \Pi$. Т.е. для натуральных склерономных систем с обычным потенциалом сил функция Гамильтона $H$ представляет собой полную механическую энергию.

\subsubsection*{Интеграл Якоби}

Найдём полную производную $H$ по времени,
\begin{equation*}
    \frac{d H}{d t} = \frac{\partial H}{\partial q^i} \dot{q}^i + \frac{\partial H}{\partial p_i} \dot{p}_i + \frac{\partial h}{\partial t} = 
    \frac{\partial H}{\partial q^i} \frac{\partial H}{\partial p_i} - \frac{\partial H}{\partial p_i} \frac{\partial H}{\partial q^i} + \frac{\partial H}{\partial t} = \frac{\partial H}{\partial t},
    \hspace{0.5cm} \Rightarrow \hspace{0.5cm} 
    \frac{d H}{d t} = \frac{\partial h}{\partial t}.
\end{equation*}
Система называется \textit{обобщенно консервативной}, если $\partial H / \partial t = 0$, т.е $H(q^i, p_i) = h$, собственно, $H$ называют \textit{обобщенной полной энергией}, а поледнее равенство -- \textit{обобщенным интегралом энергии}.


\begin{to_def} 
    Для натуральной системы с обычным потенциалом сил, если $\partial H/ \partial t =0$, то
    \begin{equation*}
         H = T_2 - T_0 + \Pi = h = \const.
     \end{equation*} 
     Соотноешние, где $h$ -- произвольная постоянная, называют \textit{интегралом Якоби}.
\end{to_def}

Есть и другая формулировка для интеграла Якоби голономной склерономной системы. Действительно, при $\partial L / \partial t = 0$, интеграл Якоби перейдёт в
\begin{equation*}
    \frac{\partial H}{\partial t} = 0,
    \hspace{0.25cm} \Rightarrow \hspace{0.25cm} 
    \frac{d }{d t} \left(
        \frac{\partial L}{\partial \dot{q}^i} \dot{q}^i
    \right) = 0,
    \hspace{0.25cm} \Rightarrow \hspace{0.25cm} 
    \frac{\partial L}{\partial \dot{q}^i} \dot{q}^i = \const.
\end{equation*}



\subsubsection*{Уравнения Уиттекера}


Если $\partial H / \partial t = 0$, то $H(q, p) = h$, где $h = \const$ определяемая из н.у. В $2n$-мерном пространстве $q, \ p$ интеграл Якобми задаёт гиперповерхность, рассмотрим движение с $H = h$.

Такое движение описывается системой с $2n-2$ уравнений, причём она может быть записана в виде канонических уравнений. Пусть $\partial H / \partial p_1 \neq 0$, тогда
\begin{equation*}
    p_1 = - K(q^1, \ldots, q^n, p_2, \ldots, p_n, h),
    \hspace{0.5cm} \Rightarrow \hspace{0.5cm} 
    \left\{\begin{aligned}
        \dot{q}^i = \frac{\partial H}{\partial p_i}, \\
        \dot{p}_j = - \frac{\partial H}{\partial q^j}        
    \end{aligned}\right.
    \hspace{0.5cm} \Rightarrow \hspace{0.5cm} 
    \frac{d q^j}{d q^1} = \frac{
    \left(\dfrac{\partial H}{\partial p_j} \right)
    }{
    \left(\dfrac{\partial H}{\partial p_1} \right)
    },
    \hspace{0.5cm} 
    \frac{d p_j}{d q^1} = - \frac{
    \left(\dfrac{\partial H}{\partial q^j}\right)
    }{
    \left(\dfrac{\partial H}{\partial p_1} \right)
    },
\end{equation*}
для $j = 2,\ 3,\ \ldots,\ n$. Подставляя $p_1$ получим
\begin{align*}
    &\frac{\partial H}{\partial q^j} - \frac{\partial H}{\partial p_1} \frac{\partial K}{\partial q^j} = 0,
    &(j = 2, \ 3, \ \ldots, n);
    \\
    &\frac{\partial H}{\partial p_j} - \frac{\partial H}{\partial p_1} \frac{\partial K}{\partial p_j}  = 0,
    &(j = 2, \ 3, \ \ldots, n).
\end{align*}
Допиливая до надлежащего вида, окончательно находим
\begin{equation*}
    \frac{d q^j}{d q^1} = \frac{\partial K}{\partial p_j},
    \hspace{1cm} 
    \frac{d p_j}{d q_1} = - \frac{\partial K}{\partial q^j},
    \hspace{1cm} 
    (j = 2, \ 3, \ \ldots, n).
\end{equation*}
Эти уравнения описывают движения системы при $H = h = \const$, и называются \textit{уравнениями Уиттекера}. 

\subsubsection*{Уравнения Якоби}


Уравнения Уиттекера имеют структуру уравнений Гамильтона, соответственно их можно зписать в виде уравнений типа Лагранжа, при гессиане $K$ по $p$ неравным 0. Пусть $P$ -- преобразование Лежандра функции $K$ по $p_j$ ($j = 2, \ 3,\ \ldots,\ n$). Тогда
\begin{equation*}
    P = P(q^2, \ldots, q^n, \tilde q^2, \ldots, \tilde q^n, q^1, h) = \sum_{j = 2}^{n} \tilde q^j p_j - K,
\end{equation*}
где $\tilde q^{j} = d q^j / d q^1$. Величины $p_j$ выражаются через $\tilde q^2, \ \ldots, \ \tilde q_n$ из уравнений 
\begin{equation*}
    \tilde q^j = \frac{\partial K}{\partial p_j}, \hspace{0.5cm} 
    (j = 2, \ 3, \ \ldots, n),
\end{equation*}
т.е. из первых $n-1$ уравнений Уиттекера. При помощи функции $P$ эти уравнения могут быть записаны в эквивалентной форме:
\begin{equation*}
    \frac{d }{d q^1} \frac{\partial P}{\partial q_j'} - \frac{\partial P}{\partial q^j} = 0
    \hspace{1cm} (j = 2,\ 3,\ \ldots,\ n).
\end{equation*}
Это уравнения типа Лагранжа, называются \textit{уравнениями Якоби}.

Преобразовывая выражение для $P$ найдём, что
\begin{equation*}
    P = \sum_{j=2}^{n} q_j \tilde q^j + p_1 = 
    \sum_{i=1}^{n} p_1 \tilde q_i = \frac{1}{\dot{q}^1} \sum_{i=1}^{n} p_i \dot{q}^i = \frac{1}{\dot{q}^1} (L+H).
\end{equation*}
Тогда в случае консервативной системы $L = T - \Pi$, $H = T + \Pi$, и\footnote{
    \red{Пара выражений в выводе опущены.}
}
\begin{equation*}
    P = \frac{2T}{\dot{q}^1},
    \hspace{0.5cm} \Rightarrow \hspace{0.5cm} 
    P = 2 \sqrt{(h-\Pi) G}.
\end{equation*}

