 \subsubsection*{Тензор напряжений}
 В недеформированном теле молекулы находятся друг с другом в механическом и тепловом равновесии. При деформировании же взаимное расположение меняется и равновесие нарушается.
 \begin{to_def}
 	В результате возникают \textit{внутренние напряжения} --- силы, стремящиеся вернуть тело в равновесие, которые обуславливаются молекулярными силами, обладающими незначительным радиусом действия.
 \end{to_def}

Выделим в теле объём и рассмотрим суммарную действующую на него силу.
\textit{С одной стороны}, эта сила может быть представлена: $\int \vc{F} d V$, для $\vc{F}$ --- силы на единицу объема.
\textit{С другой стороны}, силы, с которыми действуют различные части объёма друг на друга не приведут к появлению никакой внешней силы.
Поэтому искомая полная сила будет состоят из сил действующих на объём со стороны окружающих его частей тела. В силу пренебрежимой малости радиуса молекулярных сил, внешние силы будут представленны как суммы сил на каждый элемент поверхности объёма.
\begin{to_def}
	$\int F_i d V = \int \frac{\partial \sigma^{i k}}{\partial x^k} = \oint \sigma^{i k} d f_k$. В последнем равенстве $\sigma^{i k}$ --- \textit{тензор напряжений}(симметричный). То есть $\sigma^{i k} d f_k$ есть $i$-ая компонента силы, действующей на элемент поверхности $d \vc{f}$.
\end{to_def}
Так, на единичную площадку, перпендикулярную оси $x$, действуют нормальная к ней сила $\sigma_{x x}$ и тангенциальные $\sigma_{y x}$ и $\sigma_{z x}$.
Знак силы $\sigma^{i k} d f_k$, которая является действующей на ограниченный поверхностью объём со стороны окружающих тел --- положительный. Для напряжений же извне, перед интегралом нужно поставить знак минус.

\subsubsection*{Всесторонее и не только сжатие}
При таком сжатии на каждую единицу поверхности тела действует одинаковое по величине давление $p$, направленное везде по нормали к поверхности внутрь объёма тела. 
А на элемент $d f_i$ действует сила $-p d f_i = \sigma^{i k} d f_k$. 
Таким образом при всестороннем сжатии тензор напряжений: $\sigma^{i k} = -p \delta^{i k}$.

В общем случае ещё и диагональные элементы тензора напряжений не нуль.
То есть, помимо нормальной силы, действуют ещё и тангенциальные <<скалывающие>> напряжения, стремящиеся сдвинуть параллельные элементы поверхности друг относительно друга.

В равновесии силы внутренних напряжений должны уравновешивать друг друга, то есть:
\begin{equation*}
	F_i = 0
	\hspace*{1 cm}
	\Rightarrow
	\hspace*{1 cm}
	\frac{\partial \sigma^{i k}}{\partial x^k} = 0.
\end{equation*}
И если тело находится в поле тяжести, то в равновесии:
\begin{equation*}
	\vc{F} + \rho g = 0
	\hspace*{1 cm}
	\Rightarrow
	\hspace*{1 cm}
	\frac{\partial\sigma^{ i k}}{\partial x^k} + \rho g^i - 0.
\end{equation*}

\subsubsection*{Внешние силы}
Обычно именно внешние силы вызывают деформацию, однако они будут просто входить в граничные условия к уравнениям равновесия. 
Внешняя сила $\vc{P}$ должна компенсироваться силой $\sigma^{i k} d f_k$:
\begin{equation*}
	P^i d f = - \sigma^{i k} d f_k = 0
	\hspace*{0.4 cm}
	\Rightarrow
	\hspace*{0.4 cm}
	d f_k = n_k d f
	\hspace*{0.4 cm}
	\Rightarrow
	\hspace*{0.4 cm}
	\sigma^{i k} n_k = P^i,
\end{equation*}
где $n$ --- единичный вектор нормали к площадке. Таким образом получили условие, которое должно выполняться на всей поверхности находящегося в равновесии тела.