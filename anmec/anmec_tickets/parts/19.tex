\subsubsection*{Уравнение Бине}


\begin{to_def} 
    \textit{Полем центральных сил} называется поле, в котором сила действующая на точку :
    $
    \vc{F}(\vc{r}) = F(r) {\vc{r}}/{r}.
    $
\end{to_def}

Логично перейти к $(r, \varphi, \theta)$. Тогда
$$
    m \vc{\mathrm{w}} = F(r) \frac{\vc{r}}{r} 
    \hspace{0.5cm} \Rightarrow \hspace{0.5cm} 
    \left\{\begin{aligned}
        m \vc{\mathrm{w}}_r &= F(r) \\
        m \vc{\mathrm{w}}_\varphi &= 0 \\
        m \vc{\mathrm{w}}_\theta &= 0
    \end{aligned}\right.
    \hspace{0.5cm} \Rightarrow \hspace{0.5cm} 
    \vc{\mathrm{w}}_\theta = - r^2\ddot{\theta} - 2 r \dot{r} \dot{\theta} + r^2 \dot{\varphi}^2 \sin \theta \cos \theta  = 0,
$$
и, учитывая, что $\theta(0) = \pi / 2$, $\dot{\theta}(0)=0$, тогда $\theta(t)=\pi/2$, это с точки зрения диффуров. А с точки зрения физики кинетический момент сохраняется, то есть
$$
    \vc{\mathrm{K}}_0 = m \vc{r} \times \vc{v} = \const.
    \hspace{0.5cm} \Rightarrow \hspace{0.5cm} 
    \vc{r}, \vc{v} \in \text{постоянной плоскости.}
$$
Тогда $\theta$ мы можем просто выбросить. 
Приходим к системе уравнений
\begin{equation}
    \left\{\begin{aligned}
        m \left( \ddot{r} - r \dot{\varphi}^2 \right) &= F(r) \\
        m \frac{d}{dt} (r^2 \dot{\varphi}) &= 0
    \end{aligned}\right.
    \hspace{0.5cm} \Leftrightarrow \hspace{0.5cm} 
    \left\{\begin{aligned}
        m \left( \ddot{r} - r \dot{\varphi}^2 \right) &= F(r) \\
        r^2 \dot{\varphi} &= \frac{\mathrm{K}_0}{m} = \const.
    \end{aligned}\right. 
\end{equation}
Это, собственно, соотвествует закону Кеплера о \textit{сохранение секториальной скорости}.

Первое уравнение как-то не очень, перейдём от $d/dt$ к $d/d\varphi$. Тогда $r(t) \to u(\varphi) = \frac{1}{r}$ -- \textit{переменная Бине}.
$$
    \dot{r} = \frac{d(1/u)}{dt} = - c u', 
    \hspace{1cm} 
    \ddot{r} = - c^2 u^2 u'',
    \hspace{1cm} \Rightarrow \hspace{1cm} 
    -c^2 u^2 u'' - c^2 u^3 = \frac{F(u)}{m}.
$$
Таким образом пришли к следующей теореме:

\begin{to_thr}[уравнение Бине]
    Уравнение движения материальной точки в поле центральных сил может быть записано как     
    \begin{equation}
    u'' + u = \frac{-F(u)}{m c^2 u^2},
\end{equation}
    где $u(\varphi) = 1/r$.
\end{to_thr}