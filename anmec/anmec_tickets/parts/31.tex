\begin{to_def} 
    \textit{Обобщенная сила} $Q_k$ -- величина коэффициента $\partial q^k$ при вариации $\delta A$, то есть $\delta A = Q_k \delta q^k$.
\end{to_def}

\begin{to_thr}[Уравнения Лагранжа второго рода]
     Каждая механическая система характеризуется определенной функцией $L(q, \dot{q}, t)$. Для голономных системы с конфигурационном многообразием степени $n$, верно что
     \begin{equation*}
         \frac{d}{dt} \frac{\partial L}{\partial \dot{q}_k} - \frac{\partial L}{\partial q_k} = 0, \hspace{0.5cm} k = 1, \ldots, n.
     \end{equation*}
     Где для потенциальных систем $L = T - \Pi$. В более общем случае можно записать, что
     \begin{equation*}
         \left(\frac{d }{d t} \frac{\partial T}{\partial \dot{q}^k} - \frac{\partial T}{\partial q^k} - Q^k\right)\delta q^k = 0, \hspace{0.5cm} 
         Q^k = - \frac{\partial \Pi}{\partial q^k}.
     \end{equation*}
\end{to_thr}

\begin{proof}[$\triangle$]
Запишем второй закон Ньютона:
$
    \left(
        m_i \vc{\mathrm{w}}_i = \vc{F}_i + \vc{R}_i
    \right)\big|_{\cdot \d \vc{r}_i}
$
, где $\vc{R}_i$ -- реакции связи. Хотим записать уравнение в общековариантном виде.
То есть мы <<замораживаем>> время, так чтобы $\vc{R} \cdot \delta \vc{r} = 0$. На таких перемещениях работа реакция связи равна 0.
\begin{align*}
    \bigg[
        \sum m_i \left(\vc{\mathrm{w}}_i \cdot \frac{\partial \vc{r}_i}{\partial q^k} \right)
        -
        \left(\vc{F}_i \cdot \frac{\partial \vc{r}_i}{\partial q^k} \right)
        -
        \underbrace{
        \left(
            \vc{R}_i \cdot \frac{\partial \vc{r}_i}{\partial q^k} 
        \right)}_{\cdot \delta q^k \to 0}
    \bigg] \cdot \delta q^k &= 0;
    \\
    \left[
        \frac{d}{dt} \frac{\partial }{\partial \dot{q}^k} \sum  \frac{m_i v_i^2}{2} 
        -
        \frac{\partial }{\partial q^k} \sum \frac{m_i v_i^2}{2} -
        \sum \vc{F}_i \frac{\partial \vc{r}_i}{\partial q^k} 
    \right] \delta q^k &= 0
    , \hspace{0.5cm} \Rightarrow \hspace{0.5cm} 
    \sum_k
    \left[
        \frac{d}{dt} \frac{\partial T}{\partial \dot{q}^k} 
        - \frac{\partial T}{\partial q^k} - Q^k
    \right] \delta q^k = 0.
\end{align*}
Проблема остается в неголономных системах, где $\delta q^k$ не являются независимыми, получается, что уравнения Лагранжа справедливы для голономных систем.

Вспоминая, что
\begin{equation*}
    \delta  A = \sum_i \vc{F}_i \cdot \delta \vc{r}_i =
    \sum_i \left(\vc{F}_i \cdot \frac{\partial \vc{r}_i}{\partial q^k}\right) \delta q^k =
    \sum_k 
    \frac{\delta A_k}{\delta q^k} \delta q^k = Q_k \delta q^k.
\end{equation*}
Тогда пусть $\Pi (q, t) \colon Q^k = - \partial \Pi / \partial q^k$.  Тогда
\begin{equation*}
    \frac{d}{dt} \frac{\partial (T-\Pi)}{\partial \dot{q}^k} - \frac{\partial (T - \Pi)}{\partial q^k}  = 0,
    \hspace{0.5cm} \Rightarrow \hspace{0.5cm} 
    \frac{d}{dt} \frac{\partial L}{\partial \dot{q}^k} - \frac{\partial L}{\partial q^k} = 0, \hspace{0.5cm} k = 1, \ldots, n.
\end{equation*}
То есть получили систему уравнений на $2n$ переменных.
\end{proof}



