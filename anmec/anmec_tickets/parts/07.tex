\subsubsection*{Свободные и несвободные системы. Связи.}

В общем случае связь запишем, как
$$
    f(\vc{r}_\nu, \vc{v}_\nu, t) \geq 0.
$$
В частности, при $f(\vc{r}_\nu, \vc{v}_\nu, t) = 0$, связь называет \textit{двухсторонней}, или \textit{удерживающей}. При неравенстве, соотвественно, связь \textit{односторонняя}, \textit{освобождающая}.
Связь вида $f(\vc{r}_\nu, t) = 0$ называется \textit{геометрической}, \textit{конечная}, \textit{голономная}. Связь вида $f(\vc{r}_\nu, \vc{v}_\nu, t) = 0$ называется \textit{дифференциальной}, или \textit{кинематической}. Иногда кинематическая связь может быть представлена как геометрическая, такая связь называется \textit{интегрируемой}. 

\begin{to_def} 
     Если на систему материальных точек не наложены дифференциальные неинтегрируемые связи, то она называется голономной. Если же среди связей, наложенных на систему есть дифференциальные неинтегрируемые связи, то система называется неголономной.
\end{to_def}

Хотелось бы построить некоторую общую теория для случая, когда этих связей несколько. 
В частности пусть есть $r$ геометрических связей.
\begin{equation}
\label{eqs1}
    f_\alpha (\vc{r}, t) = 0, \hspace{0.5cm} (\alpha = 1, \ldots, r),
\end{equation}
И несколько дифференциальных линейных связей
\begin{equation}
\label{eqs2}
    \sum_{\nu=1}^N \vc{a}_{\beta \nu} (\vc{r}_1, \ldots, \vc{r}_N, t) \cdot \vc{v}_\nu + a_\beta  (\vc{r}_1, \ldots, \vc{r}_N, t) = 0,
    \hspace{0.5cm} (\beta = 1, \ldots, s)
\end{equation}
Стоит сказать, что
$$
    3N - r - s \geq 1.
$$

\begin{to_def} 
     Геометрические связи называются стационарными или склерономными, если $t$ не входит в их уравнения \eqref{eqs1}. Дифференциальные связи \eqref{eqs2} называются \textit{стационарными} или \textit{склерономными} если функции $\vc{a}_{\beta\nu}$ не зависят явно от $t$, а функции $a_\beta \equiv 0$. Система называется \textit{склерономной}, если она либо свободная, либо на нее наложены только стационарные связи. Система называется \textit{реономной}, если среди наложенных на нее связей есть хотя бы одна нестационарная.
\end{to_def}

\subsubsection*{Ограничения, налагаемые связями на положения, скорости, ускорения и перемещения точек системы.}

Пусть задан некоторый момент $t = t*$. Тогда \textit{возможными положениями} назовём $\vc{r}_\nu$ такие, что для них верно \eqref{eqs1}, \eqref{eqs2}. 

Какие возможны скорости?
\begin{equation}
    \label{eqs3}
    \sum_{\nu=1}^N \frac{\partial f_\alpha}{\partial \vc{r}_\nu} \cdot \vc{v}_\nu +   \frac{\partial f_\alpha}{\partial t} = 0, \hspace{0.5cm} (\alpha = 1, \ldots, r).
\end{equation}

Совокупность векторов $\vc{v}_\nu = \vc{v}_\nu^*$, удовлетворяющая линейным
уравнениям \eqref{eqs2} и \eqref{eqs3} в возможном для данного момента времени положении
системы, назовем возможными скоростями. 

Какие возможны ускорения?
\begin{align}
\label{eqs4}
    \eqref{eqs3}, \eqref{eqs2}
    \hspace{0.25cm} \overset{d / dt}{\Rightarrow} \hspace{0.25cm} 
    & \sum_{\nu=1}^N \frac{\partial f_\alpha}{\partial \vc{r}_\nu} \cdot \vc{\mathrm{w}}_\nu +
    \sum_{\nu, \mu=1}^N \frac{\partial^2 f_\alpha}{\partial \vc{r}_\nu \partial \vc{r}_\mu} \vc{v}_\mu \cdot \vc{v}_\nu + 2 \sum_{k=1}^N \frac{\partial^2 f_\alpha}{\partial t \partial \vc{r}_\nu} \vc{v}_\nu + \frac{\partial^2 f_\alpha}{\partial t^2}  = 0 
    \hspace{0.25cm} &\alpha \in [1, r]\\
\label{eqs5}
    & \sum_{\nu=1}^N \vc{a}_{\beta\nu} \cdot \vc{\mathrm{w}}_\nu + \sum_{\nu, \mu =1}^N \frac{\partial \vc{a}_{\beta\nu}}{\partial \vc{r}_\mu} \vc{v}_\mu \cdot \vc{v}_\nu + \sum_{\nu=1}^N \frac{\partial \vc{a}_{\beta\nu}}{\partial t} \cdot \vc{v}_\nu + \sum_{\nu=1}^N \frac{\partial a_\beta}{\partial \vc{r}_\nu} \cdot \vc{v}_\nu + \frac{\partial  a_\beta}{\partial t} = 0
     \hspace{0.25cm} &\beta \in [1, s]
\end{align}

Совокупность векторов $\vc{\mathrm{w}}_\nu =\vc{\mathrm{w}}_\nu^*$, удовлетворяющая линейным
уравнениям \eqref{eqs4} и \eqref{eqs5} в возможном для данного момента времени положении
системы (+скорости), назовем возможными скоростями. 

Рассмотрим возможные перемещения $\Delta \vc{r}_\nu$ системы за $\Delta t$ из её возможного положения $\vc{r}_\nu^*$ в момент $t=t^*$. Тогда 
\begin{equation}
    \Delta \vc{r}_\nu = \vc{v}^* \Delta t + \frac{1}{2} \vc{\mathrm{w}}_\nu^* (\Delta t)^2 + \ldots
    \hspace{0.5cm} (\nu = 1, \ldots, N).
\end{equation}
Пренебрегая нелинейными членами, получим, что $\Delta \vc{r}_\nu = \vc{v}_\nu^* \Delta t$. Тогда, домножив \eqref{eqs2}, \eqref{eqs3} на $\Delta t$, получим систему уравнений, которой удовлетворяют линейные по $\Delta t$ возможные перемещения:
\begin{align}
\label{eqs7}
    \sum_{\nu=1}^N \frac{\partial f_\alpha}{\partial \vc{r}_\nu} \cdot \Delta \vc{r}_\nu + \frac{\partial f_\alpha}{\partial t} \Delta t &= 0,
    \hspace{0.5cm} &(\alpha = 1, \ldots, r), \\
\label{eqs8}
    \sum_{\nu=1}^N \vc{a}_{\beta\nu} \cdot \Delta \vc{r}_\nu + a_\beta \Delta t &= 0,
    \hspace{0.5cm} &(\beta = 1, \ldots, s),
\end{align}
где функции $\vc{a}_{\beta\nu}, a_\beta$ и частные производные вычисляются при $t=t^*, \; \vc{r}_\nu = \vc{r}_\nu^*$.


\subsubsection*{Действительные и виртуальные перемещения}
Пусть задано положение системы для $t, \vc{r}, \vc{v}, \vc{\mathrm{w}}$. Тогда для $t = t^* + dt$ запишем, что
\begin{equation}
\label{eqs9}
    \vc{r}_\nu (t^* + dt) - \vc{r}_\nu (t^*) = \vc{v}_{\nu_0}^* \d t + \frac{1}{2} \vc{\mathrm{w}}^*_{\nu_0} (dt)^2 + \ldots,
\end{equation}
где $\vc{\mathrm{w}}_{\nu_0}^*$ -- ускорения точек системы при $t = t^*$. Величины \eqref{eqs9} -- \textit{действительные (истинные) перемещения} точек системы за время $dt$. Тогда получим систему уравнений, аналогичную \eqref{eqs7}, \eqref{eqs8}:
\begin{align}
\label{eqs10}
    \sum_{\nu=1}^N \frac{\partial f_\alpha}{\partial \vc{r}_\nu} \cdot d \vc{r}_\nu + \frac{\partial f_\alpha}{\partial t} d t &= 0,
    \hspace{0.5cm} &(\alpha = 1, \ldots, r), \\
\label{eqs11}
    \sum_{\nu=1}^N \vc{a}_{\beta\nu} \cdot d \vc{r}_\nu + a_\beta d t &= 0,
    \hspace{0.5cm} &(\beta = 1, \ldots, s).
\end{align}


Помимо действительных перемещений есть \textit{виртуальные}. Ими называется совокупность величин $\delta \vc{r}_\nu$, удовлетворяющая линейным однородным уравнениям
\begin{align}
\label{eqs12}
    \sum_{\nu=1}^N \frac{\partial f_\alpha}{\partial \vc{r}_\nu} \cdot \delta \vc{r}_\nu  &= 0,
    \hspace{0.5cm} &(\alpha = 1, \ldots, r), \\
\label{eqs13}
    \sum_{\nu=1}^N \vc{a}_{\beta\nu} \cdot \delta \vc{r}_\nu  &= 0,
    \hspace{0.5cm} &(\beta = 1, \ldots, s),
\end{align}
Если система склерономна, то действительное перемещение будет одним из виртуальных.

\begin{to_def} 
    \textit{Синхронное варьирование} -- переход из одного положения в другое, при фиксированном времени
    $$
        \vc{r}_\nu^* \to \vc{r}_\nu^* + \delta \vc{r}_\nu.
    $$
    При синхронном варьировании мы не рассматриваем процесс движения и сравниваем допускаемые связями
    бесконечно близкие положения (конфигурации) системы для данного фиксированного момента времени.
\end{to_def}

Рассмотрим две совокупности возможных перемещений с одним и тем же значением величины $\Delta t$. Согласно разложению по Тейлору,
\begin{align*}
    \Delta_1 \vc{r}_\nu = \vc{v}_{\nu_1}^* \Delta t + \frac{1}{2} \vc{\mathrm{w}}_{\nu_1}^* (\Delta t)^2 + \ldots, \\
    \Delta_2 \vc{r}_\nu = \vc{v}_{\nu_2}^* \Delta t + \frac{1}{2} \vc{\mathrm{w}}_{\nu_2}^* (\Delta t)^2 + \ldots, 
\end{align*}
и рассмотрим их разность
$$
        \Delta_1 \vc{r}_\nu - \Delta_2 \vc{r}_\nu = (\vc{v}_{\nu_1}^* \Delta t - \vc{v}_{\nu_2}^* \Delta t) + \left(\frac{1}{2} \vc{\mathrm{w}}_{\nu_1}^* (\Delta t)^2 - \frac{1}{2} \vc{\mathrm{w}}_{\nu_2}^* (\Delta t)^2\right) + \ldots.
$$
