\subsubsection*{Задача о траектории луча (оптический принципа Ферма $\sim$ принципа Мопертеи-Лагранжа)}

Найдём траекторию светового луча в среде с показателем преломления
\begin{equation*}
    n(z) = n_0 + n_z z.
\end{equation*}
Согласно принципу Ферма, введя $(ds)^2 = (dr)^2 + (dz)^2$, считая $dz / dr = \dot{z}$
\begin{equation*}
    \delta \left(
        \int_A^B (n_0 + n_z z) \d s
    \right) = 0,
\hspace{0.5cm} \Rightarrow \hspace{0.5cm} 
    \delta
    \left(
        \int_A^B z \sqrt{1 + \dot{z}^2} \d r
    + \frac{n_0}{n_z}  l 
    \right) = 0,
\end{equation*}
где
\begin{equation*}
    l = \int_A^B \sqrt{1 + \dot{z}^2} \d r.
\end{equation*}
Вспомнив \eqref{lstar} и \eqref{intl}, 
поймём, что решаем изопериметрическую задачу, которую уже решили в предыдущем пункте, решением является траектория по цепной линии, с $\lambda = -n_0/n_z$:
\begin{equation}
    z(r) = \frac{n_0}{n_z} + C_1 \ch \frac{r-C_2}{C_1},
\end{equation}
где $C_1$ и $C_2$ определяются из начальных условий\footnote{
    Предполагая, что мы хотим пустить луч от точки $(z_1, r_1)$ к $(z_2, r_2)$, мы сможем сделать это единственным образом, это и задаст $C_1$ и $C_2$.
}.



\subsubsection*{Задача о цепной линии}

Решим задачу о цепной линии. Пусть в точках $(x_1, y_1)$ и $(x_2, y_2)$ закреплена цепь с линейной плотностью $\rho$ и массой $M$.
Для цепной линии сначала найдём центр масс $y_0$:
\begin{equation*}
    y_0 = \frac{1}{M} \int_{x_1}^{x_2}
    y \cdot 
    \underbrace{
    \rho \sqrt{1 + (y_x')^2} \d x
    }_{
    dm
    }.
\end{equation*}
Лагранжиан системы
\begin{equation*}
    L = T - \Pi = M g \cdot y_0.
\end{equation*}
В силу независимости $L$ от $t$ верно, что
\begin{equation}
    \delta S = \delta 
    \left(\int_{t_1}^{t_2} L \d t\right)
    = 0,
    \hspace{0.5cm} \Rightarrow \hspace{0.5cm} 
    \delta L = 0, \hspace{0.5cm} \Rightarrow \hspace{0.5cm}
    \delta
    \bigg(
    \int_{x_1}^{x_2}
    \underbrace{
    y \sqrt{1 + (y_x')^2}
    }_{
    F(x)
    } \d x
    \bigg) = 0
    ,
\end{equation}
что позволяет нам решать немного другую задачу.

Мы знаем, что на $\dot{q}, q$ равносильны следующие условия
\begin{equation*}
    \frac{d }{d t} \frac{\partial L(\dot{q}, q, t)}{\partial q^i}  - \frac{\partial L}{\partial q}  = 0,
    \hspace{0.5cm} \Leftrightarrow \hspace{0.5cm} 
    \delta \left(
        \int_{t_1}^{t_2} L(\dot{q}, q, t) \d t
    \right) = 0,
\end{equation*}
при фиксированной длине нити $l$ равной
\begin{equation}
\label{intl}
    l = \int_{x_1}^{x_2} \underbrace{
    \sqrt{1+\dot{y}^2}
    }_{\varphi(x)} \d x,
\end{equation}
где $y_x' = \dot{y}$ (здесь и далее). 
Тогда введём\footnote{
    О причинах такого решения см. метод решения изопериметрической задачи.
} $L^*$
\begin{equation}
\label{lstar}
    L^* (y, x) = F(x) - \lambda \varphi(x),
\end{equation}
для которого верно, что
\begin{equation}
\label{maeq}
\delta \left(\int_{x_1}^{x_2} L^*(y, \dot{y}, x) \d x \right) = 0
\hspace{0.5cm} \Leftrightarrow \hspace{0.5cm} 
    \frac{d}{dx} \frac{\partial L^*}{\partial \dot{y}} -\frac{\partial L^*}{\partial x} = 0.
\end{equation}
Формально мы перещли к решению изопериметрической задачи. Для удобство переобозначим $L^* = L$. 
Посмотрим на $\partial L / \partial \dot{y} = L_{\dot{y}}$:
\begin{equation*}
    d L_{\dot{y}} (y, \dot{y}) = 
    \frac{\partial L_{\dot{y}}}{\partial y} \d y + 
    \frac{\partial L_{\dot{y}}}{\partial \dot{y}} \d \dot{y},
    \hspace{0.5cm} \Rightarrow \hspace{0.5cm} 
    \frac{d }{d x} \frac{\partial L}{\partial \dot{y}} -\frac{\partial L}{\partial y} =
    L_{\dot{y}, y} \dot{y} + L_{\dot{y}, \dot{y}} \ddot{y} - L_{y} = 0.
\end{equation*}
Домножив на $(-\dot{y})$ получим, как видно, полный дифференциал \rotatebox[origin=c]{270}{:)}
\begin{equation*}
    \frac{\partial L}{\partial y} \frac{d y}{d x} 
    + \underbrace{
    \frac{\partial L}{\partial \dot{y}} \frac{d \dot{y}}{d x} 
    - \bigg(
        \frac{\partial L}{\partial \dot{y}} \ddot{y}
    }_{\text{прибавил/вычел}}
     + 
        \dot{y} \bigg[
            \frac{\partial L_{\dot{y}}}{\partial y} \dot{y} + \frac{\partial L_{\dot{y}}}{\partial \dot{y}} \ddot{y}
        \bigg]
    \bigg)
    = \frac{d }{d x} \left(
        L - \dot{y} \frac{\partial L}{\partial \dot{y}} 
    \right),
\end{equation*}
откуда \eqref{maeq} может быть переписано, как
\begin{equation*}
    L - \dot{y} \frac{\partial L}{\partial \dot{y}} = C_1,
\end{equation*}
то есть да, <<энергия>> сохраняется, $x$ же явно не входит в $L^*$. 

Конкретно в нашем случае,
\begin{equation*}
    (y+\lambda) \sqrt{1 + \dot{y}^2}
    - \dot{y} (y + \lambda) \frac{\dot{y}}{\sqrt{1+\dot{y}^2}}  = C_1,
    \hspace{0.5cm} \Rightarrow \hspace{0.5cm} 
    y + \lambda = C_1 \sqrt{1 + \dot{y}^2}.
\end{equation*}
Как известно 
\href{https://ru.wikipedia.org/wiki/%D0%A8%D0%B8%D0%BD%D1%83%D1%81}{шинус} замечателен: $1 + \sh^2 \varkappa = \ch^2 \varkappa$, так что пусть $\dot{y} = \sh \varkappa$. Тогда
\begin{equation*}
    y = C_1 \ch \varkappa - \lambda.
\end{equation*}
Подставив друг в друга последних два выражения, найдём
\begin{equation*}
    \dot{y} = \frac{d y}{d \varkappa} \cdot \frac{d \varkappa}{d x} 
    = C_1 \frac{d \varkappa}{d x} 
    \sh \varkappa,
    \hspace{0.5cm} \Rightarrow \hspace{0.5cm} 
    x  = C_1 \varkappa + C_2.
\end{equation*}
Таким образом мы получаем \textit{уравнение цепной линиии}
\begin{equation}
    \boxed{
        y = C_1 \ch \frac{x-C_2}{C_1} - \lambda.
    }
\end{equation}
Константы могут быть найдены из граничных условий $y(x_1)=y_1$, $y(x_2)=y_2$
и интеграла \eqref{intl}:
\begin{equation*}
     \sh \frac{x_2 - C_2}{C_1} - \sh \frac{x_1 - C_2}{C_1} = \frac{l}{C_1}.
\end{equation*} 

