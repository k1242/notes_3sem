Пусть еть выражение новых координат $\tilde q^1,\ \ldots,\ \tilde q^n $ через старые:
\begin{equation*}
    q^i = q^i (\tilde q^1,\ \ldots,\ \tilde q^n, \tilde t),
    \hspace{1cm}
    t = t(\tilde q^1,\ \ldots,\ \tilde q^n , \tilde t).
\end{equation*}

В таком случае преобразованные уравнения движения будут иметь такую же форму:
\begin{equation*}
    \frac{d }{d \tilde t} \frac{\partial \tilde L}{\partial (\frac{d \tilde q^i}{d \tilde t} )} 
    -
    \frac{\partial \tilde L}{\partial \tilde q^i} = 0,
    \hspace{1cm} 
    \tilde L = L(q^1, \ldots, q^n, \ \dot{q}^1, \ldots, \dot{q}^n, t) \frac{d t}{d \tilde t}.
\end{equation*}

Сама ковариантность следует из вывода уравнений Лагранжа: в качестве обобщенных координат могут быть выбраны любые $n$ независимых параметров системы.