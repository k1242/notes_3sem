Напишем систему уравнений, соответствующую уравнениям Лагранжа первого рода:

\begin{equation}
    \sum_{\nu=1}^N \vc{a}_{\beta \nu} (\vc{r}_1, \ldots, \vc{r}_N, t) \cdot \vc{v}_\nu + a_\beta  (\vc{r}_1, \ldots, \vc{r}_N, t) = 0,
    \hspace{0.5cm} (\beta = 1, \ldots, s)
\end{equation}
$$
    f_\alpha (\vc{r}, t) = 0, \hspace{0.5cm} (\alpha = 1, \ldots, r),
$$
\begin{align}
    \sum_{\nu=1}^N \frac{\partial f_\alpha}{\partial \vc{r}_\nu} \cdot d \vc{r}_\nu + \frac{\partial f_\alpha}{\partial t} d t &= 0,
    \hspace{0.5cm} &(\alpha = 1, \ldots, r), \\
    \sum_{\nu=1}^N \vc{a}_{\beta\nu} \cdot d \vc{r}_\nu + a_\beta d t &= 0,
    \hspace{0.5cm} &(\beta = 1, \ldots, s).
\end{align}



Взяв выражения для виртуальных перемещений всегда можно получить силы реакции так называемым методом \textit{множителей Лагранжа}:
\begin{equation*}
    \sum_{\nu=1}^N \left(\vc{R}_\nu - \sum_{i = 1}^r \lambda_i \frac{\partial f_i}{\partial \vc{r}_\nu} - \sum_{j = 1}^s \mu_j \alpha_{j \nu}\right) \delta \vc{r}_\nu = 0
    \hspace*{1 cm}
    \Rightarrow
    \hspace*{1 cm}
    \vc{R}_\nu = \sum_{i = 1}^r \lambda_i \frac{\partial f_i}{\partial \vc{r}_\nu} + \sum_{j = 1}^s \mu_j \alpha_{j \nu}.
\end{equation*}
Полученные выражения для реакций идеальных сил через неопределенные множители Лагранжа $\lambda_i $ и $\mu_j $ можно подставить в исходное уравнение связей, получим \textit{уравнения Лагранжа первого рода}:
\begin{equation*}
    m_\nu \vc{w}_\nu = \vc{F}_\nu + \sum_{i = 1}^r \lambda_i \frac{\partial f_i}{\partial \vc{r}_\nu} + \sum_{j = 1}^s \mu_j \alpha_{j \nu}.
\end{equation*}