\subsection*{Решение задачи}


Составим матрицу $C$, записав в её столбцах координаты векторов $f_1, f_2, f_3$:
\begin{equation*}
    C = \begin{pmatrix}
        -3 & 1 & -2 \\
        1 & -2 & 1 \\
        2 & 3 & -1 \\
    \end{pmatrix}.
\end{equation*}
Вычислим определитель этой матрицы
\begin{equation*}
    \det C = \begin{vmatrix}
        -3 & 1 & -2 \\
        1 & -2 & 1 \\
        2 & 3 & -1 \\
    \end{vmatrix} = 
    \begin{vmatrix}
        0 & -5 & 1 \\
        1 & -2 & 1 \\
        0 & 7 & -3 \\
    \end{vmatrix} = 
    1 \cdot (-1)^{(2+1)} \begin{vmatrix}
        -5 & 1\\
        7 & -3 \\
    \end{vmatrix} = -8.
\end{equation*}
Так как $\det C \neq 0$, то векторы $\vc{f}_1, \ \vc{f}_2, \ \vc{f}_3$ линейно независимы, а потому могут быть приняты в качестве базиса в $\mathbb{R}^3$. Матрица $C$ невырождена, а потому имеет обратную $C^{-1}$. Найдём её:
\begin{align*}
    &A_1^1 = \begin{vmatrix}
        -2 & 1 \\
        3 & -1 \\
    \end{vmatrix}, \  
    A_1^2 = - \begin{vmatrix}
        1 & -2 \\
        3 & -1 \\
    \end{vmatrix}, \  
    A_1^3 = - \begin{vmatrix}
        1 & -2 \\
        -2 & 1
    \end{vmatrix}, \  
    A_2^1 = - \begin{vmatrix}
        1 & 1 \\
        2 & -1
    \end{vmatrix}, \  \
    A_2^2 = \begin{vmatrix}
        -3 & -2 \\
        2 & -1 \\
    \end{vmatrix}; 
    \\
    &A_2^3 = \begin{vmatrix}
        -3 & -2 \\
        1 & 1 \\
    \end{vmatrix}, \  
    A_3^1 = - \begin{vmatrix}
        1 & -2 \\
        2 & 3 \\
    \end{vmatrix}, \  
    A_3^2 = - \begin{vmatrix}
        -3 & 1 \\
        2 & 3
    \end{vmatrix}, \  
    \ \; \, A_3^3 = - \begin{vmatrix}
        -3 & 1 \\
        1 & -2
    \end{vmatrix}.
\end{align*}
Таким образом находим обратную матрицу
\begin{equation*}
    C^{-1} = \frac{1}{8} \begin{pmatrix}
        1 & 5 & 3 \\
        -3 & -7 & -1 \\
        -7 & -11 & -5 \\
    \end{pmatrix}.
\end{equation*}
В таком случае новые координаты $y_1, \ y_2, \ y_3$ вектора $\vc{x}$:
\begin{equation*}
    \begin{pmatrix}
        y_1 \\
        y_2 \\
        y_3 \\
    \end{pmatrix} = 
    \frac{1}{8} \begin{pmatrix}
        1 & 5 & 3 \\
        -3 & -7 & -1 \\
        -7 & -11 & -5 \\
    \end{pmatrix}
    \begin{pmatrix}
        -3 \\
        -2 \\
        7 \\
    \end{pmatrix} =
    \frac{1}{8} 
    \begin{pmatrix}
        -3-10+21 \\
        9+11 - 7 \\
        21 + 22 - 35 \\
    \end{pmatrix} = 
    \frac{1}{8} 
    \begin{pmatrix}
        8 \\
        16 \\
        8
    \end{pmatrix} =
    \begin{pmatrix}
        1 \\ 2 \\ 1
    \end{pmatrix}.
\end{equation*}