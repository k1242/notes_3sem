\subsubsection*{Пара слов от Льва Давидовича}


\begin{to_def} 
    \textit{Действием по Гамильтону} называют функционал вида
    \begin{equation*}
         S = \int_{t_0}^{t_1} L(\gamma(t), \dot{\gamma}(t), t) \d t.
    \end{equation*} 
    Переходя к однопараметрическому семейству кривых $\gamma(\alpha, t)$ получим \textit{вариацию действия}
    \begin{equation*}
        S = \int_{t_0}^{t_1} L(\gamma(\alpha, t), \dot{\gamma}(\alpha, t), t) \d t, 
        \hspace{0.5cm} 
        \delta S = \frac{d S}{d \alpha} \delta \alpha.
    \end{equation*}
\end{to_def}


\begin{to_thr}[принцип Гамильтона]
    Кривая $\gamma(\alpha, t)$ является экстремалью действия тогда и только тогда, когда является решением уравнений Лагранжа
     \begin{equation*}
         \delta S = 0
         \hspace{0.5cm} \Leftrightarrow \hspace{0.5cm} 
         \gamma(\alpha, t) \in \textnormal{Sol}\,
         \left(
     \frac{d}{dt} \frac{\partial L}{\partial \dot{q}^k} - \frac{\partial L}{\partial q^k} = 0
         \right)
         .
     \end{equation*}
\end{to_thr}


\begin{proof}[$\triangle$]
    Давайте просто проварьируем Лагранжиан, тогда
    \begin{align*}
        \delta S 
        &=
         \int_{t_0}^{t_1} 
        \left(
            \frac{\partial L}{\partial q^i} \frac{\partial q^i}{\partial \alpha} +
            \frac{\partial L}{\partial \dot{q}^i} \frac{\partial \dot{q}^i}{\partial \alpha}  
        \right) \delta \alpha \d t 
        =
        \int_{t_0}^{t_1} \left(
            \frac{\partial L}{\partial q^i} \delta q^i + \frac{\partial L}{\partial \dot{q}^i} \delta \dot{q}^i
        \right) \d t
        =
        \frac{\partial L}{\partial \dot{q}} \partial q \bigg|_{t_1}^{t_2}
        + \int_{t_1}^{t_2}
        \left(
            \frac{\partial L}{\partial q^i} - \frac{d }{d t} \frac{\partial L}{\partial \dot{q}^i} 
        \right) \,\delta q^i \d t
        = 0.
    \end{align*}
    таким образом уравнения Лагранжа выполнены. 
\end{proof}

