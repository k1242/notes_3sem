\begin{to_thr}[Теорма Ляпунова о неустойчивости I]
    Если в положении равновесия $\Pi(q)$ не имеет минимума и это определяется по квадратичной форме её разложения в ряд (в окрестности положения равновесия), то это положение равновесия неустойчиво.     
\end{to_thr}


\begin{to_thr}[Теорма Ляпунова о неустойчивости II]
    Если в положении равновесия $\Pi(q)$  имеет строгий максимум и это определяется по наинизшей степени её разложения в ряд (в окрестности положения равновесия), то это положение равновесия неустойчиво.     
\end{to_thr}

\red{Когда-нибудь здесь появится доказательство.}