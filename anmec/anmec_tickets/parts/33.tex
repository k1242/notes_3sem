Пусть есть также непотенциальные силы, часть обобщенных сил, соответстующих непотенциальным силам, обозначим $Q_i^*$, тогда
\begin{equation*}
    Q_1 = - \frac{\partial \Pi}{\partial q^i} + Q_i^*,
    \hspace{0.5cm} \Rightarrow \hspace{0.5cm} 
    \frac{d }{d t} \frac{\partial T}{\partial \dot{q}^i} - 
    \frac{\partial T}{\partial q^i} 
    =
    - \frac{\partial \Pi}{\partial q^i} + Q_i^*.
\end{equation*}
Найдём производную по времени от кинетической энергии
\begin{equation*}
    \frac{d T}{d t} = 
    \frac{\partial T}{\partial \dot{q}^i} \ddot{q}^i + \frac{\partial T}{\partial q^i} \dot{q}_i + \frac{\partial T}{\partial t}  =
    \frac{d }{d t} \left(
        \frac{\partial T}{\partial \dot{q}^i} \dot{q}^i
    \right) - \left(
        \frac{d }{d t} \frac{\partial T}{\partial \dot{q}^i} - \frac{\partial T}{\partial q^i} 
    \right) \dot{q}^i + \frac{\partial T}{\partial t}.
\end{equation*}
По \textit{теореме Эйлера об однороных функциях} для $f(x_1, \ldots, x_n)$ $k$-й степени верно что
\begin{equation*}
    \frac{\partial f}{\partial x^i} x^i = kf,
    \hspace{0.5cm} \Rightarrow \hspace{0.5cm} 
    \frac{\partial T}{\partial \dot{q}^i} \dot{q}^i = 2 T_2 + T_1.
\end{equation*}
В таком случае последнее равеноство перепишется, как
\begin{align*}
    \frac{d T}{d t} 
    &=
    \frac{d }{d t} (2 T_2 + T_1) + \frac{\partial \Pi}{\partial q^i} \dot{q}^i - Q_i^* \dot{q}^i + \frac{\partial T}{\partial t} 
    = \\ &= 
    \frac{d }{d t} (2 T_2 + 2 T_1 + 2 T_0) - \frac{d }{d t} (T_1 + 2 T_0) + \frac{d \Pi}{d t} - \frac{\partial \Pi}{\partial t} - Q_i^* \dot{q}^i + \frac{\partial T}{\partial t}.
\end{align*}
Таким образом мы доказали следующую теорему.

\begin{to_thr}
     Полная мехническая энергия голономной системы $E = T+ \Pi$ изменяется следующим образом:
     \begin{equation*}
         \frac{d E}{d t} = N^* + \frac{d }{d t} (T_1 + 2 T_0) + \frac{\partial \Pi}{\partial t} - \frac{\partial T}{\partial t}.
     \end{equation*}
     Где $N^* = Q^*_i \dot{q}^i$ -- мощность непотенциальных сил.
\end{to_thr}


\begin{to_def} 
    Голономная склерономная система с  $\Pi \equiv \Pi(q)$ называется \textit{консервативной}, при чём $d E / d t = 0$.
\end{to_def}

\subsubsection*{Гироскопические силы}


\begin{to_def} 
    Непотенициальные силы называют \textit{гироскопическими}, если их мощность равна $0$. 
\end{to_def}


Пусть $Q^*_i = \gamma_{ik} \dot{q}^k$. Если $\gamma_{ik} = -\gamma_{ki}$, то силы $Q^*_i$ гиросокопические, соответсвенно кососимметричность $\gamma_{ik}$ необходима и достаточна.

Более того, имеет место равеноство

\vspace{-25pt}
\begin{equation*}
    \sum_\nu \vc{F}_\nu \cdot \vc{v}_\nu = \sum_\nu \vc{F}_\nu \cdot
    \left(
        \frac{\partial \vc{r}_\nu}{\partial q^i} \dot{q}^i + \frac{\partial \vc{r}_\nu}{\partial t} 
    \right) 
    =
     \bigg(
     \overbrace{
     \sum_\nu \vc{F}_\nu \cdot \frac{\partial \vc{r}_\nu}{\partial q^i}
     }^{
     Q_i
     }
      \bigg)
     \dot{q}^i + \sum_\nu \vc{F}_\nu \cdot \frac{\partial \vc{r}_\nu}{\partial t},
     \hspace{0.25cm} \overset{\partial \vv{r}_\nu / \partial t = 0}{=}  \hspace{0.25cm} 
     \sum_\nu \vc{F}_\nu \cdot \vc{v}_\nu = Q_i \dot{q}^i.
\end{equation*}
Поэтому для склерономных систем $N^* = 0$ выражается в $\sum_\nu \vc{F}_\nu^* \cdot \vc{v}_\nu = 0$.

\subsubsection*{Диссипативные силы}

\begin{to_def} 
    Непотенциальные силы называются диссипативными, если их $N^* \leq 0$, но $N^* \not \equiv 0$. При $\Pi = \Pi(q)$ и диссипативности сил $d E / dt \leq 0$, тогда система называется диссипативной. В случае определенно-отрицательной $N^* (\dot{q})$ дисспация называется \textit{полной}, а в случае знакопостоянной отрицательной $N^*$ \textit{частичной}.
\end{to_def}

\begin{to_def} 
    \textit{Диссипативной функцией Рэлея} называется  положительная квадратичная форма $R$ такая, что
    \begin{equation*}
        R = \frac{1}{2} b_{ik} \dot{q}^i \dot{q}^k,
        \hspace{1cm}
        Q^*_i = - \frac{\partial R}{\partial \dot{q}^i}  = - b_{ik} \dot{q}^k.
    \end{equation*}
    Тогда для склерономной системы можность $N^*$ непотенциальных сил равна
    \begin{equation*}
        \sum_\nu \vc{F}_\nu^* \cdot \vc{v}_\nu = Q_i^* \dot{q}^i = - 2 R \leq 0.
    \end{equation*}
\end{to_def}




