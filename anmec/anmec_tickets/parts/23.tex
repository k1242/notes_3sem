Пусть каждой точке среды соответсвует $\xi^1, \xi^2, \xi^3$, собственно $(\xi, t)$ -- \textit{лгранжевы переменные}. \textit{Закон движения среды} в таком случае это
\begin{equation}
    \vc{r} (\xi, t),
\end{equation}
скорость же
$$
    \vc{v} = \frac{\partial \vc{r}(\xi, t)}{\partial t},
    \hspace{0.5cm} 
    \vc{\mathrm{w}} = \frac{\partial \vc{v} (\xi, t)}{\partial t},
$$
и так далее.

Альтернативно можеем задать $(x, t)$ -- эйлерово описание. Тогда
$$
    \vc{v}(x, t), \vc{\mathrm{w}}(x, t) 
    \hspace{0.5cm} \text{-- поля скоростей и ускорений.}
$$

В частности, представляя движение по шоссе, полоса 1,2,3 и участок трассы -- эйлерово описание среды.
Если же мы будем следить за каждой машиной, то это будет лагранжево описание.