\subsubsection*{Теорема о сложении скоростей}
\begin{to_thr}
	Абсолютная скорость точки равна сумме переносной и относительной скорости: $\vc{v}^a = \vc{v}^e + \vc{v}^r$.
\end{to_thr}
\begin{proof}[$\triangle$]
	Для точки $P$ в абсолютной системе координат:
	\begin{equation*}
		\vc{R} = \vc{R}_0 + \vc{r}
		\hspace*{1 cm}
		\Rightarrow
		\hspace*{1 cm}
		\vc{v}_a = \vc{\dot{R}} = \vc{\dot{R}}_0 + \vc{\dot{r}} = \underbrace{\vc{v}_0 + \vc{\omega}\times \vc{r}}_{v^e} + \underbrace{A \vc{\dot{\rho}}}_{v^r}.
	\end{equation*}
	\textit{Переносная скорость} $v^e$ --- есть скорость той точки подвижной системы координат, в которой находится $P$.
	Таким образом показали напрямую разложение.
\end{proof}

\subsubsection*{Теорема Кориолиса}
\begin{to_thr}
	Абсолютное ускорение точки равно сумме переносного, относительного и карелысого ускорения: $w^a = w^e + w^r + w^c$.
\end{to_thr}
\begin{proof}[$\triangle$]
	Для абсолютного ускорения точки, продифференцируем ещё раз:
	\begin{equation*}
		\vc{w}^a = \vc{\dot{v}}_0 + \vc{\dot{\omega}} \times \vc{\dot{r}} + \vc{\omega} \times \vc{\dot{r}} + \dot{A} \vc{\dot{\rho}} + A \vc{\ddot{\rho}} = w_0 + \vc{\varepsilon} \times \vc{r} + \vc{\omega} \times (\vc{\omega} \times \vc{r} + A \vc{\dot{\rho}}) + \dot{A} \vc{\dot{\rho}} + A \vc{\ddot{\rho}}.
	\end{equation*}
	$\vc{\varepsilon}$ --- угловое ускорение подвижной системы координат, а $A \ddot{\rho} = w^r$:
	\begin{equation*}
		w^a = \underbrace{ w_0 + \vc{\varepsilon} \times \vc{r} + \vc{\omega} \times (\vc{\omega} \times \vc{r})}_{w^e} + w^r + \omega \times A \vc{\dot{\rho}} + \dot{A} \vc{\dot{\rho}}.
	\end{equation*}
	И последние два слогаемых дадут кареолисового ускорение: $\dot{A} \vc{\dot{\rho}} = \dot{A} A^{-1} A \vc{\dot{\rho}} = \vc{\omega} \times A \vc{\dot{\rho}} $, тогда получаем $w^c = 2 \vc{\omega} \times v^r$. Итого получаем искомую формулу.
\end{proof}
