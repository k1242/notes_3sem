\begin{to_def} 
    Рассмотрим голономную (обобщенно) консервативную систему. Рассмотрим движение в $n$-мерном координатном пространстве. Рассмотрим прямые и окольные пути такие, что $H = h = \const$. При таком \textit{изоэнергетическом варьировании} $t_1-t_0$ не обязательно одинаково для прямого и окольного пути.     
\end{to_def}

\subsubsection*{Принцип Мопертюи-Лагранжа}

\begin{to_def} 
    При заданной $h$ уравнения движения могут быть записаны в форме Якоби, они также будут иметь форму уравнений Лагранжа, где $L \to P$, $t \to q_1$. По аналогии с действием $S$ по Гамильтону введём \textit{действие по Лагранжу}:
\begin{equation*}
    W = \int_{q^{1}_1}^{q_1^1} P \d q^1.
\end{equation*} 
\end{to_def}


\begin{to_thr}[Принцип Мопертюи-Лагранжа\footnote{
    Или \textit{принцип наименьшего действия Якоби}.
}]
     Среди всех кинематически возможных путей голономной консервативной системы, прямой путь выделяется тем, что для него действие по Лагранжу $W$ имеет стационарное значение $\delta W = 0$.
\end{to_thr}

Аналогично экстремальное значение будет принимать действие по Лагранжу при отсутствие кинематических фокусов в рассматриваемой области. 

Как было показано раннее функция Якоби $P$ может быть вычислена, как
\begin{equation*}
    W = \int_{q^{1}_1}^{q_1^1} P \d q^1 = \int_{q^{1}_1}^{q_1^1} \frac{2T}{\dot{q}^1} \d q^1 = \int_{t_0}^{t_1} 2 T \d t.
\end{equation*}
Важно заметить, что $T + \Pi = \const$,  а вот $t_1$ не зафиксировано. Иначе можно переписать
\begin{equation*}
    W = \int_{t_0}^{t_1} \sum_{\nu=1}^N m_\nu v_\nu^2 \d t 
    =
    \sum_{\nu=1}^N \int_{s_\nu^0}^{s_\nu^1} m_\nu v_\nu \d s_\nu,
\end{equation*}
т.е. для консервативной системы действие по Лагранжу равно сумме работ количеств движения точек системы на соответствующих их перемещениях.















