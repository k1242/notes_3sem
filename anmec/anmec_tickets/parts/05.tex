\subsubsection*{Сложение мгновенных вращений вокруг пересекающихся осей}
Пусть тело мгновенно вращается с $\vc{\omega}_1 $ относительно $O_1 x_1 y_1 z_1$, которая сама вращается с $\vc{\omega}_2$ относительно $O_a X Y Z$
Предположим, что оси вращений пересекаются в точке $A$, которая тогда обладает нулевой скоростью. Тогда наше сложное движение представляется как вращения с каким-то $\vc{\Omega}$ по оси через $A$. 

Для произвольной точки $P$ тела:
\begin{equation*}
	\vc{v}^a = \vc{\omega}_1 \times \overrightarrow{A P} + \vc{\omega}_2 \times \overrightarrow{A P} = (\vc{\omega}_1 + \vc{\omega}_2) \times \overrightarrow{A P}.
\end{equation*}
С другой стороны:
\begin{equation*}
	\vc{v}^a = \vc{\Omega} \times \overrightarrow{A P}
	\hspace*{1 cm}
	\Rightarrow
	\hspace*{1 cm}
	\vc{\Omega} = \vc{\omega}_1 + \vc{\omega}_2.
\end{equation*}
Результат выкладок выше можно обобщить и на $n$ таких вращений.

\subsubsection*{Параллельные оси и пара вращений}
Если же $\vc{\omega}_1$ и $\vc{\omega}_2$ не пересекаются --- параллельны, то рассмотрим точки лежащие в перпендикулярной к этом скоростям плоскости.
Тогда пусть прямая, по которой пересекается эта плоскость с плоскостью, в которой лежат скорости --- $A B$. На $A B$ есть точка $C$, которая остаётся неподвижной:
\begin{equation*}
	\vc{v}_C = 0 = \vc{\omega}_1 \times \overrightarrow{A C} + \vc{\omega}_2 \times \overrightarrow{B C}
	\hspace*{0.4 cm}
	\Rightarrow
	\hspace*{0.4 cm}
	\omega_1 AC = \omega_2 BC,
	\hspace*{1 cm}
	\vc{\Omega} = \vc{\omega}_1 + \vc{\omega}_2.
\end{equation*}
 
\begin{to_def}
	\textit{Пара вращений} --- совокупность двух мгновенных вращений вокруг параллельных осей с равными по модулю, но противоположными по направлению угловыми скоростями.
\end{to_def}
\begin{to_def}
	Плоскость в которой лежат $\vc{\omega}_1$ и $\vc{\omega}_2$ (пара) называют \textit{плоскостью пары}.
\end{to_def}
\begin{to_def}
	Расстояние между векторами пары $d$ называют \textit{плечо пары}. А $\vc{d} \times \vc{\omega}_2$ --- \textit{момент пары}.
\end{to_def}
Тело участвующее в паре вращений движется в итоге поступательно:
\begin{equation*}
	\vc{v} = \vc{\omega}_1 \times \overrightarrow{AP} + \vc{\omega}_2 \times \overrightarrow {B P} = \overrightarrow {A P} \times \vc{\omega}_2 - \overrightarrow {B P} \vc{\omega}_2 = \vc{d} \times \vc{\omega}.
\end{equation*}

\subsubsection*{Кинематические уравнения Эйлера}
\begin{to_def}
	Кинематическими уравнениями Эйлера называют следующую систему:
	\begin{equation*}
		\vc{\omega} = \begin{pmatrix}
			p \\ q \\ r
		\end{pmatrix}_{O xyz}
		\hspace*{1 cm}
		\leadsto
		\hspace*{1 cm}
		\left\{\begin{aligned}
			&p = \dot{\psi} \sin \theta \sin \varphi + \dot{\theta} \cos \varphi
			\\
			&q = \dot{\psi} \sin \theta \cos \varphi - \dot{\theta} \sin \varphi
			\\
			&r = \dot{\psi} \cos \theta + \dot{\varphi}
		\end{aligned}\right.
	\end{equation*}
 
\end{to_def}
 