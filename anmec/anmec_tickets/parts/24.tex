\subsubsection*{Подход к деформации}
Под влиянием приложенных внешних сил твердые тела в той или иной степени \textit{деформируются}, то есть меняют свою форму и объём.
Рассмотрим точку деформируемого тела $\vc{r}(x^1,x^2,x^3)$, которая после деформации станет $\vc{r}'$.
\begin{to_def}
	$\vc{u} = \vc{r}' - \vc{r}$ --- \textit{вектор деформации}. Координаты $y^i$ смещенной точки могут быть выражены через $x^i$, таким образом $\vc{u}(x^i)$ полностью определяет деформацию тела.
\end{to_def}

Рассмотрим две близкие точки, расстояние между ними до деформации
$d (l')^2 = (d x^1)^2 + (d x^2)^2 + (d x^3)^2$, а 
после $d `l^2 = (d y^1)^2 + (d y^2)^2 + (d y^3)^2$.
Записав через деформацию (здесь $u_i = g_{i k}u^k$):
\begin{equation*}
	d (l')^2 = (d x^i + d u_i)^2
	\hspace*{0.4 cm}
	\Rightarrow
	\hspace*{0.4 cm}
	\left(d u_i = \frac{\partial u_i}{\partial x^k} d x^k \right)
	\hspace*{0.4 cm}
	\Rightarrow
	\hspace*{0.4 cm}
	d (l')^2 = d l^2 + 2 \frac{\partial u_i}{\partial x^k} d x^i d x^k + \frac{\partial u_i}{\partial x^k} \frac{\partial u_i}{\partial x^l} d x^k d x^l.
\end{equation*}
Поменяем во втором члене индексы $i$ и $k$, а в третьем $i$ и $l$:
\begin{equation*}
	d (l')^2 = d l^2 + 2 u_{i k} d x^i d x^k,
	\hspace*{1 cm}
	\text{где }
	u_{i k} = \frac{1}{2} \left(\frac{u_i}{\partial x^k} + \frac{u_k}{\partial x^i} + \frac{u_i}{\partial x^l} \frac{u_l}{\partial x^k}\right).
\end{equation*}
Как и всякий симметричный тензор, можно привести тензор $u_{i k}$ в каждой данной точке к главным осям. 
Это значит, что в каждой данной точке можно выбрать такую систему координат --- главные оси тензора, --- в которой из всех компонент 
$u_{i k}$ отличны от нуля только диагональные компоненты $u_{1 1}, u_{2 2}, u_{3 3}$.

При малых же деформациях, за исключением редких случаем, и вектор деформации оказывается малым, тогда можем пренебречь последним членом в полученном нами значении для тензора деформации:
\begin{to_def}[Тензор деформации в малом приближении]
	\begin{equation*}
		u_{i k} = \frac{1}{2} \left( \frac{\partial u_i}{\partial x^k} + \frac{\partial u_k}{\partial x^i}\right) .
	\end{equation*}
\end{to_def}

\subsubsection*{Изменение объёма при деформации}
Относительные удлинения элементов длины вдоль направлений главных осей тензора деформации с нашей точностью: $\sqrt{1 + 2 u_{i i}} - 1 \approx u_{ii}$.

Малый элемент объёма тогда претерпит следующее изменение:
\begin{equation*}
	d V' = d V(1 + u_{1 1})(1 + u_{2 2})(1 + u_{3 3}) d V (1 + u_{1 1} + u_{2 2} + u_{3 3})
	\hspace*{0.5 cm}
	\Rightarrow
	\hspace*{0.5 cm}
	u_{i i} = \frac{d V' - d V}{d V}.
\end{equation*}
Для несжимаемого тела, тогда $u_{i i}$ --- сумма диагональных компонент тензора в главных осях --- нулевая. Такая деформация называется \textit{сдвигом}.

\subsubsection*{Тензор скорости деформации}
\begin{to_def}
	\textit{Тензором скорости деформации}  назовём просто $\dot{u}_{i j} = \frac{d u_{i j} }{ d t} = \frac{1}{2}\left(\frac{\partial v_i}{\partial x^j} + \frac{\partial v_j}{\partial x ^i}\right)$.
\end{to_def}
Тогда рассмотрим движение элемента объёма тела во времени: $\vc{v} = \vc{v}(\vc{r} + \delta \vc{r})$, до первого члена малости:
\begin{equation*}
	\vc{v} = \vc{v}_0 + \frac{\partial \vc{v}}{\partial x^j} \delta x^j
	\hspace*{0.3 cm}
	\Rightarrow
	\hspace*{0.3 cm}
	\left(\vc{v}_0 = 0\right)
	\hspace*{0.3 cm}
	\Rightarrow
	\hspace*{0.3 cm}
	v_i = \frac{\partial v_i}{\partial x^j} \delta x^j = \frac{1}{2} \left(\frac{\partial v_i}{\partial x^j} + \frac{\partial v_j}{\partial x^i} \right) \delta x^j + \frac{1}{2} \left(\frac{\partial v_i}{\partial x^j} - \frac{\partial v_j}{\partial x^i} \right) \delta x^j.
\end{equation*}
Получаем уравнение, где с помощью замены, и вернув начальную скорость, явно можем показать, что
\begin{to_thr}[Теорема Гельмгольца]
	Тензор скоростей деформации можно разложить на сумму симметричного и кососимметричного:
	\begin{equation*}
	 	\vc{v} = \vc{v}_0 + u_{i j} \delta x^j e^i + \vc{\omega} \times \delta \vc{r}.
	 \end{equation*} 	
\end{to_thr} 

\subsubsection*{Обобщенный закон Гука}

Пусть $E$ -- модуль Юнга, $\mu$ -- коэффициент Пуассона. Тогда
$$
    u_{11} = \frac{\sigma_{11}}{E}, \hspace{0.5cm} 
    u_{22} = u_{33} = - \frac{\mu}{E} \sigma_{11}.
$$
Перепишем это в виду
$$
    u_{11} = \frac{\sigma_{11}}{E}  - \frac{\mu}{E} \sigma_{22} -
    \frac{\mu}{E} \sigma_{33} = \frac{1+\mu}{E} \sigma_{11} - \frac{\mu}{E} \tr \sigma.
$$
Или, в матричном виде
$$
    \begin{pmatrix}
        u_{11} & 0 & 0\\
        0 & u_{22} & 0 \\
        0 & 0 & u_{33}
    \end{pmatrix} =
    \frac{1+\mu}{E} 
    \begin{pmatrix}
        \sigma_{11} & 0 & 0\\
        0 & \sigma_{22} & 0 \\
        0 & 0 & \sigma_{33}
    \end{pmatrix} -
    \frac{\mu}{E} \tr \sigma 
    \begin{pmatrix}
        1 & 0 & 0\\
        0 & 1 & 0 \\
        0 & 0 & 1
    \end{pmatrix}.
$$
В тензорном виде
$$
    u_{ik} = \frac{1+\mu}{E} \sigma_{ik} - \frac{\mu}{E} \delta_{ik} \tr \sigma .
$$
Выразим $u$:
$$
    \tr u = \frac{1+\mu}{E} \tr \sigma - \frac{3\mu}{E} \tr \sigma
    \hspace{0.5cm} \Rightarrow \hspace{0.5cm} 
    \tr \sigma = \frac{E}{1-2\mu} \tr u.
$$
Так и получаем \textit{обобщенный закон гука}:
\begin{equation*}
    \sigma_{ik} = \frac{E}{1+\mu} \left[
        u_{ik} + \frac{\mu}{1 - 2\mu} \delta_{ik} \tr u 
    \right]
\end{equation*}