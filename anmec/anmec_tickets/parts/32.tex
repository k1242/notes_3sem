Подставим разложение кинетической энергии в уравнения Лагранжа, оставив только слагаемые с обобщёнными ускорениями $f_j (q, \dot{q}, t) = a_{jk} \ddot{q}^j$. 
\begin{equation*}
    T = \frac{1}{2} \sum_\nu m_\nu \dot{\vc{r}}_\nu^2 = \frac{1}{2} \sum_\nu
    \left(
        \frac{\partial \vc{r}_\nu}{\partial q^j} \dot{q}^j + \frac{\partial \vc{r}_\nu}{\partial t} 
    \right)^2 = 
    \frac{1}{2} 
    \bigg[
    \underbrace{
        a_{jk} \dot{q}^j \dot{q}^k
    }_{
        2T_2
    } +
    \underbrace{
        a_j \dot{q}^j
    }_{
        2T_1
    } +
    \underbrace{
        a_0
    }_{
        2T_0
    }
    \bigg],
\end{equation*}
где коэффициенты, соответственно, равны 
\begin{equation*}
    a_{jk}(q, t) = \sum_\nu m_\nu \frac{\partial \vc{r}_\nu}{\partial q^j} \cdot \frac{\partial \vc{r}_\nu}{\partial q^k},
    \hspace{0.5cm} 
    a_j(q, t) = \sum_\nu m_\nu \frac{\partial \vc{r}_\nu}{\partial q^j} \cdot \frac{\partial \vc{r}_\nu}{\partial t},
    \hspace{0.5cm} 
    a_0 = \sum_\nu m_\nu 
    \left(
    \frac{\partial \vc{r}_\nu}{\partial t} 
            \right)^2.
\end{equation*}
Для склерономных систем $\partial \vc{r}_\nu / \partial t = 0$, соотвественно $T = a_{jk} \dot{q}^j \dot{q}^k$, при чём $a_{jk} \equiv a_{jk} (q)$.

Теперь подставим значение $T$ в уравнения Лагранжа, и получим, что
$
    a_{ik} \ddot{q}^k = f_i,
$
где $f_1 = f_1(q, \dot{q}, t)$. Уравнений в системе $n$, причём $a_{jk}$ является положительно определенной формой\footnote{
    \red{Требует отдельного доказательства.}
}, соответственно невырожденной. 

\begin{to_thr} 
    Уравнения Лагранжа второго рода разрешимы относительно обобщенных ускорений 
\end{to_thr}
