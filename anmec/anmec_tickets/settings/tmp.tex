\documentclass[twoside]{article}

% file's preambule


% connect packages
%%%%%%%%%%%%%%%%%%%%%%%%%%%%%%%%%%%%%%%%%%%%%%%%%%%%%%%%%%%%%%%%%%%%%%
\usepackage[T2A]{fontenc}                   %!? закрепляет внутреннюю кодировку LaTeX
\usepackage[utf8]{inputenc}                 %!  закрепляет кодировку utf8
\usepackage[english,russian]{babel}         %!  подключает русский и английский
\usepackage{amsmath}                        %!  |
\usepackage{amssymb,textcomp, esvect,esint} %!  |важно для формул 
\usepackage[margin=2cm]{geometry}           %!  фиксирует оступ на 2cm
\usepackage{amsfonts}                       %!  математические шрифты
\usepackage{amsthm}                         %!  newtheorem и их сквозная нумерация
\usepackage{graphicx}                       %?  графическое изменение текста
\usepackage{indentfirst}                    %   добавить indent перед первым параграфом
\usepackage{xcolor}                         %   добавляет цвета
\usepackage{enumitem}                       %!  задание макета перечня.
\usepackage[unicode, pdftex]{hyperref}      %!  оглавление для панели навигации по PDF-документу + гиперссылки
\usepackage{booktabs}                       %!  добавляет книжные линии в таблицы
\usepackage{hypcap}                         %?  адресация на картинку, а не на подпись к ней
\usepackage{abraces}                        %?  фигурные скобки сверху или снизу текста
\usepackage{caption}                        %-  позволяет корректировать caption 
\usepackage{multirow}                       %   объединение ячеек в таблицах
\usepackage{pifont}                         %!  нужен для крестика
\usepackage{cancel}                         %!  аутентичное перечеркивание текста
\usepackage{ulem}                           %!  перечеркивание текста
\usepackage{tikz}                           %!  высокоуровневые рисунки (кружочек)
\usepackage{titling}                        %-  автоматическое заглавие 
\usepackage{blindtext}                      %-  слепой текст
\usepackage{fancyhdr}                       %   добавить верхний и нижний колонтитул

\usepackage{import}
\usepackage{xifthen}
\usepackage{pdfpages}
\usepackage{transparent}

\usepackage{mathrsfs}  
%%%%%%%%%%%%%%%%%%%%%%%%%%%%%%%%%%%%%%%%%%%%%%%%%%%%%%%%%%%%%%%%%%%%%%

%%%%%%%%%%%%%%%%%%%%%%% ИНТЕГРАЛЫ %%%%%%%%%%%%%%%%%%%%%%%%%%%%%%%%%%%%
\makeatletter
\def\upintkern@{\mkern-7mu\mathchoice{\mkern-3.5mu}{}{}{}}
\def\upintdots@{\mathchoice{\mkern-4mu\@cdots\mkern-4mu}%
 {{\cdotp}\mkern1.5mu{\cdotp}\mkern1.5mu{\cdotp}}%
 {{\cdotp}\mkern1mu{\cdotp}\mkern1mu{\cdotp}}%
 {{\cdotp}\mkern1mu{\cdotp}\mkern1mu{\cdotp}}}
\newcommand{\upiint}{\DOTSI\protect\UpMultiIntegral{2}}
\newcommand{\upiiint}{\DOTSI\protect\UpMultiIntegral{3}}
\newcommand{\upiiiint}{\DOTSI\protect\UpMultiIntegral{4}}
\newcommand{\upidotsint}{\DOTSI\protect\UpMultiIntegral{0}}
\newcommand{\UpMultiIntegral}[1]{%
  \edef\ints@c{\noexpand\upintop
    \ifnum#1=\z@\noexpand\upintdots@\else\noexpand\upintkern@\fi
    \ifnum#1>\tw@\noexpand\upintop\noexpand\upintkern@\fi
    \ifnum#1>\thr@@\noexpand\upintop\noexpand\upintkern@\fi
    \noexpand\upintop
    \noexpand\ilimits@
  }%
  \futurelet\@let@token\ints@a
}
\makeatother

\DeclareFontFamily{OMX}{mdbch}{}
\DeclareFontShape{OMX}{mdbch}{m}{n}{ <->s * [0.8]  mdbchr7v }{}
\DeclareFontShape{OMX}{mdbch}{b}{n}{ <->s * [0.8]  mdbchb7v }{}
\DeclareFontShape{OMX}{mdbch}{bx}{n}{<->ssub * mdbch/b/n}{}

\DeclareSymbolFont{uplargesymbols}{OMX}{mdbch}{m}{n}
\SetSymbolFont{uplargesymbols}{bold}{OMX}{mdbch}{b}{n}
\DeclareMathSymbol{\upintop}{\mathop}{uplargesymbols}{82}
\DeclareMathSymbol{\upointop}{\mathop}{uplargesymbols}{"49}

\DeclareFontEncoding{MDB}{}{}
\DeclareFontFamily{MDB}{mdbch}{}
\DeclareFontShape{MDB}{mdbch}{m}{n}{ <->s * [0.8]  mdbchrmb }{}
\DeclareFontShape{MDB}{mdbch}{b}{n}{ <->s * [0.8]  mdbchbmb }{}
\DeclareFontShape{MDB}{mdbch}{bx}{n}{<->ssub * mdbch/b/n}{}
\DeclareFontSubstitution{MDB}{cmr}{m}{n}
\DeclareSymbolFont{mathdesignB}{MDB}{mdbch}{m}{n}%
\SetSymbolFont{mathdesignB}{bold}{MDB}{mdbch}{b}{n}%
\DeclareMathSymbol{\upintclockwise}{\mathop}{mathdesignB}{128}
\DeclareMathSymbol{\upointclockwise}{\mathop}{mathdesignB}{130}
\DeclareMathSymbol{\upointctrclockwise}{\mathop}{mathdesignB}{132}
\DeclareMathSymbol{\upoiint}{\mathop}{mathdesignB}{134}
\DeclareMathSymbol{\upoiiint}{\mathop}{mathdesignB}{136}

\makeatletter
\newcommand{\upint}{\DOTSI\upintop\ilimits@}
\newcommand{\upoint}{\DOTSI\upointop\ilimits@}
\makeatother

%%%%%%%%%%%%%%%%%%%%%%%%%%%%%%%%%%%%%%%%%%%%%%%%%%%%%%%%%%%%%%%%%%%%%%



% add (renew) commands
%%%%%%%%%%%%%%%%%%%%%%%%%%%%%%%%%%%%%%%%%%%%%%%%%%%%%%%%%%%%%%%%%%%%%%
\renewcommand{\Im}{\mathop{\mathrm{Im}}\nolimits}
\renewcommand{\Re}{\mathop{\mathrm{Re}}\nolimits}
\renewcommand{\div}{\mathop{\mathrm{div}}\nolimits}
\renewcommand{\d}{\, d}
\renewcommand{\leq}{\leqslant}
\renewcommand{\geq}{\geqslant}

\newcommand{\vc}[1]{\mbox{\boldmath $#1$}}
\newcommand{\T}{^{\text{T}}}

\newcommand{\grad}{\mathop{\mathrm{grad}}\nolimits}
\newcommand{\rot}{\mathop{\mathrm{rot}}\nolimits}
\newcommand{\diag}{\mathop{\mathrm{diag}}\nolimits}
\newcommand{\Ker}{\mathop{\mathrm{Ker}}\nolimits}
\newcommand{\Spec}{\mathop{\mathrm{Spec}}\nolimits}
\newcommand{\sign}{\mathop{\mathrm{sign}}\nolimits}
\newcommand{\tr}{\mathop{\mathrm{tr}}\nolimits}
\newcommand{\rg}{\mathop{\mathrm{rg}}\nolimits}

\newcommand{\const}{\text{const}}
\newcommand{\red}[1]{\textcolor{red}{#1}}
\newcommand{\xmark}{\ding{55}}

\renewcommand{\int}{\upint}
\renewcommand{\oint}{\upoint}
%%%%%%%%%%%%%%%%%%%%%%%%%%%%%%%%%%%%%%%%%%%%%%%%%%%%%%%%%%%%%%%%%%%%%%


% some tikz commands
%%%%%%%%%%%%%%%%%%%%%%%%%%%%%%%%%%%%%%%%%%%%%%%%%%%%%%%%%%%%%%%%%%%%%%
\makeatletter %%%%%%%%%%%%%%% КРУЖОЧЕК %%%%%%%%%%%%%%%%%%%%%%%%%%%%%%%
\newcommand*{\encircled}[1]{\relax\ifmmode\mathpalette
\@encircled@math{#1}\else\@encircled{#1}\fi}
\newcommand*{\@encircled@math}[2]{\@encircled{$\m@th#1#2$}}
\newcommand*{\@encircled}[1]{%
  \tikz[baseline,anchor=base]{\node[draw,circle,outer sep=0pt,
                                        inner sep=.2ex] {#1};}}
\makeatother

\usepackage{arydshln} %%%%%%%%%%%%%%% ЛИНИИ В МАТРИЧКЕ %%%%%%%%%%%%%%%
\makeatletter
  \renewcommand*\env@matrix[1][*\c@MaxMatrixCols c]{%
    \hskip -\arraycolsep
    \let\@ifnextchar\new@ifnextchar
  \array{#1}}
\makeatother
%%%%%%%%%%%%%%%%%%%%%%%%%%%%%%%%%%%%%%%%%%%%%%%%%%%%%%%%%%%%%%%%%%%%%%


% create environment
%%%%%%%%%%%%%%%%%%%%%%%%%%%%%%%%%%%%%%%%%%%%%%%%%%%%%%%%%%%%%%%%%%%%%%
\newtheorem{to_thr}{Thr}[section]
\newtheorem{to_suj}[to_thr]{Suj}
\newtheorem{to_lem}[to_thr]{Lem}
\newtheorem{to_law}[to_thr]{Law}
\newtheorem{to_com}[to_thr]{Com}
\newtheorem{to_con}[to_thr]{Con}
\theoremstyle{definition}
\newtheorem{to_def}[to_thr]{Def}

\newenvironment{description*}
{
    \begin{description}
        \setlength{\itemsep}{1pt}
        \setlength{\parskip}{1pt}
        }
    {\end{description}
}
%%%%%%%%%%%%%%%%%%%%%%%%%%%%%%%%%%%%%%%%%%%%%%%%%%%%%%%%%%%%%%%%%%%%%%

\newcommand{\incfig}[1]{%
    % \def\svgwidth{\columnwidth}
    \import{./figures/}{#1.pdf_tex}
}

% add some colors
%%%%%%%%%%%%%%%%%%%%%%%%%%%%%%%%%%%%%%%%%%%%%%%%%%%%%%%%%%%%%%%%%%%%%%
\definecolor{grey}{HTML}{666666}
\definecolor{linkcolor}{HTML}{0000CC}
\definecolor{urlcolor}{HTML}{006600}
\hypersetup{
    pdfstartview=FitH,  
    linkcolor=linkcolor,
    urlcolor=urlcolor, 
    colorlinks=true,
    citecolor=blue}
%%%%%%%%%%%%%%%%%%%%%%%%%%%%%%%%%%%%%%%%%%%%%%%%%%%%%%%%%%%%%%%%%%%%%%


% add page header
%%%%%%%%%%%%%%%%%%%%%%%%%%%%%%%%%%%%%%%%%%%%%%%%%%%%%%%%%%%%%%%%%%%%%%
\pagestyle{fancy}
\fancyhf{}
\fancyhead[RE,LO]{\textsc{Ф\raisebox{-1.5pt}{и}з\TeX}}
% \fancyhead[LE,RO]{ОбщеФиз}
\fancyhead[CO,CE]{\leftmark}
\fancyfoot[LE,RO]{\textcolor{grey}{\texttt{\thepage}}}
%%%%%%%%%%%%%%%%%%%%%%%%%%%%%%%%%%%%%%%%%%%%%%%%%%%%%%%%%%%%%%%%%%%%%%


\begin{document}

% set skip of equation length 

\setlength{\abovedisplayskip}{3pt}
\setlength{\abovedisplayshortskip}{3pt}
\setlength{\belowdisplayskip}{3pt}
\setlength{\belowdisplayshortskip}{3pt}

\numberwithin{equation}{section}

% input files

% document's head
\vspace{2cm}

\begin{center}
    \LARGE \textsc{Конспект второго тома курса теоретической физики <<Теория поля>>}
\end{center}

\hrule

\begin{flushright}
    \begin{tabular}{rr}
    % written by:
        \textbf{Авторы}: 
        & Хоружий Кирилл \\
        & Примак Евгений \\
        &\\
    % date:
        \textbf{От}: &
        \textit{\today}\\
    \end{tabular}
\end{flushright}

\thispagestyle{empty}
\tableofcontents
\newpage

%%%%%%%%%%%%%%%%%%%%%%%%%%%%%%%%%%%%%%%%%%%%%%%%%%%%%%%%%%%%%%%%%%%%%%%%%%%%%%%%%%%
\section*{Римановы и полуримановы многообразия}
\setcounter{section}{10}
\addcontentsline{toc}{section}{Римановы и полуримановы многообразия}
%%%%%%%%%%%%%%%%%%%%%%%%%%%%%%%%%%%%%%%%%%%%%%%%%%%%%%%%%%%%%%%%%%%%%%%%%%%%%%%%%%%

\sbsnum{48}{Риманова структура на многообразии}
\begin{to_def}
	Римановой структурой на гладком $M$ называется задание квадратичной формы $g_p >0$ на касательном пространстве $T_p M$, гладко зависящее от точки $p$. Полуриманова структура --- это задание невырожденной, но не обязательно положительно определённой квадратичной формы.
\end{to_def}

Будем отождествлять квадратичную форму с симметричным скалярным произведением:
\begin{equation*}
	g_{i,j} = g \left(\frac{\partial}{\partial x_i}, \frac{\partial}{\partial x_j}\right),
	\hspace*{1 cm}
	g = g_{i j} \d x^i \otimes \d x^j.
\end{equation*}
На $\forall M$ (гладком) $\exists g$. Достаточно взять локальное конечное разбиение единицы $\{p_i\}$, подчинённое картам $\{U_i\}$, в каждой карте построить $g_i$ (например: стандартная риманова структура $\delta_{i j} \d x^i \otimes \d x^j $) и положить $g = \sum_i p_i g_i$.
Эта сумма будет локально конечна и в любой точке будет давать $g>0$, так как сумма неотрицательно определённых форм, хотя бы одна из которых положительно определена, будет положительно определена.

На вложенном $M \subset \mathbb{R}^N$ можно просто ограничить стандартную риманову структуру с евклидова пространства на его подмногообразие $M$:
\begin{equation*}
	 i\colon M \rightarrow \mathbb{R}^n, \hspace*{1 cm}  
	 g = i^* g_0.
\end{equation*}

В таком случае, если локальные координаты $M$ --- это $u_{1}, \ldots, u_n $, то риманова структура задаётся в координатах как
\begin{equation}
	g_{i j} = g_0\left(\frac{\partial r}{\partial u_i} \cdot \frac{\partial r}{\partial u_j}\right) = \left(\frac{\partial r}{\partial u_i} \cdot \frac{\partial r}{\partial u_j}\right),	
\end{equation}
где $g_{0}$, $(\cdot)$ --- евклидово скалярное произведение.

борелевское

гомотопичные 

\sbsnum{49}{Риманов объём, объём на поризведении}
\begin{to_def}
	Плотность меры --- это функция в каждых локальных координатах, которая при заменах координат преобразуется почти как диф-форма высшего ранга, но умножается при замене координат на модуль якобиана обратной замены, а не на якобиан без модуля. Её интеграл уже не зависит от ориентации.
\end{to_def}

\begin{to_lem}[Формула риманова объёма]
	Для (полу)римановой структуры $g$ формула
	\begin{equation*}
		\textnormal{vol}_g = \sqrt{|\det g|} \d x^1 \wedge \ldots \wedge \d x^n,
	\end{equation*}
	где $\det g$ подразумевает $\det (g_{i j})$, кореектно определяет плотность меры.
\end{to_lem}

В случае ориентированного многообразия $\textnormal{vol}_{g}$ можно считать формой высшей степени, положительной  относительно выбранной ориентации.

Для двух римановых многообразий $M$ и $N$ их произведение $M \times N$ можно  тоже считать римановым многообразием по формуле
\begin{equation*}
	g_{M \times N}(X,Y) = g_{M}(p_*X, p_*Y) + g_{N}(q_*X, q_*Y),
\end{equation*}
где $p \colon M \times N \mapsto M$ и $q \colon M \times N \mapsto N$ --- естественные проекции.

В матричном виде на произведении координат $g_{M\times N}$ будет $\oplus$ матриц $g_M $ и $g_N$. Так как детерминант прямой суммы матриц равен произведению детерминантов исходных, для риманова произведения (напр. борелевских)подмножеств:
\begin{equation*}
	X \subseteq M, Y \subseteq N \colon
	\textnormal{vol}_{M \times N} (X \times Y) = \textnormal{vol}_M X \cdot  \textnormal{vol}_N Y.
\end{equation*}
Свойство произведения в некотором смысле обосновывает естественность выбора риманова объёма.

\begin{to_tas}
	Евклидова структура на $\mathbb{R}^n$ является произведением $n$ римановых структур прямой $\mathbb{R}^1$.
\end{to_tas}

\sbsnum{50}{Частыне случаи}
Рассмотрим частный случай риманова объёма --- площадь двумерной поверхности в евклидовом пространстве, то есть интеграл от $\text{vol}_g$ по этой поверхности. Заметим, что если поверхность задана параметрически и положительно ориентированные параметры на поверхности --- $(u,v)$, то для индуцированной с $\mathbb{R}^n$ римановой структуры
\begin{equation*}
	\text{vol}_g = \sqrt{|r_u'|^2 |r_v'|^2 - (r_u' \cdot r_v')^2} \d u \wedge \d v.
\end{equation*}

В трёхмерном случае эту формулу можно продолжить как
\begin{equation*}
	\text{vol}_g = \sqrt{|r_u'|^2 |r_v'|^2 - (r_u' \cdot r_v')^2} \d u \wedge \d v = |[r_u' \times r_v']| \d u \wedge \d v.
\end{equation*}

\sbsnum{51}{Twinkle twinkle little star}
 \begin{to_def}
 	Риманова метрика на гладком $M$ --- симметричное положительно определённое невырождение тензорное поле $(g_{i j}) \in \mathbb{T}_2^0(M^n)$. 
 \end{to_def}

 \begin{to_def}
 	$\sharp \colon \mathbb{T}_1^0(M^n) \mapsto \mathbb{T}_0^1(M^n)$ (диез) --- операция поднятия индекса: $\alpha_i \mapsto g^{i j} \alpha_j$.
 \end{to_def}

 \begin{to_def}
 	$\flat \colon \mathbb{T}^1_0(M^n) \mapsto \mathbb{T}^0_1(M^n)$ (бемоль) --- операция опускания индекса: $v^i \mapsto g_{i j}v^j$.
 \end{to_def}

Таким образом форма (полу)римановой структуры может рассматриваться как отображение $g\colon T_pM \otimes T_pM \mapsto \mathbb{R}$.
Композиция $g$ и двух поднятий индексов на её аргументов даёт билинейное отображение $\tilde{g}\colon T_p^*M \otimes T_p^*M \mapsto \mathbb{R}$. 

\begin{to_tas}[Коши-Буняковский]
	$\alpha(X)^2 \leq g(X,X) \cdot \tilde{g} (\alpha,\alpha)$.
\end{to_tas}

\begin{to_def}
	В присутствии (полу)римановой структуры $g$ на ориентированном многообразии $M^n$ (чтобы $\text{vol}_{g}$) можно было считать элементом $\Omega^n(M)$) формула 
	\begin{equation*}
		\alpha \wedge *\beta = \tilde{g}(\alpha,\beta) \text{vol}_g, \; \forall \alpha \in \Omega^k(M),	
	\end{equation*}
	корректно определяет линейный оператор $*\colon \Omega^k(M) \mapsto \Omega^{n-k}(M)$ --- звёздочку Ходжа.

	Точнее можно сказать, что звёздочка определяется как композиция изоморфизмов в каждой точке:
	\begin{equation*}
		\Omega_p^k(M) \longrightarrow \left(\Omega_p^k(M)\right)^* \longrightarrow \Omega_p^{n-k}(M).
	\end{equation*}
\end{to_def}





\end{document} Ну, что? Погнали техать?\begin{to_def} 
    \textit{Обобщенная сила} $Q_k$ -- величина коэффициента $\partial q^k$ при вариации $\delta A$, то есть $\delta A = Q_k \delta q^k$.
\end{to_def}

\begin{to_thr}[Уравнения Лагранжа второго рода]
     Каждая механическая система характеризуется определенной функцией $L(q, \dot{q}, t)$. Для голономных системы с конфигурационном многообразием степени $n$, верно что
     \begin{equation*}
         \frac{d}{dt} \frac{\partial L}{\partial \dot{q}_k} - \frac{\partial L}{\partial q_k} = 0, \hspace{0.5cm} k = 1, \ldots, n.
     \end{equation*}
     Где для потенциальных систем $L = T - \Pi$. В более общем случае можно записать, что
     \begin{equation*}
         \left(\frac{d }{d t} \frac{\partial T}{\partial \dot{q}^k} - \frac{\partial T}{\partial q^k} - Q^k\right)\delta q^k = 0, \hspace{0.5cm} 
         Q^k = - \frac{\partial \Pi}{\partial q^k}.
     \end{equation*}
\end{to_thr}

\begin{proof}[$\triangle$]
Запишем второй закон Ньютона:
$
    \left(
        m_i \vc{\mathrm{w}}_i = \vc{F}_i + \vc{R}_i
    \right)\big|_{\cdot \d \vc{r}_i}
$
, где $\vc{R}_i$ -- реакции связи. Хотим записать уравнение в общековариантном виде.
То есть мы <<замораживаем>> время, так чтобы $\vc{R} \cdot \delta \vc{r} = 0$. На таких перемещениях работа реакция связи равна 0.
\begin{align*}
    \bigg[
        \sum m_i \left(\vc{\mathrm{w}}_i \cdot \frac{\partial \vc{r}_i}{\partial q^k} \right)
        -
        \left(\vc{F}_i \cdot \frac{\partial \vc{r}_i}{\partial q^k} \right)
        -
        \underbrace{
        \left(
            \vc{R}_i \cdot \frac{\partial \vc{r}_i}{\partial q^k} 
        \right)}_{\cdot \delta q^k \to 0}
    \bigg] \cdot \delta q^k &= 0;
    \\
    \left[
        \frac{d}{dt} \frac{\partial }{\partial \dot{q}^k} \sum  \frac{m_i v_i^2}{2} 
        -
        \frac{\partial }{\partial q^k} \sum \frac{m_i v_i^2}{2} -
        \sum \vc{F}_i \frac{\partial \vc{r}_i}{\partial q^k} 
    \right] \delta q^k &= 0
    , \hspace{0.5cm} \Rightarrow \hspace{0.5cm} 
    \sum_k
    \left[
        \frac{d}{dt} \frac{\partial T}{\partial \dot{q}^k} 
        - \frac{\partial T}{\partial q^k} - Q^k
    \right] \delta q^k = 0.
\end{align*}
Проблема остается в неголономных системах, где $\delta q^k$ не являются независимыми, получается, что уравнения Лагранжа справедливы для голономных систем.

Вспоминая, что
\begin{equation*}
    \delta  A = \sum_i \vc{F}_i \cdot \delta \vc{r}_i =
    \sum_i \left(\vc{F}_i \cdot \frac{\partial \vc{r}_i}{\partial q^k}\right) \delta q^k =
    \sum_k 
    \frac{\delta A_k}{\delta q^k} \delta q^k = Q_k \delta q^k.
\end{equation*}
Тогда пусть $\Pi (q, t) \colon Q^k = - \partial \Pi / \partial q^k$.  Тогда
\begin{equation*}
    \frac{d}{dt} \frac{\partial (T-\Pi)}{\partial \dot{q}^k} - \frac{\partial (T - \Pi)}{\partial q^k}  = 0,
    \hspace{0.5cm} \Rightarrow \hspace{0.5cm} 
    \frac{d}{dt} \frac{\partial L}{\partial \dot{q}^k} - \frac{\partial L}{\partial q^k} = 0, \hspace{0.5cm} k = 1, \ldots, n.
\end{equation*}
То есть получили систему уравнений на $2n$ переменных.
\end{proof}



Подставим разложение кинетической энергии в уравнения Лагранжа, оставив только слагаемые с обобщёнными ускорениями $f_j (q, \dot{q}, t) = a_{jk} \ddot{q}^j$. 
\begin{equation*}
    T = \frac{1}{2} \sum_\nu m_\nu \dot{\vc{r}}_\nu^2 = \frac{1}{2} \sum_\nu
    \left(
        \frac{\partial \vc{r}_\nu}{\partial q^j} \dot{q}^j + \frac{\partial \vc{r}_\nu}{\partial t} 
    \right)^2 = 
    \frac{1}{2} 
    \bigg[
    \underbrace{
        a_{jk} \dot{q}^j \dot{q}^k
    }_{
        2T_2
    } +
    \underbrace{
        a_j \dot{q}^j
    }_{
        2T_1
    } +
    \underbrace{
        a_0
    }_{
        2T_0
    }
    \bigg],
\end{equation*}
где коэффициенты, соответственно, равны 
\begin{equation*}
    a_{jk}(q, t) = \sum_\nu m_\nu \frac{\partial \vc{r}_\nu}{\partial q^j} \cdot \frac{\partial \vc{r}_\nu}{\partial q^k},
    \hspace{0.5cm} 
    a_j(q, t) = \sum_\nu m_\nu \frac{\partial \vc{r}_\nu}{\partial q^j} \cdot \frac{\partial \vc{r}_\nu}{\partial t},
    \hspace{0.5cm} 
    a_0 = \sum_\nu m_\nu 
    \left(
    \frac{\partial \vc{r}_\nu}{\partial t} 
            \right)^2.
\end{equation*}
Для склерономных систем $\partial \vc{r}_\nu / \partial t = 0$, соотвественно $T = a_{jk} \dot{q}^j \dot{q}^k$, при чём $a_{jk} \equiv a_{jk} (q)$.

Теперь подставим значение $T$ в уравнения Лагранжа, и получим, что
$
    a_{ik} \ddot{q}^k = f_i,
$
где $f_1 = f_1(q, \dot{q}, t)$. Уравнений в системе $n$, причём $a_{jk}$ является положительно определенной формой\footnote{
    \red{Требует отдельного доказательства.}
}, соответственно невырожденной. 

\begin{to_thr} 
    Уравнения Лагранжа второго рода разрешимы относительно обобщенных ускорений 
\end{to_thr}
Пусть есть также непотенциальные силы, часть обобщенных сил, соответстующих непотенциальным силам, обозначим $Q_i^*$, тогда
\begin{equation*}
    Q_1 = - \frac{\partial \Pi}{\partial q^i} + Q_i^*,
    \hspace{0.5cm} \Rightarrow \hspace{0.5cm} 
    \frac{d }{d t} \frac{\partial T}{\partial \dot{q}^i} - 
    \frac{\partial T}{\partial q^i} 
    =
    - \frac{\partial \Pi}{\partial q^i} + Q_i^*.
\end{equation*}
Найдём производную по времени от кинетической энергии
\begin{equation*}
    \frac{d T}{d t} = 
    \frac{\partial T}{\partial \dot{q}^i} \ddot{q}^i + \frac{\partial T}{\partial q^i} \dot{q}_i + \frac{\partial T}{\partial t}  =
    \frac{d }{d t} \left(
        \frac{\partial T}{\partial \dot{q}^i} \dot{q}^i
    \right) - \left(
        \frac{d }{d t} \frac{\partial T}{\partial \dot{q}^i} - \frac{\partial T}{\partial q^i} 
    \right) \dot{q}^i + \frac{\partial T}{\partial t}.
\end{equation*}
По \textit{теореме Эйлера об однороных функциях} для $f(x_1, \ldots, x_n)$ $k$-й степени верно что
\begin{equation*}
    \frac{\partial f}{\partial x^i} x^i = kf,
    \hspace{0.5cm} \Rightarrow \hspace{0.5cm} 
    \frac{\partial T}{\partial \dot{q}^i} \dot{q}^i = 2 T_2 + T_1.
\end{equation*}
В таком случае последнее равеноство перепишется, как
\begin{align*}
    \frac{d T}{d t} 
    &=
    \frac{d }{d t} (2 T_2 + T_1) + \frac{\partial \Pi}{\partial q^i} \dot{q}^i - Q_i^* \dot{q}^i + \frac{\partial T}{\partial t} 
    = \\ &= 
    \frac{d }{d t} (2 T_2 + 2 T_1 + 2 T_0) - \frac{d }{d t} (T_1 + 2 T_0) + \frac{d \Pi}{d t} - \frac{\partial \Pi}{\partial t} - Q_i^* \dot{q}^i + \frac{\partial T}{\partial t}.
\end{align*}
Таким образом мы доказали следующую теорему.

\begin{to_thr}
     Полная мехническая энергия голономной системы $E = T+ \Pi$ изменяется следующим образом:
     \begin{equation*}
         \frac{d E}{d t} = N^* + \frac{d }{d t} (T_1 + 2 T_0) + \frac{\partial \Pi}{\partial t} - \frac{\partial T}{\partial t}.
     \end{equation*}
     Где $N^* = Q^*_i \dot{q}^i$ -- мощность непотенциальных сил.
\end{to_thr}


\begin{to_def} 
    Голономная склерономная система с  $\Pi \equiv \Pi(q)$ называется \textit{консервативной}, при чём $d E / d t = 0$.
\end{to_def}

\subsubsection*{Гироскопические силы}


\begin{to_def} 
    Непотенициальные силы называют \textit{гироскопическими}, если их мощность равна $0$. 
\end{to_def}


Пусть $Q^*_i = \gamma_{ik} \dot{q}^k$. Если $\gamma_{ik} = -\gamma_{ki}$, то силы $Q^*_i$ гиросокопические, соответсвенно кососимметричность $\gamma_{ik}$ необходима и достаточна.

Более того, имеет место равеноство

\vspace{-25pt}
\begin{equation*}
    \sum_\nu \vc{F}_\nu \cdot \vc{v}_\nu = \sum_\nu \vc{F}_\nu \cdot
    \left(
        \frac{\partial \vc{r}_\nu}{\partial q^i} \dot{q}^i + \frac{\partial \vc{r}_\nu}{\partial t} 
    \right) 
    =
     \bigg(
     \overbrace{
     \sum_\nu \vc{F}_\nu \cdot \frac{\partial \vc{r}_\nu}{\partial q^i}
     }^{
     Q_i
     }
      \bigg)
     \dot{q}^i + \sum_\nu \vc{F}_\nu \cdot \frac{\partial \vc{r}_\nu}{\partial t},
     \hspace{0.25cm} \overset{\partial \vv{r}_\nu / \partial t = 0}{=}  \hspace{0.25cm} 
     \sum_\nu \vc{F}_\nu \cdot \vc{v}_\nu = Q_i \dot{q}^i.
\end{equation*}
Поэтому для склерономных систем $N^* = 0$ выражается в $\sum_\nu \vc{F}_\nu^* \cdot \vc{v}_\nu = 0$.

\subsubsection*{Диссипативные силы}

\begin{to_def} 
    Непотенциальные силы называются диссипативными, если их $N^* \leq 0$, но $N^* \not \equiv 0$. При $\Pi = \Pi(q)$ и диссипативности сил $d E / dt \leq 0$, тогда система называется диссипативной. В случае определенно-отрицательной $N^* (\dot{q})$ дисспация называется \textit{полной}, а в случае знакопостоянной отрицательной $N^*$ \textit{частичной}.
\end{to_def}

\begin{to_def} 
    \textit{Диссипативной функцией Рэлея} называется  положительная квадратичная форма $R$ такая, что
    \begin{equation*}
        R = \frac{1}{2} b_{ik} \dot{q}^i \dot{q}^k,
        \hspace{1cm}
        Q^*_i = - \frac{\partial R}{\partial \dot{q}^i}  = - b_{ik} \dot{q}^k.
    \end{equation*}
    Тогда для склерономной системы можность $N^*$ непотенциальных сил равна
    \begin{equation*}
        \sum_\nu \vc{F}_\nu^* \cdot \vc{v}_\nu = Q_i^* \dot{q}^i = - 2 R \leq 0.
    \end{equation*}
\end{to_def}




Путь существует функция $V (q, \dot{q}, t)$ такая, что обобщенные силы $Q_i$ определяются по формулам
\begin{equation*}
    Q_i = \frac{d }{d t} \frac{\partial V}{\partial \dot{q}^i} - \frac{\partial V}{\partial q^i}.
\end{equation*}
Тогда функция $V$ называется обобщенным потенциалом. Действительно, при $L = T - V$ уравнения движения запишутся в той же форме. Дифференцируя по времени выясним, что
\begin{equation*}
    Q_i = \frac{\partial^2 V}{\partial \dot{q}^i \partial \dot{q}^k} \ddot{q}^k + f_i,
\end{equation*}
где $f_i \equiv f_i ( q,\dot{q}, t)$. Но так как зависимость $Q_i(\ddot{q})$ это странно, то
\begin{equation*}
    V =  A_i(q, t) \dot{q}^i + V_0(q, t).
\end{equation*}
Тогда обобщенные силы
\begin{equation*}
    Q_i = \frac{d A_i}{d t} - \frac{\partial }{\partial q^i} \left(
        A_k \dot{q}^k  + V_0
    \right) = - \frac{\partial V_0}{\partial q^i} + \frac{\partial A_i}{\partial t} +
    \left(
        \frac{\partial A_i}{\partial q^k} - \frac{\partial A_k}{\partial q^i} 
    \right) \dot{q}^k.
\end{equation*}
Если $\partial A_i / \partial t = 0$, то $Q_i$ складываются из потенциальных $\partial V_0 / \partial q_i$ и гироскопических $Q_i^* = \gamma_{ik} \dot{q}^k$, где $\gamma_{ik} = \partial_k A_i - \partial_i A_k$. Если система склерономна и $V_0 \neq V_0(t)$, то $T+V_0$ остается постоянной.

В случае существования обобщенного потенциала $L$ всё так же многочлен второй степени относительно $q, \ \dot{q}$, при чём $L_2 = T_2$, так что уравнения остаются разрешимы относительно обобщенных ускорений.









\begin{to_def} 
    \textit{Действием по Гамильтону} называют функционал вида
    \begin{equation*}
         S = \int_{t_0}^{t_1} L(\gamma(t), \dot{\gamma}(t), t) \d t.
    \end{equation*} 
    Переходя к однопараметрическому семейству кривых $\gamma(\alpha, t)$ получим \textit{вариацию действия}
    \begin{equation*}
        S = \int_{t_0}^{t_1} L(\gamma(\alpha, t), \dot{\gamma}(\alpha, t), t) \d t, 
        \hspace{0.5cm} 
        \delta S = \frac{d S}{d \alpha} \delta \alpha.
    \end{equation*}
\end{to_def}


\begin{to_thr}[принцип Гамильтона]
    Кривая $\gamma(\alpha, t)$ является экстремалью действия тогда и только тогда, когда является решением уравнений Лагранжа
     \begin{equation*}
         \delta S = 0
         \hspace{0.5cm} \Leftrightarrow \hspace{0.5cm} 
         \gamma(\alpha, t) \in \textnormal{Sol}\,
         \left(
     \frac{d}{dt} \frac{\partial L}{\partial \dot{q}^k} - \frac{\partial L}{\partial q^k} = 0
         \right)
         .
     \end{equation*}
\end{to_thr}


\begin{proof}[$\triangle$]
    Давайте просто проварьируем Лагранжиан, тогда
    \begin{align*}
        \delta S 
        &=
         \int_{t_0}^{t_1} 
        \left(
            \frac{\partial L}{\partial q^i} \frac{\partial q^i}{\partial \alpha} +
            \frac{\partial L}{\partial \dot{q}^i} \frac{\partial \dot{q}^i}{\partial \alpha}  
        \right) \delta \alpha \d t 
        =
        \int_{t_0}^{t_1} \left(
            \frac{\partial L}{\partial q^i} \delta q^i + \frac{\partial L}{\partial \dot{q}^i} \delta \dot{q}^i
        \right) \d t
        =
        \frac{\partial L}{\partial \dot{q}} \partial q \bigg|_{t_1}^{t_2}
        + \int_{t_1}^{t_2}
        \left(
            \frac{\partial L}{\partial q^i} - \frac{d }{d t} \frac{\partial L}{\partial \dot{q}^i} 
        \right) \,\delta q^i \d t
        = 0.
    \end{align*}
    таким образом уравнения Лагранжа выполнены. 
\end{proof}

% в первых билет засунуть конфигурационное многообразие
\definecolor{grey}{HTML}{666666}
\definecolor{linkcolor}{HTML}{0000CC}
\definecolor{urlcolor}{HTML}{006600}
\hypersetup{
    pdfstartview=FitH,  
    linkcolor=linkcolor,
    urlcolor=urlcolor, 
    colorlinks=true,
    citecolor=blue}% add (renew) commands

\renewcommand{\Im}{\mathop{\mathrm{Im}}\nolimits}
\renewcommand{\Re}{\mathop{\mathrm{Re}}\nolimits}
\renewcommand{\d}{\, d}
\renewcommand{\leq}{\leqslant}
\renewcommand{\geq}{\geqslant}
\renewcommand{\L}{\mathcal{L}}


\newcommand{\vc}[1]{\mbox{\boldmath $#1$}}
\newcommand{\T}{^{\text{T}}}

\newcommand{\incfig}[1]{%
    \def\svgwidth{\columnwidth}
    \import{./figures/}{#1.pdf_tex}
}

\newcommand{\diag}{\mathop{\mathrm{diag}}\nolimits}
\newcommand{\id}{\mathop{\mathrm{id}}\nolimits}
\newcommand{\grad}{\mathop{\mathrm{grad}}\nolimits}
\renewcommand{\div}{\mathop{\mathrm{div}}\nolimits}
\newcommand{\rot}{\mathop{\mathrm{rot}}\nolimits}
\newcommand{\Ker}{\mathop{\mathrm{Ker}}\nolimits}
\newcommand{\Spec}{\mathop{\mathrm{Spec}}\nolimits}
\newcommand{\sign}{\mathop{\mathrm{sign}}\nolimits}
\newcommand{\tr}{\mathop{\mathrm{tr}}\nolimits}
\newcommand{\rg}{\mathop{\mathrm{rg}}\nolimits}

\newcommand{\const}{\text{const}}
\newcommand{\red}[1]{\textcolor{red}{#1}}
\newcommand{\xmark}{\ding{55}}

\newcommand{\sbsnum}[2]{
    \setcounter{subsection}{\the\numexpr #1 - 1 \relax}
    \subsection{#2}
}
\newcommand{\secnum}[2]{
    \section*{#2}
    \setcounter{section}{#1}
    \addcontentsline{toc}{section}{#2}
}


\newcommand{\set}[2]{{#1}^{1}, \ldots, {#1}^{#2}}
\newtheorem{to_thr}{Thr}[subsection]
\newtheorem{to_suj}[to_thr]{Suj}
\newtheorem{to_lem}[to_thr]{Lem}
\newtheorem{to_com}[to_thr]{Com}
\newtheorem{to_con}[to_thr]{Con}
\theoremstyle{definition}
\newtheorem{to_def}[to_thr]{Def}
\newtheorem{to_tas}[to_thr]{Task}
\newtheorem{to_exm}[to_thr]{Exm}


\newenvironment{itemize*}
{
    \begin{itemize}
        \setlength{\itemsep}{1pt}
        \setlength{\parskip}{1pt}}
    {\end{itemize}
}

\newenvironment{enumerate*}
{
    \begin{enumerate}
        \setlength{\itemsep}{1pt}
        \setlength{\parskip}{1pt}}
    {\end{enumerate}
}

\newenvironment{description*}
{
    \begin{description}
        \setlength{\itemsep}{1pt}
        \setlength{\parskip}{1pt}}
    {\end{description}
}% add page header

\pagestyle{fancy}
\fancyhf{}
\fancyhead[RE,LO]{\textsc{Ф\raisebox{-1.5pt}{и}з\TeX}}
\fancyhead[LE,RO]{Ж\raisebox{-1.5pt}{и}К}
% \fancyhead[CO,CE]{\leftmark}
\fancyfoot[LE,RO]{\textcolor{grey}{\texttt{\thepage}}}

% document's head
% \phantom{42}

\begin{center}
    \LARGE \textsc{Билеты к экзамену по <<Аналитической Механике>>, ФОПФ}
\end{center}

\hrule

\phantom{42}

\begin{flushright}
    \begin{tabular}{rr}
    % written by:
        \textbf{Авторы}: 
        & Хоружий Кирилл \\
        & Примак Евгений \\
        &\\
    % date:
        \textbf{От}: &
        \textit{\today}\\
    \end{tabular}
\end{flushright}

\thispagestyle{empty}
\tableofcontents
\newpage% input files

% document's head
\vspace{2cm}

\begin{center}
    \LARGE \textsc{Конспект второго тома курса теоретической физики <<Теория поля>>}
\end{center}

\hrule

\begin{flushright}
    \begin{tabular}{rr}
    % written by:
        \textbf{Авторы}: 
        & Хоружий Кирилл \\
        & Примак Евгений \\
        &\\
    % date:
        \textbf{От}: &
        \textit{\today}\\
    \end{tabular}
\end{flushright}

\thispagestyle{empty}
\tableofcontents
\newpage


\sbsnum{31}{Уравнение Лагранжа второго рода}
\begin{to_lem}[Разбиение единицы в окрестности компакта на многообразии]
     Пусть $M$ -- гладкое многообразие, а $K \subseteq M$ -- его компактное подмножество. Для любого покрытия $\{U_\alpha\}_\alpha$ компакта $K$ открытыми множествами найдётся набор неотрицательных гладких функций $\{\rho_\alpha\}_\alpha$ с компактными носителями $\mathrm{supp}\, \rho_\alpha$ таких, что
\begin{equation*}
    \forall \alpha \ \textnormal{supp}\, \rho_\alpha \subset U_\alpha,
\end{equation*}
    только конечное число из них отлично от нуля и
    $\sum_\alpha \rho_\alpha (x) \equiv 1$
    в некоторой окрестности $K$.
\end{to_lem}

\begin{to_def} 
    Интеграл дифференциальной формы $\nu \in \Omega_{\text{c}}^n (M)$ с компактным носителем по ориентированному $n$-мерному многообразию $M$ определяется с помощью разбиения единицы в окрестности носителя $\nu$ 
    \begin{equation*}
         \rho_1 + \ldots + \rho_m = 1,
     \end{equation*} 
     подчиненного некоторому набору положительно ориентрированных карт как
     \begin{equation*}
         \int_M \nu = \sum_i \int_M \rho_i \nu_i,
     \end{equation*}
     где интегралы справа рассматриваются в координатных картах, содержащих носители соответствующих $\rho_i$.
\end{to_def}

\begin{to_lem} 
    Определение интеграла не зависит от выбора системы положительных карт в данной ориентации и подчиненного им разбиения единциы. 
\end{to_lem}





\sbsnum{32}{Разрешимость уравнений Лагранжа}
Подставим разложение кинетической энергии в уравнения Лагранжа, оставив только слагаемые с обобщёнными ускорениями $f_j (q, \dot{q}, t) = a_{jk} \ddot{q}^j$. 
\begin{equation*}
    T = \frac{1}{2} \sum_\nu m_\nu \dot{\vc{r}}_\nu^2 = \frac{1}{2} \sum_\nu
    \left(
        \frac{\partial \vc{r}_\nu}{\partial q^j} \dot{q}^j + \frac{\partial \vc{r}_\nu}{\partial t} 
    \right)^2 = 
    \frac{1}{2} 
    \bigg[
    \underbrace{
        a_{jk} \dot{q}^j \dot{q}^k
    }_{
        2T_2
    } +
    \underbrace{
        a_j \dot{q}^j
    }_{
        2T_1
    } +
    \underbrace{
        a_0
    }_{
        2T_0
    }
    \bigg],
\end{equation*}
где коэффициенты, соответственно, равны 
\begin{equation*}
    a_{jk}(q, t) = \sum_\nu m_\nu \frac{\partial \vc{r}_\nu}{\partial q^j} \cdot \frac{\partial \vc{r}_\nu}{\partial q^k},
    \hspace{0.5cm} 
    a_j(q, t) = \sum_\nu m_\nu \frac{\partial \vc{r}_\nu}{\partial q^j} \cdot \frac{\partial \vc{r}_\nu}{\partial t},
    \hspace{0.5cm} 
    a_0 = \sum_\nu m_\nu 
    \left(
    \frac{\partial \vc{r}_\nu}{\partial t} 
            \right)^2.
\end{equation*}
Для склерономных систем $\partial \vc{r}_\nu / \partial t = 0$, соотвественно $T = a_{jk} \dot{q}^j \dot{q}^k$, при чём $a_{jk} \equiv a_{jk} (q)$.

Теперь подставим значение $T$ в уравнения Лагранжа, и получим, что
$
    a_{ik} \ddot{q}^k = f_i,
$
где $f_1 = f_1(q, \dot{q}, t)$. Уравнений в системе $n$, причём $a_{jk}$ является положительно определенной формой\footnote{
    \red{Требует отдельного доказательства.}
}, соответственно невырожденной. 

\begin{to_thr} 
    Уравнения Лагранжа второго рода разрешимы относительно обобщенных ускорений 
\end{to_thr}


\sbsnum{33}{Изменение полной мехнической энергии голономной системы}
 % \renewcommand{\labelenumi}{\Roman{enumii}}
\begin{enumerate}[label={$\mathbb{R}^\text{\arabic*}$.}]
    \item Формула Стокса для ориентированной кривой с началом в точке $p$ и концом в точке $q$ сводится к 
    \begin{equation*}
        \int_\gamma \d f = f(q) - f(p).
    \end{equation*}
    \item Для компактного множества $G \subset \mathbb{R}^2$ с гладкой границей, ориентированного так, что при движении по $\partial G$ множество $G$ оказывается слева, верна \textit{формула Грина}
    \begin{equation*}
        \int_{\partial G} P \d x + Q \d y = \int_G
        \left(
            \frac{\partial Q}{\partial x} - \frac{\partial P}{\partial y} 
        \right) \d x \wedge d y.
    \end{equation*}
    \item Для компактного множества $G \subset \mathbb{R}^3$ с гладкой границей (край в $\mathbb{R}^3$) верна \textit{формула Гаусса-Остроградского}
    \begin{equation*}
        \int_{\partial G} P \d y \wedge d z 
        + Q \d z \wedge dx 
        + R \d x \wedge dy = 
        \int_G
        \left(
            \frac{\partial P}{\partial x} +
            \frac{\partial Q}{\partial y} +
            \frac{\partial R}{\partial z} 
        \right) \d x \wedge d y \wedge d z.
    \end{equation*}
\end{enumerate}


Кривую можно считать не бесконечно гладкой, а всего лишь кусочно непрерывно дифференцируемой, формула всё равно остаётся верной. С помощью предельного перехода также обобщается случай с $\simeq \mathbb{R}^2$ до множества с кусочно $C^2$ границей.

Вообще формула Стокса верна не только для вложенных двумерных многообразий, но и для всякого образа гладкого отображения $f \colon D \mapsto  \mathbb{R}^3$ области $D \subset \mathbb{R}^2$ с кусочно гладкой границей, если интегралы мы понимаем как интегралы обратных образов $f^*(\alpha)$ и $f^*(d\alpha)$ по $\partial D$ и $D$ соответственно. Для практических применений полезно ослабить условие гладкости $f$ до $C^2$ (в интеграле, в координатном представлении, используются производные $f$ не более чем первого порядка).


% Кривую 



\sbsnum{34}{Обобщенный потенциал и первые интегралы лагранжевых систем}
Физический потенциал силового поля в математический терминах означает поиск $f \in C^\infty (M) \colon \d f = \alpha$ для заданной силы $\alpha \in \Omega^1(M)$

\begin{to_thr}
	Необходимым и достаточным условием наличия потенциала у непрерывной $\alpha \in \Omega^1(M)$, для гладкого $M$, является независимость $\int_\gamma \alpha$ от выбора между двумя точками кусочно-гладкой кривой $\gamma$.

	Эквивалентно можно потребовать равенства нулю интегралов по всем замкнутым кусочно-гладким кривым.
	\label{thr_7.1}
\end{to_thr}

Удобным на практике необходимым условием существования потенциала у $\alpha \in \Omega^1(M)$ является $\d \alpha = 0$ (т.к. $\d (\d u) = 0$).
Однако этого не достаточно, так например в открытой $U = \mathbb{R}^2 \backslash \{ 0 \}$:
\begin{equation*}
	\alpha = \frac{x \d y - y \d x}{x^2 + y^2}
	\hspace*{0.5 cm} \leadsto \hspace*{0.5 cm}
	\d \alpha = 0,
	\hspace*{0.5 cm} \text{ но } \hspace*{0.5 cm}
	\oint_{S^1} \alpha = 2 \pi.
\end{equation*}

Далее в качестве упражнений оставлены следующие важные замечания:

\begin{to_tas}[Порядок точки относительно кривой]
Для замкнутой кусочно-гладкой $\gamma \in \mathbb{R}^2$, не проходящей через начало координат определим порядок начала координат относительно кривой:
\begin{equation*}
	w (\gamma, 0) - \frac{1}{2 \pi} \int_{\gamma} \frac{x \d y - y \d x}{x^2 + y^2},
\end{equation*}
и он не меняется при непрерывных деформациях кривой, при которых она не проходит через начало координат.
\end{to_tas}

\begin{to_tas}
	Порядок начала координат относительно кривой является целым.
\end{to_tas}

\begin{to_tas}
	Порядок начала координат относительно не проходящей через него нечётной кривой является нечётным числом. ($\gamma \colon \mathbb{S}^1 \rightarrow \mathbb{R}^2, \; \gamma(-u) = - \gamma(u)$).
\end{to_tas}

\begin{to_tas}
	Для замкнутой кривой на плоскости с всюду не нулевой скоростью $\int k(s) \d s = 2 \pi N, \; N \in \mathbb{Z}$.
\end{to_tas}

\begin{to_tas}[Лемма Жордана]
	Замкнутая кусочно-гладкая кривая $\gamma \subset \mathbb{R}^2$ без самопересечений делит плоскость на две связные части внутреннюю и внешнюю (можно усложнить и сформулировать для непрерывных кривых).
\end{to_tas}


% \sbsnum{36}{Принцип наименьшего действия}
% В силу теоремы Пуанкаре (\ref{thr_poin}) любая замкнутая форма на многообразии локально точна, однако склеивать их в точную на всём пространство нам будут мешать дырки, как это случалось в задаче из нашего задания.
Связь между устройством многообразия и взаимоотношением  замкнутых и точных форм на нём описывается группами (ко)гомологий де Рама.

Замкнутые и точные формы на $M$ образуют линейные пространства $Z^k(M)$ и $B^k(M)$ соответственно.

\begin{to_def}[Когомологии де Рама] или группа $k$-мерных когомологий многообразия $M$:
	\begin{equation*}
		H^k(M) = Z^k(M)/B^k(M)
	\end{equation*}
\end{to_def}

\begin{to_def}
	Если формы $\alpha_1, \alpha_2$ отличаются на точную форму, то говорят, что они гомологичны.
	\label{def_7.16}
\end{to_def}
Таким образом если замкнутые $\alpha_1, \alpha_2$ гомологичны, то они лежат в одном классе когомологии.

По скольку $Z^k (M)$ есть $\Ker d \colon \Omega^k(M) \rightarrow \Omega^{k+1}(M)$, а $B^k (M)$ есть $\Im d \colon \Omega^{k-1}(M) \rightarrow \Omega^{k}(M)$, то часто переписывают:

\begin{to_def}
	Когомологии де Рама гладкого $M$ --- это факторпространства
	\begin{equation*}
		H_{DR}^k (M) = \frac{\Ker d \colon \Omega^k(M) \rightarrow \Omega^{k+1}(M)}{\Im d \colon \Omega^{k-1}(M) \rightarrow \Omega^{k}(M)}.
	\end{equation*}
	\label{def_7.17}
\end{to_def}

\begin{to_lem}[Лемма Пуанкаре]
	\begin{equation*}
		H^p(\mathbb{R}^k) = 0 \text{ при } k>0
		\hspace*{1 cm}  \hspace*{1 cm}
		H^k(\mathbb{R}^k) \sim \mathbb{R} \text{ при } k=0
	\end{equation*}
	\label{lem_7.19}
\end{to_lem}\newcommand{\dmat}[4]{
  \ifthenelse{
    \equal{#1}{3}
  }{
\begin{pmatrix}
    #2 & 0 & 0 \\
    0 & #3 & 0 \\
    0 & 0 & #4 \\
\end{pmatrix}
  }{
  \ifthenelse{
      \equal{#1}{2}
    }{
  \begin{pmatrix}
      #2 & 0 \\
      0 & #3 \\
  \end{pmatrix}
    }{
      \text{\textcolor{red}{error}}
    }
  }
}

\newcommand{\skmat}[4]{
  \ifthenelse{
    \equal{#1}{3}
  }{
\begin{pmatrix}
    0 & -#4 & #3 \\
    #4 & 0 & -#2 \\
    -#3 & #2 & 0 \\
\end{pmatrix}
  }{
  \ifthenelse{
      \equal{#1}{2}
    }{
  \begin{pmatrix}
      0 & #2 \\
      -#2 & 0 \\
  \end{pmatrix}
    }{
      \text{\textcolor{red}{error}}
    }
  }
}
\usepackage[T2A]{fontenc}                   %!? закрепляет внутреннюю кодировку LaTeX
\usepackage[utf8]{inputenc}                 %!  закрепляет кодировку utf8
\usepackage[english,russian]{babel}         %!  подключает русский и английский
\usepackage[margin=1.7cm]{geometry}         %!  фиксирует оступ на 2cm

\usepackage[unicode, pdftex]{hyperref}      %!  оглавление для панели навигации по PDF-документу + гиперссылки

\usepackage{amsthm}                         %!  newtheorem и их сквозная нумерация
\usepackage{hypcap}                         %?  адресация на картинку, а не на подпись к ней
\usepackage{caption}                        %-  позволяет корректировать caption 
\usepackage{fancyhdr}                       %   добавить верхний и нижний колонтитул
\usepackage{wrapfig}                        %!  обтекание таблиц и рисунков

\usepackage{amsmath}                        %!  |
\usepackage{amssymb,textcomp, esvect,esint} %!  |важно для формул 
\usepackage{amsfonts}                       %!  математические шрифты
\usepackage{mathrsfs}                       %  добавит красивые E, H, L
\usepackage{ulem}                           %!  перечеркивание текста
\usepackage{abraces}                        %?  фигурные скобки сверху или снизу текста
\usepackage{pifont}                         %!  нужен для крестика
\usepackage{cancel}                         %!  аутентичное перечеркивание текста
\usepackage{esvect}                         %  добавит вектора стрелочками

\usepackage{graphicx}                       %?  графическое изменение текста
\usepackage{indentfirst}                    %   добавить indent перед первым параграфом
\usepackage{xcolor}                         %   добавляет цвета
\usepackage{enumitem}                       %!  задание макета перечня.

\usepackage{booktabs}                       %!  добавляет книжные линии в таблицы
\usepackage{multirow}                       %   объединение ячеек в таблицах

\usepackage{tikz}                           %!  высокоуровневые рисунки (кружочек)

\usepackage{import}                         %   |
\usepackage{xifthen}                        %   |
\usepackage{pdfpages}                       %   | вставка рисунков pdf_tex
\usepackage{transparent}                    %   |

\setlength{\headheight}{12.52pt}            % избегать warning

\usepackage{chngcntr}
\renewcommand\thesubsection{\arabic{subsection}}
\counterwithout{equation}{section}

\renewcommand{\theequation}{\arabic{equation}}
% \usepackage[upint]{stix}% file's preambule
%%%%%%%%%%%%%%%%%%%



% connect packages
\usepackage[T2A]{fontenc}
\usepackage[utf8]{inputenc}
\usepackage[english]{babel}
\usepackage{hyperref}     % ТАК_НУЖНО
\hypersetup{unicode=true} % ТАК_НУЖНО
\usepackage{amsmath}
\usepackage{amssymb,textcomp, esvect,esint}
\usepackage{amsfonts}
\usepackage{amsthm}
\usepackage{graphicx}
\usepackage{indentfirst}
\usepackage{xcolor}
\usepackage{enumitem}
\usepackage{booktabs}
\usepackage{caption}
\usepackage{listings}
\usepackage{tikz}

\usepackage{esvect}
\usepackage{movie15}
\usepackage{animate}

% create environment

\newtheorem{to_thr}{Thr}[subsection]
\newtheorem{to_suj}[to_thr]{Suj}
\newtheorem{to_lem}[to_thr]{Lem}
\newtheorem{to_com}[to_thr]{Com}
\newtheorem{to_con}[to_thr]{Con}
\theoremstyle{definition}
\newtheorem{to_def}[to_thr]{Def}


\newenvironment{itemize*}
{
    \begin{itemize}
        \setlength{\itemsep}{1pt}
        \setlength{\parskip}{1pt}}
    {\end{itemize}
}

\newenvironment{enumerate*}
{
    \begin{enumerate}
        \setlength{\itemsep}{1pt}
        \setlength{\parskip}{1pt}}
    {\end{enumerate}
}

\newenvironment{description*}
{
    \begin{description}
        \setlength{\itemsep}{1pt}
        \setlength{\parskip}{1pt}}
    {\end{description}
}

% document palette

\definecolor{grey}{HTML}{666666}
\definecolor{linkcolor}{HTML}{0000CC}
\definecolor{urlcolor}{HTML}{006600}
\hypersetup{
    pdfstartview=FitH,  
    linkcolor=linkcolor,
    urlcolor=urlcolor, 
    colorlinks=true,
    citecolor=blue}

% add (renew) commands
% add (renew) commands

\renewcommand{\Im}{\mathop{\mathrm{Im}}\nolimits}
\renewcommand{\Re}{\mathop{\mathrm{Re}}\nolimits}
\renewcommand{\d}{\, d}
\renewcommand{\leq}{\leqslant}
\renewcommand{\geq}{\geqslant}
\renewcommand{\L}{\mathcal{L}}


\newcommand{\vc}[1]{\mbox{\boldmath $#1$}}
\newcommand{\T}{^{\text{T}}}

\newcommand{\incfig}[1]{%
    \def\svgwidth{\columnwidth}
    \import{./figures/}{#1.pdf_tex}
}

\newcommand{\diag}{\mathop{\mathrm{diag}}\nolimits}
\newcommand{\id}{\mathop{\mathrm{id}}\nolimits}
\newcommand{\grad}{\mathop{\mathrm{grad}}\nolimits}
\renewcommand{\div}{\mathop{\mathrm{div}}\nolimits}
\newcommand{\rot}{\mathop{\mathrm{rot}}\nolimits}
\newcommand{\Ker}{\mathop{\mathrm{Ker}}\nolimits}
\newcommand{\Spec}{\mathop{\mathrm{Spec}}\nolimits}
\newcommand{\sign}{\mathop{\mathrm{sign}}\nolimits}
\newcommand{\tr}{\mathop{\mathrm{tr}}\nolimits}
\newcommand{\rg}{\mathop{\mathrm{rg}}\nolimits}

\newcommand{\const}{\text{const}}
\newcommand{\red}[1]{\textcolor{red}{#1}}
\newcommand{\xmark}{\ding{55}}

\newcommand{\sbsnum}[2]{
    \setcounter{subsection}{\the\numexpr #1 - 1 \relax}
    \subsection{#2}
}
\newcommand{\secnum}[2]{
    \section*{#2}
    \setcounter{section}{#1}
    \addcontentsline{toc}{section}{#2}
}


\newcommand{\set}[2]{{#1}^{1}, \ldots, {#1}^{#2}}

\catcode`\^ = 13 \def^#1{\sp{#1}{}} % чтобы штрих нормально работал

% add page header
% add page header

\pagestyle{fancy}
\fancyhf{}
% \fancyhead[RE,LO]{\textsc{Ф\raisebox{-1.5pt}{и}з\TeX}}
\fancyhead[LE,RO]{Ж\raisebox{-1.5pt}{и}К}
% \fancyhead[CO,CE]{\leftmark}
\fancyfoot[LE,RO]{\textcolor{grey}{\texttt{\thepage}}}



% matrixes shortcuts 
\newcommand{\dmat}[4]{
  \ifthenelse{
    \equal{#1}{3}
  }{
\begin{pmatrix}
    #2 & 0 & 0 \\
    0 & #3 & 0 \\
    0 & 0 & #4 \\
\end{pmatrix}
  }{
  \ifthenelse{
      \equal{#1}{2}
    }{
  \begin{pmatrix}
      #2 & 0 \\
      0 & #3 \\
  \end{pmatrix}
    }{
      \text{\textcolor{red}{error}}
    }
  }
}

\newcommand{\skmat}[4]{
  \ifthenelse{
    \equal{#1}{3}
  }{
\begin{pmatrix}
    0 & -#4 & #3 \\
    #4 & 0 & -#2 \\
    -#3 & #2 & 0 \\
\end{pmatrix}
  }{
  \ifthenelse{
      \equal{#1}{2}
    }{
  \begin{pmatrix}
      0 & #2 \\
      -#2 & 0 \\
  \end{pmatrix}
    }{
      \text{\textcolor{red}{error}}
    }
  }
}

% additional symbols and commands


\DeclareRobustCommand{\tmpsim}{ %%%%%%%%%%%%%% ~ < %%%%%%%%%%%%%%%%%%%
  \mathbin{\text{
      \raisebox{-1pt}{
            \hspace{-4.5pt} \rotatebox{-26}{\scalebox{0.8}[0.7]{$\sim$}}
        }
  }}
}
\def\lesim{{
    \setbox0\hbox{$\ <\ $}
    \rlap{\hbox to \wd0{\hss$\tmpsim$\hss}}\box0
}}
%%%%%%%%%%%%%%%%%%%%%%%%%%%%%%%%%%%%%%%%%%%%%%%%%%%%%%%%%%%%%%%%%%%%%%


\def\letuscom{%%%%%%%%%%%%%%%%%%%%%% ПУСТЬ %%%%%%%%%%%%%%%%%%%%%%%%%%
\mathord{\setbox0=\hbox{$\exists$}%
     \hbox{\kern 0.125\wd0%
           \vbox to \ht0{%
              \hrule width 0.75\wd0%
              \vfill%
              \hrule width 0.75\wd0}%
           \vrule height \ht0%
           \kern 0.125\wd0}%
   }%
}
\newcommand{\letus}{\raisebox{-1.2pt}{$\letuscom$}}
%%%%%%%%%%%%%%%%%%%%%%%%%%%%%%%%%%%%%%%%%%%%%%%%%%%%%%%%%%%%%%%%%%%%%%


\usepackage{arydshln} %%%%%%%%%%%%%%% ЛИНИИ В МАТРИЧКЕ %%%%%%%%%%%%%%%
\makeatletter
  \renewcommand*\env@matrix[1][*\c@MaxMatrixCols c]{%
    \hskip -\arraycolsep
    \let\@ifnextchar\new@ifnextchar
  \array{#1}}
\makeatother
%%%%%%%%%%%%%%%%%%%%%%%%%%%%%%%%%%%%%%%%%%%%%%%%%%%%%%%%%%%%%%%%%%%%%%


\makeatletter %%%%%%%%%%%%%%% КРУЖОЧЕК %%%%%%%%%%%%%%%%%%%%%%%%%%%%%%%
\newcommand*{\encircled}[1]{\relax\ifmmode\mathpalette
\@encircled@math{#1}\else\@encircled{#1}\fi}
\newcommand*{\@encircled@math}[2]{\@encircled{$\m@th#1#2$}}
\newcommand*{\@encircled}[1]{%
  \tikz[baseline,anchor=base]{\node[draw,circle,outer sep=0pt,
                                        inner sep=.2ex] {#1};}}
\makeatother
%%%%%%%%%%%%%%%%%%%%%%%%%%%%%%%%%%%%%%%%%%%%%%%%%%%%%%%%%%%%%%%%%%%%%%


\makeatletter
\def\upintkern@{\mkern-7mu\mathchoice{\mkern-3.5mu}{}{}{}}
\def\upintdots@{\mathchoice{\mkern-4mu\@cdots\mkern-4mu}%
 {{\cdotp}\mkern1.5mu{\cdotp}\mkern1.5mu{\cdotp}}%
 {{\cdotp}\mkern1mu{\cdotp}\mkern1mu{\cdotp}}%
 {{\cdotp}\mkern1mu{\cdotp}\mkern1mu{\cdotp}}}
\newcommand{\upiint}{\DOTSI\protect\UpMultiIntegral{2}}
\newcommand{\upiiint}{\DOTSI\protect\UpMultiIntegral{3}}
\newcommand{\upiiiint}{\DOTSI\protect\UpMultiIntegral{4}}
\newcommand{\upidotsint}{\DOTSI\protect\UpMultiIntegral{0}}
\newcommand{\UpMultiIntegral}[1]{%
  \edef\ints@c{\noexpand\upintop
    \ifnum#1=\z@\noexpand\upintdots@\else\noexpand\upintkern@\fi
    \ifnum#1>\tw@\noexpand\upintop\noexpand\upintkern@\fi
    \ifnum#1>\thr@@\noexpand\upintop\noexpand\upintkern@\fi
    \noexpand\upintop
    \noexpand\ilimits@
  }%
  \futurelet\@let@token\ints@a
}
\makeatother

\DeclareFontFamily{OMX}{mdbch}{}
\DeclareFontShape{OMX}{mdbch}{m}{n}{ <->s * [0.8]  mdbchr7v }{}
\DeclareFontShape{OMX}{mdbch}{b}{n}{ <->s * [0.8]  mdbchb7v }{}
\DeclareFontShape{OMX}{mdbch}{bx}{n}{<->ssub * mdbch/b/n}{}

\DeclareSymbolFont{uplargesymbols}{OMX}{mdbch}{m}{n}
\SetSymbolFont{uplargesymbols}{bold}{OMX}{mdbch}{b}{n}
\DeclareMathSymbol{\upintop}{\mathop}{uplargesymbols}{82}
\DeclareMathSymbol{\upointop}{\mathop}{uplargesymbols}{"48}

\DeclareFontEncoding{MDB}{}{}
\DeclareFontFamily{MDB}{mdbch}{}
\DeclareFontShape{MDB}{mdbch}{m}{n}{ <->s * [0.8]  mdbchrmb }{}
\DeclareFontShape{MDB}{mdbch}{b}{n}{ <->s * [0.8]  mdbchbmb }{}
\DeclareFontShape{MDB}{mdbch}{bx}{n}{<->ssub * mdbch/b/n}{}
\DeclareFontSubstitution{MDB}{cmr}{m}{n}
\DeclareSymbolFont{mathdesignB}{MDB}{mdbch}{m}{n}%
\SetSymbolFont{mathdesignB}{bold}{MDB}{mdbch}{b}{n}%
\DeclareMathSymbol{\upintclockwise}{\mathop}{mathdesignB}{128}
\DeclareMathSymbol{\upointclockwise}{\mathop}{mathdesignB}{130}
\DeclareMathSymbol{\upointctrclockwise}{\mathop}{mathdesignB}{132}
\DeclareMathSymbol{\upoiint}{\mathop}{mathdesignB}{134}
\DeclareMathSymbol{\upoiiint}{\mathop}{mathdesignB}{136}

\makeatletter
\renewcommand{\int}{\DOTSI\upintop\ilimits@}
\renewcommand{\oint}{\DOTSI\upointop\ilimits@}
\makeatother









% set skip of equation length 

\setlength{\abovedisplayskip}{3pt}
\setlength{\abovedisplayshortskip}{3pt}
\setlength{\belowdisplayskip}{3pt}
\setlength{\belowdisplayshortskip}{3pt}

\numberwithin{equation}{section}

\DeclareRobustCommand{\tmpsim}{ %%%%%%%%%%%%%% ~ < %%%%%%%%%%%%%%%%%%%
  \mathbin{\text{
      \raisebox{-1pt}{
            \hspace{-4.5pt} \rotatebox{-26}{\scalebox{0.8}[0.7]{$\sim$}}
        }
  }}
}
\def\lesim{{
    \setbox0\hbox{$\ <\ $}
    \rlap{\hbox to \wd0{\hss$\tmpsim$\hss}}\box0
}}
%%%%%%%%%%%%%%%%%%%%%%%%%%%%%%%%%%%%%%%%%%%%%%%%%%%%%%%%%%%%%%%%%%%%%%


\def\letuscom{%%%%%%%%%%%%%%%%%%%%%% ПУСТЬ %%%%%%%%%%%%%%%%%%%%%%%%%%
\mathord{\setbox0=\hbox{$\exists$}%
     \hbox{\kern 0.125\wd0%
           \vbox to \ht0{%
              \hrule width 0.75\wd0%
              \vfill%
              \hrule width 0.75\wd0}%
           \vrule height \ht0%
           \kern 0.125\wd0}%
   }%
}
\newcommand{\letus}{\raisebox{-1.2pt}{$\letuscom$}}
%%%%%%%%%%%%%%%%%%%%%%%%%%%%%%%%%%%%%%%%%%%%%%%%%%%%%%%%%%%%%%%%%%%%%%


\usepackage{arydshln} %%%%%%%%%%%%%%% ЛИНИИ В МАТРИЧКЕ %%%%%%%%%%%%%%%
\makeatletter
  \renewcommand*\env@matrix[1][*\c@MaxMatrixCols c]{%
    \hskip -\arraycolsep
    \let\@ifnextchar\new@ifnextchar
  \array{#1}}
\makeatother
%%%%%%%%%%%%%%%%%%%%%%%%%%%%%%%%%%%%%%%%%%%%%%%%%%%%%%%%%%%%%%%%%%%%%%


\makeatletter %%%%%%%%%%%%%%% КРУЖОЧЕК %%%%%%%%%%%%%%%%%%%%%%%%%%%%%%%
\newcommand*{\encircled}[1]{\relax\ifmmode\mathpalette
\@encircled@math{#1}\else\@encircled{#1}\fi}
\newcommand*{\@encircled@math}[2]{\@encircled{$\m@th#1#2$}}
\newcommand*{\@encircled}[1]{%
  \tikz[baseline,anchor=base]{\node[draw,circle,outer sep=0pt,
                                        inner sep=.2ex] {#1};}}
\makeatother
%%%%%%%%%%%%%%%%%%%%%%%%%%%%%%%%%%%%%%%%%%%%%%%%%%%%%%%%%%%%%%%%%%%%%%


